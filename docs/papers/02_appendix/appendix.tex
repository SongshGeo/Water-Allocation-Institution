
\documentclass{article}
\usepackage{hyperref}
\usepackage{graphicx}
\usepackage{setspace}
\usepackage[a4paper, total={7in, 10in}]{geometry}
\usepackage{amsmath}

\setstretch{1.523}

\newtheorem{ass}{Assumption}
\newtheorem{case}{Case}
%% make sure you have the nature.cls and naturemag.bst files where
%% LaTeX can find them

\bibliographystyle{naturemag}

\title{Supplementary Materials for

Institutional shifts and water sustainability of the Yellow River Basin
}

%% Notice placement of commas and superscripts and use of &
%% in the author list

 \author{Shuang Song$^{1,2}$, Huiyu Wen$^3$, *Shuai Wang$^{1,2}$, Graeme S. Cumming$^{4}$ \& Bojie Fu$^{1, 2}$}


\begin{document}

\maketitle

\begin{abstract}
    % 这个补充材料以下述逻辑组织而成:
    The supplementary material is organized logically as follows:
    % 展示流域水资源分配制度在全球的推行情况并详细介绍黄河流域的制度变化,旨在说明黄河流域为什么是一个独一无二的准自然实验。
    (1) Show the implementation of the water resource allocation system in the basin around the world and introduce the institutional changes in the Yellow River Basin in detail, aiming to explain why the Yellow River Basin is a unique quasi-natural experiment.
    % 展示定量分析方法的技术路线图,并详细介绍定量分析方法中的细节部分。
    (2) Show the technical roadmap of quantitative analysis method, and introduce the details of quantitative analysis method in detail.
    % 基于我们建立的一般经济模型,对一些可能的情景进行进一步分析。
    (3) Further extensions based on our general economic model.
\end{abstract}

\section*{S1. Detailed introduction of the institutions}
%! Author = songshgeo
%! Date = 2022/3/10

We aim to abstract the water allocating institutions from the description in official documents with necessary context into SES building blocks (Figure~\ref{framework})
Widespread building blocks in SES are the key to the functioning of structures, and a network-based description is a widely used way to depict them by abstracting links and nodes \cite{bodin2017a,kluger2020,guerrero2015}.

\begin{figure}
	\centering
	\includegraphics[width=0.9\linewidth]{diagrams/framework.jpg}
	\caption{
		Framework for understanding linkages between SES structures and outcomes.
		\textbf{a.} The general framework for analyzing social-ecological systems (SESs) (adapted from Ostrom \cite{ostrom2009}). Institutions embedded in SESs may reshape structures by changing the interactions between core subsystems, resulting in different outcomes.
        Three typical types of abstracted SES structures are shown as \textbf{b.}, \textbf{c.} and \textbf{d.} (adapted from Bodin, 2017)\cite{bodin2017b}. Red circles indicate social actors, and green ones indicate ecological components. Connection (ties between two ecological components), collaboration (ties between two social actors), or management (ties between a social actor and an ecological component) exist when gray lines link two units. According to empirical evidence, the gray dashed lines show aligned SES structures that are more likely to achieve a desirable outcome.
        }
    \label{framework}
\end{figure}

% 水资源分配方案在全世界范围内都是流域管理的普遍制度。
Water allocation institutions are widespread in large river basin management programs throughout the world (see \textit{Appendix} Figure~\ref{fig:world}) \cite{speed2013}.
This was the first basin in China for which a water resource allocation institution was created, and institutional shifts can be traced through several documents released by the Chinese government (at the national level)\cite{wang2019a}:
\begin{itemize}
    \item \textbf{1982}: The provinces and the Yellow River Water Conservancy Commission (YRCC) are required to develop a water resource plan for the Yellow River \cite{wang2019, wang2019a}.
    \item \textbf{1987}: Implementation of the Allocation Plan. (\href{http://www.gov.cn/zhengce/content/2011-03/30/content_3138.htm#}{http://www.mwr.gov.cn}, last access: \today).
    \item \textbf{1998}: Implementation of unified regulation. (\href{http://www.mwr.gov.cn/ztpd/2013ztbd/2013fxkh/fxkhswcbcs/cs/flfg/201304/t20130411_433489.html}{http://www.mwr.gov.cn}, last access: \today).
    % 各省按要求编制新的黄河流域水资源规划,将水资源额度分配进一步细化。
    \item \textbf{2008}: Provinces are asked to draw up new water resources plans for the YRB to further refine water allocations \cite{wang2019,wang2019a}.
    \item \textbf{2021}: A call for redesigning the water allocation institution (\href{http://www.ccgp.gov.cn/cggg/zygg/gkzb/202107/t20210721_16591901.htm}{http://www.ccgp.gov.cn}, last access: \today).
\end{itemize}

% 在上述文件中,1982年的文件标志着设计分水制度尝试的开始,2008年标志着该制度走向成熟(完全建立起流域-省-市区的多级水资源分配和统一调度)。
Since 1982, administrations attemptted to design a quota institution, and the 2008 document marked the maturity of the scheme (complete establishment of basin-level, provincial, and district water quotas).
Between the period, two significant institutional shits can be analyzed by using the 1987 (87-WAS) and 1998 (98-UBR) documents.

% It is worth noting that, although the essential reason for these institutions was the mismatch between the spatial and temporal distribution of water resources as well as social and economic water demands, the direct reason for their introduction was the depletion of the Yellow River.

The official documents in 1987 (\href{http://www.gov.cn/zhengce/content/2011-03/30/content_3138.htm#}{http://www.mwr.gov.cn}, last access: \today) convey the following key points:

\begin{itemize}
	% 该政策面向的目标是各省(区域),黄委会没有被提及
	\item The policy is aimed at related provinces (or regions at the same administrative level).
	% 政策制定的首要考虑是解决断流问题
	\item Depletion of the river is identified as the first consideration of this institution.
	% 各省被鼓励在此配额下制定自己的用水计划
	\item Provinces are encouraged to develop their water use plans based on a quota system.
	% 水资源供给无法满足需求对相关省(地区)是普遍现象。
	\item Water in short supply is a common phenomenon in relevant provinces (regions).
\end{itemize}

The official documents in 1998
(\href{http://www.mwr.gov.cn/ztpd/2013ztbd/2013fxkh/fxkhswcbcs/cs/flfg/201304/t20130411_433489.html}{http://www.mwr.gov.cn}, last access: \today) convey the following key points:

\begin{itemize}
	% 除了说明政策针对的各省区之外,明确指出其用水需要黄河水利委员会进行申报,并由其组织和监管
	\item The document points out that not only provinces and autonomous regions involved in water resources management (see \textit{Article 3}), the provinces’ and regions’ water use shall be declared, organized, and supervised by the YRCC (\textit{Article 11 and Chapter III to Chapter V, and Chapter VII}).
	% 本研究(\textit{ Article 1})首先考虑的是上、中、下游用水的总体规划。
	\item Creating the overall plan of water use in the upper, middle, and lower reaches is identified as the first consideration of this institution (\textit{Article 1}).
	% 各省需要
	\item With the same quota as used in the 1987 policy, provinces were encouraged to further distribute their quota into lower-level administrations (see \textit{Article 6 and Article 41}).
	% 强调以总量确定供给,以供给决定需求。
	\item They emphasize that supply is determined by total quantity, and water use should not exceed the quota proposed in 1987 (see \textit{Article 2}).
\end{itemize}

\begin{figure*}[!htb]
    \centering
    \includegraphics[width=12cm]{/Users/songshgeo/Documents/Pycharm/WAInstitution_YRB_2021/figs/diagrams/world_institutions}
	\caption{
		Overview of water allocation institutions.
		% 世界已有水资源分配制度的大河流域,其中黄河流域最早于1987年提出资源分配方案,后于1998年更改为统一调度方案。
		\textbf{A.} Major river basins in the world with water resource allocation systems (shaded red); the YRB first proposed a resource allocation scheme in 1987 (designed since 1983) and then changed to a unified regulation scheme in 1998 (designed in 1997 but implemented in 1998) \cite{speed2013}.
		% 不同的水资源分配制度设计模式,中国黄河流域是典型的自上而下。
		\textbf{B.} Different water resource allocation system design patterns; the YRB is typical of a top-down system.
		% 流域分水制度的演化。这种多层次的制度设计有其历史变化过程。
		\textbf{C.} The four periods of institutional evolution of water allocation of the YRB.
	}
    \label{fig:world}
\end{figure*}

% 基于上述分析,我们抽象出了两次制度转变之后的SES结构变化如正文的图1C所示。
Based on the above documents, we abstracted the structural changes of SES (see \textit{Appendix S2}) after the two institutional changes, as shown in Figure~\ref{fig:structure}~C.

\begin{table*}
    \centering
    \small
    \caption{Water quotas assigned in the 87-WAS}\label{tab:quota}
	\resizebox{\linewidth}{!}{
    \begin{tabular}{p{0.24\linewidth}llllp{0.1\linewidth}lllll}
	\hline
	Items (water volume, billion $m^3$)                    & Qinghai & Sichuan & Gansu   & Ningxia & Inner Mongolia & Shanxi  & Shaanxi & Henan   & Shandong & Jinji  \\
	\hline
	Demands in water plan                                                & 35.7    & 0       & 73.5    & 60.5    & 148.9          & 115     & 60.8    & 111.8   & 84       & 6      \\
	Quota designed in 1983                                               & 14      & 0       & 30      & 40      & 62             & 43      & 52      & 58      & 75       & 0      \\
	Quota assigned in 1987                                               & 14.1    & 0.4     & 30.4    & 40.0    & 58.6           & 38.0    & 43.1    & 55.4    & 70.0     & 20     \\
	Average water consumption from the Yellow River from 1987-2008       & 12.03   & 0.25$^a$   & 25.80   & 36.58   & 61.97          & 21.16   & 11.97   & 34.30   & 77.87    & 5.85$^a$  \\
	Proportion of water from the Yellow River in total water consumption & 48.12\% & 0.10$^b$\%  & 30.79\% & 58.45\% & 47.82\%        & 73.55\% & 44.39\% & 24.77\% & 34.41\%  & 3.11\%$^b$ \\
    \hline
    \end{tabular}}
	\footnotesize[a]\leftline{{Calculated by data from 2004 to 2017.}}\\
	\footnotesize[b]{\leftline{The share is too small, thus the provinces (or region) Sichuan and Jinji not to be considered in this study.}}
\end{table*}


\section*{S2. The technical roadmap and details of quantitative analyzing}
%! Author = songshgeo
%! Date = 2022/3/19


% % 找到具解释力的变量是构造合成控制法稳健的关键。
% Explanatory variables are the key to constructing a robust synthetic control method.
% % 我们共使用了用水量密切相关的26个变量,这些变量的数据集已在先前的研究中被用来解释中国的用水量变化
% We used a total of $24$ variables related to water consumption Table~\ref{tab:variables}, which datasets have been used in previous studies to explain changes in water use in China \cite{zhou2020}.
% % 由于这些变量间存在自相关,我们通过肘部法供选择了5个主成分作为DSC的输入,前人研究表明PCA方法的结合能够增强合成控制法的稳健性

% In addition, we selected $5$ principal components as input by the elbow method because selection in autocorrelated variables reduces dimensions and then enhances the robustness of the DSC (Figure~\ref{fig:elbow}).

% There are two approaches to validity testing of the DSC: (1) comparing the post-treated and pre-treated reconstructions and (2) testing robustness through placebo analysis.
% For (1), differences between each province and their synthetic are significant in post-treated periods and small in pre-treated periods (Figure~\ref{fig:87panel} and figure~\ref{fig:98panel}), which show good reconstructions of their water use changes' estimation.
% For (2), we applied the in-place placebo analysis described by \cite{abadie2010}. In most provinces, ratios of post-MSPE to pre-MSPE are higher than the median of other placebo units, which suggests the institutional shifts in treated time (1987 and 1998 here) influenced them more than most of the other provinces (figure~\ref{fig:87placebo}, figure~\ref{fig:98placebo}, Table~\ref{tab:DSC_summary}).

% \begin{figure*}[!bh]
%     \includegraphics[width=0.9\linewidth]{outputs/87panel.pdf}
%     \centering
%     \caption{Comparations between YRB' provinces and their synthetic controls around the 87-WAS.}
%     \label{fig:87panel}
% \end{figure*}

% \begin{figure*}
%     \includegraphics[width=0.9\linewidth]{outputs/98panel.pdf}
%     \centering
%     \caption{Comparations between YRB' provinces and their synthetic controls around the 98-UBR.}
%     \label{fig:98panel}
% \end{figure*}


% \begin{figure*}
%     \includegraphics[width=0.9\linewidth]{outputs/87placebo.pdf}
%     \centering
%     \caption{Gaps in change in water use between provinces outside the YRB and their synthetic control, around the 87-WAS, excluding the provinces with high pre-treatment RMSPE (more than $3$ times of treated units' RMSPE).}
%     \label{fig:87placebo}
% \end{figure*}

% \begin{figure*}
%     \includegraphics[width=0.9\linewidth]{outputs/98placebo.pdf}
%     \centering
%     \caption{Gaps in change in water use between provinces outside the YRB and their synthetic control, around the 98-UBR, excluding the provinces with high pre-treatment RMSPE (more than $3$ times of treated units' RMSPE)}
%     \label{fig:98placebo}
% \end{figure*}


\begin{table*}[!ht]
	\caption{Variables and their categories for water use predictions}
	\scriptsize
	\label{tab:variables}
	\resizebox{\linewidth}{!}{
	\begin{tabular}{lllll}
	\hline
	Sector &
	  Category &
	  Unit &
	  Description &
	  Variables \\ \hline
	Agriculture &
	  Irrigation Area &
	  thousand ha &
	  \begin{tabular}[c]{@{}l@{}}Area equipped for irrgiation by different \\ crop:\end{tabular} &
	  \begin{tabular}[c]{@{}l@{}}Rice, \\ Wheat, \\ Maize, \\ Fruits, \\ Others.\end{tabular} \\ \hline
	Industry &
	  \begin{tabular}[c]{@{}l@{}}Industrial gross \\ value added\end{tabular} &
	  Billion Yuan &
	  Industrial GVA by industries &
	  \begin{tabular}[c]{@{}l@{}}Textile, \\ Papermaking, \\ Petrochemicals, \\ Metallurgy, \\ Mining, \\ Food, \\ Cements, \\ Machinery, \\ Electronics, \\ Thermal electrivity, \\ Others.\end{tabular} \\
	 &
	  \begin{tabular}[c]{@{}l@{}}Industrial water \\ use efficiency\end{tabular} &
	  \% &
	  \begin{tabular}[c]{@{}l@{}}The ratio of recycled water and evaporated \\ water to total industrial water use\end{tabular} &
	  \begin{tabular}[c]{@{}l@{}}Ratio of industrial water recycling, \\ Ratio of industrial water evaporated.\end{tabular} \\ \hline
	Services &
	  \begin{tabular}[c]{@{}l@{}}Services gross \\ value added\end{tabular} &
	  Billion Yuan &
	  GVA of service activities &
	  Services GVA \\ \hline
	Domestic &
	  Urban population &
	  Million Capita &
	  Population living in urban regions. &
	  Urban pop \\
	 &
	  Rural population &
	  Million Capita &
	  Population living in rural regions. &
	  Rural pop \\
	 &
	  Livestock population &
	  Billion KJ &
	  \begin{tabular}[c]{@{}l@{}}Livestock commodity calories summed from \\ 7 types of animal.\end{tabular} &
	  Livestock \\ \hline
	Environment &
		  Temperature & $K$ & Near surface air temperature & Temperature \\
			& Precipitation & $mm$ & Annual accumulated precipitation & Precipitation \\ \hline
	\end{tabular}}
\end{table*}


\begin{figure*}[!h]
    \includegraphics[width=0.9\linewidth]{outputs/elbow.pdf}
    \centering
    \caption{Choose number of pricipal components by Elbow method, $5$ pricipal components already capture $89.63\%$ explained variance.}\label{fig:elbow}
\end{figure*}


\section*{S3. Further analysis regarding the general economic model}
\input{03_theoretical_model.tex}

\bibliography{/Users/songshgeo/Documents/Pycharm/WAInstitution_YRB_2021/docs/papers/WAInstitution_YRB_2021}
\end{document}
