%Model extension部分
Further analysis of the general economic model

Using the general economic model (see the Methods section in the main text), we also explored the response of stakeholders to water quota policies. We considered two additional scenarios for stakeholders, one that considered technology growth and one that considered different valuations through time (via the discount rate) of economic benefits and ecological costs. In the following scenarios, the cost is assumed to be untransferable, which could be fully allocated to the one incurring the water use. Explaining plausible scenarios for these stakeholders will help us better understand the causes of the water overuse and potential solutions. We argue that the water overuse remains robust even if a complete and equitable system.

% technology growth
    \begin{case_appendix}Forward-looking decentralized decision, taken ecology cost into considerations

    Even if the negative externality of water overuse is eliminated by``fair" ecology cost of $\frac{x_{i,0}}{x_{i,0} + \begin{matrix} \sum x_{-i,0} \end{matrix}} \cdot Q \cdot C$, it is possible that the future growth opportunities and ``remote" ecological costs provide enough incentive for the sprint.  The water overuse has the value of future economical benefits by slacking the water use constraint in the future. The heterogeneous production efficiency is omitted in this section, and we set A=1.

(a) technology growth

Assume that there is an exogenous technology growth rate of g in the scenario of $N$ provinces bargaining for water use under total quota $Q$, with unit price of output $P$, unit cost $C$, and discount factor $\beta$. For simplicity, consider a finite-period water use optimization:

$ max \quad P \cdot (1+g)^t ln(1+x_{i,0})-\frac{C}{N}+\beta^t \begin{matrix} \sum_{t=1}^T [P \cdot (1+g)^t ln(x_{i,t}+1)-C \cdot x_{i,t}] \end{matrix}$

$s.t. \quad x_{i,t} \leq Q \cdot \frac{x_{i,0}}{x_{i,0} + \begin{matrix} \sum x_{-i,0} \end{matrix}} \quad for \quad \forall t$

We depict the relationship of multi-period profit and water use $x_{i,0}$ in different horizons in Figure 4%%此处引用图片
,and thus find out the optimal water use pattern under technology growth. The higher marginal output of water might create enough incentive to set off the untransferable cost since a higher allocated quota provides growth option value. As the provincial decision is under a longer horizon, there is a greater sprint effect due to higher accumulated yield and relatively tighter water use constraints over time.

% \begin{figure}[H]
%     \centering
%     \includegraphics[width=0.7\linewidth]{Fig3.jpg}
% \end{figure}

% \begin{fig}
% Multi-period optimization of optimal water use under technology growth

% \tiny Notes: The figure depicts the relationship of multi-period benefits of province $i$ and water use under Case 3 with technology growth. Assume F(x)=ln(1+x), N=8, P=1, C=0.5, $\beta$=0.7, g=0.2, and Q=8.
% \end{fig}

(b) Economic benefits and ``remote'' ecological costs with different discount factors

Assume that there is a high discount rate for economical benefits and a low discount rate for ecological costs, in the scenario of $N$ provinces bargaining for water use under total quota $Q$, with unit price of output $P$, unit cost $C$, discount factor $\beta^{economy}$ and $\beta^{ecology}$ ($\beta^{economy} > \beta^{ecology}$). For simplicity, consider the following finite-period water use optimization, noting the water use of province $i$ at period $t$:

$ max \quad P \cdot ln(1+x_{i,0})-\frac{C}{N}+\beta_1^t \begin{matrix} \sum_{t=1}^T [P \cdot ln(x_{i,t}+1)]  \end{matrix} - \beta_2^t \begin{matrix} \sum_{t=1}^T [C \cdot x_{i,t}] \end{matrix}$

$s.t. \quad x_{i,t} \leq Q \cdot \frac{x_{i,0}}{x_{i,0} + \begin{matrix} \sum x_{-i,0} \end{matrix}} \quad for \quad \forall t$

We depict the relationship of multi-period net income and water use $x_{i,0}$ in different horizons in Figure 4%此处引用图片
, and thus find out the optimal water use pattern under ``remote'' ecological costs. The higher discounted ecological costs might create enough incentive to set off the untransferable cost. As the provincial decision is under a longer horizon, there is a greater sprint effect due to a higher accumulated yield.

% \begin{figure}[H]
%     \centering
%     \includegraphics[width=0.7\linewidth]{Fig4.jpg}
% \end{figure}

% \begin{fig}
% Multi-period optimization of water use under ``remote'' ecological cost

% \tiny Notes: The figure depicts the relationship of multi-period benefits of province $i$ and water use under Case 3 with ``remote'' ecological cost. Assume F(x)=ln(1+x), N=8, P=1, C=0.5, $\beta_{economy}$=0.7, $\beta_{ecology}$=0.3, and Q=8.
% \end{fig}

\end{case_appendix}
