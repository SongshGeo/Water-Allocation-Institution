%! Author = songshgeo
%! Date = 2022/3/19

% 为了使用合成控制法预测没有制度变化下的用水变化趋势,需要使用影响用水的社会经济数据作为自变量输入。
In order to use the synthetic control method to predict the trend of water use without institutional change, socioeconomic data affecting water use were used as the input independent variables. All variables used are listed in Table~\ref{tab:variables}.
% 这些变量涉及了省级单元在利用水资源时需要考虑的主要经济因素(农业、工业、服务业与居民生活),它们与因变量(水资源使用)的相关性如图所示。
These variables refer to major economic factors (agriculture, industry, service industry, and domestic) that provincial units need to take into account when using water resources; their correlation coefficients with the dependent variable (water resource use) are shown in Figure~\ref{fig:linear} A.

% 此外,合成控制法有一个假设是自变量的线性组合可以有效预测因变量,因此我们还对数据线性拟合的效果进行了测试。
In addition, the synthetic control method assumes that a linear combination of independent variables can effectively predict the dependent variables.
% 将数据集区分为80%的训练样本和20%的测试样本,我们使用训练样本构建了多元线性模型对用水量进行了预测,并使用测试数据集去检验模型拟合效果。
We therefore divided the dataset into two groups, training samples $(80\%)$ and test samples $(20\%)$, and used the training samples to build a multivariate linear model to predict the water consumption. We then used the test dataset to test the model-fitting effect.
% 结果表明拟合优度R2超过了0.8,因变量可以很好的被自变量的线性组合所解释,该数据集可以用于控制合成法。
Results show that the goodness of fit $R^2$ exceeded $0.8$; thus, the dependent variable is well-explained by the linear combination of independent variables, and the dataset can be used in the synthetic control method (Figure~\ref{fig:linear}B).


\begin{table}[!h]
	\caption{Variables and their categories for water use predictions}
	\scriptsize
	\label{tab:variables}
	\begin{tabular}{lllll}
	\hline
	Economic sector &
	  Category &
	  Unit &
	  Description &
	  Variables \\ \hline
	Agriculture &
	  Irrigation Area &
	  thousand ha &
	  \begin{tabular}[c]{@{}l@{}}Area equipped for irrgiation by different \\ crop:\end{tabular} &
	  \begin{tabular}[c]{@{}l@{}}Rice, \\ Wheat, \\ Maize, \\ Fruits, \\ Others.\end{tabular} \\ \hline
	Industry &
	  \begin{tabular}[c]{@{}l@{}}Industrial gross \\ value added\end{tabular} &
	  Billion Yuan &
	  Industrial GVA by industries &
	  \begin{tabular}[c]{@{}l@{}}Textile, \\ Papermaking, \\ Petrochemicals, \\ Metallurgy, \\ Mining, \\ Food, \\ Cements, \\ Machinery, \\ Electronics, \\ Thermal electrivity, \\ Others.\end{tabular} \\
	 &
	  \begin{tabular}[c]{@{}l@{}}Industrial water \\ use efficiency\end{tabular} &
	  \% &
	  \begin{tabular}[c]{@{}l@{}}The ratio of recycled water and evaporated \\ water to total industrial water use\end{tabular} &
	  \begin{tabular}[c]{@{}l@{}}Ratio of industrial water recycling, \\ Ratio of industrial water evaporated.\end{tabular} \\ \hline
	Services &
	  \begin{tabular}[c]{@{}l@{}}Services gross \\ value added\end{tabular} &
	  Billion Yuan &
	  GVA of service activities &
	  Services GVA \\ \hline
	Domestic &
	  Urban population &
	  Million Capita &
	  Population living in urban regions. &
	  Urban pop \\
	 &
	  Rural population &
	  Million Capita &
	  Population living in rural regions. &
	  Rural pop \\
	 &
	  Livestock population &
	  Billion KJ &
	  \begin{tabular}[c]{@{}l@{}}Livestock commodity calories summed from \\ 7 types of animal.\end{tabular} &
	  Livestock \\ \hline
	\end{tabular}
\end{table}

\begin{figure}[!thb]
	\centering
	\includegraphics[width=12cm]{/Users/songshgeo/Documents/Pycharm/WAInstitution_YRB_2021/figs/outputs/linear}
	\caption{
		\textbf{A.} Correlation between independent variables and the dependent variable (water use).
		\textbf{B.} Linear relationship between the independent variables and the dependent variable trained by a linear model.
	}
\label{fig:linear}
\end{figure}
