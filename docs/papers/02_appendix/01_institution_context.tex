%! Author = songshgeo
%! Date = 2022/3/10

% 水资源分配方案在全世界范围内都是流域管理的普遍制度。
Water allocation schemes are widespread in large river basin management programs throughout the world (see \textit{Supplementary Material Figure~\ref{fig:world}}) \cite{speed2013}.
% 其中中国的黄河流域以典型的自上而下进行制度改革,这种模式会对制度产生迅速的影响,使我们能够定量地估计高层制度设计变化对用水的净影响。
The Yellow River Basin (YRB) in China has a typical top-down approach to institutional reform, which produces rapid institutional impacts and allows researchers to quantitatively estimate the net impact of changes in high-level institutional design on water use (see \textit{Supplementary Material Figure~\ref{fig:framework}}).
% 作为中国最早实施水资源分配制度的流域,黄河的制度变化可以在几个水利部发布的文件中得以窥见
This was the first basin in China for which a water resource allocation institution was created, and institutional shifts can be traced through several documents released by the Chinese government (at the national level):
\begin{itemize}
    \item \textbf{1982}: The provinces and the Yellow River Water Conservancy Commission (YRCC) are required to develop a water resource plan for the Yellow River \cite{wang2019, wang2019a}.
    \item \textbf{1987}: Implementation of the Allocation Plan. (\href{http://www.gov.cn/zhengce/content/2011-03/30/content_3138.htm#}{http://www.mwr.gov.cn}, last access: \today).
    \item \textbf{1998}: Implementation of unified regulation. (\href{http://www.mwr.gov.cn/ztpd/2013ztbd/2013fxkh/fxkhswcbcs/cs/flfg/201304/t20130411_433489.html}{http://www.mwr.gov.cn}, last access: \today).
    % 各省按要求编制新的黄河流域水资源规划,将水资源额度分配进一步细化。
    \item \textbf{2008}: Provinces are asked to draw up new water resources plans for the YRB to further refine water allocations \cite{wangReviewImplementationYellow2019, wangThingsCurrentSignificance2019}.
    \item \textbf{2021}: A call for redesigning the water allocation institution (\href{http://www.ccgp.gov.cn/cggg/zygg/gkzb/202107/t20210721_16591901.htm}{http://www.ccgp.gov.cn}, last access: \today).
\end{itemize}

% 在上述文件中,1982年的文件标志着设计分水制度尝试的开始,2008年标志着该制度走向成熟(完全建立起流域-省-市区的多级水资源分配和统一调度)。
The 1982 document marked the beginning of the attempt to design a water allocation institution, and the 2008 document marked the maturity of the system (complete establishment of basin-level, provincial, and district water allocation and unified regulation). Currently, a major overhaul is in the planning stages. Major shifts of the institution can be analyzed by using the 1987 and 1998 documents. It is worth noting that, although the essential reason for these institutions was the mismatch between the spatial and temporal distribution of water resources as well as social and economic water demands, the direct reason for their introduction was the drying-up of the Yellow River \cite{wangReviewImplementationYellow2019}.

\textbf{The official documents in 1987 clearly convey the following key points:}
\href{http://www.gov.cn/zhengce/content/2011-03/30/content_3138.htm#}{http://www.mwr.gov.cn}, last access: \today.
\begin{itemize}
	% 该政策面向的目标是各省(区域),黄委会没有被提及
	\item The policy is aimed at related provinces (or regions), and the YRCC is not mentioned.
	% 政策制定的首要考虑是解决断流问题
	\item Drying-up of the river is identified as the first consideration of this institution.
	% 各省被鼓励在此配额下制定自己的用水计划
	\item Provinces are encouraged to develop their own water use plans based on a quota system.
	% 水资源供给无法满足需求对相关省(地区)是普遍现象。
	\item Water in short supply is a common phenomenon in relevant provinces (regions).
\end{itemize}

\textbf{The official documents in 1998 clearly convey the following key points:}
\href{http://www.mwr.gov.cn/ztpd/2013ztbd/2013fxkh/fxkhswcbcs/cs/flfg/201304/t20130411_433489.html}{http://www.mwr.gov.cn}, last access: \today.
\begin{itemize}
	% 除了说明政策针对的各省区之外,明确指出其用水需要黄河水利委员会进行申报,并由其组织和监管
	\item The document clearly points out that not only provinces and autonomous regions involved in water resources management (see \textit{Article 3}), the provinces’ and regions’ water use shall be declared, organized, and supervised by the YRCC (\textit{Article 11 and Chapter III to Chapter V, and Chapter VII}).
	% 本研究(\textit{ Article 1})首先考虑的是上、中、下游用水的总体规划。
	\item Creating the overall plan of water use in the upper, middle, and lower reaches is identified as the first consideration of this institution (\textit{Article 1}).
	% 各省需要
	\item With the same quota as used in the 1987 policy, provinces were encouraged to further distribute their quota into lower-level administrations (see \textit{Article 6 and Article 41}).
	% 强调以总量确定供给,以供给决定需求。
	\item They emphasize that supply is determined by total quantity, and water use should not exceed the quota proposed in 1987 (see \textit{Article 2}).
\end{itemize}

% 基于上述分析,我们抽象出了如正文图2中所展示的流域水资源分配制度的运行结构,并将研究时段选择在1975到2008之间。
% On the basis of the above analysis, we abstracted the operational structure of the water resource allocation institution of the YRB as shown in Figure 2 of the main text, focusing on the period between 1975 and 2008. By comparing the net effects of three different institutional structures split by the two institutional shifts, we were able to reach a stronger understanding of the influence of structural alignments under the same basin (previous structure-based analysis usually focus on a certain type of systems but with different geographic contexts).

\begin{figure*}[!htb]
    \centering
    \includegraphics[width=12cm]{/Users/songshgeo/Documents/Pycharm/WAInstitution_YRB_2021/figs/diagrams/world_institutions}
	\caption{
		Overview of water allocation institutions.
		% 世界已有水资源分配制度的大河流域,其中黄河流域最早于1987年提出资源分配方案,后于1998年更改为统一调度方案。
		\textbf{A.} Major river basins in the world with existing water resource allocation systems (shaded red); the YRB first proposed a resource allocation scheme in 1987 (designed in 1983) and then changed to a unified regulation scheme in 1998 (designed in 1997 but implemented in 1998).
		% 不同的水资源分配制度设计模式,中国黄河流域是典型的自上而下。
		\textbf{B.} Different water resource allocation system design patterns; the YRB is typical of a top-down system with multiple levels.
		% 流域分水制度的演化。这种多层次的制度设计有其历史变化过程。
		\textbf{C.} The four periods of institutional evolution of water allocation of the YRB.
	}
    \label{fig:world}
\end{figure*}

% \begin{figure}[!htb]
%     \centering
% 	\includegraphics[width=12cm]{/Users/songshgeo/Documents/Pycharm/WAInstitution_YRB_2021/figs/diagrams/YRB_scheme}
% 	\caption{
% 		% 中国自上而下的水资源分配制度模式(以黄河流域为例)
% 		A top-down water resources allocation scheme in China (a case study of the YRB).
% 		% 共有十个省或地区从黄河获取地表水资源,其中8个省的依赖性较强(see Supplementary Material Fig)。
% 		\textbf{A.} A total of 10 provinces or regions withdraw surface water resources from the Yellow River, of which 8 are highly dependent on the river (see \textit{Supplementary Material S2}).
% 		% 中国政府是下发流域管理政策的最高权威,其政策常常能够很快自上而下全面推行。
% 		\textbf{B.} The Chinese government is the ultimate authority in issuing watershed management policies, which are often quickly implemented from top down.
% 		% 流域管理机构(在黄河是黄河水利委员会)主要负责顺从国家的政策方针来进行流域的日常管理工作。
% 		\textbf{C.} The basin-level agency (here, the YRCC) is primarily responsible for the river-related management of the basin in accordance with national policy guidelines.
% 		% 利益相关者是各省
% 		\textbf{D.} The water management system directly affects the process by which local governments (major stakeholders) plan and use water resources for development. Although only surface water (Sur.) is usually traced and restricted, it can also influence groundwater through related hydro-processes such as recharge.
% 	}
% 	\label{fig:framework}
% \end{figure}


% \begin{table*}
%     \centering
%     \small
%     \caption{Policies and regulations above YRB level which affected the whole basin in water utilization}\label{tab:policies}
%     \begin{minipage}{\linewidth}
%     \resizebox{\linewidth}{!}{
%     \begin{tabular}{p{0.6\linewidth}rp{0.35\linewidth}}
%         Name & Year & Agency \\
%         \midrule
%         1. Water Law of PRC & 1988 & National People's Congress of the PRC \\
%         2. Water Law of PRC -revised 1 & 2009 & National People's Congress of the PRC \\
%         3. Water Law of PRC -revised 2 & 2016 & National People's Congress of the PRC \\
%         4. Regulations on the Administration of Water Drawing Licences and The Collection of water resource fees & 2006 & State Council of the PRC \\
%         5. Regulations on the Administration of Water Drawing Licences and The Collection of water resource fees -revised 1 & 2017 & State Council of the PRC \\
%         6. Regulations on the Allocation of Water in the Yellow River & 2006 & State Council of the PRC \\
%         7. Yellow River water supply distribution scheme & 1987 & State Council of the PRC \\
%         8. Measures for the Administration of Water Drawing Permits & 2008 & Ministry of Water Resources of the PRC \\
%         9. Measures for the Administration of Water Drawing Permits -revised 1 & 2015 & Ministry of Water Resources of the PRC \\
%         10. Measures for the Administration of Water Drawing Permits -revised 2 & 2017 & Ministry of Water Resources of the PRC \\
%         11. Regulations on the Allocation of Water in the Yellow River & 2006 & State Council of the PRC \\
%         12. Annual distribution of available water supply of the Yellow River and mainstream water dispatching scheme & 1998 & Ministry of Water Resources of the PRC \\
%         13. The Yellow River water dispatching management measures & 1998 & Ministry of Water Resources \\
%         14. Measures for the Implementation of the Yellow River Water Rights Conversion Management & 2004 & Ministry of Water Resources \\
%         15. Regulations on the Administration of Water Drawing Licences and The Collection of water resource fees & 2006 & State Council of the PRC \\
%         16. Measures for the implementation of the water drawing Permit system & 1993 \\ State Council of the PRC \\
%         17. Measures for the demonstration and management of water resources in construction projects & 2002 & Ministry of Water Resources of the PRC \\
%         18. Implementation Opinions on the Reform of Water Conservancy Project Management System & 2006 & State Council of the PRC \\
%         \bottomrule
%     \end{tabular}}

%     \footnotesize[1]{If a policy was proposed by multiple legacies, we only show the highest one.}
%     \end{minipage}
% \end{table*}
