\documentclass{nsr}


\usepackage{amsmath,graphicx,array}
\usepackage{dcolumn,soul}%
\let\openbox\relax
\usepackage{amsthm}
\usepackage[figuresright]{rotating}%
\usepackage{algorithm, algorithmicx, algpseudocode}
\usepackage{listings}%
\usepackage{hyperref}

\makeatletter
%%
\def\uns{\ifmmode\,\else$\,$\fi}%
\newtheorem{theorem}{Theorem}
\newtheorem{construction}{Construction}
\newtheorem{estimate}{Estimate}
\newtheorem{lemma}{Lemma}
\newtheorem{corollary}{Corollary}
\newtheorem{result}{Result}
\newtheorem{algth}{Algorithm}
\newtheorem{proposition}{Proposition}
\newtheorem{hypothesis}{Hypothesis}
\newtheorem{experiment}{Experiment}
\newtheorem{definition}{Definition}
\newtheorem{condition}{Condition}
\newtheorem{property}{Property}
\newtheorem{problem}{Problem}
\newtheorem{fact}{Fact}
\newtheorem{assumption}{Assumption}
\newtheorem{eexample}{Example}
\newtheorem{model}{Model}
%%
\makeatother

\jvol{XX}
\jnum{X}
\jyear{Year}
\doi{10.1093/nsr/XXXX}
\received{XX XX Year}
\revised{XX XX Year}
\accepted{XX XX Year}

\markboth{One, Two, and Three}{One, Two, and Three}


\begin{document}

\dhead{RESEARCH ARTICLE}
\subhead{EARTH SCIENCES}
\newtheorem{ass}{Assumption}   % My command. Assumptions.
\newtheorem{case}{Case}  % My command. Cases

\bibliographystyle{nsr}

\title{Institutional shifts and water sustainability of the Yellow River Basin
}  %! Notice Input


\author{Shuang Song$^{1,2}$}
\author{Huiyu Wen$^3$}
\author{*Shuai Wang$^{1,2}$}
\author{Xutong Wu$^{1,2}$}
\author{Graeme S. Cumming$^{4}$}
\author{Bojie Fu$^{1,2,5}$}


\affil{$^1$State Key Laboratory of Earth Surface Processes and Resource Ecology,
     Faculty of Geographical Science,
     Beijing Normal University,
     Beijing 100875,
     P.R. China}

\affil{$^2$Institute of Land Surface System and Sustainability,
     Faculty of Geographical Science,
     Beijing Normal University,
     Beijing 100875,
     P.R. China}

\affil{$^3$School of Finance,
     Renmin University of China,
     Beijing 100875,
     P.R. China}

\affil{$^4$ARC Centre of Excellence for Coral Reef Studies,
     James Cook University,
     Townsville 4811,
     QLD, Australia}

\affil{$^5$The research for this article was financed by....... The authors thank....... for insightful comments and ....... for expert research assistance. A supplementary online appendix is available with this article at the \em{National Science Review} website.}

\authornote{\textbf{Corresponding authors.} Email: shuaiwang@bnu.edu.cn}
\authornote{Shuang Song and Huiyu Wen equally contributed to this work.}

\abstract[ABSTRACT]{
    Increasing competition for water is leading to depletion of freshwater globally and calls for an urgent transformation of water governance. To better understand how institutions contribute to water governance, we quantified institutional shifts for the Yellow River Basin (YRB).  The YRB is a valuable case study because it is one of the most anthropogenically altered large river basins. It was first overdrawn, then dried up, and finally has been successfully restored. Our results suggest that two institutional shifts, the Water Allocation Scheme that began in 1987 (87-WAS) and the Unified Basinal Regulation that took over in 1998 (98-UBR), framed different social-ecological system (SES) structures. During the decade following the introduction of the 87-WAS, observed water use of the YRB increased by $8.57\%$ more than expected while 98-UBR successfully decreased total water use, ultimately. Specifically, 87-WAS stimulated increased water use in some provinces (e.g., Inner Mongolia, Henan, and Shandong), but 98-UBR regulated nearly all provinces. A mathematical economic model supports the hypothesis that regional variations were driven by SES structural changes. The quasi-natural experimental setting of the YRB and its significant structural changes over time offer deep insights into the links between SESs structures and outcomes, providing valuable guidelines for SESs worldwide that are facing water depletion.

}  %! Notice input

\keywords{Yellow River, water use, water governance, social-ecological system, institutional fit
}  %! Notice input
\maketitle

\section{INTRODUCTION}\label{intro}
% 水竞争的重要性
Widespread freshwater scarcity and overuse challenge the sustainability of large river basins, resulting in systematic risks to economies, societies, and ecosystems globally \cite{distefano2017, dolan2021, xu2020b, mekonnen2016}.
In the context of future climate change, the gap between supply and demand for water resources in large river basins is expected to become increasingly more prominent \cite{florke2018, yoon2021}.
Those river basin systems successfully supporting sustainable water resource use are structurally well-aligned with water provisioning and social-ecological demands, without inefficient competition or overuses \cite{wang2019d}.
However, balancing the water demands of ecosystems and development in heavily human-dominated river basins is a challenge because human activities and water are intertwined in their structures as complex social-ecological systems (SES) \cite{huggins2022,konar2019}.

For governing river basin systems, their SES structures can be reshaped by institutions, such as policies, laws, and norms \cite{young2008,cumming2020b}.
Representing all relative governance practices, institutions include interplays between social actors, ecological units, or between social and ecological system elements
\cite{lien2020, bodin2017b}.
Understanding how these complex interplays is crucial for developing strategies to effectively manage natural resources and enhance the resilience of social-ecological systems \cite{kluger2020}.
However, the best approach to designing effective institutions remains an open question \cite{agrawal2003, persha2011, agrawal2001}.

Effective (``matched'' or ``fit'') institutions operate at appropriate spatial, temporal, and functional scales to manage and balance different relationships and interactions between human and water systems, supporting (but not guaranteeing) the sustainability of SES \cite{epstein2015, wang2019d}.
Some institutional advances have had desirable water governance outcomes (e.g., the Ecological Water Diversion Project in Heihe River Basin, China \cite{wang2019d}, and collaborative water governance systems in Europe \cite{green2013}).
However, imposing institutional changes on a large, complex river basin may create or destroy hundreds of connections between social agents and ecological units, where matched social-ecological structures are not ubiquitous.
Two particular weaknesses in existing knowledge of institutional matches include understanding: (i) the causal links between SES structures and outcomes; (ii) details of the underlying processes, and especially the coordination of the incentives of different participants, that result from an institutional lack of match.
These weaknesses limit understanding of institutional design, and hinder approaches toward institutional matches for improving the sustainability of river basin systems.

To better understand how water governance institutions can be designed to match their social-ecological context, we take the Yellow River Basin (YRB), China, as an example \textit{\nameref{sec:yrb}} to dive into causal links between SES structures and outcomes.
We used data on changes in official documents following two institutional shifts (the 87-WAS and the 98-UBR) to describe comparable changes in the SES structures associated with the YRB from 1979 to 2008 \textit{\nameref{sec:structures}}.
We then used a method called `Differenced Synthetic Control (DSC)' \cite{arkhangelsky2021}, which considers economic growth and natural background, to estimate theoretical water use scenarios without institutional shifts (\textit{\nameref{sec:DSC}} and \textit{\nameref{secS2}}).
This approach allowed us to create a counterfactual against which to explore the mechanisms linking SESs structure and outcomes for a deeper understanding of the potential role of institutions in water governance worldwide.
Finally, we further developed an approach for marginal benefits analysis, to interpret the underlying processes of match and mismatched institutions based on SESs structures (\textit{\nameref{sec:model}}).



\begin{figure}
	\centering
	\includegraphics[width=16pc]{../../figs/diagrams/framework.jpg}
	\caption{
		Framework for understanding linkages between SES structures and outcomes.
		\textbf{a.} The general framework for analyzing social-ecological systems (SESs) (adapted from Ostrom \cite{ostromGeneralFrameworkAnalyzing2009}). Institutions embedded in SESs may reshape structures by changing the interactions between core subsystems, resulting in different outcomes.
        Three typical types of abstracted SES structures are shown as \textbf{b.}, \textbf{c.} and \textbf{d.} (adapted from Bodin, 2017)\cite{bodinCollaborativeenvironmentalgovernance2017}. Red circles indicate social actors, and green ones indicate ecological components. Connection (ties between two ecological components), collaboration (ties between two social actors), or management (ties between a social actor and an ecological component) exist when two units are linked by gray lines. The gray dashed lines show aligned SES structures that are more likely to result in a desirable outcome according to empirical evidence.
	}
    \label{fig:framework}
\end{figure}

\section{Context of institutional shifts}\label{institution}
%! Author = songshgeo
%! Date = 2022/3/10
% 黄河是世界第五长河,它的流域也是中国文明的摇篮。
The Yellow River, whose basin is the cradle of Chinese civilization, is the fifth-longest river in the world. It supports $9.7\%$ of China’s irrigation, with only $2.6\%$ of its total water resources (data from \href{http://www.yrcc.gov.cn}{http://www.yrcc.gov.cn}, last access: 28 February 2021).
% 然而,经过沿江各省多年的免费取水(图~\ref{fig:structure} A和B),到20世纪80年代,黄河地表水耗水量接近径流量的10倍,并不断上升
However, after years of free access to water by provinces along the river (Figure~\ref{fig:structure} A and B), surface water consumption of the Yellow River was close to $80\%$ of its runoff by the 1980s and rising~\cite{wangYellowRiverwater2019,songSedimenttransportincreasing2020}.
% 自1972年以来,黄河径流量的减少破坏了黄河的生态,制约了黄河的经济发展
Reductions in runoff after 1972 damaged the ecology of the YRB and restricted its economic development~\cite{wangYellowRiverwater2019}.
% 因此,在中国典型的自上而下的制度结构下(附录图S1-B),黄河流域不同层次提出了相对完整的水资源分配规定(图1a)。
% Therefore, through typical top-down institutional structures in China (see \textit{Supplementary Material S1}), relatively integrated water allocation regulations were successively proposed across different levels in the YRB.
% % 这些机构包括国家政府、流域管理机构、省、市,甚至地区(见图1 B-D)。
These include the national government, the basin management agency, provinces, cities, and even districts.
% 这些在不同制度发展阶段相继出台的政策,引发了长江经济带经济发展结构的突变,并产生了不同的结果(见附录A和图S1-C)。
These policies at different stages of institutional development triggered abrupt changes in the SES structure of the YRB with different outcomes.

% 0世纪70年代以来,中国政府向黄河水利委员会下达指令,要求黄河水利委员会设计配水方案,同时要求黄河沿岸各省进行水资源规划
In 1982, the Chinese government issued instructions to the Yellow River Water Conservancy Commission (YRCC), the basin agency of the YRB, requiring it to design a water allocation scheme and at the same time requiring the provinces along the Yellow River to carry out water resources planning (see \textit{Supplementary Material S1})~\cite{wangReviewImplementationYellow2019}.
% 经过多方讨论与权衡,中国政府于1987年为相关省份分配了水资源配额,要求沿黄各省(区)贯彻执行
The Chinese government started to assign water quotas to the relevant provinces in 1987, but did not create a unit to coordinate water division between them (Figure~\ref{fig:structure} \textbf{A} and \textbf{C}).
% 这一时期长江水利枢纽的监督任务是编制长江水利枢纽用水量统计公报,并与定额进行对比分析。
The mandate of the YRCC during this period was only to report on and analyze water consumption in the YRB~\cite{wangReviewImplementationYellow2019}.
% 随着断流的进一步恶化,1998年中国政府推进了相关政策的改革,要求所有省份在取用水资源时必须向黄委会申请许可,黄委会得以直接对各省的用水实施监管。
However, since reductions in river flow indicated an unintended SES outcome (Figure~\ref{fig:structure} E), the Chinese government pushed for a policy reform in 1998 that required all provinces to apply for licenses to use water from the YRCC, allowing the council to directly regulate their water use (see \textit{Supplementary Material S1} and Figure~\ref{fig:structure} A and D).
% 由于革新后的政策成功遏制了断流,2008年该分配政策被进一步细化,相关各省都进一步设置了更细致的分配方案,并最终形成了黄河流域如今的水资源分配格局。
The 1998 policy succeeded in curbing water extraction (Figure~\ref{fig:structure} E), and it was further refined in 2008.
The relevant provinces created a more detailed allocation plan and finally formed the present water allocation institutions of the YRB (see \textit{Supplementary Material S1}).
% 因此,在1975-2008长达33年的时间里,从没有分水政策到以两种不同模式监管下的分水政策,黄河流域实际上先后存在着三种不同的社会-生态结构(Fig 1)。
Therefore, in our study period (from 1975 to 2008), the system shifted between three different SES structures (Figure~\ref{fig:structure} \textbf{B} to \textbf{D}).
% 在它们之中,与预期相反地,1987年至1998年的流域SES结构下黄河的生态急剧恶化,表明了制度的失配。
The sharp and unintended decline in the ecological condition of the Yellow River from 1987 to 1998 indicates an institutional mismatch during this period (Figure~\ref{fig:structure} the shadowed time periods).


\begin{figure}[!ht]
    \centering
    \includegraphics[width=16pc]{../../figs/diagrams/structure.jpg}
	\caption{
		% 黄河流域的制度变迁与经济社会结构差异。
		Institutional shifts and related SES structures in the Yellow River Basin (YRB). See \textit{Supplementary Material S1} for detailed introduction for the institutions.
		% 国家政府在1987年和1998年先后两次出台了改变流域制度的政策,而黄河水利委员会、利益攸关的各省在不同制度时期具备的功能是不同的。
		\textbf{A.} The national government changed YRB management policies and institutions in 1987 and 1998. As a result, the Yellow River Conservancy Commission (YRCC) and the provinces acted differently in different periods. Three different SES structures existed successively in the YRB.
		% 没有任何政策限制,各利益相关者此时期可以从单向但联通的河流生态单元内自由取水
		\textbf{B.} 1975–1987: Without any constraints, water resources were freely accessible to each stakeholder (the provinces in this case, denoted by red circles) from a one-way but connected ecological unit (the Yellow River, denoted by the blue rectangle).
		% 在政策1之后,每个使用者都被分配了能够开采河流地表水资源的配额,而黄委会的工作是对配额的使用进行统计与汇报。
		\textbf{C.} 1987–1998: After the implementation of policy 1 in 1987, each user was assigned a quota to withdraw surface water resources, and the YRCC (yellow triangle) was tasked with reporting on water quota use.
		% 第二个政策之后,利益相关者取水需要向黄委会申请,黄委会则根据用水配额来批发许可。因此此时黄委会与利益相关者之间产生了直接的双向联系,监督他们对资源配额的取用。
		\textbf{D.} 1998–2008: After the implementation of policy 2, stakeholders had to apply for water resources from the YRCC, which then licensed water use according to the quota. Under this institution, the YRCC had direct two-way connections between provinces and ecological components.
		% 由于政策1与政策2都是为了解决黄河断流的生态问题而提出的,因此在结果与政策预期严重相反的第二时期(灰色阴影时段),是制度导致了错配的社会-生态系统结构。
		\textbf{E.} A timeline of the Yellow River and drying conditions. The size of the circles indicates the length of section that dried up (km), and the y-axis indicates the length of the drying period. Both policy 1 and policy 2 were put forward to solve this ecological crisis. The mismatch created by policy 1 is clearly correlated with the unintended outcomes shown in the second (gray-shaded) period.
	}
	\label{fig:structure}
\end{figure}


\section{RESULTS}\label{results}
\subsection{SPRINT EFFECT}\label{phenomenon}
%! Author = songshgeo
%! Date = 2022/3/10
% \MakeUppercase{\subsection{Cascading effects of the institutional shifts}}

% \subsection{ISS IMPACT ON WATER USE OF THE YRB}
\subsection{ISs impact on water use of the YRB}
\label{result-2}
% 结果一:展示制度转变带来的用水量变化

\label{result-1-p2}
Here, we use Differenced Synthetic Control (DSC) method, which considers economic growth and natural background, to estimate theoretical water use scenarios without basinal policy interferences (\textbf{Methods}; \textit{S2 in Supplementary Material}).
Our results suggest that the institutional shift in 1987 (87-WAS) stimulated the provinces to withdraw more water than would have been used without the interference (Figure~\ref{main_results}A).
From 1988 to 1998, while the estimation of water use only suggests $956.38 km^3$, the observed water use of the YRB provinces reached $1038.36 km^3$ in sum, $8.57\%$ increased.
However, after the institution shifted again in 1998 (98-UBR), the trend of increasing water use appeared to be effectively suppressed. From 1998 to 2008, the total observed water use decreased by $0.49 km^3$ per year, while the estimation of water use still suggests $1.03 km^3$ increases (Figure~\ref{main_results} B).
The increased water uses after 87-WAS aligns with the fact that badly drying-up of the surface streamflow from 1987 to 1998, which was an obvious touchstone of river degradation and environmental crisis (Figure~\ref{main_results}C).
On the other hand, the environmental crisis of river drying up was effectively resolved after the 98-UBR, though the density of droughts still increased for decades (from $0.47$ after 87-WAS to $0.62$ after 98-UBR on average) (Figure~\ref{main_results}C).
In line with previous literature had reported; therefore, the institution shift of 98-UBR contributed a lot to the successful water governance. %! citation

\begin{figure*}[!h]
    \centering
    \includegraphics[width=32pc]{outputs/main_results2.pdf}
    \caption{
        Effects of two institutional shifts on water resources use and allocation in the Yellow River Basin (YRB).
        \textbf{A.} water uses of the YRB before and after the institutional shift in 1987 (87-WAS);
        \textbf{B.} water uses of the YRB before and after the institutional shift in 1998 (98-UBR). While the blue lines are statistic water use data, the grey ones are the estimation from the Differenced Synthetic Control method with economic and environmental background controlled.
        \textbf{C.} Drought intensity in the YRB and drying up events of the Yellow River. The size of the grey bubbles denotes the length of a drying upstream.
    }
    \label{main_results}
\end{figure*}


% \subsection{REGIONAL DIFFERENCES IN RESPONSES TO THE IS}
\subsection{Regional differences in responses to the ISs}
\label{result-3}
% 结果2部分:展示区域相应差异

Differences between stakeholders in responses to institutional shifts are vital to understanding the mechanism between structures and outcomes.
Our results show that the proportion of accelerated water use in each province after the decade of 87-WAS (the proportion of actual water use exceeding the predicted water use by the model) has a significant correlation ($p<0.05$, see \textbf{Methods}) to the Yellow River water use in each province (Figure~\ref{upset}A).
Furthermore, while no evident impacts for most provinces (no more than $10\%$ differences, the apparent acceleration effects were only prominent in the big water-using provinces (e.g., Neimeng, Henan, and Shandong. Figure~\ref{upset}B).
In particular, Neimeng and Shandong, both provinces that exceeded the prescribed water uses of the 87-WAS, used $44.25\%$ and $25.69\%$ more water uses than the prediction from 1987 to 1998, respectively.
Furthermore, the satisfaction of each province with the water allocation stipulated by the 87-WAS (expressed by the difference between the actual water allocation of each province and the expected planning value) has little significant correlation to the acceleration.
By contrast, after the 98-UBR, except Shaanxi (which has always been abundant in water quota) had an evident ($17.53\%$) increase in water use, almost all provinces have seen significant declines in water use ($-12.5\%$ on average).
However, neither the satisfaction nor Yellow River water use correlates with the declines after the 98-UBR.

\begin{figure*}[!h]
    \centering
    \includegraphics[width=32pc]{outputs/upset_87.pdf}
    \caption{
        \textbf{A.} The partial correlation coefficient between wate uses (WU) of Yellow River (YR), unsatisfied ratio (compared with requirements in water plan and supply in the 87-WAS), and the average accelerated ratio.
        \textbf{B.} Average accelerated ratio of water uses for each province in the YRB during the decade after 87-WAS (from 1987 to 1998).
        \textbf{Mian users:} Major water consumption provinces (over the median).
        \textbf{Overused:} violate the 87-WAS in average water uses.
    }
    \label{upset}
\end{figure*}


\begin{figure}
    \centering
    \includegraphics[width=16pc]{../../figs/outputs/main_results.jpg}
    \caption{
        Effects of two institutional shifts on water resources use and allocation in the Yellow River Basin (YRB). (\textbf{A} to \textbf{C}: Entering the mismatched SES structure in 1987; \textbf{D} to \textbf{E}: exiting the mismatch in 1998. \textbf{A} and \textbf{D}: Impact of the first institutional shift on water use trends in the YRB. Blue points are actual water use, and gray points are predicted use under a scenario without any institutional shift (see \textit{Methods}). \textbf{B} and \textbf{E}: Impact of the institutional shift on total water use. Dark bars indicate the difference between the actual and predicted water use in specific study periods. Gray bars are the expected water use, simulated by setting up placebo experiments (null models, see \textit{Methods}). \textbf{C} and \textbf{F}: Impact of institutional shift on water allocation equity (see \textit{Methods}). Red lines indicate the index calculated from actual water use data; gray lines indicate predicted water use under a scenario with unconstrained water use.
    }
    \label{fig:main_results}
\end{figure}


\subsection{CAUSES}\label{causes}
%! Author = songshgeo
%! Date = 2022/3/10
Theoretically, our economic model suggests that different kinds of institutional shifts should lead to different optimal water uses (Figure~\ref{fig:economic_model}).
Furthermore, our analysis indicated that the cause of the sprint effect in this case was incentive distortion.
Compared with the decentralized water allocation institution in place before 1987, the presence of central management (by the YRCC in this case, after 1998) can effectively reduce marginal ecological costs (see Table~\ref{tab:cases} the \textbf{methods} for a detailed mathematical formula).
% 在不匹配条件下,边际成本减去边际收益增加,导致用水量增加,xx的差异反映了水配额模拟的非预期冲刺效应。
The unintended sprint effect (from 1987 to 1998) was caused by both declining marginal costs (a shift from a fixed unit cost to an irrelevant cost) and increasing marginal returns due to future water use benefits (see Table~\ref{tab:cases} the \textbf{methods} for a detailed mathematical formula).
% 因此,这种制度引发了一种与可持续用水意图背道而驰的激励扭曲。
The institution thus triggered an incentive distortion that ran counter to the intention of sustainable water use. Further, the strength of the sprint effect was positively correlated with the size and time horizon of the water use quota (Figure~\ref{fig:economic_model} \textbf{Panel B}).
% 这些理论上的预测正如我们在黄河流域所观察到的一样,不匹配的配额制度导致了短跑效应的出现,而中心化的制度结构结束了这一非预期的现象。
% These theoretical predictions, in line with phenomenon observed in the YRB, indicating the mismatched institution leads to incentive distortion of provinces for pre-empting resources by `sprinting', according to their expectations for the future.
% 这些理论预测进一步从YRB观察到的现象出发,表明不匹配的配额制度通过提高边际效益来刺激用水量,而边际效益与放宽限制带来的未来潜在用水量的影子值有关。
% These theoretical predictions, further from the phenomenon observed in the YRB, indicating the mismatched institution with quota stimulates water use by raising marginal benefits, which is related to the shadow value of potential future water use from relaxing the constraints.
% 温博推荐的版本:
% These theoretical predictions, further from the phenomenon observed in the YRB, attributing ``sprint effect'' into raising marginal benefits -by pursuing indirectly generating shadow value of water use while converting fixed marginal costs to irrelevant cost for each individual province.


\begin{figure*}[!ht]
    \centering
    \includegraphics[width=24pc]{outputs/economic_model.jpg}
	\caption{
		% \textbf{Assumption 1:} \textit{(Production)} Assuming that water is the only input of the homogenous production function F(x) of each province. Under diminishing marginal returns assumption, and $F(x)$ is continuous, $F'(0)=\infty$, $ F'(\infty)=0$. The production output is under perfect competition, with constant unit price of P.
		% \textbf{Assumption 2:} \textit{(Cost function)} Assuming that the ecology is a unity for the whole basin, the cost of water use is equally assigned to each province under any water use. The unit cost of water is a constant C.
		% \textbf{Assumption 3:} \textit{(Multi-period setting)} There are infinite periods with constant discount factor $\beta$ lying in (0,1) with no cross-period smoothing in water uses.
		\textbf{A.} The relationship of marginal benefits and water use of province i at t = 0 for three different cases (case 1 to case 3, corresponding to the different SES structures in Figure~\ref{structure}, assuming $F(x)=ln(1+x)$, $N=8$, $P=1$, $C=0.5$, and $\beta=0.4$ as an example  (see \textit{Methods}In Case 3, water use by others is taken as a given, equal to the optimal water use for Case 2. The horizontal coordinate of each intersection of marginal benefits and the break-even line represents the optimal water use under each case.
		\textbf{Panel B.} The relation between optimal water use of province $i$ and total quota for Case 3, under time horizon of $T=5$, $T=10$, and an infinite $T$, respectively. The settings are the same as in \textbf{A}.
        }
	\label{economic_model}
\end{figure*}

%! Author = songshgeo
%! Date = 2022/3/10
%! table marginal benefits
% Document
\begin{table*}[!ht]
	\centering
	\caption{Summary of marginal returns and marginal costs for each case }
	%\footnotesize
    \setlength{\tabcolsep}{1mm}{
       \begin{tabular}{lccc}
		\hline
		& Case 1: Decentralized Ins. & Case 2: Mismatched Ins. & Case 3: Matched Ins. \\ \hline
		Marginal return & $P*F'(X)$                & $P*F'(X)+V(X)^*$            & $P*F'(X)$              \\
		Marginal cost   & $C/N$                    & $Irrelevant$              & $C$                   \\ \hline
	\end{tabular}
    }
	\centering
	\footnotesize{\leftline{$^*$Note: $V(X)$ denotes a shadow value.}}
	\label{tab:cases}
\end{table*}

\begin{figure}[!ht]
    \centering
    \includegraphics[width=16pc]{../../figs/outputs/economic_model.jpg}
	\caption{
		% \textbf{Assumption 1:} \textit{(Production)} Assuming that water is the only input of the homogenous production function F(x) of each province. Under diminishing marginal returns assumption, and $F(x)$ is continuous, $F'(0)=\infty$, $ F'(\infty)=0$. The production output is under perfect competition, with constant unit price of P.
		% \textbf{Assumption 2:} \textit{(Cost function)} Assuming that the ecology is a unity for the whole basin, the cost of water use is equally assigned to each province under any water use. The unit cost of water is a constant C.
		% \textbf{Assumption 3:} \textit{(Multi-period setting)} There are infinite periods with constant discount factor $\beta$ lying in (0,1) with no cross-period smoothing in water uses.
		\textbf{A.} The relationship of marginal benefits and water use of province i at t = 0 for three different cases (case 1 to case 3, corresponding to the different SES structures in Figure~\ref{fig:structure}, assuming $F(x)=ln(1+x)$, $N=8$, $P=1$, $C=0.5$, and $\beta=0.4$ as an example  (see \textit{Methods}In Case 3, water use by others is taken as a given, equal to the optimal water use for Case 2. The horizontal coordinate of each intersection of marginal benefits and the break-even line represents the optimal water use under each case.
		\textbf{Panel B.} The relation between optimal water use of province i and total quota for Case 3, under time horizon of $T=5$, $T=10$, and an infinite $T$, respectively. The settings are the same as in \textbf{A}.
	}
	\label{fig:economic_model}
\end{figure}

\section{DISCUSSION}\label{discussion}
%! Author = songshgeo
%! Date = 2022/3/10

% \subsection{CAUSES OF INSTITUTIONAL IMPACTS}
% \subsection{}
\label{discussion-1}
% discussion-1: 主要介绍结果的意义、合理性

Besides environmental background, our forecast by DSC takes economic factors into account under the assumptions that the production function between economic volume and water uses remained unchanged (\textit{S2 in Supplementary Material}).
It means the forecast of water use includes the part caused by the increased economic volume, while the outcomes of the economy (GDP in different sectors) of the YRB maintained a parallel trend with other regions during the period (\textit{S3 in Supplementary Material} Figure~\ref{S3-1}).
Therefore, 87-WAS did not ``have little effect'' as previous analyses suggested (cites) but led to increased water use because the difference between prediction and observation will be trivial when the shift was just a blank policy by applying the DSC method. %! Citation
Water-use intensity is another crucial factor in interpreting the differences besides the economic factors (e.g., irrigated areas and industrial outcomes) considered and controlled by the method.
In addition to the expansion of irrigation area after the 87-WAS, water uses per unit of irrigation area also rapidly widened the gap with the average level of the rest provinces. However, the industry water use intensity hardly changed (\textit{S3 in Supplementary Material} Figure~\ref{S3-2}).
As a previous report sigh: although the key to alleviating the drought is saving water in the irrigated areas, the tragedy of scrambling for water appeared in provinces and irrigated areas %! \cite{mao2000}.
In terms of the average ratio of water-saving irrigation area (refer to drip or sprinkler irrigation systems and canal lining), although there was a significant increase in the whole country after 1987, the YRB did not rapidly open a noticeable gap until about 1994 (Figure~\ref{S3-3}).
As a result, despite the irrigation area expanding, scrambling for water resources without any incentive to optimize production per unit of water resources accelerated holistic water use.
This accelerated water use was contrary to the original intention of the 87-WAS in conserving the limited water, and the failure was a barrier to the sustainability %! \cite{huangang2002}.

% 过去的研究总结出87-WAS收效甚微的几个因素:
Previous studies have summarised factors that contribute to the non-ideal effect of 87-WAS: (1) The YRCC had no right to punish the provinces for over-exploitation; (2) the water quotas were annual values, causing provinces to rob water in the dry season; (3) The YRCC can make statistics on water use in the mainstream but cannot on the tributaries, so provinces water use underreport %! \cite{huangang2002}.
However, the effects of the two institutional shifts (the 87-WAS and the 98-UBR) were significantly different, which the above reasons cannot fully explain.
Between the 98-UBR and the further refinement of the unified regulation in 2008, there was still a lack of a temporary water allocation scheme and effective monitoring of tributaries.
Moreover, without any actual punishment, provinces with high water consumption (such as Inner Mongolia and Shandong) continued to exceed the quota after 98-UBR.
As we have analyzed (Figure~\ref{structure}), the difference between the two institutional shifts is mainly reflected in the structure of linkages between social actors.
Until the institutional shift of 98-UBR, with no necessity to apply for a water permit from YRCC, there were no horizontal connections (cooperations or agreements) between the various stakeholders (provinces) directly connected to the ecological units.
Make it clear that the YRCC was responsible for regulating provincial water use; that is, each province has made it clear that in the long run, water resources are not ``internal'' but ``dependent'' on YRCC.
In that way, the YRCC, whose authority scale matches the whole river basin, also took the primary responsibilities to the river, and literature recognized the structure as a social-ecological fit that usually led to good outcomes.
Empirical studies in many different fields also indicate that the structure before 98-UBR (i.e., fragment ecological units are linked to separate social actors) is likely to be mismatched as isolated stakeholders struggle with holistically maintaining interconnected ecosystems
\cite{sayles2017,sayles2019,cai2016,bergsten2019}.
The effect of the institutional shifts once again demonstrated that it is not easy to have a win-win situation of environment and interests in complex coupled human-nature systems \cite{hegwood2022} which calls for exceptional understanding and caution to the structure of hampering sustainability \cite{bergsten2019, sayles2019}.

% \subsection{LINKING STRUCTURES WITH ISS OUTCOMES}
% \subsection{Linking structures with ISs outcomes}
\label{discussion-2}
% discussion-2: 机制解释
% 经济模型与理论解释
Differences in the pattern of the response by provinces can demonstrate the influence of social-ecological structures led by the institutional shifts.
We analyzed mathematically why the mismatched structure made limited water use holistically elusive in the institution shift of the 87-WAS but finally achieved by the 98-UBR (\textit{method} and \textit{Supplementary Material S4}).
By taking the structure before and after the two institutional shifts as different basic assumptions (before 87-WAS: free access to water; after 87-WAS but before 98-UBR: decisions on water use under quotas; after 98-UBR: unified regulation), we use the marginal benefit model to analyze the theoretical optimal water consumption of stakeholders in each scenario.
The analysis of the model also shows that 98-UBR can reduce the overall water use of the basin while 87-WAS can increase the water use of the basin when the same parameters are guaranteed but the institutional structure changes.
Before the 98-UBR, the model assumes that the separated ecological units (river reaches) link to stakeholders (related provinces) who use water to pursue their marginal benefits but have a potential political cost if they exceed the quota 87-WAS.
Our model suggests that for users who are already economically efficient (who are already using more water), greater marginal returns from water induce the acceleration of extracting resources for future economic growth (Figure~\ref{economic_model}).
Therefore, isolated stakeholders reacted to the similar marginal cost, and smaller water users have a threshold because of the political cost, so 87-WAS triggered an increased water use for the significant users.
On the contrary, the presence of central management (by the YRCC in this case, after 1998) can effectively reduce marginal ecological costs holistically as stakeholders only take corresponding responsibilities (follow the quota as possible as they can) to the YRCC (\textit{Supplementary Material S4}).
As a result, unified regulating acted the core role after the 98-UBR and reduced water use of all stakeholders (provinces) by irregular ratios.

The alignments of differences in institutional structures and outcomes here echo the hypothesis that successful governance of SES emerged by indirectly (or vertically) creating links between different stakeholders (in the YRB cases, through administration).
When links The water quotas of 87-WAS (or the initial water rights) in our case studies went through a stage of ``bargaining'' among stakeholders (from 1982 to 1987) \cite{wang2019a, wang2019d}, where each province attempted to demonstrate its development potential related to water use.
The bargaining itself was also a process towards matches between their economic volume and water shares, as studies show that the large water users (like Shandong and Henan) need more water than their quota (in the 87-WAS) if only considering the economic equity when designing the institution.
Furthermore, with information asymmetry between upper-level decision-makers and lower-level stakeholders in water use allocation, those with more current water use might have greater bargaining power.
In practice, therefore, although the affected provinces may not have directly encouraged excessive resource use because of the institutional shift, they had a more considerable incentive to show their economic potential
That aligns with the historical records that, even after the 87-WAS had already confirmed the quotas, provinces, especially water-intensive ones, challenged it by appearing to the higher central government for larger quotas.
On the contrary, after YRCC as governing agent coordinated between stakeholders since 98-UBR, the external appeal of provinces for larger quotas turned into internal innovation to improve water efficiency (e.g., drastically increased water-conserving equipment, \textit{Supplementary Material S3})
\cite{krieger1955, ostrom1990}.
Then, the YRCC, the authority for approving water applications from all stakeholders, could adjust water use quotas according to the river conditions of the whole basin.
The 98-UBR led to a structure for achieving social-ecological fits in both basins (between YRCC and the YRB) and regions (between provincial economy and their water shares).

% \subsection{LIMITATION, INSIGHTS AND IMPLICATIONS}
\subsection{Limitation, insights and implications}
\label{discussion-3}
% discussion-3: 启示、未来的展望

Agents matching the ecological scale appear widespread as motifs in SES of successful governance, whether in fisheries, forests, or groundwater management, suggesting that reducing independent stakeholders linked to fragmentation is an essential primary mechanism for a structure to produce good results.
% 由于87-WAS和98-UBR分别带来的两种结构在很多的SES中都是反复出现的构件,我们提出的机制对理解这类耦合系统至关重要
Since the structures introduced by 87-WAS and 98-UBR are recurring motifs in many SES, our proposed mechanism is crucial to understanding such coupled systems.
% 因此,我们通过黄河流域的准自然实验,探讨了社会经济结构与可持续发展(结果)之间的因果关系,为两个主要原因提供了一个有益的案例研究。
Furthermore, we explored the causal linkages between the SES structures and sustainability (outcomes) in quasi-natural experiments of the YRB, which provides an informative case study for two main reasons.
% 首先,长江流域管理的急剧结构变化使我们能够定量估计高层制度设计变化对用水的净影响。
First, the sharp structural shifts in YRB management enabled us to quantitatively estimate the net effects of changes in high-level institutional design on water use. Institutions that determine water allocation include bottom-up agreements or social norms as well as top-down quotas or regulations, with different effects on SES structure \cite{wang2019d,speed2013}; top-down regulations can trigger immediate institutional shifts and sharp SES structural changes \cite{speed2013,roland2004}.
In comparison with investigations of more gradual changes induced by bottom-up institutional shifts, exploring the impacts of a top-down change substantially diminishes potential problems of omitted variables in the quantitative analysis of SES and clarifies the causal link between SES structure and outcome.
% 其次,通过比较长江流域两次制度变迁所分裂的三种不同制度结构的净效应,我们也可以更深入地理解“盆地固定效应”下结构格局的影响。尽管流域内的社会经济单位从世界各地的大型河流流域和许多地区的水资源中受益,但很少有流域多次经历过如此激进的社会经济结构变化。
Second, we can better understand the influence of structural alignments under a fixed basin by comparing the net effects of three different institutional structures split by two institutional shifts in the YRB. Although socioeconomic units within a basin benefit from water resources in large river basins all over the world, and many locations have shown increased levels of regulation, few basins have experienced such radical SES structural changes several times (see \textit{Supplementary Material} S1). Thus, the YRB provides a valuable setting for understanding the direct impacts of changes in the SES institutional structure.
Finally, one of the limitations of our method is that it is difficult to rule out the effects of other policies over the same time breakpoints.
However, since scholars have reached a consensus on the importance of the two institutional shifts of 87-WAS and 98-UBR, the differences in their results still provide important insights for understanding water governance.

% 我们的模型结果与机制探讨加深了SES结构的理解,强化了孤立利益相关者形成的结构不利于制度解决环境问题的基本认识
Our results and discussion deepen the understanding of SES structure and strengthen the basic understanding that the mismatched structure formed by isolated stakeholders is not conducive to institutional solutions; -and then reported how another social-ecological fit structure contributed to successful water governance and sustainability.
Moreover, the subsequent success of 98-UBR has proved the importance of institutional scale matching both theoretically and practically. Therefore, it is necessary to emphasize the establishment of potentially connected building blocks between stakeholders by agents consistent with the scale of the ecological system (in this case, the basinal scale and the YRCC).
Furthermore, we applied several scenarios based on the marginal benefit model (see \textit{Supplementary Material S4}) for some further insights into sustainable water governance.
For example, water rights transfers can be another way to emerge horizontal links between stakeholders that also have the potential in resulting in better water governance.
In addition, the policymakers can also propose a more dynamic and flexible institution by increasing the frequency of quota updates that responds to changing conditions and will adapt more effectively to its SES context.

% 未来的政策建议
Calls for a redesign of water allocation institutions in the YRB in recent years also illustrate the importance of dynamic quota setting (see \textit{Supplementary Material S1}) \cite{yu2019}. Following the institutional reforms of 1998, the Yellow River has not dried up since 1999. However, given recent changes in the YRB, its rigid resource allocation scheme can no longer meet the new demands of economic development \cite{wang2019a}. As a result, the Chinese government has embarked on an ambitious plan to redesign its decades-old water allocation institution (see \textit{Supplementary Material S1}). These initiatives can benefit from our analysis by actively considering and incorporating social-ecological complexity and incentive structures when developing new approaches that avoid unsustainable outcomes. Our research provides a cautionary tale of how institutions can be a double-edged sword in attaining sustainability. Therefore, insights from the YRB can be a valuable guideline for SESs around the world facing similar governance problems \cite{cumming2020b, muneepeerakul2017, cumming2020a, leslie2015}.



\section{Methods}
% 为了量化制度变迁为黄河流域用水带来的影响,我们按附图1所示的技术路线执行了分析过程
We estimated and analyzed the net effects of two SES structural changes of water use. The actual water use of the Yellow River Basin was peroxided by the sum of the water use of the target group provinces. To quantify water use, we used synthetic control methods to estimate possible trends of water use in the absence of institutional shifts. In addition, as a robustness test, we conducted a matched placebo test (creating a “null model”) to exclude the effects of other factors that were contemporaneous with the institutional shifts. Finally, we created an economic model based on marginal revenue to provide a theoretical explanation for the observed “sprint effect” phenomenon. A brief technical overview is given in \textit{Supplementary Material S2}.

\subsection{Dataset and variables}
% 我们使用中国1978年至2012年各省的年度用水量数据集,这个公开的数据集由全国水资源利用调查得到,详细可查看。
We used China’s provincial annual water consumption dataset from 1978 to 2012. This publicly available dataset was obtained from the National Water Resources Utilization Survey; details are accessible from Zhou (2020)
\cite{zhouDecelerationChinahuman2020}.
A total of 10 provinces or regions have been directly affected by the water allocation institutional shifts in the YRB, accounting for $8.6\%$ of the total population of China (in 1990). Eight provinces have been particularly affected because of their greater dependence on the water resources from the Yellow River (see \textit{Supplementary Material S2}). Therefore, we divided the dataset into a “target group” and a “control group”, treating provinces that were greatly affected as the target group $(n=8)$ and provinces that were not affected by the institutional shifts as the potential control group $(n=20)$.

We focused on two features of water use in the YRB: total water use and diversification of water allocation. The actual water uses are given by the dataset, but when the synthetic control method is used to predict the water use of the control group, other independent influences need to be considered. Thus, we used economic features that are highly related to water use to extrapolate demand (e.g., agriculture, industry, service industry, and domestics, see \textit{Supplementary Material S2, Table 1} To measure resource allocation diversification between the upper, middle, and lower reaches, we used “entropy” as a simple index,

$$ Index_{entropy} = \sum_i{p_i *log(p_i)} $$

% 式子中的p是每个使用的水占流域总用水的比例
Where $p_i$ is the proportion of water uses for region $i$ to the total water uses in the basin. A larger index value indicates the proportion of water resources actually used is closer to the average among the upper, middle, and lower reaches.

\subsection{Synthetic Control}
Synthetic control is an effective identification strategy for estimating the net effect of historical events or policy interventions on aggregate units (such as cities, regions, and countries) by constructing a comparable control unit \cite{abadieSyntheticControlMethods2010}.
In this study, we used a comparative event approach and compared actual post-institutional shift induced water use changes with an appropriate counterfactual of what the water use change would have been.
The counterfactual was built as the optimally weighted average of provinces not exposed to the institutional shifts.
The synthetic control method generalizes the difference-in-differences estimator and allows for time-varying individual-specific unobserved heterogeneity \cite{billmeierAssessingEconomicLiberalization2013, smithresourcecurseexorcised2015}.
In practice, each of the units (i.e., provinces) in the treated group were affected by institutional shifts in 1987 and 1998, each of which was taken as the “shifted” point $t_0$ and the two steady institutions as $t$ for analyzing in each shift. The synthetic control method generates the control unit by assigning a weight matrix $W$ to units of the potential control group, so that the treated unit and its control unit are similar in each variable before $t_0$, i.e.,

$$\min(V_{i}^{t<t_0} - W_i * F_{control}^{t<t_0})$$

where $V_i$ is a vector that indicates all features of a unit $i$ of the treated group, and $F_{control}$ is a matrix that consists of all features and units of the potential control group. $W_i$ is the weight matrix for target unit $i$. We minimized the root mean square error (RMSE) by using the Synth package in R \cite{abadieSynthPackageSynthetic2011, abadieComparativePoliticsSynthetic2015}. All codes are accessible in the repository.

% 这样一来,基于降维的思想,我们构建出了一系列特征上与实验组最为接近的可比对照。
In accordance with the idea of dimensionality reduction, we constructed a series of comparable control units that were most similar in characteristics to the treated units. Because the units of the control group were not affected by the institutional shifts, after giving the same weight to the total water use of the control group $M_i * WU_{control}$, the result $W_i*WU_{control}$ could be considered a reasonable estimation of the untreated situation. The net effect of the water allocation institutional shift was then estimated by calculating the difference of water uses after the institutional shift between the treated group and the control group, compared with the water use difference before the shift.


\subsection{Placebo Test}
% 作为一种稳健性检验,安慰剂测试的必要性体现在两个原因上。
For robustness, we conducted a placebo test because the synthetic control method neglects the influences of overall changes in factors in the same year by simply dividing time periods according to institutional shifts. Three steps were required to apply the placebo test:
% (1) 对目标组中的每个省份$i$,计算所有潜在控制组与它之间特征向量的欧氏距离
(1) For each province in the target group, we calculated the Euclidean distance of vectors between all provinces in the potential control group.
% (2) 将距离由小到大排序后,选取特征向量最相似的三个省份,所有特征的均值作为 $i$ 的替代
(2) After ranking the distances, the three provinces with the most similar economic context were used to generate an average paired treatment target unit.
% (3) 对这个配对目标同样实施控制合成法(潜在控制组排除构成它的三个省)
(3) We performed the same synthetic control analysis for this paired target (i.e., the potential control group excluding the three provinces in step 2).
% 经过上述步骤,我们在理论上构建了一个相似的“区域”并实施了同样的控制合成实验。
In this way, we theoretically constructed a pseudo-treated unit and performed the same synthetic control treatments. Because these placebo tests were directed at units unaffected by the institutional shifts, the results can be regarded as a reasonable baseline expectation or null model from which to assess the changes caused by other factors.

\subsection{Economic model}
In order to understand the mechanisms underlying the empirical results, we developed a dynamic economic model to analyze how institutional change could have led to the sprint effect in water use. Specifically, we modeled individual provincial decision-making in water resources before quota execution. The analysis result implied that the underlying driver of CPR overuse was incentive distortion.

In developing the model, we highlighted the main features of the YRB, as well as the water use institutions of 1987 and 1998. We proposed three intuitive and general assumptions.

\begin{ass}
    % (生产)为了简化,由于不可替代性,水是每个省的同质生产函数10的唯一投入。$F(x)$是连续的,满足Inada条件,即$F'(x)>0, F''(x)<0$(边际收益递减假设),$F'(0)=\infty$,$ F'(\infty)=0$。产品产量处于完全竞争状态,单位价格为15。
    (Water-dependent production) For simplicity, water is assumed to be the only input of the homogenous production function $F(x)$ of each province because of its irreplaceability. $F(x)$ is continuous and satisfies the Inada Conditions, i.e., $F'(x)>0, F''(x)<0$ (the diminishing marginal returns assumption), $F'(0)=\infty$,$ F'(\infty)=0$. The production output is under perfect competition, with a constant unit price of $P$.
\end{ass}
\begin{ass}
    (Ecological cost allocation) Under the assumption that the ecology is a single entity for the whole basin involved in N provinces, the cost of water use is equally assigned to each province under any water use. The unit cost of water is a constant $C$.
\end{ass}
\begin{ass}
    (Multi-period settings) There are infinite periods with a constant discount factor $\beta$ lying in (0,1). There is no cross-period smoothing in water uses.
\end{ass}

Under the above assumptions, we can demonstrate three cases consisting of local governments in YRB to simulate their water use decision-making and water use patterns.

\begin{case} Decentralized institution:
    This case corresponds to a situation without any high-level water allocation institution (i.e., before 1987, see Figure~\ref{fig:structure} B).

    When each province independently decides on its water use, the optimal water use $\hat x_i^*$ in province $i$ satisfies:
    $$F'(x)=\frac{C}{P \cdot N}$$

    When the decisions in different periods are independent, for $t=0,1,2 \ldots$, then:
    $$\hat x_{it}^*=\hat x_i^*$$

\end{case}

\begin{case} Mismatched institution
    This case corresponds to a mismatched institution (i.e., $1987\sim1998$, see Figure~\ref{fig:structure} C).

    The water quota is determined at $t=0$ and imposed in $t=1,2,\ldots$ The total quota is a constant denoted as $Q$, and the quota for province $i$ is determined in a proportional form:
    $$Q_i=Q \cdot \frac{x_i}{x_i + \begin{matrix} \sum x_{-i} \end{matrix}}$$

    Under a scenario with decentralized decision-making with a water quota institution, given other provinces’ water use decisions remain unchanged, the optimal water use $\widetilde x_{i0}^*$ of province $i$ at $t=0$ satisfies:

    $F'(x_{i,0})=\frac{C}{P \cdot N} - \frac{\beta}{1-\beta} \cdot f(Q \cdot \frac{x_{i,0}}{\begin{matrix} x_{i,0} + \sum x_{-i,0} \end{matrix}}) \cdot Q \cdot \frac{\begin{matrix} \sum x_{-i,0} \end{matrix}}{(\begin{matrix} x_{i,0} + \sum x_{-i,0} \end{matrix})^2}$.

    When future water use is constrained by a water quota, the dynamic optimization problem of province $i$ is shown as follows:

    $max  \quad P \cdot F(x_{i,0})-\frac{C \cdot \begin{matrix} \sum x_{i,0} + x_{-i,0} \end{matrix}}{N}+\beta P \cdot F(x_{i,1})+\beta^2 P \cdot F(x_{i,2})+...$

    $=P \cdot F(x_{i,0})-C \cdot \frac{x_{i,0} + \begin{matrix} \sum x_{-i,0} \end{matrix}}{N}+\frac{\beta}{1-\beta} P \cdot F(Q \cdot \frac{x_{i,0}}{x_{i,0} + \begin{matrix} \sum x_{-i,0} \end{matrix}})$

    First-order condition: $P \cdot F'(x_{i,0})-\frac{C}{N}+\frac{\beta}{1-\beta}[P \cdot f(Q \cdot \frac{x_{i,0}}{x_{i,0} + \begin{matrix} \sum x_{-i,0} \end{matrix}}) \cdot Q \cdot \frac{\begin{matrix} \sum x_{-i,0} \end{matrix}}{(x_{i,0}+\begin{matrix} \sum  x_{-i,0} \end{matrix})^2}]=0$

    where $f(\cdot)$ is the differential function of $F(\cdot)$.

    The optimal water use in province i at t=0 $\widetilde x_{i,0}^*$ satisfies $P \cdot F'(x_{i,0})=\frac{C}{N}-\frac{\beta}{1-\beta} \cdot P \cdot f(Q \cdot \frac{x_{i,0}}{x_{i,0} + \begin{matrix} \sum x_{-i,0} \end{matrix}}) \cdot Q \cdot \frac{\begin{matrix} \sum x_{-i,0} \end{matrix}}{(x_{i,0} + \begin{matrix} \sum x_{-i,0} \end{matrix})^2}$, i.e., $F'(x_{i,0})=\frac{C}{P \cdot N} - \frac{\beta}{1-\beta} \cdot f(Q \cdot \frac{x_{i,0}}{x_{i,0} + \begin{matrix} \sum x_{-i,0} \end{matrix}}) \cdot Q \cdot \frac{\begin{matrix} \sum x_{-i,0} \end{matrix}}{(x_{i,0} + \begin{matrix} \sum x_{-i,0} \end{matrix})^2}$.

\end{case}

\begin{case} Matched institution

    This case corresponds to the institution under which the YRCC centrally managed water allocation between provinces (i.e., $1998\sim2008$, see Figure~\ref{fig:structure} D).

    When the $N$ provinces decide on water uses as unified whole (e.g., the central government completely decides and controls on the water use in each province), the optimal water use $x_i^*$ of province $i$ satisfies:

    $$F'(x)=\frac{C}{P}$$

\end{case}

We propose Proposition 1 and Proposition 2:

\textbf{Proposition 1}: Compared with the decentralized institution, a matched institution with unified management decreases total water use.

Because $F’$ is monotonically decreasing, based on a comparison of costs and benefits for stakeholders (provinces) in the three cases,

$$\widetilde x_i^*>\hat x_i^*>x_i^*$$

The result of $\hat x_i^*>x_i^*$ indicates that individual rationality would deviate from collective rationality when property rights are unclear \cite{hardinTragedyCommons2009}, because of the common-pool characteristics of water
\cite{castilla-rhoGroundwaterCommonPool2020,ostromGeneralFrameworkAnalyzing2009}.

The difference of $\widetilde x_i^*$ and $\hat x_i^*$ stems from two parts: the marginal returns effect and the marginal costs effect. First, the “shadow value” provides additional marginal returns of water use in $t=0$, wwhich increases the incentives of water overuse by encouraging bargaining for a larger quota. Second, the future cost of water use would be degraded from $\frac{P}{N}$ to an irrelevant cost.

The optimal water use under the three cases implies that mismatched institutions cause incentive distortions and lead to resource overuse.

\textbf{Proposition 2}: The quota determination of the mismatched institution increases the incentives of current water use.

The intuition for this proposition is straight-forward in that all provinces would use up their allocated quota under a relatively small $Q$. As $Q$ increases, the quota would provide higher future benefits for a pre-emptive water use strategy. Since the provincial water use decisions are exactly symmetric, total water use would increase when each province has higher incentives for current water use. This situation corresponds to a “sprint” effect, where the total water use dramatically increases in the “sprint” period.

Extensions of the model are shown in \textit{Supplementary Material S3}.


\section{CONCLUSION}\label{conclusion}
%! Author = songshgeo
%! Date = 2022/3/10

Intense water use in one of the most anthropogenic interfered large river basins, the Yellow River Basin (YRB), once led to overburdened drying up but finally had a successful restoration by sequential water governance practices.
Focusing on two water-demand institutions, 87-WAS and the 98-UBR, we quantitatively analyzed how institutional shifts played a role in the water governance achievement of the YRB.
Shifting throughout different SES structures framed by them, the observed water use of the YRB provinces had an $8.57\%$ increase than expected during the decade after the 87-WAS.
Then, water use significantly decreased by  $4.9$ billions $m^3$ per year since the 98-UBR, while the model still suggests a $10.3$ billions $m^3$ annual increase in expectation.
Finally, as differences in stakeholders' response to the institutional shifts, water use rises after the 87-WAS in provinces with more water uses (e.g., Inner Mongolia, Henan, and Shandong) while shrunk in nearly all provinces after the 98-UBR.
Since the above results closely align with interpretations from a mathematical marginal benefits model, we can link the structures (widespread building blocks) and outcomes (goals of the institution, i.e., limiting water demands) by these quasi-natural experiments of the YRB.
We demonstrate that social-ecological fits lead to successful governance where reducing independent stakeholders linked to fragmentation is an essential primary mechanism for good SES outcomes.


\appendix
\label{appendix}
%%%%%%%% -----  02_appendix -------- %%%%%%%%%%

%% Put the bibliography here, most people will use BiBTeX in
%% which case the environment below should be replaced with
%% the \bibliography{} command.
\bibliography{WAInstitution_YRB_2021}


%% Here is the endmatter stuff: Supplementary Info, etc.
%% Use \item's to separate, default label is "Acknowledgements"


\end{document}
