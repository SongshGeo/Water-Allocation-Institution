\documentclass[11pt,a4paper,roman]{moderncv}

\graphicspath{{../../../figs/}}
\linespread{1.5}
\usepackage[english]{babel}
\usepackage{ragged2e}
\usepackage{float}
\usepackage{graphicx}
\usepackage[utf8]{inputenc}


\moderncvstyle{classic}
\moderncvcolor{blue} % Bullet point color

% Page margins
\usepackage[scale=0.85]{geometry} % Page margins

% Your Information, please revise
\name{Shuai}{Wang}
\address{Haidian District, Beijing, China}{No.19, Xinjiekouwai St.}
\phone[mobile]{+86 182 1009 3639}
\email{shuaiwang@bnu.edu.cn}


\begin{document}

% Insert Olin Logo
\begin{minipage}[t]{\textwidth}
\includegraphics[width=0.40\textwidth]{bnu}
\end{minipage}


\recipient{Dear Editor Barreteau}{}
\opening{\vspace*{-2em}}
\closing{Sincerely,}{\vspace*{-2em}}
\enclosure[Enclosures]{Main manuscript, Appendix}
\makelettertitle
% https://pythonjishu.com/qggkmtfnuj
% 调整两端对齐
\justify

I am writing to submit the revised version of our manuscript entitled ``Quantifying the Effects of Institutional Shifts on Water Governance in the Yellow River Basin: A Social-ecological System Perspective'' originally submitted to Journal of Hydrology (\#HYDROL52368). We would like to express our sincere gratitude for the opportunity to revise our manuscript and for the constructive comments from the reviewers which have been instrumental in enhancing the quality of our paper.

Following the guidance provided, we have meticulously revised the manuscript, paying particular attention to the clarity of the methodological section as suggested. We have addressed all the concerns raised by Reviewer 1, including the accurate representation of mathematical equations and the logical sequence of our methodology.

Enclosed, please find the revised manuscript, marked-up copy highlighting the changes, and a detailed response to reviewers' comments. We kindly request that you consider our manuscript for publication in Journal of Hydrology.

Thank you once again for considering our work. We look forward to hearing from you soon.

Sincerely,

\textbf{Shuai Wang} (On behalf of the author team)

State Key Laboratory of Earth Surface Processes and Resource Ecology,

Faculty of Geographical Science, Beijing Normal University, Beijing 100875, China

Email: shuaiwang@bnu.edu.cn

% \newpage
% \section{Responses}

% We have followed all your suggestions in our revision. Below are the editor's previous comments, followed by our responses.

% \textbf{Comments from editor board (Translated from Chinese):}
% \textit{The topic is important and relevant to current climate change and Earth system science. This paper takes the Yellow River as an example to carry out research, which is innovative. However, NSR, as a comprehensive journal, suggests the authors to further highlight methodological innovation and wider application in the future, rather than stay at the current level of the Yellow River. It is suggested that the author revise and improve the universality of the article based on this idea and consider submitting it for review.}


% \textit{\textbf{Responses}}

% We appreciate your recognition of the innovation of our methodology and the importance of the topic. More importantly, thank you for the valuable suggestions on highlighting approaches and applications to wider readers!

% Addressing the sustainability challenges that humanity is facing in the Anthropocene requires the coupling of human and natural systems or social-ecological systems. One of the most important scientific questions is to reveal the causal links and underlying processes between the structure and function of these coupled systems. The influences of institutions on the outcomes of social-ecological systems were widely reported worldwide, but few attempts to quantify their net effects.
% Therefore, we first add a figure to demonstrate the widely accepted general framework of SES where this knowledge gap remains, then show how our approach helps to fill that \textbf{(see the new Figure 1 in the manuscript)}.

% The Yellow River Basin (YRB) is one of the most anthropogenically altered large river basins. It was first overdrawn, then dried up, and finally has been successfully restored by institutional shifts. In this manuscript, we analyzed quasi-natural experiments of the YRB in institutional changes, which offers profound insights into the links between coupled systems' structures and outcomes. Therefore, in the improved manuscript, we emphasize the reason why YRB in China can be a valuable case study:

% \textit{(\textbf{L51-L58})}\textit{
%     To better understand how water governance institutions match their social-ecological context, we take the Yellow River Basin (YRB), China, as an example (see study area) to dive into causal links between SES structures and outcomes.
%     Specifically, we focused on two institutional shifts in water allocation of the YRB: the 1987 Water Allocation Scheme (87-WAS), and the 1998 Unified Basinal Regulation (98-UBR), which reframed SES structures significantly.
%     The YRB provides an informative case for two main reasons:
%     (1) The top-down institutional shifts induced sharp changes in SES structures, enabling us to estimate their net effects quantitatively.
%     (2) Since few large river basins have experienced such radical institutional shifts more than once, this case study provides comparable natural experiments for understanding the impacts of structural changes in SESs on natural resources.
% }

% Finally, we further emphasized the universality and future directions of the article in the discussion section. For example:

% \begin{itemize}
%     \item \textit{(\textbf{L148-L149})} \textit{On the other hand, social-ecological matches can also be supported by structure effects.}
%     \item \textit{(\textbf{L157-L158})} \textit{The structural building blocks we depicted here (Figure 2) have also been reported in other SESs worldwide...}
%     \item \textit{(\textbf{L178-L180})} \textit{The structural building blocks that led to different outcomes are recurring motifs in global SESs, so our proposed mechanism is crucial to governing such coupled systems.}
%     \item ...
% \end{itemize}

% We trust that this new version will satisfy you and thanks for your time on this manuscript again.

% % \vspace{0.5cm}

% % Thank you for your consideration!

% \vspace{0.5cm}



\end{document}
