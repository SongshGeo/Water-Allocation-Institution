\documentclass[11pt,a4paper,roman]{moderncv}
\usepackage[english]{babel}
\usepackage{ragged2e}
\usepackage{float}
\usepackage{graphicx}
\usepackage[utf8]{inputenc}


\moderncvstyle{classic}
\moderncvcolor{green} % Bullet point color

% Page margins
\usepackage[scale=0.8]{geometry} % Page margins

% Your Information, please revise
\name{Shuai}{Wang}
\address{Haidian District, Beijing, China}{No.19, Xinjiekouwai St.}
\phone[mobile]{+86 182 1009 3639}
\email{shuaiwang@bnu.edu.cn}


\begin{document}

% Insert Olin Logo
\begin{minipage}[t]{\textwidth}
\includegraphics[width=0.40\textwidth]{bnu}
\end{minipage}


\recipient{Dear Editor in chief,}{}
\opening{\vspace*{-2em}}
\closing{Sincerely,}{\vspace*{-2em}}
\enclosure[Enclosures]{Main manuscript, Appendix}
\makelettertitle

I am pleased to submit an original research article entitled ``Institutional shifts and sustainable water use of the Yellow River Basin'' for consideration for publication in the special issue ``coupled human and nature system'' of \textit{National Science Review}.

It's not easy to have a win-win situation in coupled human-nature systems, so governing them needs a deep understanding of institutional mechanisms. Those river basin systems successfully supporting sustainable water resource use are structurally well-aligned with water provisioning and social-ecological demands, without inefficient competition or overuses. Despite alignments of social-ecological system (SES) structures being a crucial approach to institutional matches, an understanding of its causal links and underlying processes are still weaknesses. The YRB is a valuable case study because it is one of the most anthropogenically altered large river basins. It was first overdrawn, then dried up, and finally has been successfully restored by institutional shifts. In this manuscript, we analyzed quasi-natural experiments of the YRB in institutional changes, which offers profound insights into the links between coupled systems' structures and outcomes.

We believe this manuscript is appropriate for publication by \textit{National Science Review} because it provides insights contributing to worldwide social-ecological systems management by exploring causal linkages between SES structures and sustainability-related outcomes. Furthermore, our approach of quasi-natural experiments analysis creates a paradigm for future studies of quantitatively analyzing how human-introduced institutions impact the outcome of complex coupled human-nature systems.

This manuscript has not been published and is not under consideration for publication elsewhere.  We have no conflicts of interest to disclose. If you feel that the manuscript is appropriate for your journal, we suggest the following reviewers:

\begin{itemize}
    \item Dr. Murugesu Sivapalan is a professor at the University of Illinois. He focuses on socio-hydrology, linking water and human society with basin studies. His email is sivapala@illinois.edu
    \item Dr. Xin Li is a researcher at the Institute of Tibetan Plateau Research, Chinese Academy of Sciences. He made progress in integrated river basin studies. His email is xinli@itpcas.ac.cn
    \item Dr. Giuliano Di Baldassarre is a professor at Uppsala University. He has expertise in water management analysis of transboundary river basins. His email is Giuliano.Dibaldassarre@geo.uu.se
    \item Dr. Yongping Wei is a researcher at the University of Queensland. Her study interests contain sustainable water management. Her email is yongping.wei@uq.edu.au
    \item Dr. Fuqiang Tian is a professor at the Tsinghua Univetsity. He focuses on socio-hydrology and integrated river basin management. His email is tianfq@tsinghua.edu.cn
\end{itemize}

\vspace{0.5cm}

Thank you for your consideration!

\vspace{0.5cm}

Sincerely,

\textbf{Shuai Wang} (On behalf of the author team)

%\begin{minipage}[t]{\textwidth}
%	\includegraphics[height=1cm]{signature.jpeg}
%\end{minipage}

State Key Laboratory of Earth Surface Processes and Resource Ecology,

Faculty of Geographical Science, Beijing Normal University, Beijing 100875, China

Email: shuaiwang@bnu.edu.cn

\end{document}
