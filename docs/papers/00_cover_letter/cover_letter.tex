\documentclass[11pt,a4paper,roman]{moderncv}

\graphicspath{{../../../figs/}}
\linespread{1.5}
\usepackage[english]{babel}
\usepackage{ragged2e}
\usepackage{float}
\usepackage{graphicx}
\usepackage[utf8]{inputenc}


\moderncvstyle{classic}
\moderncvcolor{green} % Bullet point color

% Page margins
\usepackage[scale=0.8]{geometry} % Page margins

% Your Information, please revise
\name{Shuai}{Wang}
\address{Haidian District, Beijing, China}{No.19, Xinjiekouwai St.}
\phone[mobile]{+86 182 1009 3639}
\email{shuaiwang@bnu.edu.cn}


\begin{document}

% Insert Olin Logo
\begin{minipage}[t]{\textwidth}
\includegraphics[width=0.40\textwidth]{bnu}
\end{minipage}


\recipient{Dear Editor in chief,}{}
\opening{\vspace*{-2em}}
\closing{Sincerely,}{\vspace*{-2em}}
\enclosure[Enclosures]{Main manuscript, Appendix}
\makelettertitle

I am pleased to re-submit (Previous ID was ``MS-2022-716'') our original research article entitled ``Approaching social-ecological matches of river basin systems for sustainability'' for consideration for publication in the special issue ``coupled human and nature system'' of \textit{National Science Review}. We thank the editors of \textit{National Science Review} who suggested we improve the manuscript by highlighting methodological innovation and wider applications in the future. In the new manuscript, we mindfully followed all the suggestions and attached a response letter in detail to introduce how our improved manuscript highlights our contribution to help in sustainability challenges worldwide.

We trust that this new version will satisfy you and is appropriate for publication by \textit{National Science Review} now because it provides insights contributing to worldwide social-ecological systems management by exploring causal linkages between SES structures and sustainability-related outcomes. Furthermore, our approach of quasi-natural experiments analysis creates a paradigm for future studies of quantitatively analyzing how human-introduced institutions impact the outcome of complex coupled human-nature systems. Thank you once again for the time that you have spent reviewing this manuscript.

The manuscript includes 1 text (4156 words), 5 figures, 1 supplementary material (appendix with four sections). We have no conflicts of interest to disclose.
We look forward to hearing from you.

% \begin{itemize}
%     \item Dr. Murugesu Sivapalan is a professor at the University of Illinois. He focuses on socio-hydrology, linking water and human society with basin studies. His email is sivapala@illinois.edu
%     \item Dr. Xin Li is a researcher at the Institute of Tibetan Plateau Research, Chinese Academy of Sciences. He made progress in integrated river basin studies. His email is xinli@itpcas.ac.cn
%     \item Dr. Giuliano Di Baldassarre is a professor at Uppsala University. He has expertise in water management analysis of transboundary river basins. His email is Giuliano.Dibaldassarre@geo.uu.se
%     \item Dr. Yongping Wei is a researcher at the University of Queensland. Her study interests contain sustainable water management. Her email is yongping.wei@uq.edu.au
%     \item Dr. Fuqiang Tian is a professor at the Tsinghua Univetsity. He focuses on socio-hydrology and integrated river basin management. His email is tianfq@tsinghua.edu.cn
% \end{itemize}

Sincerely,

\textbf{Shuai Wang} (On behalf of the author team)

State Key Laboratory of Earth Surface Processes and Resource Ecology,

Faculty of Geographical Science, Beijing Normal University, Beijing 100875, China

Email: shuaiwang@bnu.edu.cn

% \newpage
% \section{Responses}

% We have followed all your suggestions in our revision. Below are the editor's previous comments, followed by our responses.

% \textbf{Comments from editor board (Translated from Chinese):}
% \textit{The topic is important and relevant to current climate change and Earth system science. This paper takes the Yellow River as an example to carry out research, which is innovative. However, NSR, as a comprehensive journal, suggests the authors to further highlight methodological innovation and wider application in the future, rather than stay at the current level of the Yellow River. It is suggested that the author revise and improve the universality of the article based on this idea and consider submitting it for review.}


% \textit{\textbf{Responses}}

% We appreciate your recognition of the innovation of our methodology and the importance of the topic. More importantly, thank you for the valuable suggestions on highlighting approaches and applications to wider readers!

% Addressing the sustainability challenges that humanity is facing in the Anthropocene requires the coupling of human and natural systems or social-ecological systems. One of the most important scientific questions is to reveal the causal links and underlying processes between the structure and function of these coupled systems. The influences of institutions on the outcomes of social-ecological systems were widely reported worldwide, but few attempts to quantify their net effects.
% Therefore, we first add a figure to demonstrate the widely accepted general framework of SES where this knowledge gap remains, then show how our approach helps to fill that \textbf{(see the new Figure 1 in the manuscript)}.

% The Yellow River Basin (YRB) is one of the most anthropogenically altered large river basins. It was first overdrawn, then dried up, and finally has been successfully restored by institutional shifts. In this manuscript, we analyzed quasi-natural experiments of the YRB in institutional changes, which offers profound insights into the links between coupled systems' structures and outcomes. Therefore, in the improved manuscript, we emphasize the reason why YRB in China can be a valuable case study:

% \textit{(\textbf{L51-L58})}\textit{
%     To better understand how water governance institutions match their social-ecological context, we take the Yellow River Basin (YRB), China, as an example (see study area) to dive into causal links between SES structures and outcomes.
%     Specifically, we focused on two institutional shifts in water allocation of the YRB: the 1987 Water Allocation Scheme (87-WAS), and the 1998 Unified Basinal Regulation (98-UBR), which reframed SES structures significantly.
%     The YRB provides an informative case for two main reasons:
%     (1) The top-down institutional shifts induced sharp changes in SES structures, enabling us to estimate their net effects quantitatively.
%     (2) Since few large river basins have experienced such radical institutional shifts more than once, this case study provides comparable natural experiments for understanding the impacts of structural changes in SESs on natural resources.
% }

% Finally, we further emphasized the universality and future directions of the article in the discussion section. For example:

% \begin{itemize}
%     \item \textit{(\textbf{L148-L149})} \textit{On the other hand, social-ecological matches can also be supported by structure effects.}
%     \item \textit{(\textbf{L157-L158})} \textit{The structural building blocks we depicted here (Figure 2) have also been reported in other SESs worldwide...}
%     \item \textit{(\textbf{L178-L180})} \textit{The structural building blocks that led to different outcomes are recurring motifs in global SESs, so our proposed mechanism is crucial to governing such coupled systems.}
%     \item ...
% \end{itemize}

% We trust that this new version will satisfy you and thanks for your time on this manuscript again.

% % \vspace{0.5cm}

% % Thank you for your consideration!

% \vspace{0.5cm}



\end{document}
