Increasing competition for water is leading to depletion of freshwater globally and calls for an urgent transformation of the governance system. To quantitatively analyse how institutions contributed to water governance, we focus on institutional shifts of the Yellow River Basin (YRB), one of the most anthropogenic interfered large river basins overburdened in water use, then drying up, but finally successfully restored. Our results suggest that two institutional shifts, the Water Allocation Scheme since 1987 (87-WAS) and the Unified Basinal Regulation since 1998 (98-UBR), framed different structures of social-ecological systems (SESs) in regional and basinal water use. During the decade after the 87-WAS, the observed water use of the YRB had an $8.57\%$ increase than an expectation. However, the 98-UBR significantly decreased total water use by $4.9$ billion $m^3/yr$. Specifically, the 87-WAS stimulated water use in provinces with more water uses (e.g., Inner Mongolia, Henan, and Shandong), but the 98-UBR regulated nearly all provinces. Linking our results to a mathematical marginal benefits model, we suggest that the outcomes with regional variations come from the effects of SES structural changes. These quasi-natural experiments of the YRB deepened insights on SESs structures and outcomes, thus providing a valuable guideline for SESs worldwide facing water depletion.
