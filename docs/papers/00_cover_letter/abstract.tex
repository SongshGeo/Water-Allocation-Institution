Increasing competition for water is leading to depletion of freshwater globally and calls for an urgent transformation of water governance. To better understand how institutions contribute to water governance, we quantified institutional shifts for the Yellow River Basin (YRB).  The YRB is a valuable case study because it is one of the most anthropogenically altered large river basins. It was first overdrawn, then dried up, and finally has been successfully restored. Our results suggest that two institutional shifts, the Water Allocation Scheme that began in 1987 (87-WAS) and the Unified Basinal Regulation that took over in 1998 (98-UBR), framed different social-ecological system (SES) structures. During the decade following the introduction of the 87-WAS, observed water use of the YRB increased by $8.57\%$ more than expected while 98-UBR successfully decreased total water use, ultimately. Specifically, 87-WAS stimulated increased water use in some provinces (e.g., Inner Mongolia, Henan, and Shandong), but 98-UBR regulated nearly all provinces. A mathematical economic model supports the hypothesis that regional variations were driven by SES structural changes. The quasi-natural experimental setting of the YRB and its significant structural changes over time offer deep insights into the links between SESs structures and outcomes, providing valuable guidelines for SESs worldwide that are facing water depletion.
