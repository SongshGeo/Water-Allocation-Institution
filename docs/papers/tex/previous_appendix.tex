
%DIF LATEXDIFF DIFFERENCE FILE
%DIF DEL tex/appendix.tex   Sun Mar 20 09:12:32 2022
%DIF ADD tex/appendix.tex   Sun Mar 20 09:12:32 2022
\documentclass{article}
\usepackage{hyperref}
\usepackage{graphicx}
\usepackage{setspace}
\usepackage[a4paper, total={7in, 10in}]{geometry}
\usepackage{amsmath}

\setstretch{1.523}

\newtheorem{ass}{Assumption}
\newtheorem{case}{Case}
%% make sure you have the nature.cls and naturemag.bst files where
%% LaTeX can find them

\bibliographystyle{naturemag}

\title{Supplementary Materials for

Institutional shifts and water sustainability of the Yellow River Basin
}

%% Notice placement of commas and superscripts and use of &
%% in the author list

% \author{Shuang Song$^{1,2}$, Huiyu Wen$^3$, *Shuai Wang$^{1,2}$, Graeme S. Cumming$^{4}$ \& Bojie Fu$^{1, 2}$}
%DIF PREAMBLE EXTENSION ADDED BY LATEXDIFF
%DIF UNDERLINE PREAMBLE %DIF PREAMBLE
\RequirePackage[normalem]{ulem} %DIF PREAMBLE
\RequirePackage{color}\definecolor{RED}{rgb}{1,0,0}\definecolor{BLUE}{rgb}{0,0,1} %DIF PREAMBLE
\providecommand{\DIFaddtex}[1]{{\protect\color{blue}\uwave{#1}}} %DIF PREAMBLE
\providecommand{\DIFdeltex}[1]{{\protect\color{red}\sout{#1}}}                      %DIF PREAMBLE
%DIF SAFE PREAMBLE %DIF PREAMBLE
\providecommand{\DIFaddbegin}{} %DIF PREAMBLE
\providecommand{\DIFaddend}{} %DIF PREAMBLE
\providecommand{\DIFdelbegin}{} %DIF PREAMBLE
\providecommand{\DIFdelend}{} %DIF PREAMBLE
\providecommand{\DIFmodbegin}{} %DIF PREAMBLE
\providecommand{\DIFmodend}{} %DIF PREAMBLE
%DIF FLOATSAFE PREAMBLE %DIF PREAMBLE
\providecommand{\DIFaddFL}[1]{\DIFadd{#1}} %DIF PREAMBLE
\providecommand{\DIFdelFL}[1]{\DIFdel{#1}} %DIF PREAMBLE
\providecommand{\DIFaddbeginFL}{} %DIF PREAMBLE
\providecommand{\DIFaddendFL}{} %DIF PREAMBLE
\providecommand{\DIFdelbeginFL}{} %DIF PREAMBLE
\providecommand{\DIFdelendFL}{} %DIF PREAMBLE
%DIF HYPERREF PREAMBLE %DIF PREAMBLE
\providecommand{\DIFadd}[1]{\texorpdfstring{\DIFaddtex{#1}}{#1}} %DIF PREAMBLE
\providecommand{\DIFdel}[1]{\texorpdfstring{\DIFdeltex{#1}}{}} %DIF PREAMBLE
\newcommand{\DIFscaledelfig}{0.5}
%DIF HIGHLIGHTGRAPHICS PREAMBLE %DIF PREAMBLE
\RequirePackage{settobox} %DIF PREAMBLE
\RequirePackage{letltxmacro} %DIF PREAMBLE
\newsavebox{\DIFdelgraphicsbox} %DIF PREAMBLE
\newlength{\DIFdelgraphicswidth} %DIF PREAMBLE
\newlength{\DIFdelgraphicsheight} %DIF PREAMBLE
% store original definition of \includegraphics %DIF PREAMBLE
\LetLtxMacro{\DIFOincludegraphics}{\includegraphics} %DIF PREAMBLE
\newcommand{\DIFaddincludegraphics}[2][]{{\color{blue}\fbox{\DIFOincludegraphics[#1]{#2}}}} %DIF PREAMBLE
\newcommand{\DIFdelincludegraphics}[2][]{% %DIF PREAMBLE
\sbox{\DIFdelgraphicsbox}{\DIFOincludegraphics[#1]{#2}}% %DIF PREAMBLE
\settoboxwidth{\DIFdelgraphicswidth}{\DIFdelgraphicsbox} %DIF PREAMBLE
\settoboxtotalheight{\DIFdelgraphicsheight}{\DIFdelgraphicsbox} %DIF PREAMBLE
\scalebox{\DIFscaledelfig}{% %DIF PREAMBLE
\parbox[b]{\DIFdelgraphicswidth}{\usebox{\DIFdelgraphicsbox}\\[-\baselineskip] \rule{\DIFdelgraphicswidth}{0em}}\llap{\resizebox{\DIFdelgraphicswidth}{\DIFdelgraphicsheight}{% %DIF PREAMBLE
\setlength{\unitlength}{\DIFdelgraphicswidth}% %DIF PREAMBLE
\begin{picture}(1,1)% %DIF PREAMBLE
\thicklines\linethickness{2pt} %DIF PREAMBLE
{\color[rgb]{1,0,0}\put(0,0){\framebox(1,1){}}}% %DIF PREAMBLE
{\color[rgb]{1,0,0}\put(0,0){\line( 1,1){1}}}% %DIF PREAMBLE
{\color[rgb]{1,0,0}\put(0,1){\line(1,-1){1}}}% %DIF PREAMBLE
\end{picture}% %DIF PREAMBLE
}\hspace*{3pt}}} %DIF PREAMBLE
} %DIF PREAMBLE
\LetLtxMacro{\DIFOaddbegin}{\DIFaddbegin} %DIF PREAMBLE
\LetLtxMacro{\DIFOaddend}{\DIFaddend} %DIF PREAMBLE
\LetLtxMacro{\DIFOdelbegin}{\DIFdelbegin} %DIF PREAMBLE
\LetLtxMacro{\DIFOdelend}{\DIFdelend} %DIF PREAMBLE
\DeclareRobustCommand{\DIFaddbegin}{\DIFOaddbegin \let\includegraphics\DIFaddincludegraphics} %DIF PREAMBLE
\DeclareRobustCommand{\DIFaddend}{\DIFOaddend \let\includegraphics\DIFOincludegraphics} %DIF PREAMBLE
\DeclareRobustCommand{\DIFdelbegin}{\DIFOdelbegin \let\includegraphics\DIFdelincludegraphics} %DIF PREAMBLE
\DeclareRobustCommand{\DIFdelend}{\DIFOaddend \let\includegraphics\DIFOincludegraphics} %DIF PREAMBLE
\LetLtxMacro{\DIFOaddbeginFL}{\DIFaddbeginFL} %DIF PREAMBLE
\LetLtxMacro{\DIFOaddendFL}{\DIFaddendFL} %DIF PREAMBLE
\LetLtxMacro{\DIFOdelbeginFL}{\DIFdelbeginFL} %DIF PREAMBLE
\LetLtxMacro{\DIFOdelendFL}{\DIFdelendFL} %DIF PREAMBLE
\DeclareRobustCommand{\DIFaddbeginFL}{\DIFOaddbeginFL \let\includegraphics\DIFaddincludegraphics} %DIF PREAMBLE
\DeclareRobustCommand{\DIFaddendFL}{\DIFOaddendFL \let\includegraphics\DIFOincludegraphics} %DIF PREAMBLE
\DeclareRobustCommand{\DIFdelbeginFL}{\DIFOdelbeginFL \let\includegraphics\DIFdelincludegraphics} %DIF PREAMBLE
\DeclareRobustCommand{\DIFdelendFL}{\DIFOaddendFL \let\includegraphics\DIFOincludegraphics} %DIF PREAMBLE
%DIF LISTINGS PREAMBLE %DIF PREAMBLE
\RequirePackage{listings} %DIF PREAMBLE
\RequirePackage{color} %DIF PREAMBLE
\lstdefinelanguage{DIFcode}{ %DIF PREAMBLE
%DIF DIFCODE_UNDERLINE %DIF PREAMBLE
  moredelim=[il][\color{red}\sout]{\%DIF\ <\ }, %DIF PREAMBLE
  moredelim=[il][\color{blue}\uwave]{\%DIF\ >\ } %DIF PREAMBLE
} %DIF PREAMBLE
\lstdefinestyle{DIFverbatimstyle}{ %DIF PREAMBLE
	language=DIFcode, %DIF PREAMBLE
	basicstyle=\ttfamily, %DIF PREAMBLE
	columns=fullflexible, %DIF PREAMBLE
	keepspaces=true %DIF PREAMBLE
} %DIF PREAMBLE
\lstnewenvironment{DIFverbatim}{\lstset{style=DIFverbatimstyle}}{} %DIF PREAMBLE
\lstnewenvironment{DIFverbatim*}{\lstset{style=DIFverbatimstyle,showspaces=true}}{} %DIF PREAMBLE
%DIF END PREAMBLE EXTENSION ADDED BY LATEXDIFF

\begin{document}

\maketitle
\input{/Users/songshgeo/Documents/Pycharm/WAInstitution_YRB_2021/docs/papers/authors.md}

\begin{abstract}
    % 这个补充材料以下述逻辑组织而成:
    The supplementary material is organized logically as follows:
    % 展示流域水资源分配制度在全球的推行情况并详细介绍黄河流域的制度变化,旨在说明黄河流域为什么是一个独一无二的准自然实验。
    (1) Show the implementation of the water resource allocation system in the basin around the world and introduce the institutional changes in the Yellow River Basin in detail, aiming to explain why the Yellow River Basin is a unique quasi-natural experiment.
    % 展示定量分析方法的技术路线图,并详细介绍定量分析方法中的细节部分。
    (2) Show the technical roadmap of quantitative analysis method, and introduce the details of quantitative analysis method in detail.
    % 基于我们建立的一般经济模型,对一些可能的情景进行进一步分析。
    (3) Further extensions based on our general economic model.
\end{abstract}

\section*{S1. Detailed introduction of the institutions}
%! Author = songshgeo
%! Date = 2022/3/10

% 水资源分配方案在全世界范围内都是流域管理的普遍制度。
Water allocation schemes are widespread in large river basin management programs throughout the world (see \textit{Supplementary Material Figure~\ref{fig:world}}) \cite{speedBasinwaterallocation2013}.
% 其中中国的黄河流域以典型的自上而下进行制度改革,这种模式会对制度产生迅速的影响,使我们能够定量地估计高层制度设计变化对用水的净影响。
The Yellow River Basin (YRB) in China has a typical top-down approach to institutional reform, which produces rapid institutional impacts and allows researchers to quantitatively estimate the net impact of changes in high-level institutional design on water use (see \textit{Supplementary Material Figure~\ref{fig:framework}}).
% 作为中国最早实施水资源分配制度的流域,黄河的制度变化可以在几个水利部发布的文件中得以窥见
This was the first basin in China for which a water resource allocation institution was created, and institutional shifts can be traced through several documents released by the Chinese government (at the national level):
\begin{itemize}
    \item \textbf{1982}: The provinces and the Yellow River Water Conservancy Commission (YRCC) are required to develop a water resource plan for the Yellow River \cite{wangReviewImplementationYellow2019, wangThingsCurrentSignificance2019}.
    \item \textbf{1987}: Implementation of the Allocation Plan. (\href{http://www.gov.cn/zhengce/content/2011-03/30/content_3138.htm#}{http://www.mwr.gov.cn}, last access: \today).
    \item \textbf{1998}: Implementation of unified regulation. (\href{http://www.mwr.gov.cn/ztpd/2013ztbd/2013fxkh/fxkhswcbcs/cs/flfg/201304/t20130411_433489.html}{http://www.mwr.gov.cn}, last access: \today).
    % 各省按要求编制新的黄河流域水资源规划,将水资源额度分配进一步细化。
    \item \textbf{2008}: Provinces are asked to draw up new water resources plans for the YRB to further refine water allocations \cite{wangReviewImplementationYellow2019, wangThingsCurrentSignificance2019}.
    \item \textbf{2021}: A call for redesigning the water allocation institution (\href{http://www.ccgp.gov.cn/cggg/zygg/gkzb/202107/t20210721_16591901.htm}{http://www.ccgp.gov.cn}, last access: \today).
\end{itemize}

% 在上述文件中,1982年的文件标志着设计分水制度尝试的开始,2008年标志着该制度走向成熟(完全建立起流域-省-市区的多级水资源分配和统一调度)。
The 1982 document marked the beginning of the attempt to design a water allocation institution, and the 2008 document marked the maturity of the system (complete establishment of basin-level, provincial, and district water allocation and unified regulation). Currently, a major overhaul is in the planning stages. Major shifts of the institution can be analyzed by using the 1987 and 1998 documents. It is worth noting that, although the essential reason for these institutions was the mismatch between the spatial and temporal distribution of water resources as well as social and economic water demands, the direct reason for their introduction was the drying-up of the Yellow River \cite{wangReviewImplementationYellow2019}.

\textbf{The official documents in 1987 clearly convey the following key points:}
\href{http://www.gov.cn/zhengce/content/2011-03/30/content_3138.htm#}{http://www.mwr.gov.cn}, last access: \today.
\begin{itemize}
	% 该政策面向的目标是各省(区域),黄委会没有被提及
	\item The policy is aimed at related provinces (or regions), and the YRCC is not mentioned.
	% 政策制定的首要考虑是解决断流问题
	\item Drying-up of the river is identified as the first consideration of this institution.
	% 各省被鼓励在此配额下制定自己的用水计划
	\item Provinces are encouraged to develop their own water use plans based on a quota system.
	% 水资源供给无法满足需求对相关省(地区)是普遍现象。
	\item Water in short supply is a common phenomenon in relevant provinces (regions).
\end{itemize}

\textbf{The official documents in 1998 clearly convey the following key points:}
\href{http://www.mwr.gov.cn/ztpd/2013ztbd/2013fxkh/fxkhswcbcs/cs/flfg/201304/t20130411_433489.html}{http://www.mwr.gov.cn}, last access: \today.
\begin{itemize}
	% 除了说明政策针对的各省区之外,明确指出其用水需要黄河水利委员会进行申报,并由其组织和监管
	\item The document clearly points out that not only provinces and autonomous regions involved in water resources management (see \textit{Article 3}), the provinces’ and regions’ water use shall be declared, organized, and supervised by the YRCC (\textit{Article 11 and Chapter III to Chapter V, and Chapter VII}).
	% 本研究(\textit{ Article 1})首先考虑的是上、中、下游用水的总体规划。
	\item Creating the overall plan of water use in the upper, middle, and lower reaches is identified as the first consideration of this institution (\textit{Article 1}).
	% 各省需要
	\item With the same quota as used in the 1987 policy, provinces were encouraged to further distribute their quota into lower-level administrations (see \textit{Article 6 and Article 41}).
	% 强调以总量确定供给,以供给决定需求。
	\item They emphasize that supply is determined by total quantity, and water use should not exceed the quota proposed in 1987 (see \textit{Article 2}).
\end{itemize}

% 基于上述分析,我们抽象出了如正文图2中所展示的流域水资源分配制度的运行结构,并将研究时段选择在1975到2008之间。
On the basis of the above analysis, we abstracted the operational structure of the water resource allocation institution of the YRB as shown in Figure 2 of the main text, focusing on the period between 1975 and 2008. By comparing the net effects of three different institutional structures split by the two institutional shifts, we were able to reach a stronger understanding of the influence of structural alignments under the same basin (previous structure-based analysis usually focus on a certain type of systems but with different geographic contexts).

\begin{figure}[!htb]
    \centering
    \includegraphics[width=12cm]{/Users/songshgeo/Documents/Pycharm/WAInstitution_YRB_2021/figs/diagrams/world_institutions}
	\caption{
		Overview of water allocation institutions.
		% 世界已有水资源分配制度的大河流域,其中黄河流域最早于1987年提出资源分配方案,后于1998年更改为统一调度方案。
		\textbf{A.} Major river basins in the world with existing water resource allocation systems (shaded red); the YRB first proposed a resource allocation scheme in 1987 (designed in 1983) and then changed to a unified regulation scheme in 1998 (designed in 1997 but implemented in 1998).
		% 不同的水资源分配制度设计模式,中国黄河流域是典型的自上而下。
		\textbf{B.} Different water resource allocation system design patterns; the YRB is typical of a top-down system with multiple levels.
		% 流域分水制度的演化。这种多层次的制度设计有其历史变化过程。
		\textbf{C.} The four periods of institutional evolution of water allocation of the YRB.
	}
    \label{fig:world}
\end{figure}

\begin{figure}[!htb]
    \centering
	\includegraphics[width=12cm]{/Users/songshgeo/Documents/Pycharm/WAInstitution_YRB_2021/figs/diagrams/YRB_scheme}
	\caption{
		% 中国自上而下的水资源分配制度模式(以黄河流域为例)
		A top-down water resources allocation scheme in China (a case study of the YRB).
		% 共有十个省或地区从黄河获取地表水资源,其中8个省的依赖性较强(see Supplementary Material Fig)。
		\textbf{A.} A total of 10 provinces or regions withdraw surface water resources from the Yellow River, of which 8 are highly dependent on the river (see \textit{Supplementary Material S2}).
		% 中国政府是下发流域管理政策的最高权威,其政策常常能够很快自上而下全面推行。
		\textbf{B.} The Chinese government is the ultimate authority in issuing watershed management policies, which are often quickly implemented from top down.
		% 流域管理机构(在黄河是黄河水利委员会)主要负责顺从国家的政策方针来进行流域的日常管理工作。
		\textbf{C.} The basin-level agency (here, the YRCC) is primarily responsible for the river-related management of the basin in accordance with national policy guidelines.
		% 利益相关者是各省
		\textbf{D.} The water management system directly affects the process by which local governments (major stakeholders) plan and use water resources for development. Although only surface water (Sur.) is usually traced and restricted, it can also influence groundwater through related hydro-processes such as recharge.
	}
	\label{fig:framework}
\end{figure}


\section*{S2. The technical roadmap and details of quantitative analyzing}
%! Author = songshgeo
%! Date = 2022/3/19

To quantify and interpret the institutional effects, we followed the technical route illustrated in \textit{Supplementary Material Figure~\ref{fig:roadmap}}. It was a two-step process in which we
% 基于数据使用合成控制法和安慰剂匹配法,创建可比的控制对照和期望(零模型)来分析制度变化对用水产生的影响。
(i) created a comparable control group and an expectation (a “null model”) to analyze the impact of the institutional shifts on water use by using synthetic control and placebo matching methods, and
% 创建一个基于边际收益的一般经济模型,解释我们发现的“冲刺效应”,并设计情景来进一步分析。
(ii) created a general economic model based on marginal revenue to explain the “sprint effect” we observed and designed scenarios for further analysis.

\begin{figure}[!hbt]
    \centering
    \includegraphics[width=12cm]{/Users/songshgeo/Documents/Pycharm/WAInstitution_YRB_2021/figs/diagrams/roadmap}
    \caption{
        Technical roadmap for quantitatively analyzing the “sprint effect”. WU is the the core variable we concerned -total water uses of the Yellow River Basin.
    }
    \label{fig:roadmap}
\end{figure}

% 为了使用合成控制法预测没有制度变化下的用水变化趋势,需要使用影响用水的社会经济数据作为自变量输入。
In order to use the synthetic control method to predict the trend of water use without institutional change, socioeconomic data affecting water use were used as the input independent variables (see the \textit{Methods in the main text}). All variables used are listed in Table~\ref{tab:variables}.
% 这些变量涉及了省级单元在利用水资源时需要考虑的主要经济因素(农业、工业、服务业与居民生活),它们与因变量(水资源使用)的相关性如图所示。
These variables refer to major economic factors (agriculture, industry, service industry, and domestic) that provincial units need to take into account when using water resources; their correlation coefficients with the dependent variable (water resource use) are shown in Figure~\ref{fig:linear} A.

% 此外,合成控制法有一个假设是自变量的线性组合可以有效预测因变量,因此我们还对数据线性拟合的效果进行了测试。
In addition, the synthetic control method assumes that a linear combination of independent variables can effectively predict the dependent variables.
% 将数据集区分为80%的训练样本和20%的测试样本,我们使用训练样本构建了多元线性模型对用水量进行了预测,并使用测试数据集去检验模型拟合效果。
We therefore divided the dataset into two groups, training samples $(80\%)$ and test samples $(20\%)$, and used the training samples to build a multivariate linear model to predict the water consumption. We then used the test dataset to test the model-fitting effect.
% 结果表明拟合优度R2超过了0.8,因变量可以很好的被自变量的线性组合所解释,该数据集可以用于控制合成法。
Results show that the goodness of fit $R^2$ exceeded $0.8$; thus, the dependent variable is well-explained by the linear combination of independent variables, and the dataset can be used in the synthetic control method (Figure~\ref{fig:linear}B).
% 为了估计用水量变化的期望值,我们对黄河流域各省用各自变量最接近的三个其它省(最近邻法)匹配了一个用作安慰剂实验的数据集
To estimate the expected water use changes, we used the placebo test method as a null model (see the \textit{Methods}), which gives a comparable baseline for matched datasets.
% 从图可以看出,匹配的数据集无论是数值还是变化趋势都与实际数据相似。
Figure~\ref{fig:placebo} indicates that the matched dataset is similar to the actual data both in values and change trends. This means that the differences between the predictions made from actual data and from the matched data by the synthetic control method are more likely to be result of the impacts of institutional shifts.

\begin{table}[!h]
	\caption{Variables and their categories for water use predictions}
	\scriptsize
	\label{tab:variables}
	\begin{tabular}{lllll}
	\hline
	Economic sector &
	  Category &
	  Unit &
	  Description &
	  Variables \\ \hline
	Agriculture &
	  Irrigation Area &
	  thousand ha &
	  \begin{tabular}[c]{@{}l@{}}Area equipped for irrgiation by different \\ crop:\end{tabular} &
	  \begin{tabular}[c]{@{}l@{}}Rice, \\ Wheat, \\ Maize, \\ Fruits, \\ Others.\end{tabular} \\ \hline
	Industry &
	  \begin{tabular}[c]{@{}l@{}}Industrial gross \\ value added\end{tabular} &
	  Billion Yuan &
	  Industrial GVA by industries &
	  \begin{tabular}[c]{@{}l@{}}Textile, \\ Papermaking, \\ Petrochemicals, \\ Metallurgy, \\ Mining, \\ Food, \\ Cements, \\ Machinery, \\ Electronics, \\ Thermal electrivity, \\ Others.\end{tabular} \\
	 &
	  \begin{tabular}[c]{@{}l@{}}Industrial water \\ use efficiency\end{tabular} &
	  \% &
	  \begin{tabular}[c]{@{}l@{}}The ratio of recycled water and evaporated \\ water to total industrial water use\end{tabular} &
	  \begin{tabular}[c]{@{}l@{}}Ratio of industrial water recycling, \\ Ratio of industrial water evaporated.\end{tabular} \\ \hline
	Services &
	  \begin{tabular}[c]{@{}l@{}}Services gross \\ value added\end{tabular} &
	  Billion Yuan &
	  GVA of service activities &
	  Services GVA \\ \hline
	Domestic &
	  Urban population &
	  Million Capita &
	  Population living in urban regions. &
	  Urban pop \\
	 &
	  Rural population &
	  Million Capita &
	  Population living in rural regions. &
	  Rural pop \\
	 &
	  Livestock population &
	  Billion KJ &
	  \begin{tabular}[c]{@{}l@{}}Livestock commodity calories summed from \\ 7 types of animal.\end{tabular} &
	  Livestock \\ \hline
	\end{tabular}
\end{table}

\begin{figure}[!thb]
	\centering
	\includegraphics[width=12cm]{/Users/songshgeo/Documents/Pycharm/WAInstitution_YRB_2021/figs/outputs/linear}
	\caption{
		\textbf{A.} Correlation between independent variables and the dependent variable (water use).
		\textbf{B.} Linear relationship between the independent variables and the dependent variable trained by a linear model.
	}
\label{fig:linear}
\end{figure}

\begin{figure}[!thb]
	\centering
	\includegraphics[width=12cm]{/Users/songshgeo/Documents/Pycharm/WAInstitution_YRB_2021/figs/outputs/compared_placebo}
	\caption{
		\textbf{A.} A comparison of the matched placebo dataset and the actual YRB dataset for each province.
		\textbf{B.} Trend of total water use $(10^8 m^3)$ for the matched placebo and actual datasets.
	}
\label{fig:placebo}
\end{figure}

\section*{S3. Further analysis regarding the general economic model}
%! Author = songshgeo
%! Date = 2022/3/19


% 基于构建的一般经济学模型,我们对利益相关者面对用水配额政策的响应做了进一步探讨
Using the general economic model (see the \textit{Methods in the main text}), we also explored the response of stakeholders to water quota policies.
We considered two additional scenarios for stakeholders, one that considered technology growth and one that considered different valuations through time (via the discount rate) of economic benefits and ecological costs.
In the following scenarios, the cost is assumed to be Nontransferable, which could be fully allocated to the one incurring the water use.
Explaining plausible scenarios for these stakeholders will help us better understand the causes of the sprint effect and potential solutions.
We argue that the sprint effect of water use remains robust even if a complete and equitable system.

\subsection*{Growth in technology}

Assume that there is an exogenous technology growth rate of $g$ in the scenario of $N$ provinces bargaining for water use under total quota $Q$, with unit price of output $P$, unit cost $C$ and discount factor $\beta$.
For simplicity, consider the following finite-period water use optimization:

$ \max \quad P \cdot (1+g)^t \ln(1+x_{i,0})-\frac{C}{N}+\beta^t \begin{matrix} \sum_{t=1}^T [P \cdot (1+g)^t \ln(x_{i,t}+1)-C \cdot x_{i,t}] \end{matrix}$

$s.t. \quad x_{i,t} \leq Q \cdot \frac{x_{i,0}}{x_{i,0} + \begin{matrix} \sum x_{-i,0} \end{matrix}} \quad for \quad \forall t$

We depict the relation between multi-period benefits and water use $x_{i0}$ in Figure~\ref{fig:tech_growth}to illustrate the optimal water use pattern under technology growth.
The higher marginal benefits of water might create enough incentive to offset the nontransferable costs of water overuse at $t=0$, because a higher allocated quota provides growth option value.
Because provincial decisions are under a longer time horizon, there is a greater sprint effect due to the higher accumulated yield.

\begin{figure}[!h]
	\centering
	\includegraphics[width=0.7\linewidth]{/Users/songshgeo/Documents/Pycharm/WAInstitution_YRB_2021/figs/outputs/tech_growth}
	\caption{Multi-period benefits and optimal water use under technology growth and a quota system. The figure depicts the relationship of multi-period benefits of province $i$ and water use under Case 3 with technology growth under $T$ periods. Assume $F(x)=\ln(1+x)$, $N=8$, $P=1$, $C=0.5$, $\beta=0.7$, $g=0.2$, and $Q=8$. The horizontal axis coordinates of $*$ denotes optimal water use at $t=0$ under each time horizon in decision-making.}
	\label{fig:tech_growth}
\end{figure}

\subsection*{Economic benefits and ecological costs with different discount rate}

Assume that there is high discount rate for economic benefits and a low discount rate for ecological costs, in the scenario of $N$ provinces bargaining for water use under total quota $Q$, with unit price of output $P$, unit cost $C$, discount factor $\beta^{ecology}$ and $\beta^{ecology}$.
It means that present economic profits are significantly concerned but future ecological costs are widely ignored. (In fact, if GDP are the core standard to judge officers’ performance, this is an assumption cloth to the reality.) For simplicity, consider the following finite-period water use optimization, notes the water use of province $i$ at period $t$:

\[ \max \quad P \cdot \ln(1+x_{i,0})-\frac{C}{N}+\beta_{economy}^t \begin{matrix} \sum_{t=1}^T [P \cdot \ln(x_{i,t}+1)]  \end{matrix} - \beta_{ecology}^t \begin{matrix} \sum_{t=1}^T [C \cdot x_{i,t}] \end{matrix}\]

\[s.t. \quad x_{i,t} \leq Q \cdot \frac{x_{i,0}}{x_{i,0} + \begin{matrix} \sum x_{-i,0} \end{matrix}} \quad for \quad \forall t\]

We depict the relation between multi-period benefits and water use xi0 in different time horizons in Figure~\ref{fig:remote_cost}, Using a higher discount rate for ecological costs might create enough incentive to set off the nontransferable unit cost of $C$.
Because the provincial decision is often under a longer horizon than that in baseline results in Figure 4 of the main text, there is a greater sprint effect as a result of higher accumulated yields.

\begin{figure}[h!]
	\centering
	\includegraphics[width=0.7\linewidth]{/Users/songshgeo/Documents/Pycharm/WAInstitution_YRB_2021/figs/outputs/remote_ecological_cost}
	\caption{Multi-period benefits and optimal water use when economic benefits have a high discount rate, ecological costs have a low discount rate, and a quota is implemented. The figure depicts the relation between multi-period benefits of province $i$ and water use under Case 3 under $T$ periods. Assume $F(x)=\ln(1+x)$, $N=8$, $P=1$, $C=0.5$, $\beta_{economy}=0.7$, $\beta_{ecology}=0.3$, and $Q=8$. The horizontal axis coordinates of $*$ denotes optimal water use at $t=0$ under each time horizon in decision-making.
	}
	\label{fig:remote_cost}
\end{figure}


\bibliography{WAInstitution_YRB_2021}
\end{document}
