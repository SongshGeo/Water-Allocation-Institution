%% Template for a preprint Letter or Article for submission
%% to the journal Nature.
%% Written by Peter Czoschke, 26 February 2004
%%
\documentclass{article}
\usepackage{hyperref}
\usepackage{graphicx}
\usepackage{setspace}
\usepackage[a4paper, total={7in, 10in}]{geometry}

\setstretch{1.523}

\newtheorem{ass}{Assumption}
\newtheorem{case}{Case}
%% make sure you have the nature.cls and naturemag.bst files where
%% LaTeX can find them

\bibliographystyle{naturemag}


%\usepackage{biblatex}
%% make sure you have the nature.cls and naturemag.bst files where
%% LaTeX can find them

\title{Importance of social-ecological fit for water management in large river basins: An analysis of institutional shifts in the Yellow River Basin}

%% Notice placement of commas and superscripts and use of &
%% in the author list

\author{Shuang Song$^{1,2}$, Huiyu Wen$^3$, *Shuai Wang$^{1,2}$, Graeme S. Cumming$^{4}$ \& Bojie Fu$^{1, 2}$}


\begin{document}

\maketitle
\input{../../drafts/authors.md}

\begin{abstract}
    Increasing competition for water is leading to depletion of freshwater globally and calls for an urgent transformation of water governance. To better understand how institutions contribute to water governance, we quantified institutional shifts for the Yellow River Basin (YRB).  The YRB is a valuable case study because it is one of the most anthropogenically altered large river basins. It was first overdrawn, then dried up, and finally has been successfully restored. Our results suggest that two institutional shifts, the Water Allocation Scheme that began in 1987 (87-WAS) and the Unified Basinal Regulation that took over in 1998 (98-UBR), framed different social-ecological system (SES) structures. During the decade following the introduction of the 87-WAS, observed water use of the YRB increased by $8.57\%$ more than expected while 98-UBR successfully decreased total water use, ultimately. Specifically, 87-WAS stimulated increased water use in some provinces (e.g., Inner Mongolia, Henan, and Shandong), but 98-UBR regulated nearly all provinces. A mathematical economic model supports the hypothesis that regional variations were driven by SES structural changes. The quasi-natural experimental setting of the YRB and its significant structural changes over time offer deep insights into the links between SESs structures and outcomes, providing valuable guidelines for SESs worldwide that are facing water depletion.

\end{abstract}

% Then the body of the main text appears after the intro paragraph.
% Figure environments can be left in place in the document.
% \verb|\includegraphics| commands are ignored since Nature wants
% the figures sent as separate files and the captions are
% automatically moved to the end of the document (they are printed
% out with the \verb|\end{document}| command. However, tables must
% be manually moved to the end of the document, after the addendum.
% 表需要手动移到文末
% water use; water management; social-ecological system; common pool resources; institutional analysis; collaborative governance

\section{Introduction}
Emerging competition for water is an urgent problem in water governance, with widespread water scarcity and overuse resulting in huge impacts on economies, societies, and ecosystems \cite{gleickPeakwaterlimits2010,dolanEvaluatingeconomicimpact2021,siwiCompetitionWaterGetting,ziolkowskaCompetitionWaterResources2016,distefanoArewedeep2017}.
Water is the key exclusive and competitive resource that couples socioeconomic and ecological systems (known as common-pool resources, CPRs) \cite{wutichWaterScarcitySustainability2009,ostromGeneralFrameworkAnalyzing2009,castilla-rhoSocialtippingpoints2017,castilla-rhoGroundwaterCommonPool2020}.
Conflicts of interest often occur in allocation and competitions of water resources, with water governance policies often leading to long-term changes in human–water relationships and the redistribution of benefits \cite{wangAlignmentsocialecological2019,speedBasinwaterallocation2013,chingManagingsocioecologyvery2015}.
Although governments in many of the world’s large river basins have tried to resolve competition for water through the deliberate design of new institutions, general principles underlying the successes and failures of these initiatives are poorly documented and understood.

Institutions (such as policies, laws, and norms) can influence regional sustainability by changing social-ecological system (SES) structure and dynamics \cite{youngInstitutionsenvironmentalchange2008,cummingAdvancingunderstandingnatural2020,lieninstitutionalgrammartool2020}.
These include inter-relationships and interactions between social actors, between ecological units, or between social and ecological system elements \cite{bodinCollaborativeenvironmentalgovernance2017,barnesSocialecologicalalignmentecological2019}.
Effective (“matched”) institutions operate at appropriate spatial, temporal, and functional scales to manage and balance these different relationships and interactions \cite{heWateringEnvironmentalRegulation2020,kellenbergempiricalinvestigationpollution2009}.
From the perspective of SES outcomes, matched institutions support (but do not guarantee) sustainability \cite{epsteinInstitutionalfitsustainability2015,wangAlignmentsocialecological2019}.

Some kinds of institutions have been shown to support desirable outcomes in water-centered SESs (e.g., the Ecological Water Diversion Project in Heihe River Basin, China \cite{wangAlignmentsocialecological2019} and in collaborative water governance systems in Europe \cite{greenEUWaterGovernance2013}).
At the same time, undesired and unsustainable outcomes (e.g., failures in environmental regulation of highly polluting industries or the development of “tragedy of the commons” situations when pursuing more water resources), with considerable ecological degradation, have attracted much attention \cite{saylesSocialecologicalnetworkanalysis2019,caiPollutingthyneighbor2016,castilla-rhoSocialtippingpoints2017}.
Despite widespread recognition of the rising importance of integrated water resource management in solving water competition in the world’s large river basins, relatively few studies have explored the mechanisms by which SESs respond to new institutions \cite{agrawalSustainableGovernanceCommonPool2003,pershaSocialEcologicalSynergy2011,agrawalCommonPropertyInstitutions2001}.
Two particular weaknesses in existing knowledge include understanding (1) the causal links between SES structures and outcomes; and (2) details of the underlying processes, especially the coordination of the incentives of different participants, that result from an institutional lack of fit. These weaknesses limit our understanding of institutional design, and they may reduce the speed and transfer of new knowledge and experience related to improving the sustainability of comprehensive water resources management.

In order to disentangle the relationship between SES structure and outcomes, we analyzed a case study to show how an institutional shift led to a structural mismatch that triggered unsustainable water use and unintended ecological deterioration. Two rapid shifts in institutional structure that occurred in 1987 and 1998 (see Supplementary Material S1) provide unique settings to exploit quasi-natural experiments of a large river basin, the Yellow River Basin (YRB) in China \cite{xiaDevelopmentWaterAllocation2012}. After a period of severe drying up, an institutional shift implemented in 1987 represented the beginning of attempts to control water use in the YRB through the use of quotas, with the goals of alleviating conflicts between supply and demand and achieving sustainable development.

Our results show that this initiative actually accelerated water withdrawals, resulting in an unintended “sprint effect”, where institutional mismatches created an even stronger incentive for each resource user to withdraw resources until the next major institutional shift in 1998. Our analysis contributes to a deeper understanding of the mechanisms underlying the relationships among institutions, SES structures, and outcomes. By highlighting potential concerns for ecosystem collapse under structural mismatches, our findings are consistent with the urgent calls for a more dynamic design for water use allocation to achieve sustainability.

\section{Institutions and SES structures}
% 机构可以塑造社会经济体系的结构,对其进行抽象是机械地理解结构和结果之间联系的第一步。
Because institutions may shape the structure of SESs, describing institutional structure is a first step toward understanding the mechanisms linking structures and outcomes in SESs (Figure~\ref{fig:framework}A).
% 例如,机构可以创建一种被确定为与良好的社会经济效益相关联的水平匹配结构,如果它鼓励管理相连生态成分的不同行动者之间的协作(图1B)。
For example, institutions may create a structure that encourages collaboration between the different actors managing connected ecological components (Figure~\ref{fig:framework}B), leading to sustainable outcomes.
% 例如,机构可以创建一个结构,鼓励不同行为者之间的合作,管理相连的生态成分
Similarly, institutions for vertical management may enhance multi-layered SES matching by coordinating horizontal relationships (Figure~\ref{fig:framework}C and D).
% 在实践中,一个大型、复杂的河流流域的制度变化将创造或摧毁数百种不同的联系。这些局部变化的更广泛影响可以从系统的整体行为中看到。
In practice, institutional changes in a large, complex river basin will create or destroy hundreds of different connections. The broader impacts of these local changes can be seen in the overall behavior of the system.
% 因此,我们通过黄河流域的准自然实验,探讨了社会经济结构与可持续发展(结果)之间的因果关系,为两个主要原因提供了一个有益的案例研究。
We thus explored the causal linkages between the SES structures and sustainability (outcomes) in quasi-natural experiments of the YRB, which provides an informative case study for two main reasons.
% 首先,长江流域管理的急剧结构变化使我们能够定量估计高层制度设计变化对用水的净影响。决定水分配的制度包括自下而上的协议或社会规范,以及自上而下的配额或法规,它们对社会经济结构有不同的影响;自上而下的监管可能会立即引发制度转变和SES的剧烈结构性变化。通过与由自下而上的制度转变引起的更渐进的变化相比,探索自上而下变化的影响,在对社会经济地位的定量分析中,极大地减少了来自不可观测因素的潜在干扰,提高了社会经济地位结构和结果之间因果关系的清晰度。
First, the sharp structural shifts in YRB management enabled us to quantitatively estimate the net effects of changes in high-level institutional design on water use. Institutions that determine water allocation include bottom-up agreements or social norms as well as top-down quotas or regulations, with different effects on SES structure \cite{wangAlignmentsocialecological2019,speedBasinwaterallocation2013}; top-down regulations can trigger immediate institutional shifts and sharp SES structural changes \cite{speedBasinwaterallocation2013,rolandUnderstandinginstitutionalchange2004}.
In comparison with investigations of more gradual changes induced by bottom-up institutional shifts, exploring the impacts of a top-down change substantially diminishes potential problems of omitted variables in the quantitative analysis of SES and improves the clarity of the causal link between SES structure and outcome.
% 其次,通过比较长江流域两次制度变迁所分裂的三种不同制度结构的净效应,我们也可以更深入地理解“盆地固定效应”下结构格局的影响。尽管流域内的社会经济单位从世界各地的大型河流流域和许多地区的水资源中受益,但很少有流域多次经历过如此激进的社会经济结构变化。
Second, by comparing the net effects of three different institutional structures split by two institutional shifts in the YRB, we can also reach a stronger understanding of the influence of structural alignments under a fixed basin. Although socioeconomic units within a basin benefit from water resources in large river basins all over the world and many locations have shown increased levels of regulation, few basins have experienced such radical SES structural changes several times (see \textit{Supplementary Material} S1). Thus, the YRB provides a valuable setting for understanding the direct impacts of changes in SES institutional structure.
% 因此,黄河流域的“流域固定效应”为SES结构的自我比较提供了宝贵的机会。
% Thus, the YRB provides valuable settings for understanding the direct impacts of changes in SES institutional structure.

\begin{figure}
	\centering
	\includegraphics[width=8.3cm]{../../../figs/diagrams/framework.jpg}
	\caption{
		Framework for understanding linkages between SES structures and outcomes.
		\textbf{a.} The general framework for analyzing social-ecological systems (SESs) (adapted from Ostrom \cite{ostromGeneralFrameworkAnalyzing2009}). Institutions embedded in SESs may reshape structures by changing the interactions between core subsystems, resulting in different outcomes.
        Three typical types of abstracted SES structures are shown as \textbf{b.}, \textbf{c.} and \textbf{d.} (adapted from Bodin, 2017)\cite{bodinCollaborativeenvironmentalgovernance2017}. Red circles indicate social actors, and green ones indicate ecological components. Connection (ties between two ecological components), collaboration (ties between two social actors), or management (ties between a social actor and an ecological component) exist when two units are linked by gray lines. The gray dashed lines show aligned SES structures that are more likely to result in a desirable outcome according to empirical evidence.
	}
    \label{fig:framework}
\end{figure}

\section{Context of institutional shifts}
% 黄河是世界第五长河,它的流域也是中国文明的摇篮。
The Yellow River, whose basin is the cradle of Chinese civilization, is the fifth-longest river in the world. It supports $9.7\%$ of China’s irrigation, with only $2.6\%$ of its total water resources (data from \href{http://www.yrcc.gov.cn}{http://www.yrcc.gov.cn}, last access: 28 February 2021).
% 然而,经过沿江各省多年的免费取水(图~\ref{fig:structure} A和B),到20世纪80年代,黄河地表水耗水量接近径流量的10倍,并不断上升
However, after years of free access to water by provinces along the river (Figure~\ref{fig:structure} A and B), surface water consumption of the Yellow River was close to $80\%$ of its runoff by the 1980s and rising
\cite{wangYellowRiverwater2019,songSedimenttransportincreasing2020}.
% 自1972年以来,黄河径流量的减少破坏了黄河的生态,制约了黄河的经济发展
Reductions in runoff after 1972 damaged the ecology of the YRB and restricted its economic development \cite{wangYellowRiverwater2019}.
% 因此,在中国典型的自上而下的制度结构下(附录图S1-B),黄河流域不同层次提出了相对完整的水资源分配规定(图1a)。
Therefore, through typical top-down institutional structures in China (see \textit{Supplementary Material S1}), relatively integrated water allocation regulations were successively proposed across different levels in the YRB.
% 这些机构包括国家政府、流域管理机构、省、市,甚至地区(见图1 B-D)。
These include the national government, the basin management agency, provinces, cities, and even districts.
% 这些在不同制度发展阶段相继出台的政策,引发了长江经济带经济发展结构的突变,并产生了不同的结果(见附录A和图S1-C)。
These policies at different stages of institutional development triggered abrupt changes in the SES structure of the YRB with different outcomes.

% 0世纪70年代以来,中国政府向黄河水利委员会下达指令,要求黄河水利委员会设计配水方案,同时要求黄河沿岸各省进行水资源规划
In 1982, the Chinese government issued instructions to the Yellow River Water Conservancy Commission (YRCC), the basin agency of the YRB, requiring it to design a water allocation scheme and at the same time requiring the provinces along the Yellow River to carry out water resources planning (see \textit{Supplementary Material S1})
\cite{wangReviewImplementationYellow2019}.
% 经过多方讨论与权衡,中国政府于1987年为相关省份分配了水资源配额,要求沿黄各省(区)贯彻执行
The Chinese government started to assign water quotas to the relevant provinces in 1987, but did not create a unit to coordinate water division between them (Figure~\ref{fig:structure} \textbf{A} and \textbf{C}).
% 这一时期长江水利枢纽的监督任务是编制长江水利枢纽用水量统计公报,并与定额进行对比分析。
The mandate of the YRCC during this period was only to report on and analyze water consumption in the YRB
\cite{wangReviewImplementationYellow2019}.
% 随着断流的进一步恶化,1998年中国政府推进了相关政策的改革,要求所有省份在取用水资源时必须向黄委会申请许可,黄委会得以直接对各省的用水实施监管。
However, since reductions in river flow indicated an unintended SES outcome (Figure~\ref{fig:structure} E), the Chinese government pushed for a policy reform in 1998 that required all provinces to apply for licenses to use water from the YRCC, allowing the council to directly regulate their water use (see \textit{Supplementary Material S1} and Figure~\ref{fig:structure} A and D).
% 由于革新后的政策成功遏制了断流,2008年该分配政策被进一步细化,相关各省都进一步设置了更细致的分配方案,并最终形成了黄河流域如今的水资源分配格局。
The 1998 policy succeeded in curbing water extraction (Figure~\ref{fig:structure} E), and it was further refined in 2008.
The relevant provinces created a more detailed allocation plan and finally formed the present water allocation institutions of the YRB (see \textit{Supplementary Material S1}).
% 因此,在1975-2008长达33年的时间里,从没有分水政策到以两种不同模式监管下的分水政策,黄河流域实际上先后存在着三种不同的社会-生态结构(Fig 1)。
Therefore, in our study period (from 1975 to 2008), the system shifted between three different SES structures (Figure~\ref{fig:structure} \textbf{B} to \textbf{D}).
% 在它们之中,与预期相反地,1987年至1998年的流域SES结构下黄河的生态急剧恶化,表明了制度的失配。
The sharp and unintended decline in the ecological condition of the Yellow River from 1987 to 1998 indicates an institutional mismatch during this period (Figure~\ref{fig:structure} the shadowed time periods).

\begin{figure}[!ht]
    \centering
    \includegraphics[width=12cm]{../../../figs/diagrams/structure.jpg}
	\caption{
		% 黄河流域的制度变迁与经济社会结构差异。
		Institutional shifts and related SES structures in the Yellow River Basin (YRB). See \textit{Supplementary Material S1} for detailed introduction for the institutions.
		% 国家政府在1987年和1998年先后两次出台了改变流域制度的政策,而黄河水利委员会、利益攸关的各省在不同制度时期具备的功能是不同的。
		\textbf{A.} The national government changed YRB management policies and institutions in 1987 and 1998. As a result, the Yellow River Conservancy Commission (YRCC) and the provinces acted differently in different periods. Three different SES structures existed successively in the YRB.
		% 没有任何政策限制,各利益相关者此时期可以从单向但联通的河流生态单元内自由取水
		\textbf{B.} 1975–1987: Without any constraints, water resources were freely accessible to each stakeholder (the provinces in this case, denoted by red circles) from a one-way but connected ecological unit (the Yellow River, denoted by the blue rectangle).
		% 在政策1之后,每个使用者都被分配了能够开采河流地表水资源的配额,而黄委会的工作是对配额的使用进行统计与汇报。
		\textbf{C.} 1987–1998: After the implementation of policy 1 in 1987, each user was assigned a quota to withdraw surface water resources, and the YRCC (yellow triangle) was tasked with reporting on water quota use.
		% 第二个政策之后,利益相关者取水需要向黄委会申请,黄委会则根据用水配额来批发许可。因此此时黄委会与利益相关者之间产生了直接的双向联系,监督他们对资源配额的取用。
		\textbf{D.} 1998–2008: After the implementation of policy 2, stakeholders had to apply for water resources from the YRCC, which then licensed water use according to the quota. Under this institution, the YRCC had direct two-way connections between provinces and ecological components.
		% 由于政策1与政策2都是为了解决黄河断流的生态问题而提出的,因此在结果与政策预期严重相反的第二时期(灰色阴影时段),是制度导致了错配的社会-生态系统结构。
		\textbf{E.} A timeline of the Yellow River and drying conditions. The size of the circles indicates the length of section that dried up (km), and the y-axis indicates the length of the drying period. Both policy 1 and policy 2 were put forward to solve this ecological crisis. The mismatch created by policy 1 is clearly correlated with the unintended outcomes shown in the second (gray-shaded) period.
	}
	\label{fig:structure}
\end{figure}


\section{Sprint effect induced by an institutional mismatch}
% % 这里,我们使用合成控制模型创建了假设不存在政策变化的虚拟对照组(see \textit{Methods}-1),同时用K-近临算法找到经济状况与黄河流域各省“类似”的安慰剂对照(see \textit{Methods}-2),反复对比表明两次社会-生态系统结构变迁对流域的水资源利用及分配产生了明显不同的影响。
% Here, we use synthetic control method to create a control group assuming that there is no water use constraints (\textbf{methods}), at the same time with K-nearest algorithm to match an economic state and the YRB provinces ``similar'' placebo-controlled (\textbf{methods}).
% % 然后,我们反复比较了两个SES结构(分别自1987年和1998年),以量化是否有显著不同的影响,YRB用水或分配。
% Then we repeatedly compared between two SES structures (since 1987 and 1998, respectively) to quantify if there were significantly different impacts on water uses or allocation (\textbf{methods}).
% % 我们的结果表明,1987年的结构变化刺激各省使用了远超模型预测的没有政策影响下的用水量,其增量超出安慰剂对照组xx\%,但对水资源的分配的均匀程度没有改变(see fig)。
Our results suggest that the institutional shift in 1987 stimulated the provinces to use far more water than would have been used without policy effects (Figure~\ref{fig:main_results}A), with an observed increase of $164\%$ over the expectation(Figure~\ref{fig:main_results}B).
% 然而,水资源分配的均匀性并没有改变,区域间的用水量在相同规模下增加(图~\ref{fig:main_results}C)。
However, the relative share of water use was not changed, denoting proportionally similar water use increases among the different regions (Figure~\ref{fig:main_results}C).
% 在1998年出台的政策再次改变社会-生态结构之后,用水量显著下降,总量仅为安慰剂对照组的xx%。
After the SES structure changed again in 1998, the trend of increasing water use appeared to be effectively suppressed (Figure~\ref{fig:main_results} D), with total observed water consumption decreasing by $260\%$ relative to expectations(Figure~\ref{fig:main_results} E).
% 但此阶段主要是xx,xx,xx等黄河流域的用水大省受到了遏制,因此水资源在各个省之间的比例分配变得更均匀了。
At this stage, however, the reduction in water use came mainly from the provinces with large water consumption, such as Henan and Shandong (Figure~\ref{fig:main_results} E),  so the proportion of water used by regions became more similar (Figure~\ref{fig:main_results} F).
% 上述结果表明,1998年各省之间由黄委会统一监管后分水政策才真正达到了其遏制用水、解决断流的预期。
In conclusion, the water allocation policy curbed water use in 1998, whereas the 1987–1998 institutional mismatch stimulated a notable increase in total water use in all related provinces.
% 在过去的十年里,对制度变革的“冲刺”反应似乎创造了一场竞赛,每个省份都开始使用超过他们需要的水。
Over this decade, “sprint” responses (i.e., rapid increases in resource use \cite{lueckPreemptiveHabitatDestruction2003}) to institutional change appear to have created a race in which each province began to use more water than they needed.

\begin{figure}
    \centering
    \includegraphics[width=12cm]{../../../figs/outputs/main_results.jpg}
    \caption{
        Effects of two institutional shifts on water resources use and allocation in the Yellow River Basin (YRB). (\textbf{A} to \textbf{C}: Entering the mismatched SES structure in 1987; \textbf{D} to \textbf{E}: exiting the mismatch in 1998. \textbf{A} and \textbf{D}: Impact of the first institutional shift on water use trends in the YRB. Blue points are actual water use, and gray points are predicted use under a scenario without any institutional shift (see \textit{Methods}). \textbf{B} and \textbf{E}: Impact of the institutional shift on total water use. Dark bars indicate the difference between the actual and predicted water use in specific study periods. Gray bars are the expected water use, simulated by setting up placebo experiments (null models, see \textit{Methods}). \textbf{C} and \textbf{F}: Impact of institutional shift on water allocation equity (see \textit{Methods}). Red lines indicate the index calculated from actual water use data; gray lines indicate predicted water use under a scenario with unconstrained water use.
    }
    \label{fig:main_results}
\end{figure}

\section{Incentive distortion causes the sprint effect}
Theoretically, our economic model suggests that different kinds of institutional shifts should lead to different optimal water uses (Figure~\ref{fig:economic_model}).
Furthermore, our analysis indicated that the cause of the sprint effect in this case was incentive distortion.
Compared with the decentralized water allocation institution in place before 1987, the presence of central management (by the YRCC in this case, after 1998) can effectively reduce marginal ecological costs (see Table~\ref{tab:cases} the \textbf{methods} for a detailed mathematical formula).
% 在不匹配条件下,边际成本减去边际收益增加,导致用水量增加,xx的差异反映了水配额模拟的非预期冲刺效应。
The unintended sprint effect (from 1987 to 1998) was caused by both declining marginal costs (a shift from a fixed unit cost to an irrelevant cost) and increasing marginal returns due to future water use benefits (see Table~\ref{tab:cases} the \textbf{methods} for a detailed mathematical formula).
% 因此,这种制度引发了一种与可持续用水意图背道而驰的激励扭曲。
The institution thus triggered an incentive distortion that ran counter to the intention of sustainable water use. Further, the strength of the sprint effect was positively correlated with the size and time horizon of the water use quota (Figure~\ref{fig:economic_model} \textbf{Panel B}).
% 这些理论上的预测正如我们在黄河流域所观察到的一样,不匹配的配额制度导致了短跑效应的出现,而中心化的制度结构结束了这一非预期的现象。
% These theoretical predictions, in line with phenomenon observed in the YRB, indicating the mismatched institution leads to incentive distortion of provinces for pre-empting resources by `sprinting', according to their expectations for the future.
% 这些理论预测进一步从YRB观察到的现象出发,表明不匹配的配额制度通过提高边际效益来刺激用水量,而边际效益与放宽限制带来的未来潜在用水量的影子值有关。
% These theoretical predictions, further from the phenomenon observed in the YRB, indicating the mismatched institution with quota stimulates water use by raising marginal benefits, which is related to the shadow value of potential future water use from relaxing the constraints.
% 温博推荐的版本:
% These theoretical predictions, further from the phenomenon observed in the YRB, attributing ``sprint effect'' into raising marginal benefits -by pursuing indirectly generating shadow value of water use while converting fixed marginal costs to irrelevant cost for each individual province.

\begin{table}[!ht]
	\centering
	\caption{Summary of marginal returns and marginal costs for each case }
	%\footnotesize
    \setlength{\tabcolsep}{1mm}{
       \begin{tabular}{lccc}
		\hline
		& Case 1: Decentralized Ins. & Case 2: Mismatched Ins. & Case 3: Matched Ins. \\ \hline
		Marginal return & $P*F'(X)$                & $P*F'(X)+V(X)^*$            & $P*F'(X)$              \\
		Marginal cost   & $C/N$                    & $Irrelevant$              & $C$                   \\ \hline
	\end{tabular}
    }
	\centering
	\footnotesize{\leftline{$^*$Note: $V(X)$ denotes a shadow value.}}
	\label{tab:cases}
\end{table}


\begin{figure}[!ht]
    \centering
    \includegraphics[width=12cm]{../../../figs/outputs/economic_model.jpg}
	\caption{
		% \textbf{Assumption 1:} \textit{(Production)} Assuming that water is the only input of the homogenous production function F(x) of each province. Under diminishing marginal returns assumption, and $F(x)$ is continuous, $F'(0)=\infty$, $ F'(\infty)=0$. The production output is under perfect competition, with constant unit price of P.
		% \textbf{Assumption 2:} \textit{(Cost function)} Assuming that the ecology is a unity for the whole basin, the cost of water use is equally assigned to each province under any water use. The unit cost of water is a constant C.
		% \textbf{Assumption 3:} \textit{(Multi-period setting)} There are infinite periods with constant discount factor $\beta$ lying in (0,1) with no cross-period smoothing in water uses.
		\textbf{A.} The relationship of marginal benefits and water use of province i at t = 0 for three different cases (case 1 to case 3, corresponding to the different SES structures in Figure~\ref{fig:structure}, assuming $F(x)=ln(1+x)$, $N=8$, $P=1$, $C=0.5$, and $\beta=0.4$ as an example  (see \textit{Methods}In Case 3, water use by others is taken as a given, equal to the optimal water use for Case 2. The horizontal coordinate of each intersection of marginal benefits and the break-even line represents the optimal water use under each case.
		\textbf{Panel B.} The relation between optimal water use of province i and total quota for Case 3, under time horizon of $T=5$, $T=10$, and an infinite $T$, respectively. The settings are the same as in \textbf{A}.
	}
	\label{fig:economic_model}
\end{figure}

\section{Discussion}
% 我们展示了不匹配的分配制度是如何导致激励失真导致水资源加速耗竭的(即“冲刺效应”)。“冲刺效应”是CPR系统面临的一种特殊情况,在这种情况下,制度的不匹配为每个资源使用者创造了更强的动机(扭曲),促使他们收回资源
We have shown how a mismatched allocation institution can lead to an accelerated depletion of water resources (i.e., the “sprint effect”) caused by incentive distortion. The sprint effect is a special case faced by CPR systems, where institutional mismatches create an even stronger incentive (with distortion) for each resource user to withdraw resources
\cite{ostromRevisitingCommonsLocal1999,ostromGeneralFrameworkAnalyzing2009,castilla-rhoSocialtippingpoints2017}.
% 过往研究指出制度常常是避免公共池塘资源系统的崩溃的关键,但“短跑效应”的出现表明在自上而下进行制度设计所形成的错配SES结构中,制度也可以成为系统加速崩溃的触发者。
Previous studies have suggested that institutions are often the key to avoid the collapse of a CPR system, but the emergence of a sprint effect shows that an institution with structural mismatches can also be the trigger that accelerates system collapse \cite{bodinConservationSuccessFunction2014,bodinCollaborativeenvironmentalgovernance2017,wangAlignmentsocialecological2019}.
The initial formulation of the water quota in our case studies went through a stage of “bargaining” among stakeholders (from 1982 to 1987) \cite{wangReviewImplementationYellow2019, wangThingsCurrentSignificance2019}, where each province attempted to demonstrate its development potential related to water use.
% 在高层决策者与低层利益相关者之间存在信息不对称的用水分配中,当前用水量越大的利益相关者议价能力越大。1987年以后,对各省来说,合乎逻辑的下一步是试图证明争取更大配额的合理性,而不是立即采取资源节约型改革。在实践中,尽管受影响的省份可能没有直接鼓励过度使用资源,但由于激励扭曲,他们有更大的动机对资源提取开绿灯。因此,在争夺潜在的水资源配额时,各省往往将生态成本隐藏在经济发展背后。
In water use allocation with information asymmetry between upper-level decision-makers and lower-level stakeholders, those with more current water use might have greater bargaining power. After 1987, the logical next step for provinces was to attempt to justify bargaining for larger quotas rather than immediately adopt resource-conserving transformations. In practice, although the affected provinces may not have directly encouraged excessive resource use, they had a greater incentive to give the green light to resource withdrawals because of incentive distortions \cite{kriegerProgressGroundWater1955, ostromGoverningCommonsEvolution1990}. As a result, while competing for potential water quotas, the provinces tended to hide the ecological costs behind economic development.

% 毫无疑问,随着资源竞争的日趋激烈,越来越多的SES正依赖着不同形式的制度进行资源分配(如自组织和政府干预),避免“短跑效应”的出现或将成为制度设计的关键。
There is no doubt that with increasingly fierce competition for water, more and more SESs are developing new institutions for water allocation (whether through self-organization or government intervention) \cite{anderssonVoluntaryleadershipemergence2020, wutichWaterScarcitySustainability2009}.
% 为了防止公共池资源被过度利用,总配额在强制禁止水资源过度利用的环境规制中发挥着重要作用,从而形成一个长期匹配的水资源分配机制。
Adoption of an overall quota plays an important role in preventing overuse of CPRs \cite{tilmanLocalizedprosocialpreferences2019}.
However, the negative effects of incentive distortion imply a trade-off between long-term SES benefits and current stability, and the proportion of available resources allocated under quota schemes matters when institutions change \cite{ladeRegimeshiftssocialecological2013}.
According to our analysis of plausible scenario assumptions based on our general economic model, the sprint effect will be reinforced when stakeholders anticipate that technological advances will amplify the benefits of water quotas in the future (see \textit{Supplementary Material S3}).
% 然而,如果有水权转换机制允许利益相关者之间通过交易来弥补 shadow value,当前这种错乱的动机就不会那么强。
However, if an institution allowed stakeholders to compensate for the shadow value (i.e., potential returns sacrificed due to water constraints and water scarcity) \cite{howarthAccountingvalueecosystem2002} of future water use, incentive distortion would be less devastating (e.g., through water rights transfer).
Policymakers can also weaken the sprint effect by increasing the frequency of quota updates, supporting the idea that a more dynamic institution that responds to changing conditions (see \textit{Supplementary Material S3}) will adapt more effectively to its social-ecological context.

% 近年来黄河流域面临的分水制度调整问题也说明了动态设置配额的重要性。
Calls for a redesign of water allocation institutions in the YRB in recent years also illustrate the importance of dynamic quota setting (see \textit{Supplementary Material S1}) \cite{yuAdaptabilityassessmentpromotion2019}. Following the institutional reforms of 1998, the Yellow River has not dried up since 1999. However, given recent changes in the YRB, its rigid resource allocation scheme can no longer meet the new demands of economic development \cite{wangThingsCurrentSignificance2019}. The Chinese government has embarked on an ambitious plan to redesign its decades-old water allocation institution (see \textit{Supplementary Material S1}). Other SESs around the world face similar problems in establishing successful resource allocation institutions \cite{cummingQuantifyingSocialEcologicalScale2020, muneepeerakulStrategicbehaviorsgovernance2017, cummingAdvancingunderstandingnatural2020, leslieOperationalizingsocialecologicalsystems2015}. These initiatives can benefit from our analysis by actively considering and incorporating social-ecological complexity and incentive structures when developing new approaches that avoid unsustainable outcomes. Our research provides a cautionary tale of how institutions can act as a double-edged sword when trying to attain sustainability.


\section{Methods}
% 为了量化制度变迁为黄河流域用水带来的影响,我们按附图1所示的技术路线执行了分析过程
We estimated and analyzed the net effects of two SES structural changes of water use. The actual water use of the Yellow River Basin was peroxided by the sum of the water use of the target group provinces. To quantify water use, we used synthetic control methods to estimate possible trends of water use in the absence of institutional shifts. In addition, as a robustness test, we conducted a matched placebo test (creating a “null model”) to exclude the effects of other factors that were contemporaneous with the institutional shifts. Finally, we created an economic model based on marginal revenue to provide a theoretical explanation for the observed “sprint effect” phenomenon. A brief technical overview is given in \textit{Supplementary Material S2}.

\subsection{Dataset and variables}
% 我们使用中国1978年至2012年各省的年度用水量数据集,这个公开的数据集由全国水资源利用调查得到,详细可查看。
We used China’s provincial annual water consumption dataset from 1978 to 2012. This publicly available dataset was obtained from the National Water Resources Utilization Survey; details are accessible from Zhou (2020)
\cite{zhouDecelerationChinahuman2020}.
A total of 10 provinces or regions have been directly affected by the water allocation institutional shifts in the YRB, accounting for $8.6\%$ of the total population of China (in 1990). Eight provinces have been particularly affected because of their greater dependence on the water resources from the Yellow River (see \textit{Supplementary Material S2}). Therefore, we divided the dataset into a “target group” and a “control group”, treating provinces that were greatly affected as the target group $(n=8)$ and provinces that were not affected by the institutional shifts as the potential control group $(n=20)$.

We focused on two features of water use in the YRB: total water use and diversification of water allocation. The actual water uses are given by the dataset, but when the synthetic control method is used to predict the water use of the control group, other independent influences need to be considered. Thus, we used economic features that are highly related to water use to extrapolate demand (e.g., agriculture, industry, service industry, and domestics, see \textit{Supplementary Material S2, Table 1} To measure resource allocation diversification between the upper, middle, and lower reaches, we used “entropy” as a simple index,

$$ Index_{entropy} = \sum_i{p_i *log(p_i)} $$

% 式子中的p是每个使用的水占流域总用水的比例
Where $p_i$ is the proportion of water uses for region $i$ to the total water uses in the basin. A larger index value indicates the proportion of water resources actually used is closer to the average among the upper, middle, and lower reaches.

\subsection{Synthetic Control}
Synthetic control is an effective identification strategy for estimating the net effect of historical events or policy interventions on aggregate units (such as cities, regions, and countries) by constructing a comparable control unit \cite{abadieSyntheticControlMethods2010}.
In this study, we used a comparative event approach and compared actual post-institutional shift induced water use changes with an appropriate counterfactual of what the water use change would have been.
The counterfactual was built as the optimally weighted average of provinces not exposed to the institutional shifts.
The synthetic control method generalizes the difference-in-differences estimator and allows for time-varying individual-specific unobserved heterogeneity \cite{billmeierAssessingEconomicLiberalization2013, smithresourcecurseexorcised2015}.
In practice, each of the units (i.e., provinces) in the treated group were affected by institutional shifts in 1987 and 1998, each of which was taken as the “shifted” point $t_0$ and the two steady institutions as $t$ for analyzing in each shift. The synthetic control method generates the control unit by assigning a weight matrix $W$ to units of the potential control group, so that the treated unit and its control unit are similar in each variable before $t_0$, i.e.,

$$\min(V_{i}^{t<t_0} - W_i * F_{control}^{t<t_0})$$

where $V_i$ is a vector that indicates all features of a unit $i$ of the treated group, and $F_{control}$ is a matrix that consists of all features and units of the potential control group. $W_i$ is the weight matrix for target unit $i$. We minimized the root mean square error (RMSE) by using the Synth package in R \cite{abadieSynthPackageSynthetic2011, abadieComparativePoliticsSynthetic2015}. All codes are accessible in the repository.

% 这样一来,基于降维的思想,我们构建出了一系列特征上与实验组最为接近的可比对照。
In accordance with the idea of dimensionality reduction, we constructed a series of comparable control units that were most similar in characteristics to the treated units. Because the units of the control group were not affected by the institutional shifts, after giving the same weight to the total water use of the control group $M_i * WU_{control}$, the result $W_i*WU_{control}$ could be considered a reasonable estimation of the untreated situation. The net effect of the water allocation institutional shift was then estimated by calculating the difference of water uses after the institutional shift between the treated group and the control group, compared with the water use difference before the shift.


\subsection{Placebo Test}
% 作为一种稳健性检验,安慰剂测试的必要性体现在两个原因上。
For robustness, we conducted a placebo test because the synthetic control method neglects the influences of overall changes in factors in the same year by simply dividing time periods according to institutional shifts. Three steps were required to apply the placebo test:
% (1) 对目标组中的每个省份$i$,计算所有潜在控制组与它之间特征向量的欧氏距离
(1) For each province in the target group, we calculated the Euclidean distance of vectors between all provinces in the potential control group.
% (2) 将距离由小到大排序后,选取特征向量最相似的三个省份,所有特征的均值作为 $i$ 的替代
(2) After ranking the distances, the three provinces with the most similar economic context were used to generate an average paired treatment target unit.
% (3) 对这个配对目标同样实施控制合成法(潜在控制组排除构成它的三个省)
(3) We performed the same synthetic control analysis for this paired target (i.e., the potential control group excluding the three provinces in step 2).
% 经过上述步骤,我们在理论上构建了一个相似的“区域”并实施了同样的控制合成实验。
In this way, we theoretically constructed a pseudo-treated unit and performed the same synthetic control treatments. Because these placebo tests were directed at units unaffected by the institutional shifts, the results can be regarded as a reasonable baseline expectation or null model from which to assess the changes caused by other factors.

\subsection{Economic model}
In order to understand the mechanisms underlying the empirical results, we developed a dynamic economic model to analyze how institutional change could have led to the sprint effect in water use. Specifically, we modeled individual provincial decision-making in water resources before quota execution. The analysis result implied that the underlying driver of CPR overuse was incentive distortion.

In developing the model, we highlighted the main features of the YRB, as well as the water use institutions of 1987 and 1998. We proposed three intuitive and general assumptions.

\begin{ass}
    % (生产)为了简化,由于不可替代性,水是每个省的同质生产函数10的唯一投入。$F(x)$是连续的,满足Inada条件,即$F'(x)>0, F''(x)<0$(边际收益递减假设),$F'(0)=\infty$,$ F'(\infty)=0$。产品产量处于完全竞争状态,单位价格为15。
    (Water-dependent production) For simplicity, water is assumed to be the only input of the homogenous production function $F(x)$ of each province because of its irreplaceability. $F(x)$ is continuous and satisfies the Inada Conditions, i.e., $F'(x)>0, F''(x)<0$ (the diminishing marginal returns assumption), $F'(0)=\infty$,$ F'(\infty)=0$. The production output is under perfect competition, with a constant unit price of $P$.
\end{ass}
\begin{ass}
    (Ecological cost allocation) Under the assumption that the ecology is a single entity for the whole basin involved in N provinces, the cost of water use is equally assigned to each province under any water use. The unit cost of water is a constant $C$.
\end{ass}
\begin{ass}
    (Multi-period settings) There are infinite periods with a constant discount factor $\beta$ lying in (0,1). There is no cross-period smoothing in water uses.
\end{ass}

Under the above assumptions, we can demonstrate three cases consisting of local governments in YRB to simulate their water use decision-making and water use patterns.

\begin{case} Decentralized institution:
    This case corresponds to a situation without any high-level water allocation institution (i.e., before 1987, see Figure~\ref{fig:structure} B).

    When each province independently decides on its water use, the optimal water use $\hat x_i^*$ in province $i$ satisfies:
    $$F'(x)=\frac{C}{P \cdot N}$$

    When the decisions in different periods are independent, for $t=0,1,2 \ldots$, then:
    $$\hat x_{it}^*=\hat x_i^*$$

\end{case}

\begin{case} Mismatched institution
    This case corresponds to a mismatched institution (i.e., $1987\sim1998$, see Figure~\ref{fig:structure} C).

    The water quota is determined at $t=0$ and imposed in $t=1,2,\ldots$ The total quota is a constant denoted as $Q$, and the quota for province $i$ is determined in a proportional form:
    $$Q_i=Q \cdot \frac{x_i}{x_i + \begin{matrix} \sum x_{-i} \end{matrix}}$$

    Under a scenario with decentralized decision-making with a water quota institution, given other provinces’ water use decisions remain unchanged, the optimal water use $\widetilde x_{i0}^*$ of province $i$ at $t=0$ satisfies:

    $F'(x_{i,0})=\frac{C}{P \cdot N} - \frac{\beta}{1-\beta} \cdot f(Q \cdot \frac{x_{i,0}}{\begin{matrix} x_{i,0} + \sum x_{-i,0} \end{matrix}}) \cdot Q \cdot \frac{\begin{matrix} \sum x_{-i,0} \end{matrix}}{(\begin{matrix} x_{i,0} + \sum x_{-i,0} \end{matrix})^2}$.

    When future water use is constrained by a water quota, the dynamic optimization problem of province $i$ is shown as follows:

    $max  \quad P \cdot F(x_{i,0})-\frac{C \cdot \begin{matrix} \sum x_{i,0} + x_{-i,0} \end{matrix}}{N}+\beta P \cdot F(x_{i,1})+\beta^2 P \cdot F(x_{i,2})+...$

    $=P \cdot F(x_{i,0})-C \cdot \frac{x_{i,0} + \begin{matrix} \sum x_{-i,0} \end{matrix}}{N}+\frac{\beta}{1-\beta} P \cdot F(Q \cdot \frac{x_{i,0}}{x_{i,0} + \begin{matrix} \sum x_{-i,0} \end{matrix}})$

    First-order condition: $P \cdot F'(x_{i,0})-\frac{C}{N}+\frac{\beta}{1-\beta}[P \cdot f(Q \cdot \frac{x_{i,0}}{x_{i,0} + \begin{matrix} \sum x_{-i,0} \end{matrix}}) \cdot Q \cdot \frac{\begin{matrix} \sum x_{-i,0} \end{matrix}}{(x_{i,0}+\begin{matrix} \sum  x_{-i,0} \end{matrix})^2}]=0$

    where $f(\cdot)$ is the differential function of $F(\cdot)$.

    The optimal water use in province i at t=0 $\widetilde x_{i,0}^*$ satisfies $P \cdot F'(x_{i,0})=\frac{C}{N}-\frac{\beta}{1-\beta} \cdot P \cdot f(Q \cdot \frac{x_{i,0}}{x_{i,0} + \begin{matrix} \sum x_{-i,0} \end{matrix}}) \cdot Q \cdot \frac{\begin{matrix} \sum x_{-i,0} \end{matrix}}{(x_{i,0} + \begin{matrix} \sum x_{-i,0} \end{matrix})^2}$, i.e., $F'(x_{i,0})=\frac{C}{P \cdot N} - \frac{\beta}{1-\beta} \cdot f(Q \cdot \frac{x_{i,0}}{x_{i,0} + \begin{matrix} \sum x_{-i,0} \end{matrix}}) \cdot Q \cdot \frac{\begin{matrix} \sum x_{-i,0} \end{matrix}}{(x_{i,0} + \begin{matrix} \sum x_{-i,0} \end{matrix})^2}$.

\end{case}

\begin{case} Matched institution

    This case corresponds to the institution under which the YRCC centrally managed water allocation between provinces (i.e., $1998\sim2008$, see Figure~\ref{fig:structure} D).

    When the $N$ provinces decide on water uses as unified whole (e.g., the central government completely decides and controls on the water use in each province), the optimal water use $x_i^*$ of province $i$ satisfies:

    $$F'(x)=\frac{C}{P}$$

\end{case}

We propose Proposition 1 and Proposition 2:

\textbf{Proposition 1}: Compared with the decentralized institution, a matched institution with unified management decreases total water use.

Because $F’$ is monotonically decreasing, based on a comparison of costs and benefits for stakeholders (provinces) in the three cases,

$$\widetilde x_i^*>\hat x_i^*>x_i^*$$

The result of $\hat x_i^*>x_i^*$ indicates that individual rationality would deviate from collective rationality when property rights are unclear \cite{hardinTragedyCommons2009}, because of the common-pool characteristics of water
\cite{castilla-rhoGroundwaterCommonPool2020,ostromGeneralFrameworkAnalyzing2009}.

The difference of $\widetilde x_i^*$ and $\hat x_i^*$ stems from two parts: the marginal returns effect and the marginal costs effect. First, the “shadow value” provides additional marginal returns of water use in $t=0$, wwhich increases the incentives of water overuse by encouraging bargaining for a larger quota. Second, the future cost of water use would be degraded from $\frac{P}{N}$ to an irrelevant cost.

The optimal water use under the three cases implies that mismatched institutions cause incentive distortions and lead to resource overuse.

\textbf{Proposition 2}: The quota determination of the mismatched institution increases the incentives of current water use.

The intuition for this proposition is straight-forward in that all provinces would use up their allocated quota under a relatively small $Q$. As $Q$ increases, the quota would provide higher future benefits for a pre-emptive water use strategy. Since the provincial water use decisions are exactly symmetric, total water use would increase when each province has higher incentives for current water use. This situation corresponds to a “sprint” effect, where the total water use dramatically increases in the “sprint” period.

Extensions of the model are shown in \textit{Supplementary Material S3}.


\section{Info.}
Funding was provided by the National Natural Science Foundation of China (CN) (Grant Nos. NSFC 42041007).

[Competing Interests] The authors declare that they have no competing interests.

[Correspondence] Correspondence and requests for materials should be addressed to Shuai Wang.~(email: shuaiwang@bnu.edu.cn).

[Author contributions]Shuai Wang and Bojie Fu designed this research, Shuang Song performed the research and analyzed data, and Huiyu Wen designed the economic model. Shuang Song and Huiyu Wen wrote the paper, and Graeme S. Cumming revised the manuscript and offered important advice.

[Code availability] All codes and datasets used in this research are accessible at https://github.com/SongshGeo/sochyd-transboundary-HESS. They will be made open source after this project is completed.

%% Put the bibliography here, most people will use BiBTeX in
%% which case the environment below should be replaced with
%% the \bibliography{} command.
\bibliography{WAInstitution_YRB_2021}


%% Here is the endmatter stuff: Supplementary Info, etc.
%% Use \item's to separate, default label is "Acknowledgements"


\end{document}
