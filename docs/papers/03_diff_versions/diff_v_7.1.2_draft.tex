%DIF 1a1
%DIF LATEXDIFF DIFFERENCE FILE
%DIF DEL 01_manuscript/manuscript_v-7.1.2.tex   Sat Jul 29 20:49:00 2023
%DIF ADD 01_manuscript/manuscript.tex           Sat Jul 29 19:55:43 2023
% chktex-file 46
 %DIF >
%DIF -------
\documentclass[preprint, 12pt]{elsarticle}

%DIF 3a4-5
%% Use the option review to obtain double line spacing
 %DIF >
%% \documentclass[authoryear,preprint,review,12pt]{elsarticle}
 %DIF >
%DIF -------

%DIF 4a7-14
%% Use the options 1p,twocolumn; 3p; 3p,twocolumn; 5p; or 5p,twocolumn
 %DIF >
%% for a journal layout:
 %DIF >
%% \documentclass[final,1p,times]{elsarticle}
 %DIF >
%% \documentclass[final,1p,times,twocolumn]{elsarticle}
 %DIF >
%% \documentclass[final,3p,times]{elsarticle}
 %DIF >
%% \documentclass[final,3p,times,twocolumn]{elsarticle}
 %DIF >
%% \documentclass[final,5p,times]{elsarticle}
 %DIF >
%% \documentclass[final,5p,times,twocolumn]{elsarticle}
 %DIF >
%DIF -------

%DIF 5a16-18
%% For including figures, graphicx.sty has been loaded in
 %DIF >
%% elsarticle.cls. If you prefer to use the old commands
 %DIF >
%% please give \usepackage{epsfig}
 %DIF >
%DIF -------

%DIF 6a20
%% The amssymb package provides various useful mathematical symbols
 %DIF >
%DIF -------
\usepackage{amssymb}
\usepackage{amsmath, graphicx, array}
%DIF 8c23-24
%DIF < \usepackage{dcolumn, soul}\let\openbox\relax
%DIF -------
\usepackage{dcolumn, soul}%
 %DIF >
\let\openbox\relax
 %DIF >
%DIF -------
\usepackage{amsthm}
%DIF 10-11c26-29
%DIF < \usepackage[figuresright]{rotating}\usepackage{algorithm, algorithmicx, algpseudocode}
%DIF < \usepackage{listings}\usepackage{hyperref}
%DIF -------
\usepackage[figuresright]{rotating}%
 %DIF >
\usepackage{algorithm, algorithmicx, algpseudocode}
 %DIF >
\usepackage{listings}%
 %DIF >
\usepackage{hyperref}
 %DIF >
%DIF -------
\usepackage{geometry}
\usepackage{tabularx}

\geometry{a4paper,left=2.0cm,right=2.0cm,top=2.5cm,bottom=2.5cm}
%DIF 16a34
% 行距
 %DIF >
%DIF -------
\linespread{1.5}
%DIF 17a36-37
% \usepackage[nofiglist,notablist]{endfloat}
 %DIF >
% \def\uns{\ifmmode\,\else$\,$\fi}%
 %DIF >
%DIF -------
\newtheorem{defn}{Definition}
\newtheorem{thm}{Theorem}
\newtheorem{ass}{Assumption}
\newtheorem{prop}{Proposition}
\newtheorem{fig}{Fig.}
\newtheorem{case}{Case}
\newtheorem{case_appendix}{Case}
\newtheorem{example}{Example}[section]
\renewcommand{\proofname}{\textbf{Proof}}
\newtheorem{property}{Property}
\newtheorem{remark}{Remark}
%DIF 28a49
%\usepackage{enumerate}
 %DIF >
%DIF -------
\usepackage{enumitem}
\usepackage{float}
\usepackage{multirow}
\usepackage{lineno}
\usepackage{booktabs}
\usepackage{diagbox}
%DIF 34a56
%%
 %DIF >
%DIF -------

%DIF 35a58-64
% \jvol{XX}
 %DIF >
% \jnum{X}
 %DIF >
% \jyear{Year}
 %DIF >
% \doi{10.1093/nsr/XXXX}
 %DIF >
% \received{XX XX Year}
 %DIF >
% \revised{XX XX Year}
 %DIF >
% \accepted{XX XX Year}
 %DIF >
%DIF -------

%DIF 36a66
% \markboth{One, Two, and Three}{One, Two, and Three}
 %DIF >
%DIF -------
\graphicspath{{../../../figs/}}
\journal{Jounral of Hydrology}

%DIF PREAMBLE EXTENSION ADDED BY LATEXDIFF
%DIF UNDERLINE PREAMBLE %DIF PREAMBLE
\RequirePackage[normalem]{ulem} %DIF PREAMBLE
\RequirePackage{color}\definecolor{RED}{rgb}{1,0,0}\definecolor{BLUE}{rgb}{0,0,1} %DIF PREAMBLE
\providecommand{\DIFaddtex}[1]{{\protect\color{blue}\uwave{#1}}} %DIF PREAMBLE
\providecommand{\DIFdeltex}[1]{{\protect\color{red}\sout{#1}}}                      %DIF PREAMBLE
%DIF SAFE PREAMBLE %DIF PREAMBLE
\providecommand{\DIFaddbegin}{} %DIF PREAMBLE
\providecommand{\DIFaddend}{} %DIF PREAMBLE
\providecommand{\DIFdelbegin}{} %DIF PREAMBLE
\providecommand{\DIFdelend}{} %DIF PREAMBLE
\providecommand{\DIFmodbegin}{} %DIF PREAMBLE
\providecommand{\DIFmodend}{} %DIF PREAMBLE
%DIF FLOATSAFE PREAMBLE %DIF PREAMBLE
\providecommand{\DIFaddFL}[1]{\DIFadd{#1}} %DIF PREAMBLE
\providecommand{\DIFdelFL}[1]{\DIFdel{#1}} %DIF PREAMBLE
\providecommand{\DIFaddbeginFL}{} %DIF PREAMBLE
\providecommand{\DIFaddendFL}{} %DIF PREAMBLE
\providecommand{\DIFdelbeginFL}{} %DIF PREAMBLE
\providecommand{\DIFdelendFL}{} %DIF PREAMBLE
%DIF HYPERREF PREAMBLE %DIF PREAMBLE
\providecommand{\DIFadd}[1]{\texorpdfstring{\DIFaddtex{#1}}{#1}} %DIF PREAMBLE
\providecommand{\DIFdel}[1]{\texorpdfstring{\DIFdeltex{#1}}{}} %DIF PREAMBLE
%DIF COLORLISTINGS PREAMBLE %DIF PREAMBLE
\RequirePackage{listings} %DIF PREAMBLE
\RequirePackage{color} %DIF PREAMBLE
\lstdefinelanguage{DIFcode}{ %DIF PREAMBLE
%DIF DIFCODE_UNDERLINE %DIF PREAMBLE
  moredelim=[il][\color{red}\sout]{\%DIF\ <\ }, %DIF PREAMBLE
  moredelim=[il][\color{blue}\uwave]{\%DIF\ >\ } %DIF PREAMBLE
} %DIF PREAMBLE
\lstdefinestyle{DIFverbatimstyle}{ %DIF PREAMBLE
	language=DIFcode, %DIF PREAMBLE
	basicstyle=\ttfamily, %DIF PREAMBLE
	columns=fullflexible, %DIF PREAMBLE
	keepspaces=true %DIF PREAMBLE
} %DIF PREAMBLE
\lstnewenvironment{DIFverbatim}{\lstset{style=DIFverbatimstyle}}{} %DIF PREAMBLE
\lstnewenvironment{DIFverbatim*}{\lstset{style=DIFverbatimstyle,showspaces=true}}{} %DIF PREAMBLE
%DIF END PREAMBLE EXTENSION ADDED BY LATEXDIFF

\begin{document}
\begin{frontmatter}
%DIF >  \normalsize
%DIF >  \dhead{RESEARCH ARTICLE}
%DIF >  \subhead{EARTH SCIENCES}

%DIF >  \bibliographystyle{../nsr}
%DIF >  社会水文学专刊地址
%DIF >  https://www.sciencedirect.com/journal/journal-of-hydrology/about/call-for-papers#grounded-sociohydrology
\title{Quantifying the Effects of Institutional Shifts on Water Governance in the Yellow River Basin: A Social-ecological System Perspective}


\author[inst1]{Shuang Song}
\author[inst2]{Huiyu Wen}
\author[inst1]{*Shuai Wang}
\author[inst1]{Xutong Wu}

\author[inst3]{Graeme S. Cumming}
\author[inst1]{Bojie Fu}


\affiliation[inst1]{
     State Key Laboratory of Earth Surface Processes and Resource Ecology,
     Faculty of Geographical Science,
     Beijing Normal University,
     Beijing 100875,
     P.R. China}

%DIF >  \affil{stitute of Land Surface System and Sustainability,
%DIF >       Faculty of Geographical Science,
%DIF >       Beijing Normal University,
%DIF >       Beijing 100875,
%DIF >       P.R. China}

\DIFaddbegin

\DIFaddend \affiliation[inst2]{School of Finance,
     Renmin University of China,
     Beijing 100875,
     P.R. China}

\affiliation[inst3]{
     ARC Centre of Excellence for Coral Reef Studies,
     James Cook University,
     Townsville 4811,
     QLD, Australia}

%DIF >  \affil{e research for this article was financed by the National Natural Science Foundation of China (CN) (Grant Nos. NSFC 42041007). A supplementary online appendix is available with this article at the \em{National Science Review} website.}

\DIFaddbegin

%DIF >  \authornote{\textbf{Corresponding authors.} Email: shuaiwang@bnu.edu.cn}
%DIF > \authornote{Shuang Song and Huiyu Wen equally contributed to this work.}
%DIF >  \renewcommand\linenumberfont{\normalfont\bfseries\small}
\DIFaddend \begin{abstract}
     Water governance in river basins worldwide faces challenges due to complex socio-economic and environmental factors. In the Yellow River Basin (YRB), two major institutional shifts, the 1987 Water Allocation Scheme (87-WAS) and the 1998 Unified Basin Regulation (98-UBR), aimed to address water allocation and usage issues. This study quantifies the net effects of these institutional shifts on water use within the YRB and analyzes the underlying reasons for their success or failure.
     We employ a Differenced Synthetic Control method to assess the impacts of the institutional shifts. Our analysis reveals that the 87-WAS unexpectedly increased water use by $5.75\%$, while the 98-UBR successfully reduced water use as anticipated. Our research highlights the role of institutional structures in governance policies, demonstrating that the mismatched structure of the 87-WAS led to increased competition and exploitation of water resources, while the 98-UBR, with its scale-matched, basin-wide authority and stronger connections between stakeholders, resulted in improved water governance.
     Our study underscores the importance of designing institutions that are consistent with the scale of the ecological system, promote cooperation among stakeholders, and adapt to changing social-ecological system (SES) contexts. As outdated and inflexible water quotas may no longer meet the demands of sustainable development in the YRB, governments and policymakers must consider the potential consequences of institutional shifts and their impact on water use and sustainability.
\end{abstract}

%DIF > %Graphical abstract
%DIF >  \begin{graphicalabstract}
%DIF >       \includegraphics{outputs/main_results2.pdf}
%DIF >  \end{graphicalabstract}

\DIFaddbegin

%DIF > %Research highlights
%DIF >  \begin{highlights}
%DIF >       \item Research highlight 1
%DIF >       \item Research highlight 2
%DIF >  \end{highlights}

\DIFaddend \begin{keyword}
     water use~\sep~water governance~\sep~social-ecological system~\sep~institutions~\sep~Yellow River
\end{keyword}

\end{frontmatter}
\newpage
\linenumbers
\section{Introduction}\label{sec:introduction}
%DIF >  水竞争的重要性
Widespread freshwater scarcity and overuse challenge the sustainability of large river basins, resulting in systematic risks to economies, societies, and ecosystems globally~\cite{distefano2017, dolan2021, xu2020b, mekonnen2016}.
Amidst climate change, mismatches between supply and demand for water resources are expected to become increasingly more prominent~\cite{florke2018, yoon2021}.
Consequently, large river basins are progressively seeking effective water governance solutions by coordinating stakeholders, providing water resources, and ensuring the sustainable allocation of shared water resources~\cite{wang2019d}.
In this way, hydrological processes are tightly intertwined with societies, forming a social-ecological system (SES) at a basin scale with complex socio-hydrological feedback.

Institutions encompass the interplay between social actors, ecological units, and their interactions~\cite{young2008, lien2020, bodin2017b, wang2022g} (Figure~\ref{fig:framework}~a).
These interactions constitute a type of SES structure, where effective institutions operate at appropriate spatial, temporal, and functional scales to manage and balance different interactions, contributing to sustainability~\cite{epstein2015, wang2019d} (Figure~\ref{fig:framework}~b).
%DIF >  (Figure~\ref{fig:framework}~\textbf{a}).
%DIF >  Governing river basin systems involves reshaping their SES structures through institutions such as policies, laws, and norms~\cite{,cumming2020b}.
While some institutional advances have led to effective water governance outcomes (e.g., the Ecological Water Diversion Project in Heihe River Basin, China~\cite{wang2019d}, and collaborative water governance systems in Europe~\cite{green2013}), imposing institutional shifts may create or destroy connections and effectiveness is not ubiquitous~\cite{loos2022}.
For example, the Colorado River once experienced severe water shortage, and institutions led to various shortage magnitudes for different stakeholders even under the same water demand levels~\cite{hadjimichael2020}.
Therefore, examining when and how an institution leads to effective water governance can bring crucial insights for the sustainability of river basins.

%DIF >  match & mismatch
Recent studies have explored diverse effects of institutions on river basin governance~\cite{bouckaert2022, vallury2022, loch2020, kirchhoff2016}, while the current analysis is more about interpreting outcomes after the institutional changes but cannot compare how scenarios would be without these institutional changes.
Moreover, understanding how different SES structures influence institutional effectiveness is challenging due to the complexity and dynamics of socio-hydrological systems~\cite{bodin2017b}.
Thus, knowledge gaps lie in the limited understanding of effective alignments between institutional shifts and SES structures, hindering the design of effective policies to promote sustainable river basin governance.
To fill these knowledge gaps, we study the Yellow River basin, the fifth-largest river worldwide and one of the most anthropogenically altered river basins, to quantitatively measure the effects of changing SES structures.

\begin{figure}[!ht]
	\centering
	\includegraphics[width=0.5\linewidth]{diagrams/framework.png}
	\caption{
		Illustration for understanding institutional shifts and SES structural changes. \textbf{a.} In the general framework for analyzing social-ecological systems (SESs), (Adapted from Ostrom, 2008~\cite{ostrom2009}). Institutional shifts can change interactions within the SES and reframe the structures.  \textbf{b.} We aim to examine how institutional shifts effect river basin governance by structuring SES.}\label{fig:framework}
\end{figure}

%DIF >  黄河的介绍
In the 1980s, intense water use, accounting for about $80\%$ of the Yellow River surface water, caused consecutive drying-up crises of runoff, leading to wetland shrinkage, agriculture reduction, and scrambles for water~\cite{wohlfart2016}.
To alleviate water stress, Chinese authorities implemented several ambitious water management policies in the Yellow River Basin (YRB), such as the South-to-North Water Diversion Project and the Water Resources Allocation Institutions~\cite{long2020, wang2019d}.
In this study, we specifically examined two significant institutional shifts in water allocation of the YRB\: the 1987 Water Allocation Scheme (87-WAS) and the 1998 Unified Basinal Regulating (98-UBR).
\DIFdelbegin \DIFdel{the 1987 Water Allocation Scheme (87-WAS) and the 1998 Unified Basinal Regulating (98-UBR).
}\DIFdelend Instead of focusing on engineering and increasing water supply, the 87-WAS (which assigned water quotas for provinces in the YRB) and the 98-UBR (under which provinces had to obtain permits from the Yellow River Conservancy Commission, YRCC, an authority at a basin level) mainly aimed to limit water demands~\cite{bouckaert2022, speed2013}.
These institutional shifts can offer valuable insights for two main reasons:
(1) the top-down institutional shifts suddenly led to transformations of SES structures, allowing us to quantitatively estimate their net effects; and (2) the two institutional shifts within the same river basin provide rare comparable quasi-natural experiments.

In this study, we portrayed changes of SES structures throughout the YRB's institutional shifts (the 87-WAS and the 98-UBR) and quantitatively investigated their consequences, followed by a discussion on the effectiveness of institutional shifts.
Specifically, we first used the descriptions of official documents following the two institutional shifts to abstract the interactions between main stakeholders and their river segment units for interpreting SES structure changes between 1979 and 2008.
Next, and perhaps most importantly, we employed the ``Differenced Synthetic Control (DSC)'' method~\cite{arkhangelsky2021}, which accounts for economic growth and natural background, to estimate theoretical water use volumes under scenarios absent of institutional shifts.
Finally, in the discussion, we linked the effectiveness of institutional shifts to the portrayed structures, by comparing the YRB's case to previous SES structure studies and developing a marginal benefits analysis.


\section{Study area and institutional contexts}\label{sec:yrb}

The YRB, cradle of Chinese civilization, is located in north-central China and spans ten province-level regions whose socio-economic development heavily depends on water from the Yellow River.
As a semi-arid and arid region, the YRB's annual precipitation varies from about 100 to 1,000 mm and increases from the northwest to the southeast, while the annual pan evaporation varies from about 700 to 1,800 mm~\cite{wang2022e}.
Together, the YRB supports $35.63\%$ of China's irrigation and $30\%$ of its population while containing only $2.66\%$ of its water resources (data from \href{http://www.yrcc.gov.cn}{http://www.yrcc.gov.cn}, last access: \today).
Hence, over-withdrawing water from the Yellow River became an urgent concern when the river began to dry up in the early 1970s.
Among the policies proposed to address the problem, a series of water resource allocation institutions aimed to limit water use for each region with specific quotas, which were regarded as some of the most important solutions.
However, few attempts have been made to quantitatively assess how the YRB's water allocation scheme contributed to water governance, while other engineering solutions have been carefully evaluated~\cite{long2020}.
%DIF >  出现报错是因为最后的大写结尾LaTeX无法判断是缩写还是一个句子结束,这里是句子结束所以加 \\
%DIF >  https://tex.stackexchange.com/questions/55105/when-should-i-use-intersentence-spacing

%DIF >  Despite the essential reason for these institutions was the mismatch between the spatial and temporal distribution of water resources as well as social and economic water demands, the direct reason for their introduction was the depletion of the Yellow River.
\DIFaddbegin

\DIFaddend The YRB was the first basin in China for which water resource allocation institutions were created, and institutional shifts can be traced through several regulating documents released by the Chinese government (at the national level):
(1) In $1980s$, \DIFaddbegin \DIFadd{the central government }\DIFaddend proposed to develop a water resource allocation institution for the Yellow River~\cite{wang2019d, wang2019e}.
(2) In $1987$, the Water Allocation Scheme was implemented (\href{http://www.gov.cn/zhengce/content/2011-03/30/content_3138.htm#}{http://www.mwr.gov.cn}, last access: \today).
(3) In $1998$, the Unified Basinal Regulation was implemented (\href{http://www.mwr.gov.cn/ztpd/2013ztbd/2013fxkh/fxkhswcbcs/cs/flfg/201304/t20130411_433489.html}{http://www.mwr.gov.cn}, last access: \today).
%DIF >  各省按要求编制新的黄河流域水资源规划,将水资源额度分配进一步细化。
(4) In $2008$, provinces were asked to draw up new water resources plans for the YRB to further refine water allocations~\cite{wang2019d,wang2019e}.
(5) In $2021$, there was a call for redesigning the water allocation institution (\href{http://www.ccgp.gov.cn/cggg/zygg/gkzb/202107/t20210721_16591901.htm}{http://www.ccgp.gov.cn}, last access: \today).

Our study period therefore ranges from $1980$ (when water quotas were proposed) to $2008$, when a regulating system with quotas was fully established at basin, provincial, and district levels.
During this period, two significant institutional shifts can be analyzed using documents from $1987$ (87-WAS) and $1998$ (98-UBR), which split the study period into three sections: from $1980$ to $1987$ (before 87-WAS), from $1988$ to $1997$ (after 87-WAS and before 98-UBR), and from $1998$ to $2007$ (after 98-UBR).
%DIF >  Those efforts led to ecological restoration of wetlands and the estuarine delta. Drying up has been avoided for over $20$ years, which is widely considered a substantial management achievement.


\section{Methods}\label{sec:methods}
%DIF > ! Author = songshgeo
%DIF > ! Date = 2022/3/10

In the methodology section, we first utilize the descriptions of official documents following the two institutional shifts to abstract the interactions of SES into structures as \DIFdelbegin \DIFdel{point-axis networks }\DIFdelend \DIFaddbegin \DIFadd{organizational diagrams }\DIFaddend during different periods of time.
Next, we introduce the dataset we used here and employ the Principal Components Analysis (PCA) method to reduce the dimensionality of variables affecting the total water use.
We then estimate the net effects of the two institutional shifts on total water use, changing trends, and differences in the YRB's provinces using the Differenced Synthetic Control (DSC) method~\cite{arkhangelsky2021}.
Finally, we present the \DIFdelbegin \DIFdel{robustness }\DIFdelend \DIFaddbegin \DIFadd{efficiency }\DIFaddend tests approach for the DSC model.
%DIF >  Finally, for discussion, we developed a marginal benefit analysis based on identified SES structures to provide the observed pattern of water use changes with a theoretical interpretation.

\subsection{Portraying structures}\label{sec:structures}
\DIFdelbegin \DIFdel{A point-axis network type structure of SES }\DIFdelend \DIFaddbegin

\DIFadd{An organizational diagram }\DIFaddend is widely used to depict \DIFdelbegin \DIFdel{them }\DIFdelend \DIFaddbegin \DIFadd{SES structures }\DIFaddend by abstracting links and nodes \DIFaddbegin \DIFadd{from the real-world interactions}\DIFaddend ~\cite{wang2022g,bodin2017a,kluger2020,guerrero2015}.
We apply the \DIFdelbegin \DIFdel{network approach}\DIFdelend \DIFaddbegin \DIFadd{analysis of the organizational diagrams}\DIFaddend ~\cite{bodin2017b} to portray SES structures by abstracting relationships between ecological units (river reaches), stakeholders (provinces), and the administrative unit at the basin scale (the Yellow River Conservancy Commission) into structural patterns from official documents.
\DIFdelbegin \DIFdel{The network-based approach abstracts connections between entities into links according to their interactions~\mbox{%DIFAUXCMD
\cite{bodin2017a,kluger2020,guerrero2015}}\hskip0pt%DIFAUXCMD
, so we }\DIFdelend \DIFaddbegin \DIFadd{We }\DIFaddend examined the official documents of the two institutional shifts (87-WAS and 98-UBR) to portray \DIFdelbegin \DIFdel{these interactions }\DIFdelend \DIFaddbegin \DIFadd{the organizational diagrams }\DIFaddend in this study\DIFaddbegin \DIFadd{~\mbox{%DIFAUXCMD
\cite{bodin2017a,kluger2020,guerrero2015}}\hskip0pt%DIFAUXCMD
}\DIFaddend .
It is important to note that it can result in very different structures when basin-scale regulatory entity (YRCC) is responsible for river reach regulation, or have direct authority to interact with provincial units.

\subsection{Dataset and preprocessing}\label{sec:dataset}
The data of water consumption surveys conducted by the Ministry of Water Resources were taken as the observed values throughout the years.
Then, to estimate the water use of the YRB by assuming there were no effects from institutional shifts, we focused on variables from five categories (environmental, economic, domestic, and technological) water use factors. Their specific items and origins are listed in~\ref{secS2}~Table~\ref{tab:variables}.
Among the total $31$ data-accessible provinces (or regions) assigned quotas in the 87-WAS and the 98-UBR, we dropped Sichuan, Tianjin and Beijing (together, Jinji) because of their trivial water use from the YRB (see Table~\ref{tab:quota}).
%DIF >  We then divided the dataset into a ``target group'' and a ``control group'', treating provinces involved in water quota as the target group $(n=8)$ and other provinces as the control group $(n=20)$ for applying the DSC.\\

Using the normalized data of all variables, we performed the PCA reduction to capture $89.63\%$ explained variance by \DIFdelbegin \DIFdel{$5$ }\DIFdelend \DIFaddbegin \DIFadd{$D = 5$ }\DIFaddend principal components.
Previous study has proved that combining PCA and DSC can \DIFdelbegin \DIFdel{raise the robustness of }\DIFdelend \DIFaddbegin \DIFadd{lead to a more robust }\DIFaddend causal inference~\cite{bayani2021}.
We first applied the Zero-Mean normalization (unit variance), as the variables' units are far different. Then, we apply PCA to the multi-year average of each province, using the Elbow method to decide the number of the principal components (\textit{Appendix~\nameref{secS2}~Figure\ref{fig:elbow}}).
Finally, we transform the dataset and input the dimensions-reduced output into the DSC model.

\subsection{Differenced Synthetic Control}\label{sec:DSC}
\DIFdelbegin \DIFdel{Using the }\DIFdelend \DIFaddbegin

\DIFadd{The }\DIFaddend Differenced Synthetic Control (DSC) method\DIFdelbegin \DIFdel{, we can estimate water use under the scenarios of }\DIFdelend \DIFaddbegin \DIFadd{~\mbox{%DIFAUXCMD
\cite{arkhangelsky2021} }\hskip0pt%DIFAUXCMD
is a tool we use to estimate how water use might have evolved if there had been }\DIFaddend no institutional shift.
\DIFdelbegin \DIFdel{The DSC method is an effective identification strategy for estimating the net effect of historical events or policy interventions on aggregate units (such as cities, regions, and countries) by constructing a comparable control unit}\DIFdelend \DIFaddbegin \DIFadd{Think of it as creating an alternate reality or a ``what-if'' scenario to compare with what actually happened}\DIFaddend ~\cite{abadie2010, abadie2015, hill2021}.
\DIFdelbegin \DIFdel{This approach enables us to establish a counterfactual basis for exploring the consequences and incentives related to policy changes.
}%DIFDELCMD <

%DIFDELCMD < %%%
\DIFdel{This method aims }\DIFdelend \DIFaddbegin \DIFadd{The key idea behind this method is }\DIFaddend to evaluate the effects of policy \DIFdelbegin \DIFdel{change that are not random across units but focuses on some of them (i.e., }\DIFdelend \DIFaddbegin \DIFadd{changes that mainly affect certain units (in this case, the }\DIFaddend institutional shifts in the \DIFdelbegin \DIFdel{YRB here).
By re-weighting units to match the pre-trend for the treated and control units, the DSC method imputes post-treatment control outcomes for the treated units by constructing a synthetic }\DIFdelend \DIFaddbegin \DIFadd{Yellow River Basin or YRB).
The method creates a ``synthetic'' }\DIFaddend version of the \DIFdelbegin \DIFdel{treated units equal to a convex combination of control units. Therefore, the syntheticand actual version difference can be estimated as a net effect for a treated unit}\DIFdelend \DIFaddbegin \DIFadd{affected units by combining information from other similar but unaffected units. This ``synthetic'' version serves as a control group, which we can compare with the actual affected units.
The DSC method, therefore, is a powerful tool as it allows us to control for unobserved factors that can change over time, providing more robust results}\DIFaddend .

In practice, all treated units (i.e., provinces) were affected by institutional shifts in 1987 and 1998, each taken as the ``shifted'' time \DIFdelbegin \DIFdel{$t_0$ }\DIFdelend \DIFaddbegin \DIFadd{$T_0$ }\DIFaddend within two individually analyzed periods \DIFdelbegin \DIFdel{$T$}\DIFdelend \DIFaddbegin \DIFadd{$1, 2, \dots, T_0, T_0 + 1, \dots, T$}\DIFaddend : from 1979 to 1998; from 1987 to 2008.
We \DIFdelbegin \DIFdel{include each province }\DIFdelend \DIFaddbegin \DIFadd{separately include each of the eight provinces }\DIFaddend in the YRB (\DIFdelbegin \DIFdel{$n=8$, }\DIFdelend see \textit{\nameref{sec:dataset}}) as \DIFdelbegin \DIFdel{the treated unit separately, as multiple treated unitsapproach had been widely applied}\DIFdelend \DIFaddbegin \DIFadd{individual treated units}\DIFaddend ~\cite{abadie2021}.
Then, we consider the $J+1$ units observed in time periods \DIFdelbegin \DIFdel{$T = {1,2 \cdots , T}$ with }\DIFdelend \DIFaddbegin \DIFadd{$1, \dots, T$ where }\DIFaddend the remaining $J=20$ units \DIFdelbegin \DIFdel{are }\DIFdelend \DIFaddbegin \DIFadd{represent }\DIFaddend untreated provinces from outside \DIFdelbegin \DIFdel{.
We define $T_0$ to represent the number of pre-treatment periods ($1,\ldots,t_0$) and $T_1$ the number post-treatment periods ($t_0, \ldots, T$), such that $T = T_0+ T_1$.}\DIFdelend \DIFaddbegin \DIFadd{the YRB.\
%DIF >  We define $T_0$ to represent the number of pre-treatment periods ($1,\ldots,t_0$) and $T_1$ the number post-treatment periods ($t_0, \ldots, T$), such that $T = T_0+ T_1$.
}\DIFaddend The treated unit is exposed to the institutional shift in every post-treatment period \DIFdelbegin \DIFdel{$T_0$}\DIFdelend \DIFaddbegin \DIFadd{$T_0 +1, \dots, T$}\DIFaddend , unaffected by the institutional shift in all preceding periods \DIFdelbegin \DIFdel{$T_1$}\DIFdelend \DIFaddbegin \DIFadd{$1, 2, \dots, T_0$}\DIFaddend .
Then, any weighted average of the control units is a synthetic control and can be represented by a (\DIFdelbegin \DIFdel{$J * 1$}\DIFdelend \DIFaddbegin \DIFadd{$J \times 1$}\DIFaddend ) vector of weights $\mathbf{W} = (w_{1}, \ldots ,w_{J})$, with $w_j \in (0, 1)$ \DIFdelbegin \DIFdel{.
Among them, by introduce a ($k * k$) diagonal, matrix }\DIFdelend \DIFaddbegin \DIFadd{and $w_1 + \cdots  + w_{J} = 1$.
We denote $k$ is $T * D$ number of covariates, in which $D$ is number of dimensions of the dataset (i.e., $D = 5$ in this case).
Introducing a ($k \times 1$) non-negative vector }\DIFaddend $\mathbf{V}$ \DIFaddbegin \DIFadd{($\mathbf{V} = v_{1}, \ldots ,v_{k}$) and $v_1 + \cdots  + v_{k} = 1$ }\DIFaddend that signifies the relative importance of each \DIFdelbegin \DIFdel{covariant, the DSC method procedure for }\DIFdelend \DIFaddbegin \DIFadd{covariate.
%DIF >  接下来的目标就是求一个权重矩阵W,让它来表示控制组各单位的权重,使这些权重构成我们需要的“合成”版本的控制组
The next goal is }\DIFaddend finding the optimal synthetic control (\DIFdelbegin \DIFdel{$W$) is expressed as follows:
}\DIFdelend \DIFaddbegin \DIFadd{$\mathbf{W}$) which represents the ``synthetic'' versions of the affected provinces in the YRB.\
}\DIFaddend

\begin{equation}
    \mathbf{W^{*}(V)}=\DIFdelbegin %DIFDELCMD < \underset{\mathbf{W} \in \mathcal{W}}{\operatorname{minimize}}%%%
\DIFdelend \DIFaddbegin \underset{\mathbf{W} \in \mathcal{W}}{\operatorname{argmin}}\DIFaddend \left(\mathbf{X}_{\mathbf{1}}-\mathbf{X}_{\mathbf{0}} \mathbf{W}\right)^{\prime} \mathbf{V}\left(\mathbf{X}_{\mathbf{1}}-\mathbf{X}_{\mathbf{0}} \mathbf{W}\right)
\end{equation}

\DIFdelbegin \DIFdel{where }\DIFdelend \DIFaddbegin \DIFadd{Here, $\mathbf{X_1}$ is the pre-treatment average of each variable in the dataset for the treated unit, while $\mathbf{X_0}$ is a ($k \times J$) matrix containing the pre-treatment characteristics for each of the $J$ control units; }\DIFaddend $\mathbf{W}^{*}(V)$ is the vector of weights $\mathbf{W}$ that minimizes the difference between the pre-treatment characteristics of the treated unit and the synthetic control, given $\mathbf{V}$. That is, $\mathbf{W^{*}}$ depends on the choice of $\mathbf{V}$ –hence the notation \DIFdelbegin \DIFdel{$\mathbf{W*(V)}$}\DIFdelend \DIFaddbegin \DIFadd{$\mathbf{W^{*}(V)}$}\DIFaddend . Therefore, we choose $\mathbf{V^{*}}$ to be the $\mathbf{V}$ that results in \DIFdelbegin \DIFdel{$\mathbf{W*(V)}$ }\DIFdelend \DIFaddbegin \DIFadd{$\mathbf{{W}^{*}(V)}$ }\DIFaddend that minimizes the following expression:

\begin{equation}
    \mathbf{V}^{*}=\underset{\mathbf{V} \in \mathcal{V}}{\operatorname{argmin}}\left(\mathbf{Z}_{1}-\mathbf{Z}_{0} \mathbf{W}^{*}(\mathbf{V})\right)^{\prime}\left(\mathbf{Z}_{1}-\mathbf{Z}_{0} \mathbf{W}^{*}(\mathbf{V})\right)
\end{equation}

That is the minimum difference between the \DIFdelbegin \DIFdel{outcome of the treated unit }\DIFdelend \DIFaddbegin \DIFadd{water uses of treated units }\DIFaddend and the synthetic \DIFdelbegin \DIFdel{control }\DIFdelend \DIFaddbegin \DIFadd{controls }\DIFaddend in the pre-treatment period, where $\mathbf{Z}_{1}$ is a \DIFdelbegin \DIFdel{($1*T_0$) }\DIFdelend matrix containing every observation of the \DIFdelbegin \DIFdel{outcome }\DIFdelend \DIFaddbegin \DIFadd{water use }\DIFaddend for the treated unit in the pre-treatment period \DIFaddbegin \DIFadd{$T_0$}\DIFaddend .
Similarly, \DIFdelbegin \DIFdel{let }\DIFdelend $\mathbf{Z}_{0}$ \DIFdelbegin \DIFdel{be a ($k * T_0$) matrix containing the outcome }\DIFdelend \DIFaddbegin \DIFadd{is a ($J \times T_0$) matrix contains the water use }\DIFaddend for each control unit in the pre-treatment period\DIFdelbegin \DIFdel{, and $k$ is the number of variables in the datasets.
}\DIFdelend \DIFaddbegin \DIFadd{.
%DIF >  https://github.com/OscarEngelbrektson/SyntheticControlMethods/issues/18 这里和README不一样,因为它有问题
}\DIFaddend The DSC method generalizes the difference-in-differences estimator and allows for time-varying individual-specific unobserved heterogeneity, with \DIFdelbegin \DIFdel{double robustnessproperties}\DIFdelend \DIFaddbegin \DIFadd{better robustness}\DIFaddend ~\cite{billmeier2013, smith2015}.
\DIFaddbegin \DIFadd{In this study, we adopted the minimization by the ``Synthetic Control Methods'' Python library (version 1.1.17)~\mbox{%DIFAUXCMD
\cite{engelbrektson2023}}\hskip0pt%DIFAUXCMD
.
}\DIFaddend

\DIFdelbegin %DIFDELCMD < \subsection{Robustness analysis}%%%
\DIFdelend \DIFaddbegin \subsection{Validating results}\DIFaddend \label{sec:robustness}

Two primary methods can be employed to \DIFdelbegin \DIFdel{test the robustness }\DIFdelend \DIFaddbegin \DIFadd{validate the efficiency }\DIFaddend of the DSC approach.

Firstly, the reconstruction effect on inferred variables (water consumption here) before and after treatment (the interventions of 87-WAS and 98-UBR) can be compared.
If there are small gaps between the predicted and observed values before treatment, and a large gap after treatment, it indicates that the policy intervention's effect is apparent.
In this study, to determine whether the intervention effect is significant, the paired sample $T$ test is used to calculate statistics, comparing the model predictions and actual observation data in the periods before and after institutional interventions for both the 87-WAS in $1987$ and 98-UBR in $1998$.
\DIFdelbegin \DIFdel{A robust synthetic control model will show }\DIFdelend \DIFaddbegin \DIFadd{An effective result would be one where }\DIFaddend a significant difference \DIFaddbegin \DIFadd{is observed }\DIFaddend after treatment but not before treatment. \DIFaddbegin \DIFadd{If this is not the case, it implies that the institutional changes were ineffective for the treated units.
}\DIFaddend

Secondly, placebo \DIFdelbegin \DIFdel{experiments }\DIFdelend \DIFaddbegin \DIFadd{tests }\DIFaddend are another common way to evaluate the effectiveness of synthetic control methods\DIFaddbegin \DIFadd{~\mbox{%DIFAUXCMD
\cite{abadie2010}}\hskip0pt%DIFAUXCMD
}\DIFaddend .
Placebo units are selected from the control unit pool and substituted for the treated unit, applying the synthetic control method to the placebo unit using the same data and parameters as the treated unit.
If the synthetic control method is effective, there should be a clear difference between the placebo unit and the control unit since the placebo unit should not be affected by the intervention.
%DIF >  Placebo tests can be used to assess the effectiveness of the synthetic control method and detect any bias or confounding factors in the analysis.
In this study, we adopt the placebo test step suggested by Abadie when proposing the synthetic control method~\cite{abadie2010} and utilize the Python library of the differential synthetic control method for the placebo test.
If the ratio of the root mean square error (see Equation~\ref{ch5:eq:RMSE}) in the pre-synthesis period is significantly higher for most provinces (again using the $T$ test to determine the significance of the difference) than the results of other placebo units, \DIFdelbegin \DIFdel{it would suggest that the Yellow River Basin }\DIFdelend \DIFaddbegin \DIFadd{the provinces in the YRB }\DIFaddend was more significantly affected than most other provinces during the treatment periods ($1987$ and $1998$), i.e., \DIFdelbegin \DIFdel{the results are more robust}\DIFdelend \DIFaddbegin \DIFadd{more effective}\DIFaddend .

\begin{equation}
    \label{ch5:eq:RMSE}
    \text{RMSE} = \sqrt{\frac{1}{n}\sum_{i=1}^{n}{(y_i-\hat{y}_i)}^2}
\end{equation}

Where $n$ is the observed number, $y_i$ is the actual value, and $\hat{y}_i$ is the predicted value.


\section{Results}\label{sec:results}
%DIF > ! Author = songshgeo
%DIF > ! Date = 2022/3/10

%DIF >  \subsection{INSTITUTIONAL SHIFTS AND STRUCTURES}
\subsection{Institutional shifts and structures}\label{results-1}

\begin{figure*}[!t]
	\includegraphics[width=\linewidth]{diagrams/diagram.pdf}
	\caption{
		%DIF >  黄河流域的制度变迁与经济社会结构差异。
		Institutional shifts and related SES structures in the Yellow River Basin (YRB).
		\textbf{A.} The YRB crosses $10$ provinces or the same-level administrative regions, $8$ of which heavily rely on the water resources from the YRB (Table~\ref{tab:quota}). The national administrations hold ultimate authority in issuing water governance policies, which are often implemented by the basin-level agency (the Yellow River Conservancy Commission, YRCC) and each province-level agency.
		\textbf{B.} Provincial administrative agencies are the major stakeholders. Since the 87-WAS, with surface water withdrawal from the Yellow River restricted by specific quotas, each stakeholder plans and uses water resources for development. However, natural hydrological processes are interconnected. Although the institutions focus mainly on surface water (Sur.), they can also influence groundwater inside (Gro.) or water resources outside (Sur.\ and Gro.') through systematic socio-hydrological processes within the YRB.\\ The YRCC only monitors water withdrawals at that time.
		\textbf{C.} Institutional shifts and subsequent structural changes (details in \textit{\nameref{sec:yrb}}). (1) From 1979 to 1987, water resources were freely accessible to each stakeholder (denoted by red circles) from the connected ecological unit (the reach of the Yellow River, denoted by the blue circles). (2) After 1987-WAS, the YRCC (the yellow circles) monitored (the dot-line links) river reaches with water use quotas. (3) Since the 98-UBR, stakeholders have had to apply for water use licenses from the YRCC (the connections between the red and yellow circles).
	}\label{fig:structure}
\end{figure*}

%DIF >  制度变动综述
Until the 87-WAS, provincial regions in the YRB had unrestricted access to the Yellow River water resources for development, despite geographic and temporal differences between freshwater demand and availability.
The YRCC had no links to the provinces regarding water use before 1987, and the provinces could connect directly to the Yellow River reaches (Figure~\ref{fig:structure}~C).
Following the 87-WAS, national authorities proposed allocating specific water quotas among the provinces, and the YRCC's duty became to report actual water use volumes in each reach.
As it was the first time the YRCC's responsibilities included water use, this introduced new links between the YRCC and the river (i.e., ecological nodes Figure~\ref{fig:structure}~C).
The 98-UBR further reinforced the YRCC's responsibilities for integrated water use management.
Since $1998$, provinces have been required to submit their annual water use plans for water use licenses to the YRCC instead of freely accessing the Yellow River water.
Consequently, the YRCC has been directly linked to the provinces since then (Figure~\ref{fig:structure}C).
%DIF >  As result, the two institutional shifts reshaped SES structures, leading to three general structures linked by social actors and ecological nodes (Figure~\ref{fig:structure}~C).
Key points of the official documents supporting the structural changes above can be found in supplementary material~\textit{\nameref{secS1}}.

%DIF >  Table generated by Excel2LaTeX from sheet 'Sheet2'
\begin{table}[htbp]\footnotesize
	\centering
	\caption{Water quotas assigned for provincial regions in the YRB}\label{tab:quota}
	  \begin{tabularx}{\textwidth}{p{3cm}XXXXX}
	  \toprule
	  Provincial regions & \multicolumn{1}{l}{Water planning$^a$} & \multicolumn{1}{l}{Proposal in 1983$^b$} & \multicolumn{1}{l}{Scheme in 1987$^c$} & \multicolumn{1}{l}{Avg. WU$^d$} & \multicolumn{1}{l}{Ratio ($\%$)$^e$} \\
	  \midrule
	  Qinghai & 35.70  & 14.00  & 14.10  & 12.03  & 48.12  \\
	  Sichuan & 0.00  & 0.00  & 0.40  & 0.25  & 0.10  \\
	  Gansu & 73.50  & 30.00  & 30.40  & 25.80  & 30.79  \\
	  Ningxia & 60.50  & 40.00  & 40.00  & 36.58  & 58.45  \\
	  Inner Mongolia & 148.90  & 62.00  & 58.60  & 61.97  & 47.82  \\
	  Shanxi & 115.00  & 43.00  & 38.00  & 21.16  & 73.55  \\
	  Shaanxi & 60.80  & 52.00  & 43.10  & 11.97  & 44.39  \\
	  Henan & 111.80  & 58.00  & 55.40  & 34.30  & 24.77  \\
	  Shandong & 84.00  & 75.00  & 70.00  & 77.87  & 34.41  \\
	  Jinji & 6.00  & 0.00  & 20.00  & 5.85  & 3.11  \\
	  \bottomrule
	  \multicolumn{6}{p{\textwidth}}{$^a$ In 1982, each provincial region proposed their water use plans.}\\
	  \multicolumn{6}{p{\textwidth}}{$^b$ In 1983, the Yellow River Conservancy Commission (YRCC) proposed these initial water quotas.}\\
	  \multicolumn{6}{p{\textwidth}}{$^c$ In 1987, the quotas agreed by state department (Ministry of Water Resources).}\\
	  \multicolumn{6}{p{\textwidth}}{$^d$ Average water use (WU) from the Yellow River for each region. Because of missing data, Sichuan and Jinji were calculated by data from 2004 to 2017.}\\
	  \multicolumn{6}{p{\textwidth}}{$^e$ Ratio of the average water use (WU) from the Yellow River to provincial total water uses.}\\
	  \end{tabularx}\\
\end{table}%DIF >


%DIF > ! Author = songshgeo
%DIF > ! Date = 2022/3/10
%DIF >  \MakeUppercase{\subsection{Cascading effects of the institutional shifts}}
\DIFaddbegin

%DIF >  \subsection{}

%DIF > \subsection{INSTITUTIONAL SHIFTS IMPACT ON WATER USE OF THE YRB}
\DIFaddend \subsection{Institutional shifts impact on water use}\label{result-2}
%DIF >  结果一:展示制度转变带来的用水量变化

\begin{figure*}[!htb]
	\centering
	\includegraphics[width=0.9\linewidth]{outputs/main_results2.pdf}
	\caption{
	Effects of two institutional shifts on water resources use and allocation in the Yellow River Basin (YRB).
	\textbf{A.} Water uses of the YRB before and after the institutional shift in 1987 (87-WAS);
	\textbf{B.} Water uses of the YRB before and after the institutional shift in 1998 (98-UBR). Blue lines are statistics derived from water use data; grey lines are estimates from the Differenced Synthetic Control method with economic and environmental background controlled;
	\textbf{C.} Drought intensity in the YRB and drying up events of the Yellow River. The size of the grey bubbles denotes the length of drying upstream.
	}\label{fig:main_results}
\end{figure*}

%DIF >  黄河流域的总用水量在反事实推断模型和实际观测值在两次制度变化后呈现出差异显著,在之前此差异则较小且不显著(见图\ref{ch5:fig:main_results}A和B),这表明其用水变化的估计重建良好。
The total water use of the YRB exhibited a significant difference between the counterfactual prediction and the actual observed value after the two institutional shifts, while the difference was small and insignificant before (see Figures~\ref{fig:main_results}A and B). This indicates that the estimated reconstruction of water use change was \DIFdelbegin \DIFdel{robust}\DIFdelend \DIFaddbegin \DIFadd{effective}\DIFaddend .
Figure~\ref{fig:main_results}A suggests that the 87-WAS prompted the provinces to withdraw even more water than would have been used without an institutional shift (Figure~\ref{fig:main_results}A).
%DIF >  NOTE: 这里是每年 xxx 立方米
From 1988 to 1998, on average, while the estimation of annual water use only suggests \DIFdelbegin \DIFdel{$887.05~km^3$ }\DIFdelend \DIFaddbegin \DIFadd{$887.05$ }\DIFaddend billion $m^3$, the observed water use of the YRB provinces reached $938.06$ billion $m^3$ (an increase of $5.75\%$).
However, after the 98-UBR, trends of increasing water use appeared to be effectively suppressed.
From 1998 to 2008, the total observed water use decreased by $6.6$ billion $m^3/yr$ per year, while the estimation of water use still suggests $5.5$ billion $m^3/yr$ increases (Figure~\ref{fig:main_results} B).
The increased water uses after 87-WAS align with the severe dry-up of the surface streamflow from $1987$ to $1998$, a clear indicator of river degradation and environmental crisis (Figure~\ref{fig:main_results}C).
On the other hand, the 98-UBR ended river depletion, despite subsequent increases in drought intensity (from $0.47$ after 87-WAS to $0.62$ after 98-UBR on average) (Figure~\ref{fig:main_results}C).

%DIF > \subsection{REGIONAL DIFFERENCES IN RESPONSES TO INSTITUTIONAL SHIFTS}
\subsection{Heterogeneous effects and interpretation}\label{result-3}

\begin{figure*}[!htb]
	\centering
	\includegraphics[width=0.9\linewidth]{outputs/fig3.pdf}
	\caption{
		Regulating differences for provinces in the YRB.\\
		Red (the 87-WAS) and green (the 98-UBR) bars denote an increased or decreased ratio for actual water use relative to the estimate from the model in the decade after the institutional shift.
		The grey bars indicate the proportions of actual water use for each province relative to their total water use in the decade after the institutional shift.
		The triangles mark the water quotas assigned under the institution, converted to ratios by dividing by their sum.
	}\label{fig:regulating}
\end{figure*}

Our results demonstrate that there are differences in the response patterns of the two changes in the water resources allocation system.
In Figure~\ref{fig:regulating}, the red bar chart (87-WAS) and the green bar chart (98-UBR) respectively represent the increase or decrease ratio of actual water consumption compared to the estimated water use of the DSC model within ten years after the institutional shifts.
The gray bar chart shows the ratio of actual water use by provinces to their total water use in the decade after the two changes; The triangle marks indicate the ratio of the theoretical water resource quota of the province to the total available water in the YRB.\
In the ten years after the 87-WAS, the proportion of water consumption increase (or decrease) compared to that estimated by the DSC model was positively correlated with the proportion of water consumption taken from the YRB at present (partial correlation coefficient was $0.64$\DIFdelbegin \DIFdel{, Figure~\ref{fig:regulating}}\DIFdelend ).
From 1987 to 1998, some provinces with high water consumption (e.g., Inner Mongolia and Henan) also showed significant increases in water consumption (Figure~\ref{fig:regulating} and Table~\ref{tab:DSC_summary}), with the average water consumption in four major users (Shandong, Inner Mongolia, Henan, and Ningxia) exceeding the predicted value by $32.14\%$.
However, from 1998 to 2008, almost all provinces experienced a decrease in water consumption (by an average of $16.54\%$).
In addition, the water consumption of each province has a negative correlation with the proportion of water taken from the Yellow River Basin (partial correlation coefficient is $-0.51$).


%DIF >  Table generated by Excel2LaTeX from sheet 'Sheet1'
\begin{table}[!htbp]\footnotesize
	\centering
	\caption{Pre and post treatment root mean squared prediction error (RMSE) for YRB's provinces}\label{tab:DSC_summary}
	\begin{tabularx}{0.8\textwidth}{XXXXXXX}
	  \toprule
			& \multicolumn{3}{c}{87-WAS} & \multicolumn{3}{c}{98-UBR} \\
  \cmidrule{2-7}    province  & \multicolumn{1}{c}{post/pre} & To avg. & \multicolumn{1}{c}{sig.} & \multicolumn{1}{c}{post/pre} & To avg.   & \multicolumn{1}{c}{sig.} \\
	  \midrule
	  Qinghai & 5.26  & =     & FALSE & 5.89  & >     & TRUE \\
	  Gansu & 10.37  & >     & TRUE  & 9.55  & >     & TRUE \\
	  Ningxia & 5.81  & =     & FALSE & 6.83  & >     & TRUE \\
	  Inner Mongolia & 7.11  & >     & TRUE  & 1.60  & <     & TRUE \\
	  Shanxi & 1.72  & <     & TRUE  & 5.60  & >     & TRUE \\
	  Shaanxi & 3.05  & <     & TRUE  & 3.01  & >     & TRUE \\
	  Henan & 20.66  & >     & TRUE  & 1.18  & <     & TRUE \\
	  Shandong & 4.54  & =     & FALSE & 4.14  & >     & TRUE \\
	  \bottomrule
	  \end{tabularx}%DIF >
	\label{tab:addlabel}%DIF >
\end{table}%DIF >
\DIFaddbegin



\DIFaddend \section{Discussion}\label{sec:discussion}
%DIF > ! Author = songshgeo
%DIF > ! Date = 2022/3/10

%DIF >  \subsection{CAUSES OF INSTITUTIONAL IMPACTS}
%DIF >  \subsection{}
%DIF >  discussion-1:
%DIF >  用水量的上升、下降-结果解读
%DIF >  制度对社会生态系统的结果产生影响在世界范围内都很普遍,
The impacts of institutional shifts on the governing effects of social-ecological systems (SESs) have \DIFdelbegin \DIFdel{been widely reported worldwide, but few attempts have been made }\DIFdelend \DIFaddbegin \DIFadd{attracted global attention, yet efforts }\DIFaddend to quantify their net effects \DIFaddbegin \DIFadd{remain sparse}\DIFaddend ~\cite{cumming2020a}.
Our \DIFdelbegin \DIFdel{case study }\DIFdelend \DIFaddbegin \DIFadd{investigation }\DIFaddend of the YRB's water governance \DIFdelbegin \DIFdel{suggests that }\DIFdelend \DIFaddbegin \DIFadd{reveals vary effects of nuanced-differences institutional shifts: }\DIFaddend while the 98-UBR \DIFdelbegin \DIFdel{decreased }\DIFdelend \DIFaddbegin \DIFadd{led to an expected decrease in }\DIFaddend total water use\DIFdelbegin \DIFdel{as expected}\DIFdelend , the 87-WAS \DIFdelbegin \DIFdel{unexpectedly }\DIFdelend \DIFaddbegin \DIFadd{surprisingly }\DIFaddend increased it by $5.75\%$\DIFdelbegin \DIFdel{, a comparison of which can yield insights into }\DIFdelend \DIFaddbegin \DIFadd{.
This comparison offers insightful perspectives on }\DIFaddend the effectiveness of governance \DIFdelbegin \DIFdel{.
Firstly, the results challenge previous analyses (i.e., suggesting that 87-WAS ``had little practical effect'') because theoretically, there should be few gaps between actual and synthetic water use in the YRB if no effect is present~\mbox{%DIFAUXCMD
\cite{abadie2015,hill2021}}\hskip0pt%DIFAUXCMD
.
However, the significant net effect indicated by our analysis suggests }\DIFdelend \DIFaddbegin \DIFadd{because it suggests a significant net effect on increased water use following the implementation of this policy, in addition to the previous reports and comments suggesting }\DIFaddend that the 87-WAS was \DIFdelbegin \DIFdel{followed by increased water use even after controlling for environmental and economic variables (see }\textit{%DIFDELCMD < \nameref{secS2}%%%
} %DIFAUXCMD
\DIFdel{Table~\ref{tab:variables})}\DIFdelend \DIFaddbegin \DIFadd{``out-of-control''~\mbox{%DIFAUXCMD
\cite{wang2019d, departmentofearthsciences1999}}\hskip0pt%DIFAUXCMD
}\DIFaddend .
In contrast, the 98-UBR reduced surface water competition, so many studies attributed the streamflow restoration mainly to the successful introduction of \DIFdelbegin \DIFdel{this institution}\DIFdelend \DIFaddbegin \DIFadd{it}\DIFaddend ~\cite{chen2021,huangang2002,an2007}.

\DIFdelbegin \DIFdel{Examining the unexpected }\DIFdelend \DIFaddbegin \DIFadd{The unanticipated consequence of the }\DIFaddend 87-WAS policy \DIFdelbegin \DIFdel{, we found it shared a similar structure with }\DIFdelend \DIFaddbegin \DIFadd{echoes the structural challenges reported in }\DIFaddend many other SES governance failures\DIFdelbegin \DIFdel{, supporting the hypothesis that specific mismatched structures can rapidly exhaust }\DIFdelend \DIFaddbegin \DIFadd{.
This suggests a general pattern where specific misaligned structures can precipitate the rapid depletion of }\DIFaddend common resources~\cite{kellenberg2009,cai2016,barnes2019}.
\DIFdelbegin \DIFdel{Generally, these }\DIFdelend \DIFaddbegin \DIFadd{These }\DIFaddend structure-based failures \DIFaddbegin \DIFadd{often }\DIFaddend occur when social actors \DIFdelbegin \DIFdel{freely access }\DIFdelend \DIFaddbegin \DIFadd{have unregulated access to }\DIFaddend linked resource units\DIFdelbegin \DIFdel{(like the institution before }\DIFdelend \DIFaddbegin \DIFadd{, a feature prevalent in the institution prior to }\DIFaddend 1987\DIFdelbegin \DIFdel{), while the monitoring duty of the YRCC after 87-WAS was a sign that water quota was valuable to pursue (between 1987 and 1998).
This conjecture aligns with the increased water use after 87-WAS and the concerns about frequently scrambling for water in some provinces during this period~\mbox{%DIFAUXCMD
\cite{mao2000, bouckaert2022}}\hskip0pt%DIFAUXCMD
}\DIFdelend \DIFaddbegin \DIFadd{~\mbox{%DIFAUXCMD
\cite{wang2019c}}\hskip0pt%DIFAUXCMD
.
When the central government attempted to curtail this free access by introducing water quotas, they were met with water demands from stakeholders' proposals that far exceeded expectations (Table~\ref{tab:quota})}\DIFaddend .
A previous study \DIFdelbegin \DIFdel{analyzed reasons for the non-ideal }\DIFdelend \DIFaddbegin \DIFadd{attributed the suboptimal }\DIFaddend effect of 87-WAS \DIFdelbegin \DIFdel{~\mbox{%DIFAUXCMD
\cite{huangang2002}}\hskip0pt%DIFAUXCMD
}\DIFdelend \DIFaddbegin \DIFadd{to the lack of enforcement and control mechanisms~\mbox{%DIFAUXCMD
\cite{huangang2002}}\hskip0pt%DIFAUXCMD
.
Taken together, it underpins a hypothesis that in the absence of enforcement}\DIFaddend , \DIFdelbegin \DIFdel{where core concerns were the lack of enforcementand controlling approaches, while major stakeholders kept arguing they needed more quotas from 1983 to the 1990s.
It indicates that it was reasonable for stakeholders to pursue more water quota by withdrawing more water, beyond their economic growth.
Our results align with this hypothesis since the correlation between current water use and changed (increased or decreased) water use was significant after the 87-WAS but not after the 98-UBR (Figure~\ref{fig:regulating}).
In addition, through a theoretically marginal benefit analysis, this ``major users use more'' pattern can be inferred from a simple assumption that stakeholders can expect water quota's value in the short future, also supporting the above hypothesis (see }\textit{%DIFDELCMD < \nameref{secS4}%%%
}%DIFAUXCMD
\DIFdel{).
}\DIFdelend \DIFaddbegin \DIFadd{stakeholders might have exploited the system by increasing water withdrawals to secure more water quotas for their economic prospects.
}\DIFaddend

\DIFdelbegin \DIFdel{Besides our results, the above hypothesis also aligns with }\DIFdelend \DIFaddbegin \DIFadd{This hypothesis can be further substantiated by }\DIFaddend two reported facts:
(1) \DIFdelbegin \DIFdel{The water quotas of }\DIFdelend \DIFaddbegin \DIFadd{There were not only surges of total water uses following the }\DIFaddend 87-WAS\DIFdelbegin \DIFdel{(or the initial water rights) went through }\DIFdelend \DIFaddbegin \DIFadd{, but also scrambles for water reported in several provinces during this period~\mbox{%DIFAUXCMD
\cite{mao2000, bouckaert2022}}\hskip0pt%DIFAUXCMD
.
(2) From 1983 to the 1990s, the stakeholders persistently argued for increased the water quotas, when is }\DIFaddend a stage of ``bargaining''\DIFdelbegin \DIFdel{among stakeholders (from 1982 to 1987) and the bargaining arguments even persisted years after 1987~\mbox{%DIFAUXCMD
\cite{wang2019e, wang2019d}}\hskip0pt%DIFAUXCMD
.
During this process, each province attempted to demonstrate its development potential related to water use, to match water shares to their economy because the major water users (like Shandong and Henan) needed more water than their original quota (if only considering economic potentials when designing the institution)~\mbox{%DIFAUXCMD
\cite{zuo2020}}\hskip0pt%DIFAUXCMD
.
(2) During the }\DIFdelend \DIFaddbegin \DIFadd{~\mbox{%DIFAUXCMD
\cite{wang2019e, wang2019d}}\hskip0pt%DIFAUXCMD
;
(3) During this }\DIFaddend ``bargaining'' \DIFdelbegin \DIFdel{, more significant stakeholders had considerable
incentives to pursue more water quotas, which aligned with the fact that major water users }\DIFdelend \DIFaddbegin \DIFadd{stage, the stakeholders who had more economic profits }\DIFaddend submitted appeals to the higher central government for larger shares~\cite{wang2019e, wang2019d}.
\DIFdelbegin \DIFdel{This means provinces with higher current water use have greater bargaining power in water use allocation.
}\DIFdelend

\DIFdelbegin \DIFdel{On the other hand}\DIFdelend \DIFaddbegin \DIFadd{Our results also corroborate some intuitive deductions of the hypothesis.
Firstly, we found significant correlations between current and changed water use after the 87-WAS, which suggests that the key stakeholders (such as Neimeng, Henan, and Shandong), were more likely to be affected by the institutional change.\
Secondly, a theoretical marginal benefit analysis (see }\textit{\nameref{secS4}}\DIFadd{) suggests that this ``major users are effected more'' pattern can be inferred from a simple assumption that stakeholders anticipate future value in water quotas, thereby lending further support to the above hypothesis.
Finally}\DIFaddend , since the YRCC could \DIFaddbegin \DIFadd{forcibly }\DIFaddend coordinate stakeholders by water quota licenses \DIFdelbegin \DIFdel{according to water conditions }\DIFdelend for the entire YRB after 98-UBR, the external appeals of provinces for larger quotas turned into internal innovation to improve water efficiency (e.g., drastically increased water-conserving equipment)~\cite{krieger1955, ostrom1990}.
\DIFdelbegin \DIFdel{Similar, proportional decreased water use of provinces and the theoretical minimal water use of marginal benefit model indicated this policy lead to successful governance as expected (see }%DIFDELCMD < \nameref{result-3}%%%
\DIFdel{)}\DIFdelend \DIFaddbegin

\DIFadd{On the flip side, the apparent success of the 98-UBR institutional transformation has received consistent acclaim, particularly for its role in restoring the previously dry river~\mbox{%DIFAUXCMD
\cite{wang2019e, wang2019d}}\hskip0pt%DIFAUXCMD
.
Our findings suggest that the 98-UBR led to a proportional decrease in water use across provinces, indeed indicative of an immediate and tangible effect}\DIFaddend .
However, \DIFdelbegin \DIFdel{since }\DIFdelend \DIFaddbegin \DIFadd{it's essential to recognize that }\DIFaddend the 98-UBR \DIFdelbegin \DIFdel{only regulated }\DIFdelend \DIFaddbegin \DIFadd{focused solely on regulating }\DIFaddend surface water use, \DIFdelbegin \DIFdel{many clues suggested the institution shift may cause broader influences, including estimated }\DIFdelend \DIFaddbegin \DIFadd{which hints at potential broader implications.
Notably, some evidence suggests that this institutional shift might have resulted in }\DIFaddend increased groundwater withdrawals \DIFdelbegin \DIFdel{after 98-UBR in many intensive water use regions}\DIFdelend \DIFaddbegin \DIFadd{in regions with intensive water usage following the 98-UBR}\DIFaddend ~\cite{sun2022b}.
\DIFdelbegin \DIFdel{With limited }\DIFdelend \DIFaddbegin \DIFadd{Unfortunately, the limited availability of }\DIFaddend eligible data on groundwater use \DIFdelbegin \DIFdel{, related assessmentis }\DIFdelend \DIFaddbegin \DIFadd{constrains a comprehensive assessment, leaving this aspect }\DIFaddend beyond the scope of \DIFdelbegin \DIFdel{this studybut remains quite important, }\DIFdelend \DIFaddbegin \DIFadd{the current study.
Nonetheless, this consideration remains highly relevant, especially }\DIFaddend as similar water quota policies \DIFdelbegin \DIFdel{started }\DIFdelend \DIFaddbegin \DIFadd{have begun }\DIFaddend to be implemented nationally since the \DIFaddbegin \DIFadd{turn of the }\DIFaddend 21st century.

\DIFdelbegin \DIFdel{The }\DIFdelend \DIFaddbegin \DIFadd{To provide an intuitive understanding of the profound impact of the Institutional shifts, we can turn to the insights shared by a representative of the Hetao Irrigation District in Neimeng.
As a primary stakeholder, the district's representative voiced the struggle to adapt under the 98-UBR policy which strictly enforced water quotas in our surveys.
``The water allocated to us is far from enough'', he revealed with a desire on more water quota: ``And it's not like in the past when we could actually over use, it is very strictly controlled now.''
``Under a limited quota, of course there are conflicts between users time to time, which depends on leaderships of the water-user associates'', he reflected: ``-farmers may have their own solutions, such as switching to sunflower, which is more water-efficient, or using shallow groundwater when is available.''
Simultaneously, the district looked forward to future projects, such as the ``South-to-North Water Diversion'' Western Route Project, which they hoped would increase their water quotas and allow for expansion of their irrigation area.
The desire of water in Neimeng wasn't without controversy. Stakeholders in other lower reaches argued that the Hetao Irrigation District was consuming too much water from the Yellow River.
}

\DIFadd{The above analysis with a real-world example underscores the crucial role of institutions in water governance by shaping SES structures.
The }\DIFaddend structural pattern we depicted here (Figure~\ref{fig:structure}) has also been reported in other SESs worldwide~\cite{kluger2020,guerrero2015,bodin2012}.
Before 98-UBR, fragmented ecological units were linked to separate social actors, which more likely led to lower effectiveness because isolated actors generally struggle to maintain interconnected ecosystems holistically~\cite{sayles2017,sayles2019,cai2016,bergsten2019}.
Institutional re-alignments since 98-UBR enhanced the responsibilities of the basin-scale authority (YRCC) and led to effectiveness in runoff restoration, which are usually named scale match or institutional match of SESs~\cite{cumming2020a,wang2019d}.
Taken together, the comparison demonstrates the challenge of finding win-win situations in coupled social-ecological systems~\cite{hegwood2022} and the need to more deeply understand the role of institutions in water governance~\cite{bergsten2019, sayles2019}.

Our approach has some inevitable limitations.
First, the contributions of economic growth and institutional shifts are difficult to distinguish because of intertwined causality (institutional changes can also influence the relative economic variables);
and second, when applying the DSC method, it is difficult to rule out the effects of other policies over the same time breakpoints (1987 and 1998).
Our quasi-experiment approach nonetheless provides evidence supporting the view that there was a change in water use trajectory following the YRB's unique institutional shifts and offers insights into water governance (and particularly the importance of having a scale-matched, basin-wide authority for water allocation solutions~\cite{bodin2017b, ostrom2009, reyers2018}).
Moreover, the ultimate success of the 98-UBR institutional shift theoretically and practically proved the importance of social-ecological fit.
For sustainability in the future, it is necessary to emphasize the need to strengthen connections between stakeholders through agents consistent with the scale of the ecological system.
For example, water rights transfers may be another way to build horizontal links between stakeholders that also have the potential to result in better water governance.
In addition, policymakers can propose more dynamic and flexible institutions to increase the adaptation of stakeholders to a changing SES context~\cite{reyers2018}.

The structural pattern that led to different effectiveness is widespread in global SESs, making our proposed mechanism crucial to governing such coupled systems.
Our analysis suggests that these initiatives can lead to unexpected results because of mismatched structure when shifting to another institution~\cite{bodin2017b}.
Better governance calls for more institutional analysis while China has embarked to redesign its decades-old water allocation scheme.
Our research provides insights into how institutions can affect achieving successful river basin governance when socio-hydrological interplays frame SES structures~\cite{muneepeerakul2017, leslie2015, hegwood2022}.


\section{Conclusion}\label{sec:conclusion}
%DIF > ! Author = songshgeo
%DIF > ! Date = 2022/3/10

In this study, we examined the effects of two major institutional shifts in water governance within the Yellow River Basin (YRB): the 1987 Water Allocation Scheme (87-WAS) and the 1998 Unified Basin Regulation (98-UBR). By employing a \DIFdelbegin \DIFdel{Difference-in-Differences approach with Synthetic Control }\DIFdelend \DIFaddbegin \DIFadd{Differenced Synthetic Control (DSC) approach}\DIFaddend , we quantified the net effects of these institutional shifts on water use within the YRB.\
Our results showed that the 87-WAS unexpectedly increased water use by $5.75\%$, contrary to its intended goals, while the 98-UBR successfully reduced water use as anticipated. The analysis \DIFdelbegin \DIFdel{revealed }\DIFdelend \DIFaddbegin \DIFadd{suggested }\DIFaddend that the structural patterns of the institutions played a critical role in their effectiveness. The mismatched structure of the 87-WAS led to increased competition and exploitation of water resources, while the 98-UBR, with its scale-matched, basin-wide authority and stronger connections between stakeholders, resulted in improved water governance.

In conclusion, our research contributes to a better understanding of the role of institutions in SES governance, particularly in the context of water management. By identifying the key factors that influence the success or failure of institutional shifts, we provide valuable insights for the design of effective and sustainable water governance policies. Future research should continue to explore the intricacies of institutions in SES governance and investigate the potential impacts of additional policies and institutional shifts on water use and sustainability.


\textbf{Authors Contribution}\\
Shuai Wang and BF designed this research. Shuang Song performed the study and analysed data. Shuang Song and Huiyu Wen wrote the paper. Xutong Wu, Cumming S. Graeme, and HW revised and polished the manuscript and gave significant advice.

\textbf{Acknowledgments}\\
This research has been supported by the National Natural Science Foundation of China (grant no. 42041007) and the Fundamental Research Funds for the Central Universities\DIFaddbegin \DIFadd{.

}\DIFaddend

\DIFaddbegin \textbf{\DIFadd{Declaration of generative AI and AI-assisted technologies in the writing process}}

\DIFadd{During the preparation of this work the authors used ChatGPT 4.0 in order to polish sentences. After using this tool/service, the authors reviewed and edited the content as needed and take full responsibility for the content of the publication.

}

\DIFaddend \begin{thebibliography}{10}
\expandafter\ifx\csname url\endcsname\relax
  \def\url#1{\texttt{#1}}\fi
\expandafter\ifx\csname urlprefix\endcsname\relax\def\urlprefix{URL }\fi
\expandafter\ifx\csname href\endcsname\relax
  \def\href#1#2{#2} \def\path#1{#1}\fi

\bibitem{distefano2017}
T.~Distefano, S.~Kelly, Are we in deep water? {{Water}} scarcity and its limits
  to economic growth, Ecological Economics 142 (2017) 130--147.
\newblock \href {http://dx.doi.org/10.1016/j.ecolecon.2017.06.019}
  {\path{doi:10.1016/j.ecolecon.2017.06.019}}.

\bibitem{dolan2021}
F.~Dolan, J.~Lamontagne, R.~Link, M.~Hejazi, P.~Reed, J.~Edmonds, Evaluating
  the economic impact of water scarcity in a changing world, Nature
  Communications 12~(1) (2021) 1915.
\newblock \href {http://dx.doi.org/10.1038/s41467-021-22194-0}
  {\path{doi:10.1038/s41467-021-22194-0}}.

\bibitem{xu2020b}
Z.~Xu, Y.~Li, S.~N. Chau, T.~Dietz, C.~Li, L.~Wan, J.~Zhang, L.~Zhang, Y.~Li,
  M.~G. Chung, J.~Liu, Impacts of international trade on global sustainable
  development, Nature Sustainability\href
  {http://dx.doi.org/10.1038/s41893-020-0572-z}
  {\path{doi:10.1038/s41893-020-0572-z}}.

\bibitem{mekonnen2016}
M.~M. Mekonnen, A.~Y. Hoekstra, Four billion people facing severe water
  scarcity, Science Advances 2~(2) (2016) e1500323.
\newblock \href {http://dx.doi.org/10.1126/sciadv.1500323}
  {\path{doi:10.1126/sciadv.1500323}}.

\bibitem{florke2018}
M.~Fl{\"o}rke, C.~Schneider, R.~I. McDonald, Water competition between cities
  and agriculture driven by climate change and urban growth, Nature
  Sustainability 1~(1) (2018) 51--58.
\newblock \href {http://dx.doi.org/10.1038/s41893-017-0006-8}
  {\path{doi:10.1038/s41893-017-0006-8}}.

\bibitem{yoon2021}
J.~Yoon, C.~Klassert, P.~Selby, T.~Lachaut, S.~Knox, N.~Avisse, J.~Harou,
  A.~Tilmant, B.~Klauer, D.~Mustafa, K.~Sigel, S.~Talozi, E.~Gawel,
  J.~{Medell{\'i}n-Azuara}, B.~Bataineh, H.~Zhang, S.~M. Gorelick, A coupled
  human\textendash natural system analysis of freshwater security under climate
  and population change, Proceedings of the National Academy of Sciences
  118~(14) (2021) e2020431118.
\newblock \href {http://dx.doi.org/10.1073/pnas.2020431118}
  {\path{doi:10.1073/pnas.2020431118}}.

\bibitem{wang2019d}
Y.~Wang, S.~Peng, j.~Wu, G.~Ming, G.~Jiang, H.~Fang, C.~Chen, {Review of the
  Implementation of the Yellow River Water Allocation Scheme for Thirty Years},
  Yellow River 41~(9) (2019) 6--19.
\newblock \href {http://dx.doi.org/10.3969/j.issn.1000-1379.2019.09.002}
  {\path{doi:10.3969/j.issn.1000-1379.2019.09.002}}.

\bibitem{young2008}
O.~R. Young, L.~A. King, H.~Schroeder (Eds.), Institutions and Environmental
  Change: Principal Findings, Applications, and Research Frontiers, {MIT
  Press}, {Cambridge, Mass}, 2008.

\bibitem{lien2020}
A.~M. Lien, The institutional grammar tool in policy analysis and applications
  to resilience and robustness research, Current Opinion in Environmental
  Sustainability 44 (2020) 1--5.
\newblock \href {http://dx.doi.org/10.1016/j.cosust.2020.02.004}
  {\path{doi:10.1016/j.cosust.2020.02.004}}.

\bibitem{bodin2017b}
{\"O}.~Bodin, M.~L. Barnes, R.~R. McAllister, J.~C. Rocha, A.~M. Guerrero,
  Social\textendash{{Ecological Network Approaches}} in {{Interdisciplinary
  Research}}: {{A Response}} to {{Bohan}} et al. and {{Dee}} et al., Trends in
  Ecology \& Evolution 32~(8) (2017) 547--549.
\newblock \href {http://dx.doi.org/10.1016/j.tree.2017.06.003}
  {\path{doi:10.1016/j.tree.2017.06.003}}.

\bibitem{wang2022g}
K.~Wang, Z.~Cai, Y.~Xu, F.~Zhang, Hexagonal cyclical network structure and
  operating mechanism of the social-ecological system, Ecological Indicators
  141 (2022) 109099.
\newblock \href {http://dx.doi.org/10.1016/j.ecolind.2022.109099}
  {\path{doi:10.1016/j.ecolind.2022.109099}}.

\bibitem{epstein2015}
G.~Epstein, J.~Pittman, S.~M. Alexander, S.~Berdej, T.~Dyck, U.~Kreitmair,
  K.~J. Rathwell, S.~{Villamayor-Tomas}, J.~Vogt, D.~Armitage, Institutional
  fit and the sustainability of social\textendash ecological systems, Current
  Opinion in Environmental Sustainability 14 (2015) 34--40.
\newblock \href {http://dx.doi.org/10.1016/j.cosust.2015.03.005}
  {\path{doi:10.1016/j.cosust.2015.03.005}}.

\bibitem{green2013}
O.~Green, A.~Garmestani, H.~{van Rijswick}, A.~Keessen, {{EU Water
  Governance}}: {{Striking}} the {{Right Balance}} between {{Regulatory
  Flexibility}} and {{Enforcement}}?, Ecology and Society 18~(2).
\newblock \href {http://dx.doi.org/10.5751/ES-05357-180210}
  {\path{doi:10.5751/ES-05357-180210}}.

\bibitem{loos2022}
J.~R. Loos, K.~Andersson, S.~Bulger, K.~C. Cody, M.~Cox, A.~Gebben, S.~M.
  Smith, Individual to collective adaptation through incremental change in
  {{Colorado}} groundwater governance, Frontiers in Environmental Science 10.
\newblock \href {http://dx.doi.org/10.3389/fenvs.2022.958597}
  {\path{doi:10.3389/fenvs.2022.958597}}.

\bibitem{hadjimichael2020}
A.~Hadjimichael, J.~Quinn, P.~Reed, Advancing {{Diagnostic Model Evaluation}}
  to {{Better Understand Water Shortage Mechanisms}} in {{Institutionally
  Complex River Basins}}, Water Resources Research 56~(10) (2020)
  e2020WR028079.
\newblock \href {http://dx.doi.org/10.1029/2020WR028079}
  {\path{doi:10.1029/2020WR028079}}.

\bibitem{bouckaert2022}
F.~W. Bouckaert, Y.~Wei, J.~Pittock, V.~Vasconcelos, R.~Ison, River basin
  governance enabling pathways for sustainable management: {{A}} comparative
  study between {{Australia}}, {{Brazil}}, {{China}} and {{France}}, Ambio
  51~(8) (2022) 1871--1888.
\newblock \href {http://dx.doi.org/10.1007/s13280-021-01699-4}
  {\path{doi:10.1007/s13280-021-01699-4}}.

\bibitem{vallury2022}
S.~Vallury, H.~C. Shin, M.~A. Janssen, R.~{Meinzen-Dick}, S.~Kandikuppa, K.~R.
  Rao, R.~Chaturvedi, Assessing the institutional foundations of adaptive water
  governance in {{South India}}, Ecology and Society 27~(1) (2022) art18.
\newblock \href {http://dx.doi.org/10.5751/ES-12957-270118}
  {\path{doi:10.5751/ES-12957-270118}}.

\bibitem{loch2020}
A.~Loch, D.~Adamson, N.~P. Dumbrell, The {{Fifth Stage}} in {{Water
  Management}}: {{Policy Lessons}} for {{Water Governance}}, Water Resources
  Research 56~(5) (2020) e2019WR026714.
\newblock \href {http://dx.doi.org/10.1029/2019WR026714}
  {\path{doi:10.1029/2019WR026714}}.

\bibitem{kirchhoff2016}
C.~J. Kirchhoff, L.~Dilling, The role of {{U}}.{{S}}. states in facilitating
  effective water governance under stress and change, Water Resources Research
  52~(4) (2016) 2951--2964.
\newblock \href {http://dx.doi.org/10.1002/2015WR018431}
  {\path{doi:10.1002/2015WR018431}}.

\bibitem{ostrom2009}
E.~Ostrom, A {{General Framework}} for {{Analyzing Sustainability}} of
  {{Social-Ecological Systems}}, Science 325~(5939) (2009) 419--422.
\newblock \href {http://dx.doi.org/10.1126/science.1172133}
  {\path{doi:10.1126/science.1172133}}.

\bibitem{wohlfart2016}
C.~Wohlfart, C.~Kuenzer, C.~Chen, G.~Liu, Social-ecological challenges in the
  {{Yellow River}} basin ({{China}}): A review, Environmental Earth Sciences
  75~(13) (2016) 1066.
\newblock \href {http://dx.doi.org/10.1007/s12665-016-5864-2}
  {\path{doi:10.1007/s12665-016-5864-2}}.

\bibitem{long2020}
D.~Long, W.~Yang, B.~R. Scanlon, J.~Zhao, D.~Liu, P.~Burek, Y.~Pan, L.~You,
  Y.~Wada, South-to-{{North Water Diversion}} stabilizing {{Beijing}}'s
  groundwater levels, Nature Communications 11~(1) (2020) 3665.
\newblock \href {http://dx.doi.org/10.1038/s41467-020-17428-6}
  {\path{doi:10.1038/s41467-020-17428-6}}.

\bibitem{speed2013}
R.~Speed, {Asian Development Bank}, Basin Water Allocation Planning:
  Principles, Procedures, and Approaches for Basin Allocation Planning, {Asian
  Development Bank, GIWP, UNESCO, and WWF-UK}, {Metro Manila, Philippines},
  2013.

\bibitem{arkhangelsky2021}
D.~Arkhangelsky, S.~Athey, D.~A. Hirshberg, G.~W. Imbens, S.~Wager, Synthetic
  {{Difference-in-Differences}}, American Economic Review 111~(12) (2021)
  4088--4118.
\newblock \href {http://dx.doi.org/10.1257/aer.20190159}
  {\path{doi:10.1257/aer.20190159}}.

\bibitem{wang2022e}
Y.~Wang, S.~Wang, W.~Zhao, Y.~Liu, The increasing contribution of potential
  evapotranspiration to severe droughts in the {{Yellow River}} basin, Journal
  of Hydrology 605 (2022) 127310.
\newblock \href {http://dx.doi.org/10.1016/j.jhydrol.2021.127310}
  {\path{doi:10.1016/j.jhydrol.2021.127310}}.

\bibitem{wang2019e}
Z.~Wang, Z.~Zheng, {Things and Current Significance of the Yellow River Water
  Allocation Scheme in 1987}, Yellow River 41~(10) (2019) 109--127.
\newblock \href {http://dx.doi.org/10.3969/j.issn.1000-1379.2019.10.019}
  {\path{doi:10.3969/j.issn.1000-1379.2019.10.019}}.

\bibitem{bodin2017a}
{\"O}.~Bodin, B.~I. Crona, Social {{Networks}}: {{Uncovering
  Social}}\textendash{{Ecological}} ({{Mis}})matches in {{Heterogeneous Marine
  Landscapes}}, in: S.~E. Gergel, M.~G. Turner (Eds.), Learning {{Landscape
  Ecology}}: {{A Practical Guide}} to {{Concepts}} and {{Techniques}},
  {Springer}, {New York, NY}, 2017, pp. 325--340.

\bibitem{kluger2020}
L.~C. Kluger, P.~Gorris, S.~Kochalski, M.~S. Mueller, G.~Romagnoni, Studying
  human\textendash nature relationships through a network lens: {{A}}
  systematic review, People and Nature 2~(4) (2020) 1100--1116.
\newblock \href {http://dx.doi.org/10.1002/pan3.10136}
  {\path{doi:10.1002/pan3.10136}}.

\bibitem{guerrero2015}
A.~Guerrero, {\"O}.~Bodin, R.~McAllister, K.~Wilson, Achieving
  social-ecological fit through bottom-up collaborative governance: An
  empirical investigation, Ecology and Society 20~(4).
\newblock \href {http://dx.doi.org/10.5751/ES-08035-200441}
  {\path{doi:10.5751/ES-08035-200441}}.

\bibitem{bayani2021}
M.~Bayani, Robust {{PCA Synthetic Control}}, {{SSRN Scholarly Paper}} 3920293,
  {Social Science Research Network}, {Rochester, NY} (Sep. 2021).

\bibitem{abadie2010}
A.~Abadie, A.~Diamond, J.~Hainmueller, Synthetic {{Control Methods}} for
  {{Comparative Case Studies}}: {{Estimating}} the {{Effect}} of
  {{California}}'s {{Tobacco Control Program}}, Journal of the American
  Statistical Association 105~(490) (2010) 493--505.
\newblock \href {http://dx.doi.org/10.1198/jasa.2009.ap08746}
  {\path{doi:10.1198/jasa.2009.ap08746}}.

\bibitem{abadie2015}
A.~Abadie, A.~Diamond, J.~Hainmueller, Comparative {{Politics}} and the
  {{Synthetic Control Method}}: {{Comparative Politics}} and the {{Synthetic
  Control Method}}, American Journal of Political Science 59~(2) (2015)
  495--510.
\newblock \href {http://dx.doi.org/10.1111/ajps.12116}
  {\path{doi:10.1111/ajps.12116}}.

\bibitem{hill2021}
A.~D. Hill, S.~G. Johnson, L.~M. Greco, E.~H. O'Boyle, S.~L. Walter,
  Endogeneity: {{A Review}} and {{Agenda}} for the {{Methodology-Practice
  Divide Affecting Micro}} and {{Macro Research}}, Journal of Management 47~(1)
  (2021) 105--143.
\newblock \href {http://dx.doi.org/10.1177/0149206320960533}
  {\path{doi:10.1177/0149206320960533}}.

\bibitem{abadie2021}
A.~Abadie, Using {{Synthetic Controls}}: {{Feasibility}}, {{Data
  Requirements}}, and {{Methodological Aspects}}, Journal of Economic
  Literature 59~(2) (2021) 391--425.
\newblock \href {http://dx.doi.org/10.1257/jel.20191450}
  {\path{doi:10.1257/jel.20191450}}.

\bibitem{billmeier2013}
A.~Billmeier, T.~Nannicini, Assessing {{Economic Liberalization Episodes}}: {{A
  Synthetic Control Approach}}, The Review of Economics and Statistics 95~(3)
  (2013) 983--1001.
\newblock \href {http://dx.doi.org/10.1162/REST_a_00324}
  {\path{doi:10.1162/REST_a_00324}}.

\bibitem{smith2015}
B.~Smith, The resource curse exorcised: {{Evidence}} from a panel of countries,
  Journal of Development Economics 116~(C) (2015) 57--73.
\newblock \href {http://dx.doi.org/10.1016/j.jdeveco.2015.04.001}
  {\path{doi:10.1016/j.jdeveco.2015.04.001}}.

\DIFaddbegin \bibitem{engelbrektson2023}
\DIFadd{O.~Engelbrektson,
  }\href{https://github.com/OscarEngelbrektson/SyntheticControlMethods}{\DIFadd{Synthetic
  }{\DIFadd{Control}} {\DIFadd{Methods}}\DIFadd{: }{\DIFadd{A}} {\DIFadd{Python}} \DIFadd{package for causal inference using
  synthetic controls}} \DIFadd{(Feb. 2023).
}\newline\urlprefix\url{https://github.com/OscarEngelbrektson/SyntheticControlMethods}

\DIFaddend \bibitem{cumming2020a}
G.~S. Cumming, G.~Epstein, Landscape sustainability and the landscape ecology
  of institutions, Landscape Ecology 35~(11) (2020) 2613--2628.
\newblock \href {http://dx.doi.org/10.1007/s10980-020-00989-8}
  {\path{doi:10.1007/s10980-020-00989-8}}.

\DIFaddbegin \bibitem{departmentofearthsciences1999}
{\DIFadd{Department of Earth Sciences}}\DIFadd{, Countermeasures and suggestions on alleviating
  }{\DIFadd{Yellow}} {\DIFadd{River}} \DIFadd{drying up, Advance in Earth Sciences~(1) (1999) 3--5.
}

\DIFaddend \bibitem{chen2021}
C.~Chen, G.~{Jia-jia}, S.~{Da-jun}, {Water resources allocation and
  re-allocation of the Yellow River Basin}, Resources Science 43~(04) (2021)
  799--812.

\bibitem{huangang2002}
W.~Y.-h. {Hu An-gang}, {Institutional failure is an important reason for the
  depletion of the Yellow River}, Review of Economic Research~(63) (2002) 31.
\newblock \href {http://dx.doi.org/10.16110/j.cnki.issn2095-3151.2002.63.035}
  {\path{doi:10.16110/j.cnki.issn2095-3151.2002.63.035}}.

\bibitem{an2007}
A.~{Xin-dai}, S.~Qing, C.~{Yong-qi}, {Prospect of water right system
  establishment in Yellow River Basin}, CHINA WATER RESOURCES~(19) (2007)
  66--69.

\bibitem{kellenberg2009}
D.~K. Kellenberg, An empirical investigation of the pollution haven effect with
  strategic environment and trade policy, Journal of International Economics
  78~(2) (2009) 242--255.
\newblock \href {http://dx.doi.org/10.1016/j.jinteco.2009.04.004}
  {\path{doi:10.1016/j.jinteco.2009.04.004}}.

\bibitem{cai2016}
H.~Cai, Y.~Chen, Q.~Gong, Polluting thy neighbor: {{Unintended}} consequences
  of {{China}}'s pollution reduction mandates, Journal of Environmental
  Economics and Management 76 (2016) 86--104.
\newblock \href {http://dx.doi.org/10.1016/j.jeem.2015.01.002}
  {\path{doi:10.1016/j.jeem.2015.01.002}}.

\bibitem{barnes2019}
M.~L. Barnes, {\"O}.~Bodin, T.~R. McClanahan, J.~N. Kittinger, A.~S. Hoey,
  O.~G. Gaoue, N.~A.~J. Graham, Social-ecological alignment and ecological
  conditions in coral reefs, Nature Communications 10~(1) (2019) 2039.
\newblock \href {http://dx.doi.org/10.1038/s41467-019-09994-1}
  {\path{doi:10.1038/s41467-019-09994-1}}.

\DIFaddbegin \bibitem{wang2019c}
\DIFadd{S.~Wang, B.~Fu, O.~Bodin, J.~Liu, M.~Zhang, X.~Li, Alignment of social and
  ecological structures increased the ability of river management, Science
  Bulletin 64~(18) (2019) 1318--1324.
}\newblock \href {http://dx.doi.org/10.1016/j.scib.2019.07.016}
  {\path{doi:10.1016/j.scib.2019.07.016}}\DIFadd{.
}

\DIFaddend \bibitem{mao2000}
M.~{Shou-long}, {Institutional analysis under the depletion of the Yellow
  River}, Chinese \& Foreign Corporate Culture~(20) (2000) 58--61.

\DIFdelbegin \bibitem{zuo2020}
\DIFdel{Z.~}%DIFDELCMD < {%%%
\DIFdel{Qi-ting}%DIFDELCMD < }%%%
\DIFdel{, W.~}%DIFDELCMD < {%%%
\DIFdel{Bin-bin}%DIFDELCMD < }%%%
\DIFdel{, Z.~Wei, M.~}%DIFDELCMD < {%%%
\DIFdel{Jun-xia}%DIFDELCMD < }%%%
\DIFdel{, }%DIFDELCMD < {%%%
\DIFdel{A method of water
  distribution in transboundary rivers and the new calculation scheme of the
  Yellow River water distribution}%DIFDELCMD < }%%%
\DIFdel{, Resources Science 42~(01) (2020) 37--45.
}%DIFDELCMD < \newblock %%%
\href {http://dx.doi.org/10.18402/resci.2020.01.04}
  {%DIFDELCMD < \path{doi:10.18402/resci.2020.01.04}%%%
}%DIFAUXCMD
\DIFdel{.
}%DIFDELCMD <

%DIFDELCMD < %%%
\DIFdelend \bibitem{krieger1955}
J.~H. Krieger, Progress in {{Ground Water Replenishment}} in {{Southern
  California}}, Journal (American Water Works Association) 47~(9) (1955)
  909--913.
\newblock \href {http://arxiv.org/abs/41254171} {\path{arXiv:41254171}}, \href
  {http://dx.doi.org/10.1002/j.1551-8833.1955.tb19237.x}
  {\path{doi:10.1002/j.1551-8833.1955.tb19237.x}}.

\bibitem{ostrom1990}
E.~Ostrom, Governing the {{Commons}}: {{The Evolution}} of {{Institutions}} for
  {{Collective Action}}, Political {{Economy}} of {{Institutions}} and
  {{Decisions}}, {Cambridge University Press}, {Cambridge}, 1990.
\newblock \href {http://dx.doi.org/10.1017/CBO9780511807763}
  {\path{doi:10.1017/CBO9780511807763}}.

\bibitem{sun2022b}
M.~Sun, F.~Zhang, F.~Duarte, C.~Ratti, Understanding architecture age and style
  through deep learning, Cities 128 (2022) 103787.
\newblock \href {http://dx.doi.org/10.1016/j.cities.2022.103787}
  {\path{doi:10.1016/j.cities.2022.103787}}.

\bibitem{bodin2012}
{\"O}.~Bodin, M.~Teng{\"o}, Disentangling intangible social\textendash
  ecological systems, Global Environmental Change 22~(2) (2012) 430--439.
\newblock \href {http://dx.doi.org/10.1016/j.gloenvcha.2012.01.005}
  {\path{doi:10.1016/j.gloenvcha.2012.01.005}}.

\bibitem{sayles2017}
J.~S. Sayles, J.~A. Baggio, Social\textendash ecological network analysis of
  scale mismatches in estuary watershed restoration, Proceedings of the
  National Academy of Sciences 114~(10) (2017) E1776--E1785.
\newblock \href {http://dx.doi.org/10.1073/pnas.1604405114}
  {\path{doi:10.1073/pnas.1604405114}}.

\bibitem{sayles2019}
J.~S. Sayles, Social-ecological network analysis for sustainability sciences: A
  systematic review and innovative research agenda for the future, Environ.
  Res. Lett. (2019) 19\href {http://dx.doi.org/10.1088/1748-9326/ab2619}
  {\path{doi:10.1088/1748-9326/ab2619}}.

\bibitem{bergsten2019}
A.~Bergsten, T.~S. Jiren, J.~Leventon, I.~Dorresteijn, J.~Schultner,
  J.~Fischer, Identifying governance gaps among interlinked sustainability
  challenges, Environmental Science \& Policy 91 (2019) 27--38.
\newblock \href {http://dx.doi.org/10.1016/j.envsci.2018.10.007}
  {\path{doi:10.1016/j.envsci.2018.10.007}}.

\bibitem{hegwood2022}
M.~Hegwood, R.~E. Langendorf, M.~G. Burgess, Why win\textendash wins are rare
  in complex environmental management, Nature Sustainability (2022) 1--7\href
  {http://dx.doi.org/10.1038/s41893-022-00866-z}
  {\path{doi:10.1038/s41893-022-00866-z}}.

\bibitem{reyers2018}
B.~Reyers, C.~Folke, M.-L. Moore, R.~Biggs, V.~Galaz, Social-{{Ecological
  Systems Insights}} for {{Navigating}} the {{Dynamics}} of the
  {{Anthropocene}}, Annual Review of Environment and Resources 43~(1) (2018)
  267--289.
\newblock \href {http://dx.doi.org/10.1146/annurev-environ-110615-085349}
  {\path{doi:10.1146/annurev-environ-110615-085349}}.

\bibitem{muneepeerakul2017}
R.~Muneepeerakul, J.~M. Anderies, Strategic behaviors and governance challenges
  in social-ecological systems, Earth's Future 5~(8) (2017) 865--876.
\newblock \href {http://dx.doi.org/10.1002/2017EF000562}
  {\path{doi:10.1002/2017EF000562}}.

\bibitem{leslie2015}
H.~M. Leslie, X.~Basurto, M.~Nenadovic, L.~Sievanen, K.~C. Cavanaugh, J.~J.
  {Cota-Nieto}, B.~E. Erisman, E.~Finkbeiner, G.~{Hinojosa-Arango},
  M.~{Moreno-B{\'a}ez}, S.~Nagavarapu, S.~M.~W. Reddy,
  A.~{S{\'a}nchez-Rodr{\'i}guez}, K.~Siegel, J.~J. {Ulibarria-Valenzuela},
  A.~H. Weaver, O.~{Aburto-Oropeza}, Operationalizing the social-ecological
  systems framework to assess sustainability, Proceedings of the National
  Academy of Sciences 112~(19) (2015) 5979--5984.
\newblock \href {http://dx.doi.org/10.1073/pnas.1414640112}
  {\path{doi:10.1073/pnas.1414640112}}.

\end{thebibliography}
\bibliography{../mybib}

\bibliographystyle{elsarticle-num}\label{bib}

%DIF > %%%%%%% -----  02_appendix -------- %%%%%%%%%%
\newpage
\appendix\label{appendix}

\section{Key points in the documents of 87-WAS and 98-UBR}\label{secS1}
\renewcommand{\thefigure}{A\arabic{figure}}
\renewcommand{\thetable}{A\arabic{table}}
\setcounter{figure}{0}
\setcounter{table}{0}
%DIF > ! Author = songshgeo
%DIF > ! Date = 2022/3/10

%DIF >  \begin{figure}[!bh]
%DIF >  	\centering
%DIF >  	\includegraphics[width=0.9\linewidth]{diagrams/framework.jpg}
%DIF >  	\caption{
%DIF >  		Framework for understanding linkages between SES structures and outcomes.
%DIF >  		\textbf{a.} The general framework for analyzing social-ecological systems (SESs) (adapted from Ostrom \cite{ostrom2009}). Institutions embedded in SESs may reshape structures by changing the interactions between core subsystems, resulting in different outcomes.
%DIF >          Three typical types of abstracted SES structures are shown as \textbf{b.}, \textbf{c.} and \textbf{d.} (adapted from Bodin, 2017)\cite{bodin2017b}. Red circles indicate social actors, and green ones indicate ecological components. Connection (ties between two ecological components), collaboration (ties between two social actors), or management (ties between a social actor and an ecological component) exist when gray lines link two units. According to empirical evidence, the gray dashed lines show aligned SES structures that are more likely to achieve a desirable outcome.
%DIF >          }
%DIF >      \label{framework}
%DIF >  \end{figure}
\DIFaddbegin

%DIF >  水资源分配方案在全世界范围内都是流域管理的普遍制度。
%DIF >  Water allocation institutions are widespread in large river basin management programs throughout the world (see \textit{Appendix} Figure~\ref{fig:world})~\cite{speed2013}.
%DIF >  This was the first basin in China for which a water resource allocation institution was created, and institutional shifts can be traced through several documents released by the Chinese government (at the national level)\cite{wang2019e}:
%DIF >  \begin{itemize}
%DIF >      \item \textbf{1982}: The provinces and the Yellow River Water Conservancy Commission (YRCC) are required to develop a water resource plan for the Yellow River \cite{wang2019d, wang2019e}.
%DIF >      \item \textbf{1987}: Implementation of the Allocation Plan. (\href{http://www.gov.cn/zhengce/content/2011-03/30/content_3138.htm#}{http://www.mwr.gov.cn}, last access: \today).
%DIF >      \item \textbf{1998}: Implementation of unified regulation. (\href{http://www.mwr.gov.cn/ztpd/2013ztbd/2013fxkh/fxkhswcbcs/cs/flfg/201304/t20130411_433489.html}{http://www.mwr.gov.cn}, last access: \today).
%DIF >      % 各省按要求编制新的黄河流域水资源规划,将水资源额度分配进一步细化。
%DIF >      \item \textbf{2008}: Provinces are asked to draw up new water resources plans for the YRB to further refine water allocations \cite{wang2019d,wang2019e}.
%DIF >      \item \textbf{2021}: A call for redesigning the water allocation institution (\href{http://www.ccgp.gov.cn/cggg/zygg/gkzb/202107/t20210721_16591901.htm}{http://www.ccgp.gov.cn}, last access: \today).
%DIF >  \end{itemize}

%DIF >  在上述文件中,1982年的文件标志着设计分水制度尝试的开始,2008年标志着该制度走向成熟(完全建立起流域-省-市区的多级水资源分配和统一调度)。

\DIFaddend The official documents in 1987 (\href{http://www.gov.cn/zhengce/content/2011-03/30/content_3138.htm#}{http://www.mwr.gov.cn}, last access: \today) convey the following key points:

\begin{itemize}
	%DIF >  该政策面向的目标是各省(区域),黄委会没有被提及
	\item The policy is aimed at related provinces (or regions at the same administrative level).
	%DIF >  政策制定的首要考虑是解决断流问题
	\item Depletion of the river is identified as the first consideration of this institution.
	%DIF >  各省被鼓励在此配额下制定自己的用水计划
	\item Provinces are encouraged to develop their water use plans based on a quota system.
	%DIF >  水资源供给无法满足需求对相关省(地区)是普遍现象。
	\item Water in short supply is a common phenomenon in relevant provinces (regions).
\end{itemize}

The official documents in 1998
(\href{http://www.mwr.gov.cn/ztpd/2013ztbd/2013fxkh/fxkhswcbcs/cs/flfg/201304/t20130411_433489.html}{http://www.mwr.gov.cn}, last access: \today) convey the following key points:

\begin{itemize}
	%DIF >  除了说明政策针对的各省区之外,明确指出其用水需要黄河水利委员会进行申报,并由其组织和监管
	\item The document points out that not only provinces and autonomous regions involved in water resources management (see \textit{Article 3}), the provinces’ and regions’ water use shall be declared, organized, and supervised by the YRCC (\textit{Article 11 and Chapter III to Chapter V, and Chapter VII}).
	%DIF >  本研究(\textit{ Article 1})首先考虑的是上、中、下游用水的总体规划。
	\item Creating the overall plan of water use in the upper, middle, and lower reaches is identified as the first consideration of this institution (\textit{Article 1}).
	%DIF >  各省需要
	\item With the same quota as used in the 1987 policy, provinces were encouraged to further distribute their quota into lower-level administrations (see \textit{Article 6 and Article 41}).
	%DIF >  强调以总量确定供给,以供给决定需求。
	\item They emphasize that supply is determined by total quantity, and water use should not exceed the quota proposed in 1987 (see \textit{Article 2}).
\end{itemize}

%DIF >  \begin{figure*}[!htb]
%DIF >      \centering
%DIF >      \includegraphics[width=12cm]{diagrams/world_institutions.pdf}
%DIF >  	\caption{
%DIF >  		Overview of water allocation institutions.
%DIF >  		% 世界已有水资源分配制度的大河流域,其中黄河流域最早于1987年提出资源分配方案,后于1998年更改为统一调度方案。
%DIF >  		\textbf{A.} Major river basins in the world with water resource allocation systems (shaded red); the YRB first proposed a resource allocation scheme in 1987 (designed since 1983) and then changed to a unified regulation scheme in 1998 (designed in 1997 but implemented in 1998) \cite{speed2013}.
%DIF >  		% 不同的水资源分配制度设计模式,中国黄河流域是典型的自上而下。
%DIF >  		\textbf{B.} Different water resource allocation system design patterns; the YRB is typical of a top-down system.
%DIF >  		% 流域分水制度的演化。这种多层次的制度设计有其历史变化过程。
%DIF >  		\textbf{C.} The four periods of institutional evolution of water allocation of the YRB.
%DIF >  	}\label{fig:world}
%DIF >  \end{figure*}
\DIFaddbegin

%DIF >  基于上述分析,我们抽象出了两次制度转变之后的SES结构变化如正文的图1C所示。
%DIF >  Based on the above documents, we abstracted the structural changes of SES after the two institutional changes, as shown in the main text \textbf{Figure 2C}.


\DIFaddend \newpage
\section{Data source and method details}\label{secS2}
\renewcommand{\thefigure}{B\arabic{figure}}
\renewcommand{\thetable}{B\arabic{table}}
\setcounter{figure}{0}
\setcounter{table}{0}
%DIF > ! Author = songshgeo
%DIF > ! Date = 2022/3/19


%DIF >  % 找到具解释力的变量是构造合成控制法稳健的关键。
%DIF >  Explanatory variables are the key to constructing a robust synthetic control method.
%DIF >  % 我们共使用了用水量密切相关的26个变量,这些变量的数据集已在先前的研究中被用来解释中国的用水量变化
%DIF >  We used a total of $24$ variables related to water consumption Table~\ref{tab:variables}, which datasets have been used in previous studies to explain changes in water use in China \cite{zhou2020}.
%DIF >  % 由于这些变量间存在自相关,我们通过肘部法供选择了5个主成分作为DSC的输入,前人研究表明PCA方法的结合能够增强合成控制法的稳健性
\DIFaddbegin

%DIF >  In addition, we selected $5$ principal components as input by the elbow method because selection in autocorrelated variables reduces dimensions and then enhances the robustness of the DSC (Figure~\ref{fig:elbow}).

%DIF >  There are two approaches to validity testing of the DSC: (1) comparing the post-treated and pre-treated reconstructions and (2) testing robustness through placebo analysis.
%DIF >  For (1), differences between each province and their synthetic are significant in post-treated periods and small in pre-treated periods (Figure~\ref{fig:87panel} and figure~\ref{fig:98panel}), which show good reconstructions of their water use changes' estimation.
%DIF >  For (2), we applied the in-place placebo analysis described by \cite{abadie2010}. In most provinces, ratios of post-MSPE to pre-MSPE are higher than the median of other placebo units, which suggests the institutional shifts in treated time (1987 and 1998 here) influenced them more than most of the other provinces (figure~\ref{fig:87placebo}, figure~\ref{fig:98placebo}, Table~\ref{tab:DSC_summary}).

%DIF >  \begin{figure*}[!bh]
%DIF >      \includegraphics[width=0.9\linewidth]{outputs/87panel.pdf}
%DIF >      \centering
%DIF >      \caption{Comparations between YRB' provinces and their synthetic controls around the 87-WAS.}
%DIF >      \label{fig:87panel}
%DIF >  \end{figure*}

%DIF >  \begin{figure*}
%DIF >      \includegraphics[width=0.9\linewidth]{outputs/98panel.pdf}
%DIF >      \centering
%DIF >      \caption{Comparations between YRB' provinces and their synthetic controls around the 98-UBR.}
%DIF >      \label{fig:98panel}
%DIF >  \end{figure*}


%DIF >  \begin{figure*}
%DIF >      \includegraphics[width=0.9\linewidth]{outputs/87placebo.pdf}
%DIF >      \centering
%DIF >      \caption{Gaps in change in water use between provinces outside the YRB and their synthetic control, around the 87-WAS, excluding the provinces with high pre-treatment RMSPE (more than $3$ times of treated units' RMSPE).}
%DIF >      \label{fig:87placebo}
%DIF >  \end{figure*}

%DIF >  \begin{figure*}
%DIF >      \includegraphics[width=0.9\linewidth]{outputs/98placebo.pdf}
%DIF >      \centering
%DIF >      \caption{Gaps in change in water use between provinces outside the YRB and their synthetic control, around the 98-UBR, excluding the provinces with high pre-treatment RMSPE (more than $3$ times of treated units' RMSPE)}
%DIF >      \label{fig:98placebo}
%DIF >  \end{figure*}


\DIFaddend \begin{table*}[!ht]
	\caption{Variables and their categories for water use predictions}
	\scriptsize
	\label{tab:variables}
	\resizebox{\linewidth}{!}{
	\begin{tabular}{lllll}
	\hline
	Sector &
	  Category &
	  Unit &
	  Description &
	  Variables \\ \hline
	Agriculture &
	  Irrigation Area &
	  thousand ha &
	  \begin{tabular}[c]{@{}l@{}}Area equipped for irrgiation by different \\ crop:\end{tabular} &
	  \begin{tabular}[c]{@{}l@{}}Rice, \\ Wheat, \\ Maize, \\ Fruits, \\ Others.\end{tabular} \\ \hline
	Industry &
	  \begin{tabular}[c]{@{}l@{}}Industrial gross \\ value added\end{tabular} &
	  Billion Yuan &
	  Industrial GVA by industries &
	  \begin{tabular}[c]{@{}l@{}}Textile, \\ Papermaking, \\ Petrochemicals, \\ Metallurgy, \\ Mining, \\ Food, \\ Cements, \\ Machinery, \\ Electronics, \\ Thermal electrivity, \\ Others.\end{tabular} \\
	 &
	  \begin{tabular}[c]{@{}l@{}}Industrial water \\ use efficiency\end{tabular} &
	  \% &
	  \begin{tabular}[c]{@{}l@{}}The ratio of recycled water and evaporated \\ water to total industrial water use\end{tabular} &
	  \begin{tabular}[c]{@{}l@{}}Ratio of industrial water recycling, \\ Ratio of industrial water evaporated.\end{tabular} \\ \hline
	Services &
	  \begin{tabular}[c]{@{}l@{}}Services gross \\ value added\end{tabular} &
	  Billion Yuan &
	  GVA of service activities &
	  Services GVA \\ \hline
	Domestic &
	  Urban population &
	  Million Capita &
	  Population living in urban regions. &
	  Urban pop \\
	 &
	  Rural population &
	  Million Capita &
	  Population living in rural regions. &
	  Rural pop \\
	 &
	  Livestock population &
	  Billion KJ &
	  \begin{tabular}[c]{@{}l@{}}Livestock commodity calories summed from \\ 7 types of animal.\end{tabular} &
	  Livestock \\ \hline
	Environment &
		  Temperature & $K$ & Near surface air temperature & Temperature \\
			& Precipitation & $mm$ & Annual accumulated precipitation & Precipitation \\ \hline
	\end{tabular}}
\end{table*}


\begin{figure*}[!h]
    \includegraphics[width=0.9\linewidth]{outputs/elbow.pdf}
    \centering
    \caption{Choose number of pricipal components by Elbow method, $5$ pricipal components already capture $89.63\%$ explained variance.}\label{fig:elbow}
\end{figure*}


%DIF >  \section{S2: Methods in details}\label{secS3}
%DIF >  \graphicspath{{../../../figs/}}

\begin{figure*}
    \includegraphics[width=0.7\linewidth]{outputs/economy.pdf}
    \centering
    \caption{test}
    \label{S3-1}
\end{figure*}


\begin{figure*}
    \includegraphics[width=0.7\linewidth]{outputs/S3_WUI.pdf}
    \centering
    \caption{test}
    \label{S3-2}
\end{figure*}


\begin{figure*}
    \includegraphics[width=0.7\linewidth]{outputs/S3_wci.pdf}
    \centering
    \caption{test}
    \label{S3-3}
\end{figure*}


\DIFaddbegin

\DIFaddend \newpage
\section{Marginal benefit model for water use}\label{secS4}
\renewcommand{\thefigure}{C\arabic{figure}}
\renewcommand{\thetable}{C\arabic{table}}
\setcounter{figure}{0}
\setcounter{table}{0}
%DIF > ! Author = songshgeo
%DIF > ! Date = 2022/3/19


%DIF >  \subsection{Structure-based marginal benefit analysis}
%DIF >  \label{result-4}
\DIFaddbegin

\DIFaddend For interpretation of the pattern of provincial water uses, we compared the theoretical marginal returns and optimal water use under three different structural cases (case 1 to case 3, corresponding to different SES structures in Figure~\ref{fig:structure}~C).

Assuming that water is the factor input with decreasing marginal output of each province, results show that varying incentives for water use in each province derive from the relationship between the benefits and costs of water use.
As a benchmark, case 1 analogy to a decentralized stakeholders situation and lead to medium-level water use.
In case 2, each stakeholder expects that current water use helps bargain for a favorable water quota in the face of institutional shift (see \textit{\nameref{secS4}}), which can intensify the incentive to use water, leading to higher water use.
Furthermore, the water users with higher capability are more stimulated by the institutional shift and away from the theoretically optimal water use under a unified allocation.
After water-use decisions are consolidated into unified management (case 3), marginal benefits analysis suggests the lowest water use among the cases.


\begin{figure}[!htb]
	\centering
	\includegraphics[width=0.6\linewidth]{outputs/economic_model.pdf}
	\caption{
		The proposed relationship of marginal benefits and water use of individual province under varying cases (case 1 to case 3, corresponding to the different SES structures in Figure~\ref{fig:structure}~C) Major water users' theoretically optimal water use is also larger (see the proofs below.)}
\end{figure}

Below are the detailed theoretical model derivation process, where we started from proposing three intuitive and general assumptions:

\begin{ass}
(Water-dependent production) Because of irreplaceability, water is assumed to be the only input of the production function with two types of production efficiency. The production function of a high-incentive province is $A_HF(x)$, and the production function of a low-incentive province is $A_LF(x)$ ($A_H>A_L$). F(x) is continuous, $F'(0)=\infty$, $ F'(\infty)=0$, $F'(x)>0$, and $F''(x)<0$. The production output is under perfect competition, with a constant unit price of $P$.
\end{ass}

\begin{ass}
 (Ecological cost allocation) Under the assumption that the ecology is a single entity for the whole basin involved in $N$ provinces, the cost of water use is equally assigned to each province under any water use. The unit cost of water is a constant $C$.
\end{ass}
\begin{ass}
(Multi-period settings) There are infinite periods with a constant discount factor $\beta$ lying in (0,1). There is no cross-period smoothing in water use.
\end{ass}

Under the above assumptions, we can demonstrate three cases consisting of local governments in a whole basin to simulate their water use decision-making and water use patterns.

 %DIF > case1
\begin{case} before 1987: This case corresponds to a situation without any high-level water allocation institution.

When each province independently decides on its water use, the optimal water use $x_i^*$ in province $i$ satisfies:

 $AF'(x)=\frac{C}{P}$,

 where $A_H$ and $A_L$ denote high-incentive and low-incentive provinces, respectively.

 When the decisions in different periods are independent, for $t$ = $0, 1, 2 \cdots$, then:

 $x_{it}^* = x_i^*$

 \end{case}

 %DIF > case2
 \begin{case} from 1987 to 1998: This case corresponds to an SES structure where fragmented stakeholders are linked to unified river reaches.

 The water quota is determined at $t$=0 and imposed in $t$=1,2, \ldots Under the subjective expectation of each province that current water use may influence the future water allocation determined by high-level authorities, the total quota is a constant denoted as Q, and the quota for province $i$ is determined in a proportional form:

 $Q_i=Q \cdot \frac{x_i}{x_i + \begin{matrix}\sum{x_{-i}} \end{matrix}}$.

Under a scenario with decentralized decision-making with a water quota, given other provinces' decisions on water use remain unchanged, the optimal water use of province $i$ at $t$=0 satisfies:

$AF'(x_{i,0})=\frac{C}{P \cdot N} - \frac{\beta}{1-\beta} \cdot A \cdot f(Q \cdot \frac{x_{i,0}}{\begin{matrix} x_{i,0} + \sum x_{-i,0} \end{matrix}}) \cdot Q \cdot \frac{\begin{matrix} \sum x_{-i,0} \end{matrix}}{(\begin{matrix} x_{i,0} + \sum x_{-i,0} \end{matrix})^2}$,

where $A_H$ denotes a high-incentive province and $A_L$ denotes a low-incentive province.

\end{case}

 %DIF > case3
\begin{case} after 1998: This case corresponds to the institution under which water use in a basin is centrally managed.

 When the $N$ provinces decide on water use as a unified whole (e.g., the central government completely decides and controls the water use in each province), the optimal water use $x_i^*$ of province $i$ satisfies:

$F'(x)=\frac{C}{P}$.

\end{case}

We propose Proposition 1 and Proposition 2:

Proposition 1: Compared with the decentralized institution, a institution with unified management decreases total water use.

The optimal water use under the three cases implies that mismatched institutions cause incentive distortions and lead to resource overuse.


Proposition 2: Water overuse is higher among provinces with high water use incentives than low- water use incentives under a mismatched institution.

The intuition for this proposition is straightforward in that all provinces would use up their allocated quota under a relatively small $Q$. As production efficiency increases, the marginal benefits of a unit quota increase, and the quota would provide higher future benefits for a pre-emptive water use strategy. Provinces with high production efficiency have higher optimal water use values under the decentralized decision. The divergence in water use would be exaggerated when the water quota is expected to be implemented with greater competition.

%DIF > Appendix的模型和proposition推导细节部分
%DIF > 放在Appendix
\DIFaddbegin

%DIF > case1

\DIFaddend When the N provinces decide on water uses as a unity, the marginal cost is C, equal to its fixed unit cost.
The water use of province $i$ aims to maximize $P\cdot A\cdot F(x)-C$.
Hence, $x_i^*$ satisfies $P \cdot A\cdot F'(x)=C$, i.e., $AF'(x)=\frac{C}{P}$, where A denotes $A_H$ for a high-incentive province and $A_L$ for a low-incentive province.

When each of the N provinces independently decides on its water use, the marginal cost of water use would be $\frac{C}{N}$ as a result of cost-sharing with others.
Hence, the optimal water use in province i at period t, denoted as $\hat x_i^*$, satisfies $P \cdot A \cdot F'(x_{it})=\frac{C}{N}$, i.e., $A \cdot F'(x)=\frac{C}{P \cdot N}$.
Since $F'$ is monotonically decreasing, $\hat x_{it}^*>x_i^*$.

When the water quota would constrain future water use, the dynamic optimization problem of province i is shown as follows. In $t=1,2,\cdots$, there would be no relevant cost when the quota is bound that each province takes ongoing costs of $\frac{P \cdot Q}{N}$ regardless of the allocation. Therefore, it is sufficient to consider only the total water quota is less than total water use in Case 2 since a ``too large'' quota doesn't make sense for ecological policies.

$max  \quad P \cdot A \cdot F(x_{i,0})-\frac{C \cdot \begin{matrix} \sum x_{i,0} + x_{-i,0} \end{matrix}}{N}+\beta P \cdot A \cdot F(x_{i,1})+\beta^2 P \cdot A \cdot F(x_{i,2})+...$

$=P \cdot A \cdot F(x_{i,0})-C \cdot \frac{x_{i,0} + \begin{matrix} \sum x_{-i,0} \end{matrix}}{N}+\frac{\beta}{1-\beta} P \cdot A \cdot F(Q \cdot \frac{x_{i,0}}{x_{i,0} + \begin{matrix} \sum x_{-i,0} \end{matrix}})$

First-order condition: $P \cdot A \cdot F'(x_{i,0})-\frac{C}{N}+\frac{\beta}{1-\beta}[P \cdot A \cdot f(Q \cdot \frac{x_{i,0}}{x_{i,0} + \begin{matrix} \sum x_{-i,0} \end{matrix}}) \cdot Q \cdot \frac{\begin{matrix} \sum x_{-i,0} \end{matrix}}{(x_{i,0}+\begin{matrix} \sum  x_{-i,0} \end{matrix})^2}]=0$

where $f(\cdot)$ is the differential function of $F(\cdot)$.

The optimal water use in province i at t=0 $\widetilde x_{i,0}^*$ satisfies $P \cdot A \cdot F'(x_{i,0})=\frac{C}{N}-\frac{\beta}{1-\beta} \cdot P \cdot A \cdot f(Q \cdot \frac{x_{i,0}}{x_{i,0} + \begin{matrix} \sum x_{-i,0} \end{matrix}}) \cdot Q \cdot \frac{\begin{matrix} \sum x_{-i,0} \end{matrix}}{(x_{i,0} + \begin{matrix} \sum x_{-i,0} \end{matrix})^2}$,
i.e.,
$A \cdot F'(x_{i,0})=\frac{C}{P \cdot N} - \frac{\beta}{1-\beta} \cdot A \cdot f(Q \cdot \frac{x_{i,0}}{x_{i,0} + \begin{matrix} \sum x_{-i,0} \end{matrix}}) \cdot Q \cdot \frac{\begin{matrix} \sum x_{-i,0} \end{matrix}}{(x_{i,0} + \begin{matrix} \sum x_{-i,0} \end{matrix})^2}$.

Since $F'>0$ and $F''<0$, $\widetilde x_i^*>\hat x_i^*>x_i^*$, taken others' water use $x_{-i,0}$ as given. Since the provincial water use decisions are exactly symmetric, total water use would increase when each province has higher incentives for current water use.

%DIF > Proposition 1
Proof of Proposition 1:

Because $F'>0$ and $F''(x)<0$ is monotonically decreasing, based on a comparison of costs and benefits for stakeholders (provinces) in the three cases,

$\widetilde x_i^*>\hat x_i^*>x_i^*$.

The result of $\hat x_i^*>x_i^*$ indicates that individual rationality would deviate from collective rationality under unclear property rights where a water user is fully responsible for the relevant costs. The result of $\hat x_i^*>x_i^*$

The difference between $ x_i^*$ and $\hat x_i^*$ stems from two parts: the effect of the marginal returns and the effect of the marginal costs. First, the ``shadow value'' provides additional marginal returns of water use in $t$ = 0, which increases the incentives of water overuse by encouraging bargaining for a larger quota. Second, the future cost of water use would be degraded from $\frac{P}{N}$ to an irrelevant cost.

%DIF > Proposition 2
Proof of Proposition 2:

Since $A_H>A_L$, $F'(x_H)<F'(x_L)$,
Eq.(xxx) %DIF > 此处引用:$AF'(x_{i,0})=\frac{C}{P \cdot N} - \frac{\beta}{1-\beta} \cdot A \cdot f(Q \cdot \frac{x_{i,0}}{\begin{matrix} x_{i,0} + \sum x_{-i,0} \end{matrix}}) \cdot Q \cdot \frac{\begin{matrix} \sum x_{-i,0} \end{matrix}}{(\begin{matrix} x_{i,0} + \sum x_{-i,0} \end{matrix})^2}$
implies a positive relation between $x_{i0}$ and A, when $\beta, P, C, Q$, and other provinces' water use are taken as given.

The difference between $\widetilde x_i^*$ and $\hat x_i^*$ (i.e., $\frac{\beta}{1-\beta} \cdot A \cdot f(Q \cdot \frac{x_{i,0}}{x_{i,0} + \begin{matrix} \sum x_{-i,0} \end{matrix}}) \cdot Q \cdot \frac{\begin{matrix} \sum x_{-i,0} \end{matrix}}{(x_{i,0} + \begin{matrix} \sum x_{-i,0} \end{matrix})^2}$) represents the incentive of water overuse derived from an expectation of water quota allocation. The incentive of water overuse increases by A.

\DIFaddbegin


%DIF >  \section{Model extensions}\label{secS5}
%DIF >  \renewcommand{\thefigure}{D\arabic{figure}}
%DIF >  \renewcommand{\thetable}{D\arabic{table}}
%DIF >  \setcounter{figure}{0}
%DIF >  \setcounter{table}{0}
%DIF >  %Model extension部分
Further analysis of the general economic model

Using the general economic model (see the Methods section in the main text), we also explored the response of stakeholders to water quota policies. We considered two additional scenarios for stakeholders, one that considered technology growth and one that considered different valuations through time (via the discount rate) of economic benefits and ecological costs. In the following scenarios, the cost is assumed to be untransferable, which could be fully allocated to the one incurring the water use. Explaining plausible scenarios for these stakeholders will help us better understand the causes of the water overuse and potential solutions. We argue that the water overuse remains robust even if a complete and equitable system.

% technology growth
    \begin{case_appendix}Forward-looking decentralized decision, taken ecology cost into considerations

    Even if the negative externality of water overuse is eliminated by``fair" ecology cost of $\frac{x_{i,0}}{x_{i,0} + \begin{matrix} \sum x_{-i,0} \end{matrix}} \cdot Q \cdot C$, it is possible that the future growth opportunities and ``remote" ecological costs provide enough incentive for the sprint.  The water overuse has the value of future economical benefits by slacking the water use constraint in the future. The heterogeneous production efficiency is omitted in this section, and we set A=1.

(a) technology growth

Assume that there is an exogenous technology growth rate of g in the scenario of $N$ provinces bargaining for water use under total quota $Q$, with unit price of output $P$, unit cost $C$, and discount factor $\beta$. For simplicity, consider a finite-period water use optimization:

$ max \quad P \cdot (1+g)^t ln(1+x_{i,0})-\frac{C}{N}+\beta^t \begin{matrix} \sum_{t=1}^T [P \cdot (1+g)^t ln(x_{i,t}+1)-C \cdot x_{i,t}] \end{matrix}$

$s.t. \quad x_{i,t} \leq Q \cdot \frac{x_{i,0}}{x_{i,0} + \begin{matrix} \sum x_{-i,0} \end{matrix}} \quad for \quad \forall t$

We depict the relationship of multi-period profit and water use $x_{i,0}$ in different horizons in Figure 4%%此处引用图片
,and thus find out the optimal water use pattern under technology growth. The higher marginal output of water might create enough incentive to set off the untransferable cost since a higher allocated quota provides growth option value. As the provincial decision is under a longer horizon, there is a greater sprint effect due to higher accumulated yield and relatively tighter water use constraints over time.

% \begin{figure}[H]
%     \centering
%     \includegraphics[width=0.7\linewidth]{Fig3.jpg}
% \end{figure}

% \begin{fig}
% Multi-period optimization of optimal water use under technology growth

% \tiny Notes: The figure depicts the relationship of multi-period benefits of province $i$ and water use under Case 3 with technology growth. Assume F(x)=ln(1+x), N=8, P=1, C=0.5, $\beta$=0.7, g=0.2, and Q=8.
% \end{fig}

(b) Economic benefits and ``remote'' ecological costs with different discount factors

Assume that there is a high discount rate for economical benefits and a low discount rate for ecological costs, in the scenario of $N$ provinces bargaining for water use under total quota $Q$, with unit price of output $P$, unit cost $C$, discount factor $\beta^{economy}$ and $\beta^{ecology}$ ($\beta^{economy} > \beta^{ecology}$). For simplicity, consider the following finite-period water use optimization, noting the water use of province $i$ at period $t$:

$ max \quad P \cdot ln(1+x_{i,0})-\frac{C}{N}+\beta_1^t \begin{matrix} \sum_{t=1}^T [P \cdot ln(x_{i,t}+1)]  \end{matrix} - \beta_2^t \begin{matrix} \sum_{t=1}^T [C \cdot x_{i,t}] \end{matrix}$

$s.t. \quad x_{i,t} \leq Q \cdot \frac{x_{i,0}}{x_{i,0} + \begin{matrix} \sum x_{-i,0} \end{matrix}} \quad for \quad \forall t$

We depict the relationship of multi-period net income and water use $x_{i,0}$ in different horizons in Figure 4%此处引用图片
, and thus find out the optimal water use pattern under ``remote'' ecological costs. The higher discounted ecological costs might create enough incentive to set off the untransferable cost. As the provincial decision is under a longer horizon, there is a greater sprint effect due to a higher accumulated yield.

% \begin{figure}[H]
%     \centering
%     \includegraphics[width=0.7\linewidth]{Fig4.jpg}
% \end{figure}

% \begin{fig}
% Multi-period optimization of water use under ``remote'' ecological cost

% \tiny Notes: The figure depicts the relationship of multi-period benefits of province $i$ and water use under Case 3 with ``remote'' ecological cost. Assume F(x)=ln(1+x), N=8, P=1, C=0.5, $\beta_{economy}$=0.7, $\beta_{ecology}$=0.3, and Q=8.
% \end{fig}

\end{case_appendix}


\DIFaddend

\end{document}
