%! Author = songshgeo
%! Date = 2022/3/10

% % 这里,我们使用合成控制模型创建了假设不存在政策变化的虚拟对照组(see \textit{Methods}-1),同时用K-近临算法找到经济状况与黄河流域各省“类似”的安慰剂对照(see \textit{Methods}-2),反复对比表明两次社会-生态系统结构变迁对流域的水资源利用及分配产生了明显不同的影响。
% Here, we use synthetic control method to create a control group assuming that there is no water use constraints (\textbf{methods}), at the same time with K-nearest algorithm to match an economic state and the YRB provinces ``similar'' placebo-controlled (\textbf{methods}).
% % 然后,我们反复比较了两个SES结构(分别自1987年和1998年),以量化是否有显著不同的影响,YRB用水或分配。
% Then we repeatedly compared between two SES structures (since 1987 and 1998, respectively) to quantify if there were significantly different impacts on water uses or allocation (\textbf{methods}).
% % 我们的结果表明,1987年的结构变化刺激各省使用了远超模型预测的没有政策影响下的用水量,其增量超出安慰剂对照组xx\%,但对水资源的分配的均匀程度没有改变(see fig)。
Our results suggest that the institutional shift in 1987 stimulated the provinces to use far more water than would have been used without policy effects (Figure~\ref{fig:main_results}A), with an observed increase of $164\%$ over the expectation(Figure~\ref{fig:main_results}B).
% 然而,水资源分配的均匀性并没有改变,区域间的用水量在相同规模下增加(图~\ref{fig:main_results}C)。
However, the relative share of water use was not changed, denoting proportionally similar water use increases among the different regions (Figure~\ref{fig:main_results}C).
% 在1998年出台的政策再次改变社会-生态结构之后,用水量显著下降,总量仅为安慰剂对照组的xx%。
After the SES structure changed again in 1998, the trend of increasing water use appeared to be effectively suppressed (Figure~\ref{fig:main_results} D), with total observed water consumption decreasing by $260\%$ relative to expectations(Figure~\ref{fig:main_results} E).
% 但此阶段主要是xx,xx,xx等黄河流域的用水大省受到了遏制,因此水资源在各个省之间的比例分配变得更均匀了。
At this stage, however, the reduction in water use came mainly from the provinces with large water consumption, such as Henan and Shandong (Figure~\ref{fig:main_results} E),  so the proportion of water used by regions became more similar (Figure~\ref{fig:main_results} F).
% 上述结果表明,1998年各省之间由黄委会统一监管后分水政策才真正达到了其遏制用水、解决断流的预期。
In conclusion, the water allocation policy curbed water use in 1998, whereas the 1987–1998 institutional mismatch stimulated a notable increase in total water use in all related provinces.
% 在过去的十年里,对制度变革的“冲刺”反应似乎创造了一场竞赛,每个省份都开始使用超过他们需要的水。
Over this decade, “sprint” responses (i.e., rapid increases in resource use \cite{lueckPreemptiveHabitatDestruction2003}) to institutional change appear to have created a race in which each province began to use more water than they needed.
