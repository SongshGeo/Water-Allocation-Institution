%! Author = songshgeo
%! Date = 2022/3/10
Theoretically, our economic model suggests that different kinds of institutional shifts should lead to different optimal water uses (Figure~\ref{fig:economic_model}).
Furthermore, our analysis indicated that the cause of the sprint effect in this case was incentive distortion.
Compared with the decentralized water allocation institution in place before 1987, the presence of central management (by the YRCC in this case, after 1998) can effectively reduce marginal ecological costs (see Table~\ref{tab:cases} the \textbf{methods} for a detailed mathematical formula).
% 在不匹配条件下,边际成本减去边际收益增加,导致用水量增加,xx的差异反映了水配额模拟的非预期冲刺效应。
The unintended sprint effect (from 1987 to 1998) was caused by both declining marginal costs (a shift from a fixed unit cost to an irrelevant cost) and increasing marginal returns due to future water use benefits (see Table~\ref{tab:cases} the \textbf{methods} for a detailed mathematical formula).
% 因此,这种制度引发了一种与可持续用水意图背道而驰的激励扭曲。
The institution thus triggered an incentive distortion that ran counter to the intention of sustainable water use. Further, the strength of the sprint effect was positively correlated with the size and time horizon of the water use quota (Figure~\ref{fig:economic_model} \textbf{Panel B}).
% 这些理论上的预测正如我们在黄河流域所观察到的一样,不匹配的配额制度导致了短跑效应的出现,而中心化的制度结构结束了这一非预期的现象。
% These theoretical predictions, in line with phenomenon observed in the YRB, indicating the mismatched institution leads to incentive distortion of provinces for pre-empting resources by `sprinting', according to their expectations for the future.
% 这些理论预测进一步从YRB观察到的现象出发,表明不匹配的配额制度通过提高边际效益来刺激用水量,而边际效益与放宽限制带来的未来潜在用水量的影子值有关。
% These theoretical predictions, further from the phenomenon observed in the YRB, indicating the mismatched institution with quota stimulates water use by raising marginal benefits, which is related to the shadow value of potential future water use from relaxing the constraints.
% 温博推荐的版本:
% These theoretical predictions, further from the phenomenon observed in the YRB, attributing ``sprint effect'' into raising marginal benefits -by pursuing indirectly generating shadow value of water use while converting fixed marginal costs to irrelevant cost for each individual province.