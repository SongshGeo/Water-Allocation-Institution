Emerging competition for water is an urgent problem in water governance, with widespread water scarcity and overuse resulting in huge impacts on economies, societies, and ecosystems \cite{gleickPeakwaterlimits2010,dolanEvaluatingeconomicimpact2021,siwiCompetitionWaterGetting,ziolkowskaCompetitionWaterResources2016,distefanoArewedeep2017}.
Water is the key exclusive and competitive resource that couples socioeconomic and ecological systems (known as common-pool resources, CPRs) \cite{wutichWaterScarcitySustainability2009,ostromGeneralFrameworkAnalyzing2009,castilla-rhoSocialtippingpoints2017,castilla-rhoGroundwaterCommonPool2020}.
Conflicts of interest often occur in allocation and competitions of water resources, with water governance policies often leading to long-term changes in human–water relationships and the redistribution of benefits \cite{wangAlignmentsocialecological2019,speedBasinwaterallocation2013,chingManagingsocioecologyvery2015}.
Although governments in many of the world’s large river basins have tried to resolve competition for water through the deliberate design of new institutions, general principles underlying the successes and failures of these initiatives are poorly documented and understood.

Institutions (such as policies, laws, and norms) can influence regional sustainability by changing social-ecological system (SES) structure and dynamics \cite{youngInstitutionsenvironmentalchange2008,cummingAdvancingunderstandingnatural2020,lieninstitutionalgrammartool2020}.
These include inter-relationships and interactions between social actors, between ecological units, or between social and ecological system elements \cite{bodinCollaborativeenvironmentalgovernance2017,barnesSocialecologicalalignmentecological2019}.
Effective (“matched”) institutions operate at appropriate spatial, temporal, and functional scales to manage and balance these different relationships and interactions \cite{heWateringEnvironmentalRegulation2020,kellenbergempiricalinvestigationpollution2009}.
From the perspective of SES outcomes, matched institutions support (but do not guarantee) sustainability \cite{epsteinInstitutionalfitsustainability2015,wangAlignmentsocialecological2019}.

Some kinds of institutions have been shown to support desirable outcomes in water-centered SESs (e.g., the Ecological Water Diversion Project in Heihe River Basin, China \cite{wangAlignmentsocialecological2019} and in collaborative water governance systems in Europe \cite{greenEUWaterGovernance2013}).
At the same time, undesired and unsustainable outcomes (e.g., failures in environmental regulation of highly polluting industries or the development of “tragedy of the commons” situations when pursuing more water resources), with considerable ecological degradation, have attracted much attention \cite{saylesSocialecologicalnetworkanalysis2019,caiPollutingthyneighbor2016,castilla-rhoSocialtippingpoints2017}.
Despite widespread recognition of the rising importance of integrated water resource management in solving water competition in the world’s large river basins, relatively few studies have explored the mechanisms by which SESs respond to new institutions \cite{agrawalSustainableGovernanceCommonPool2003,pershaSocialEcologicalSynergy2011,agrawalCommonPropertyInstitutions2001}.
Two particular weaknesses in existing knowledge include understanding (1) the causal links between SES structures and outcomes; and (2) details of the underlying processes, especially the coordination of the incentives of different participants, that result from an institutional lack of fit. These weaknesses limit our understanding of institutional design, and they may reduce the speed and transfer of new knowledge and experience related to improving the sustainability of comprehensive water resources management.

In order to disentangle the relationship between SES structure and outcomes, we analyzed a case study to show how an institutional shift led to a structural mismatch that triggered unsustainable water use and unintended ecological deterioration. Two rapid shifts in institutional structure that occurred in 1987 and 1998 (see Supplementary Material S1) provide unique settings to exploit quasi-natural experiments of a large river basin, the Yellow River Basin (YRB) in China \cite{xiaDevelopmentWaterAllocation2012}. After a period of severe drying up, an institutional shift implemented in 1987 represented the beginning of attempts to control water use in the YRB through the use of quotas, with the goals of alleviating conflicts between supply and demand and achieving sustainable development.

Our results show that this initiative actually accelerated water withdrawals, resulting in an unintended “sprint effect”, where institutional mismatches created an even stronger incentive for each resource user to withdraw resources until the next major institutional shift in 1998. Our analysis contributes to a deeper understanding of the mechanisms underlying the relationships among institutions, SES structures, and outcomes. By highlighting potential concerns for ecosystem collapse under structural mismatches, our findings are consistent with the urgent calls for a more dynamic design for water use allocation to achieve sustainability.

\section{Institutions and SES structures}
% 机构可以塑造社会经济体系的结构,对其进行抽象是机械地理解结构和结果之间联系的第一步。
Because institutions may shape the structure of SESs, describing institutional structure is a first step toward understanding the mechanisms linking structures and outcomes in SESs (Figure~\ref{fig:framework}A).
% 例如,机构可以创建一种被确定为与良好的社会经济效益相关联的水平匹配结构,如果它鼓励管理相连生态成分的不同行动者之间的协作(图1B)。
For example, institutions may create a structure that encourages collaboration between the different actors managing connected ecological components (Figure~\ref{fig:framework}B), leading to sustainable outcomes.
% 例如,机构可以创建一个结构,鼓励不同行为者之间的合作,管理相连的生态成分
Similarly, institutions for vertical management may enhance multi-layered SES matching by coordinating horizontal relationships (Figure~\ref{fig:framework}C and D).
% 在实践中,一个大型、复杂的河流流域的制度变化将创造或摧毁数百种不同的联系。这些局部变化的更广泛影响可以从系统的整体行为中看到。
In practice, institutional changes in a large, complex river basin will create or destroy hundreds of different connections. The broader impacts of these local changes can be seen in the overall behavior of the system.
% 因此,我们通过黄河流域的准自然实验,探讨了社会经济结构与可持续发展(结果)之间的因果关系,为两个主要原因提供了一个有益的案例研究。
We thus explored the causal linkages between the SES structures and sustainability (outcomes) in quasi-natural experiments of the YRB, which provides an informative case study for two main reasons.
% 首先,长江流域管理的急剧结构变化使我们能够定量估计高层制度设计变化对用水的净影响。决定水分配的制度包括自下而上的协议或社会规范,以及自上而下的配额或法规,它们对社会经济结构有不同的影响;自上而下的监管可能会立即引发制度转变和SES的剧烈结构性变化。通过与由自下而上的制度转变引起的更渐进的变化相比,探索自上而下变化的影响,在对社会经济地位的定量分析中,极大地减少了来自不可观测因素的潜在干扰,提高了社会经济地位结构和结果之间因果关系的清晰度。
First, the sharp structural shifts in YRB management enabled us to quantitatively estimate the net effects of changes in high-level institutional design on water use. Institutions that determine water allocation include bottom-up agreements or social norms as well as top-down quotas or regulations, with different effects on SES structure \cite{wangAlignmentsocialecological2019,speedBasinwaterallocation2013}; top-down regulations can trigger immediate institutional shifts and sharp SES structural changes \cite{speedBasinwaterallocation2013,rolandUnderstandinginstitutionalchange2004}.
In comparison with investigations of more gradual changes induced by bottom-up institutional shifts, exploring the impacts of a top-down change substantially diminishes potential problems of omitted variables in the quantitative analysis of SES and improves the clarity of the causal link between SES structure and outcome.
% 其次,通过比较长江流域两次制度变迁所分裂的三种不同制度结构的净效应,我们也可以更深入地理解“盆地固定效应”下结构格局的影响。尽管流域内的社会经济单位从世界各地的大型河流流域和许多地区的水资源中受益,但很少有流域多次经历过如此激进的社会经济结构变化。
Second, by comparing the net effects of three different institutional structures split by two institutional shifts in the YRB, we can also reach a stronger understanding of the influence of structural alignments under a fixed basin. Although socioeconomic units within a basin benefit from water resources in large river basins all over the world and many locations have shown increased levels of regulation, few basins have experienced such radical SES structural changes several times (see \textit{Supplementary Material} S1). Thus, the YRB provides a valuable setting for understanding the direct impacts of changes in SES institutional structure.
% 因此,黄河流域的“流域固定效应”为SES结构的自我比较提供了宝贵的机会。
% Thus, the YRB provides valuable settings for understanding the direct impacts of changes in SES institutional structure.

