\section{Reviewer \#2}\label{reviewer_2}

\subsection*{Overall comments}

\RC{} This study examines the effects of two institutional shifts in Water Governance for the Yellow River Basin - 87-WAS and 98-UBR. Results show that while 87-WAS led to an increased water use, 98 UBR successfully reduced it due to its basin wide authority and stronger connections between stakeholders. It is suggested that institutions should be designed according to ecological scale, promote cooperation with different parties involved, as well as adapting themselves accordingly. Governments and policymakers must analyze the effects of changes in policy on water usage and sustainability. The author adopted the newly developed Synthetic Difference-in-Differences published on AER. This paper is well written and strutured. However, this method is completely new to the researcher on water study. I have several suggestions wish the authors further consider them in the revised manuscript.

\AR{} We appreciate your acknowledgement of the reasonable and adaptable methodology used in this study, as well as the useful insights provided by the case analysis. We agree that the discussion could be more supportive and solid in its content of the results and we will work to revise the manuscript accordingly. Please find our point-by-point response to your specific comments below.

\subsection{\#1 Method}

\RC{}  It is suggested that the principle behind Differenced Synthetic Control, as well as the process for selecting treatment and control groups and conducting synthetic control, be explained using more accessible language. Currently, the explanation is too abstract and may be difficult for most readers of the journal to understand. To further expand on the suggestion, using more accessible language would involve breaking down complex ideas and jargon into simpler terms that are easier for readers to understand. This could involve providing more concrete examples and explanations, and avoiding technical terminology that may not be familiar to a general audience. Additionally, it may be helpful to provide step-by-step instructions or diagrams to illustrate the process of selecting treatment and control groups and conducting synthetic control. By doing so, readers who may not be familiar with the topic can more easily understand and appreciate the content of the journal.

\AR{} Thank you for your insightful suggestion on making the Differenced Synthetic Control (DSC) method explanation more accessible to a wider audience. We understand the importance of making our methodology clear to all readers, not just those with a strong background in statistical analysis. We have revised the section accordingly, breaking down complex ideas into simpler terms, providing concrete examples, and avoiding jargon. We believe these changes make the methodology more accessible to a broader audience. Your suggestion greatly helped us improve the clarity and accessibility of our methodology description:

\begin{quote}[page=6, sline=124, eline=132]
    The Differenced Synthetic Control (DSC) method~\cite{arkhangelsky2021} is a tool we use to estimate how water use might have evolved if there had been no institutional shift.
    Think of it as creating an alternate reality or a ``what-if'' scenario to compare with what actually happened~\cite{abadie2010, abadie2015, hill2021}.
    The key idea behind this method is to evaluate the effects of policy changes (in this case, the 87-WAS and the 98-UBR) that mainly affect certain units (the provinces in the YRB).
    The method creates a ``synthetic'' version of the affected units by combining information from other similar but unaffected units. This ``synthetic'' version serves as a control group, which we can compare with the actual affected units.
    The DSC method, therefore, is a powerful tool as it allows us to control for unobserved factors that can change over time.
\end{quote}

\subsection{\#2 Robustness}\label{sec:2-2}

\RC{} I'm glad to see that you have included a couple of robustness checks. However, there are several more details related to Differenced Synthetic Control that need to be tested. These may include testing for the sensitivity of the results to changes in the choice of control group, the inclusion or exclusion of certain covariates, and the specification of the synthetic control model. Other possible robustness checks may involve testing for heterogeneity of treatment effects across different subgroups or time periods, and assessing the impact of outliers or influential observations on the results.

\AR{} We appreciate your thoughtful suggestion about additional checks for our Differenced Synthetic Control (DSC) method. We now revised the term from ``robustness'' to ``efficiency'' based on the comments from the reviewer \#1, so we basically kept the previous testing approaches for the following reasons:

\AR*{} Firstly, our primary objective in this paper is not to conduct a precisely quantitative assessment. Rather, we aim to uncover the interesting differential responses underlying the two different institutional shifts. As you pointed out, readers of our paper might not expect an overload of technical details. To maintain conciseness and avoid overwhelming our readers, we have chosen to limit the number of methodologies we include.

\AR*{} Secondly, the two approaches we have employed were recommended by the developers of the DSC method~\cite{engelbrektson2023}.
The corresponding algorithms have been provided and are consistent with the DSC method. If we were to introduce further tests, we might have to develop our own algorithms. This could potentially lead to inconsistencies and might confuse readers who wish to apply our methods in their own work.

\AR*{} Lastly, in response to a suggestion from another reviewer, we have revised the terminology from ``robustness'' to ``efficiency''. We believe this terminology better communicates our intentions and the nature of our methodology to readers, reducing the need for additional emphasis on robustness.

\AR*{} We believe these reasons justify our decision. However, we appreciate your suggestion and will certainly keep it in mind for future research where more extensive robustness checks may be more appropriate.

\subsection{\#3 Results and discussion}\label{sec:2-3}

\RC{} The results, discussion, and comparisons are comprehensive, and I have no further comments at this time.

\AR{} Thank you for your positive feedback on the comprehensiveness of the results, discussion, and comparisons in our paper. We're delighted to hear that you found this section satisfactory. Your kind words are greatly encouraging and motivate us to continue to improve our work. We've still improved the discussion section, and we hope you'll find this even better.

