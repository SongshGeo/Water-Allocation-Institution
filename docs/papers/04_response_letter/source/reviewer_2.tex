\section{Reviewer \#2}\label{reviewer_2}

\subsection*{Overall comments}
% 本研究考察了黄河流域87-WAS和98-UBR两种制度变迁对水治理的影响。结果表明,87-WAS导致用水量增加,98 UBR由于其在流域范围内的权威和利益相关者之间更强的联系而成功地减少了用水量。建议根据生态规模进行制度设计,促进不同参与方的合作,并进行相应的调整。政府和决策者必须分析政策变化对水的使用和可持续性的影响。作者采用了新开发的在AER上发表的合成差中差。这篇论文写得很好,结构也很好。然而,这种方法对于水研究人员来说是全新的。我有几点建议,希望作者在改稿中进一步考虑。
\RC{} This study examines the effects of two institutional shifts in Water Governance for the Yellow River Basin - 87-WAS and 98-UBR. Results show that while 87-WAS led to an increased water use, 98 UBR successfully reduced it due to its basin wide authority and stronger connections between stakeholders. It is suggested that institutions should be designed according to ecological scale, promote cooperation with different parties involved, as well as adapting themselves accordingly. Governments and policymakers must analyze the effects of changes in policy on water usage and sustainability. The author adopted the newly developed Synthetic Difference-in-Differences published on AER. This paper is well written and strutured. However, this method is completely new to the researcher on water study. I have several suggestions wish the authors further consider them in the revised manuscript.

\AR{} We appreciate your acknowledgement of the reasonable and adaptable methodology used in this study, as well as the useful insights provided by the case analysis. We agree that the discussion could be more supportive and solid in its content of the results and we will work to revise the manuscript accordingly. Please find our point-by-point response to your specific comments below.

\subsection{\#1 Method}

\RC{}  It is suggested that the principle behind Differenced Synthetic Control, as well as the process for selecting treatment and control groups and conducting synthetic control, be explained using more accessible language. Currently, the explanation is too abstract and may be difficult for most readers of the journal to understand. To further expand on the suggestion, using more accessible language would involve breaking down complex ideas and jargon into simpler terms that are easier for readers to understand. This could involve providing more concrete examples and explanations, and avoiding technical terminology that may not be familiar to a general audience. Additionally, it may be helpful to provide step-by-step instructions or diagrams to illustrate the process of selecting treatment and control groups and conducting synthetic control. By doing so, readers who may not be familiar with the topic can more easily understand and appreciate the content of the journal.

\AR{} Thank you for your insightful suggestion on making the Differenced Synthetic Control (DSC) method explanation more accessible to a wider audience. We understand the importance of making our methodology clear to all readers, not just those with a strong background in statistical analysis. We have revised the section accordingly, breaking down complex ideas into simpler terms, providing concrete examples, and avoiding jargon. We believe these changes make the methodology more accessible to a broader audience. Your suggestion greatly helped us improve the clarity and accessibility of our methodology description:

\begin{quote}
    The Differenced Synthetic Control (DSC) method~\cite{arkhangelsky2021} is a tool we use to estimate how water use might have evolved if there had been no institutional shift.
    Think of it as creating an alternate reality or a ``what-if'' scenario to compare with what actually happened~\cite{abadie2010, abadie2015, hill2021}.
    The key idea behind this method is to evaluate the effects of policy changes that mainly affect certain units (in this case, the institutional shifts in the Yellow River Basin or YRB).
    The method creates a ``synthetic'' version of the affected units by combining information from other similar but unaffected units. This ``synthetic'' version serves as a control group, which we can compare with the actual affected units.
    The DSC method, therefore, is a powerful tool as it allows us to control for unobserved factors that can change over time, providing more robust results.
\end{quote}

\subsection{\#2 Robustness}\label{sec:2-2}
% 导言:与流域和水资源相关的指标有很多,其中一些指标具有与治理和政策相关的参数。为了突出你所制定的IWGI的新颖性,也许你可以加上对其他相关指标的审查。
\RC{} I'm glad to see that you have included a couple of robustness checks. However, there are several more details related to Differenced Synthetic Control that need to be tested. These may include testing for the sensitivity of the results to changes in the choice of control group, the inclusion or exclusion of certain covariates, and the specification of the synthetic control model. Other possible robustness checks may involve testing for heterogeneity of treatment effects across different subgroups or time periods, and assessing the impact of outliers or influential observations on the results.

\AR{} Thanks for valuable suggestion. We now added a whole paragraph to the introduction:

\subsection{\#3 Results and discussion}\label{sec:2-3}
% 方程:方程(7)中的CEM是什么?还有,水的比例指的是什么?支持信息中公式(2)中的一些术语没有解释。例如:R或WU。
\RC{} The results, discussion, and comparisons are comprehensive, and I have no further comments at this time.

\AR{} We are glad that you found our results, discussion, and comparisons comprehensive. We appreciate your time and effort in reviewing our manuscript.