% chktex-file 46
\section*{Associate Editor}\label{editor}

\subsection*{General comments}
\RC{} Dear Author,

\RC*{} Two reviewers have assessed your paper and identified a good potential but some flaws, with which I concur. I would particularly insist on Reviewer 1's comment on DSC method description and wording.

\RC*{} In addition to reviewers' comments, I would also recomment strengthening the discussion in light of the special issue topic. Your analysis of consequences of institutional shifts is indeed important and timely for grounded sociohydrology. The methodological choice mainly based on statistical quantitative analysis is innovative to better qualify changes in SocioHydrological systems, which is good. Going to grounded sociohydrology needs more of the story behind, explanation of the social and hydrological processes that drive these changes. The discussion opens some ways in this direction but is too weak on that regard. I missed examples (eg.\ facts describing the transfer from surface to groundwater withdrawal after 98), quotations (eg.\ on bargaining arguments), precisions of who are the dominant stakeholders and how they influence the sociohydrological process.

\RC*{} I hope you will be able and keen on revising the manuscript with these directions.

\RC*{} O. Barreteau in the name of invited editorial team

\AR{} Dear Dr. Barreteau and the Invited Editorial Team,

\AR*{} I would like to express my sincere gratitude for the thoughtful feedback and constructive comments provided by the reviewers and yourself. Your insights have greatly enhanced the quality of our manuscript, and I appreciate the opportunity to respond to each concern.

\AR{} DSC Method Description and Wording (Reviewer \#1's Comment)

\AR*{} Thank you for highlighting this point. We have thoroughly revised the sections related to the DSC method, providing a more detailed explanation and improving the wording. This revision ensures that the method is clearly presented and its relevance to the study is well-articulated.

\AR{} Strengthening the Discussion in Light of the Special Issue Topic:

\AR*{} Social and Hydrological Processes: We recognize the importance of delving deeper into the underlying social and hydrological processes that drive changes in the SocioHydrological systems. To address this, we have enriched our discussion by:

\AR*{} (1) Inclusion of Examples and Facts: As suggested, we have included specific examples such as the case of the Hetao Irrigation District and facts describing the transfer from surface to groundwater withdrawal after 98. These additions offer a concrete illustration of the theoretical arguments and provide contextual insights.

\AR*{} (2) Quotations on Bargaining Arguments: We have integrated relevant quotations to illuminate the bargaining arguments among stakeholders during the institutional shifts. This inclusion offers a more vivid and grounded understanding of the negotiations and stakeholder dynamics.

\AR{} I am grateful for the encouragement and the direction provided to enhance our manuscript. It has been a privilege to contribute to this special issue, and I believe that the revisions made align with the expectations and vision set forth by the editorial team.
