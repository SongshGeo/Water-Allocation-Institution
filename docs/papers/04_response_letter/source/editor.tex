% chktex-file 46
\section*{Associate Editor}\label{editor}

\subsection*{General comments}
\RC*{} Dear Author,

\RC*{} Two reviewers have assessed your paper and identified a good potential but some flaws, with which I concur. I would particularly insist on Reviewer 1's comment on DSC method description and wording.

% 除了审稿人的意见外,我还建议根据特刊主题加强讨论。你对制度变迁后果的分析,对于社会水文学来说,确实是重要而及时的。主要基于统计定量分析的方法选择是创新的,可以更好地描述社会水文系统的变化,这是好的。要深入社会水文学,需要更多背后的故事,对推动这些变化的社会和水文过程的解释。讨论在这个方向上开辟了一些途径,但在这方面过于薄弱。我漏了一些例子。1998年以后从地表水转移到地下水取水的事实),引语(如:(关于讨价还价的论据),谁是主要利益相关者的准确性以及他们如何影响社会水文过程。
\RC*{} In addition to reviewers' comments, I would also recomment strengthening the discussion in light of the special issue topic. Your analysis of consequences of institutional shifts is indeed important and timely for grounded sociohydrology. The methodological choice mainly based on statistical quantitative analysis is innovative to better qualify changes in SocioHydrological systems, which is good. Going to grounded sociohydrology needs more of the story behind, explanation of the social and hydrological processes that drive these changes. The discussion opens some ways in this direction but is too weak on that regard. I missed examples (eg.\ facts describing the transfer from surface to groundwater withdrawal after 98), quotations (eg.\ on bargaining arguments), precisions of who are the dominant stakeholders and how they influence the sociohydrological process.

\RC*{} I hope you will be able and keen on revising the manuscript with these directions.

\RC*{} O. Barreteau in the name of invited editorial team

\AR*{} We hope that these revisions address the concerns raised by both reviewers. We are confident that these changes have significantly improved the quality of our manuscript and made it more relevant and valuable to the scientific community. We look forward to your further feedback and the possibility of our manuscript being accepted for publication.
