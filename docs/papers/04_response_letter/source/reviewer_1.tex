% chktex-file 1
% chktex-file 46
\section{Reviewer \#1}\label{reviewer_1}

\subsection*{Overall comments}

% \RC{}本研究旨在利用差分综合控制(DSC)方法量化黄河流域(YRB)制度变迁(政策变化)和水资源利用的影响。这项研究在方法上似乎是合理的。这个话题很有趣:它以定量的方式将政策维度整合到对流域的评估中,可能会引起期刊读者的兴趣。也就是说,我认为手稿需要相当多的改进和仔细的修改。请参阅下文。
\RC{} This study aimed at quantifying the effect of institution shifts (policy changes) and water usage in the Yellow River Basin (YRB), using the Differenced Synthetic Control (DSC) method. The study seemed methodologically sound. The topic is interesting: it integrated the policy dimension into an assessment of a river basin in a quantitative way and could be of interest to the journal's readership. That said, I believe the manuscript needs quite a few improvements and careful revisions. Please see below.

\AR{} Thank you for your valuable feedback and for acknowledging our efforts \dots

\subsection{CLEARER METHOD DESCRIPTION}\label{sec:1-1}

\RC{} Section 3.3. Differenced Synthetic Control needs to be much clearer. For example, on lines 141-145, please double-check the description and symbols used in this passage. In particular, the multiple uses of capital letter T—as the duration of a period and as a time index—were quite confusing. Then, ``post-treatment period $T_0$.\ldots all preceding periods $T_1$'' - is this correct? Anyhow, please double-check and revise to make this passage clearer.

\RC{} In Eq 1, $X_0$ and $X_1$ were never defined.

\RC{} Using ``$*$'' to denote the dimension of a matrix (e.g., $k*k$) is a bit odd; please consider using a multiplication symbol, e.g., $k \times k$.

\AR{} We have revised the description of the DSC method to make it clearer and more comprehensive. We expanded on the purpose and process of the DSC method, provided more context on how it fits into our research, and clarified the mathematical notation. We also explained the specific variables used when we introduced $X_1, X_0, Z_1$, and $Z_0$, as well as mentioned the specific method used for minimization and the corresponding reference. We believe that these revisions have significantly improved the clarity of our method description:

\begin{quote}[page=XXX, sline=XXX, eline=XXX]
    Modification \dots
\end{quote}

\AR*{} We are grateful for your insightful feedback, which has helped us improve the quality of our manuscript. We hope that our revisions have addressed all your concerns satisfactorily.


\subsection{BETTER WORD CHOICES}\label{sec:1-2}
% 结果是总结性的。我建议提供更多关于三个指数和IWGI结果的信息。例如,三个指标的结果,以及分区域的结果。请解释IS、IP、IA和IWGI这三个指标的范围,以及不同数值的含义。这些细节可以帮助读者理解结果的合理性。
\RC{} I found some of the word choices a bit strange. For example, unless the term has been used this way in the literature, "point-axis networks" sounds strange. Otherwise, please consider alternatives like "organizational charts" or "organizational diagrams," which I believe is more appropriate.

\RC{} In a similar vein, I do not think that "network approach" or "network-based approach" is an appropriate description of what is being done here. Normally, network approaches involve concepts from complex network theories, which are not present here. Indeed, calling this a network approach may provoke unnecessary objections from network researchers. I would suggest the use of simpler terms like "analysis of the organizational charts" or something along that line.

\RC{} I disagree with how the term and concept of "robustness" is used here. Normally, a robust result is a result that holds even when the input variables and/or parameters are uncertain. What is being reported here seems to be the effectiveness or efficiency of the institutional shifts, not the robustness. If I misunderstood anything here, please clarify. As it is, I don't think that the use of the term "robustness" is appropriate here.

\RC{} Is ``placebo experiments'' the term normally used in this way? If so, can you provide some references?

\subsection{Other comments}\label{sec:1-3}
% 本研究中使用的数据似乎有不同的时间段,大多数数据来自2013年之前。近十年来,黄河的水文状况和水治理应该发生了重大变化。因此,结合近十年的结果将使这项研究更加可靠。
\RC{} I would change the first two key points into full sentences; as they are, they are only fragments, while the third and fourth points are full sentences.

\RC{} Lines 39-40: "the 1987...(98-UBR)." is repeated - please remove.

\RC{} Line 75: A subject is missing in front of "proposed to..." Perhaps, "the government"?

\RC{} Lines 169-171: "A robust synthetic control model will show a significant difference after treatment, but not before treatment." Why? This statement assumes that the treatment must have an effect; what if the treatment has no effect?

\RC{} Line 209: Are parentheses missing for ``$km^3 billion m^3$''?

\RC{} Lines 261-262: It's unclear to me how Fig. 4 supports the statement "the correlation between...98-UBR (Figure 4)."

\RC{} Somehow, I did not see Appendix C explicitly referred to in the main text. There is reference to "Marginal benefit model for water use" but this was not explicitly labeled as Appendix C.

\RC{} Line 280: "Similar" -> "Similarly"

\RC{} Line 323-324: This is the first time you used the term "difference-in-differences" - so far, you have called your method "Differenced Synthetic Control" (DSC). Why different names? Likely a typo; please double-check.

\RC{} Line 327: "revealed" is too strong; perhaps "suggested"?

\RC{} Line 437, Ref [30]: Pca should be all caps, PCA.\
