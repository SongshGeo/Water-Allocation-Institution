% chktex-file 1
% chktex-file 46
\section{Reviewer \#1}\label{reviewer_1}

\subsection*{Overall comments}


% 在大多数情况下,作者已经解决了我在原始提交中关于单词选择和各种符号混淆的评论,我对此表示赞赏。然而,差分综合控制方法的描述,如果有什么的话,变得更让我困惑!
\RC{} For the most part, the authors have addressed my comments regarding word choices and various notational confusion in the original submission, and I appreciate that. However, the description of the Differenced Synthetic Control method, if anything, becomes even more confusing to me!

% Eq 1: V怎么可能是一个向量,而不是一个矩阵?在Eq 1中的矩阵乘法只是不工作与指定的维度!X_1的维数是多少?作者指定W是一个(J x 1)向量,X_0是一个(k x J)矩阵。这意味着乘积(X_0 W)是一个(k x 1)向量,所以X_1也应该是一个(k x 1)向量。这意味着(X_1 - X_0 W)'和(X_1 - X_0 W)的维数应该是(1 × k)和(k × 1)。为了使方程1成立,V必须是一个(k × k)矩阵。我错过什么了吗?如果我误解了什么,请澄清。
\RC{} Eq 1: How can V be a vector, not a matrix? The matrix multiplication in Eq 1 just doesn't work with the dimensions specified! And what is the dimension of X_1? The authors specified that W is a (J x 1) vector and X_0 is a (k x J) matrix. This implies that the product (X_0 W) is a (k x 1) vector, so X_1 should be a (k x 1) vector as well. Which means that (X_1 - X_0 W)' and (X_1 - X_0 W) should have dimensions of (1 x k) and (k x 1). For Eq 1 to work, V must be a (k x k) matrix. Did I miss something here? If I misunderstood anything here, please clarify.

% 加上,如何确定V的项的值?公式2表明需要W来确定V?但这不是循环逻辑吗?所以你需要最优的V来确定W,但你也需要最优的W来确定最优的V?哪些优化必须先做?
\RC{} Plus, how does one determine the values of V's entries? Eq. 2 suggests that one needs W to determine V? But isn't that circular logic? So you need the optimal V to determine W, but you also need the optimal W to determine the optimal V? Which optimization must be done first?

% Z的维数不应该是(T_0 x J),而不是(J x T_0)吗?否则,它就不能从右边乘以W, W的维数是(jx1)再说一遍,我漏了什么吗?如果我误解了什么,请澄清。
\RC{} Eq. 2: Shouldn't Z's dimension be (T_0 x J), not (J x T_0)? Otherwise, it cannot be multiplied from the right by W, which has the dimension of (J x 1). Again, did I miss something here? If I misunderstood anything here, please clarify.

% 虽然我仍然认为这项研究可能会引起期刊读者的兴趣,正如我上面的沮丧所表明的那样,方法描述非常令人困惑,并且没有得到充分的解决。我强烈建议作者在下次修订时要非常小心。
\RC{} While I still think that the study can be of interest to the journal's readership, as indicated by my frustration above, the method description is very confusing and has not been adequately addressed. I strongly suggest that the authors be VERY CAREFUL in the next revision.
