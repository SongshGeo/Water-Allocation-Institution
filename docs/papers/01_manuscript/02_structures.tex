%! Author = songshgeo
%! Date = 2022/3/10

% \subsection{INSTITUTIONAL SHIFTS AND STRUCTURES}
\subsection{Institutional shifts and structures}
\label{results-1}

% 制度变动综述
The institutional shifts in the YRB in 1987 (87-WAS) and 1998 (98-UBR) were two widely recognized milestones in restricting water use among YRB's water governance practices (\textit{\nameref{secS1}}).
Until the 87-WAS, stakeholders (the provinces in the YRB) had free access to the YR water resources for development, but there were geographic and temporal differences between freshwater demand and availability.
Therefore, the YRCC had no links to the provinces regarding water use before 1987, and the provinces could link directly to the Yellow River reaches (Figure~\ref{fig:structure}~C).
To shrink water deficits, in 87-WAS, national authorities proposed in 87-WAS allocating specific water quotas between $10$ provinces (or regions) along the YR basin (Table~\ref{tab:quota}).
Simultaneously, according to the extracted information from documents of the 87-WAS issued by national ministries, the YRCC started to report water use in each reach.
As it was the first time YRCC's responsibility involved water use, we introduced new links between the YRCC and the ecological nodes (Figure~\ref{fig:structure}~C).
However, the controversial 87-WAS did not resolve water depletion.
In 1998, another strategy (98-UBR) was developed to strengthen the responsibilities of the YRCC for integrated managing water use.
Information from the 98-UBR documents demonstrated that the provinces had to apply their plan for an annual water use license to YRCC instead of direct access to the Yellow River water.
Thus, the YRCC has been linked to the provinces since 1998 (Figure~\ref{fig:structure}~C).

\begin{figure*}[!htb]
    \includegraphics[width=\linewidth]{diagrams/diagram.pdf}
	\caption{
		% 黄河流域的制度变迁与经济社会结构差异。
		Institutional shifts and related SES structures in the Yellow River Basin (YRB).
		\textbf{A.} The YBR crosses $10$ provinces or the same-level administrative regions, $8$ of which are highly relying on the water resources from the YRB (see \textit{\nameref{secS1}} Table~\ref{tab:quota}). The national administrations are the ultimate authority in issuing water governance policies, which are often implemented by basin-level agency (the Yellow River Conservancy Commission, YRCC) and each province-level agency.
		\textbf{B.} Provincial administrative agencies are the major stakeholders. Since the 87-WAS, with surface water withdrawal from the Yellow River restricted by specific quotas, each stakeholder plan and use water resources for development. However, the natural hydrological processes are connected. Although the institutions focus mainly on surface water (Sur.), it can also influence groundwater inside (Gro.) or water resources outside (Sur. and Gro.') through systematic socio-hydrological processes within the YRB. The YRCC only monitors water withdrawals at that time.
		\textbf{C.} Institutional shifts and following structures changes (details in \textit{\nameref{secS1}}). (1) From 1979 to 1987, water resources were freely accessible to each stakeholder (denoted by red circles) from the connected ecological unit (the reach of Yellow River, denoted by the blue circles). (2) After 1987-WAS, the YRCC (the yellow circles) was monitoring (the dot-line links) river reaches with the water use quota. (3) Since the 98-UBR, stakeholders have to apply for water use licenses from the YRCC (the connections between the red and yellow circles).
	}
	\label{fig:structure}
\end{figure*}
