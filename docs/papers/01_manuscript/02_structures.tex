%! Author = songshgeo
%! Date = 2022/3/10

% \subsection{INSTITUTIONAL SHIFTS AND STRUCTURES}
\subsection{Institutional shifts and structures}
\label{results-1}

% 制度变动综述
% Including the national authorities, the basin management authorities, provinces, cities, and even districts, top-down institutional structures of the YRB started to evolve up to now (\textit{S1 in Supplementary Material}).
As a pioneer in water governance shifting in China, the YRB started to explore the initial water allocating scheme in the 1970s.
The institutional shifts in 1987 (87-WAS) and 1998 (98-UBR) were two widely recognized milestones of water governance.
Until the 87-WAS, stakeholders (the provinces in the YRB) had free access to the YR water resources for development, but there were geographic and temporal differences between freshwater demand and availability.
As a compounded result of development, the provinces such as Shandong, Henan and Inner Mongolia used more water resources in the YRB with larger economies (primarily for irrigation agriculture).
For shrinking water deficits, national authorities proposed in 87-WAS allocating specific water quotas between $10$ provinces (or regions) along the YR basin.
However, the controversial scheme helped little in turning the water depletion around until another strategy attempted to strengthen the responsibilities of the YRCC in integrated water management in 1998 (the 98-UBR).
Therefore, our analysis period spans from 1975 (emergence of river depletion) to 2008 (a further polish of the 98-UBR), with the SESs shifted between three varying institutions which were different from other large river basins in China.

% 制度结构关系抽象
% Because institutions may shape the structure of SESs, describing institutional structure is a first step toward understanding the mechanisms linking structures and outcomes in SESs (Figure~\ref{framework} A).
% For example, institutions may create a structure that encourages collaboration between the different actors managing connected ecological components (Figure~\ref{framework} B), leading to sustainable outcomes.
% Similarly, institutions for vertical management may enhance multi-layered SES matching by coordinating horizontal relationships (Figure~\ref{framework} C and D).
% Empirical studies have suggested that such widespread building blocks in SES are the key to the functioning of structures, and a network model is a widely used way to depict them by abstracting links and nodes.

We selected institutional regulatory documents on water use issued by national ministries (for validation to both watershed and regional agents) and extracted the interactions between the agents involved (\textit{Supplementary Material S2}).
Before 1987, the YRCC had no links to the provinces regarding water use, and the provinces could link to the Yellow River reaches directly (Figure~\ref{structure}).
However, according to the extracted information from the 87-WAS, the YRCC started to report water use from the provinces.
Furthermore, information from the 98-UBR documents demonstrated that the provinces had to apply their plan for an annual water use licence instead of direct access to the Yellow River water.
Thus, there have been links between the YRCC and the provinces since the strengthening responsibilities of the YRCC in 1998.

\begin{figure*}[!h]
    \includegraphics[width=32pc]{diagrams/diagram.pdf}
	\caption{
		% 黄河流域的制度变迁与经济社会结构差异。
		Institutional shifts and related SES structures in the Yellow River Basin (YRB). See \textit{Supplementary Material S1} for detailed introduction for the institutions.
		% 国家政府在1987年和1998年先后两次出台了改变流域制度的政策,而黄河水利委员会、利益攸关的各省在不同制度时期具备的功能是不同的。
%		\textbf{A.} The national government changed YRB management policies and institutions in 1987 and 1998. As a result, the Yellow River Conservancy Commission (YRCC) and the provinces acted differently in different periods. Three different SES structures existed successively in the YRB.
		% 没有任何政策限制,各利益相关者此时期可以从单向但联通的河流生态单元内自由取水
%		\textbf{1975–1987:} Without any constraints, water resources were freely accessible to each stakeholder (the provinces, in this case, denoted by red circles) from a one-way but connected ecological unit (the Yellow River, denoted by the blue rectangle).
		% 在政策1之后,每个使用者都被分配了能够开采河流地表水资源的配额,而黄委会的工作是对配额的使用进行统计与汇报。
%		\textbf{1987–1998:} After the implementation of policy 1 in 1987, each user was assigned a quota to withdraw surface water resources, and the YRCC (yellow triangle) was tasked with reporting on water quota use.
		% 第二个政策之后,利益相关者取水需要向黄委会申请,黄委会则根据用水配额来批发许可。因此此时黄委会与利益相关者之间产生了直接的双向联系,监督他们对资源配额的取用。
%		\textbf{1998–2008:} After the implementation of policy 2, stakeholders had to apply for water resources from the YRCC, which then licensed water use according to the quota. Under this institution, the YRCC had direct two-way connections between provinces and ecological components.
	}
	\label{structure}
\end{figure*}
