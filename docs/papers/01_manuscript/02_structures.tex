%! Author = songshgeo
%! Date = 2022/3/10

% \subsection{INSTITUTIONAL SHIFTS AND STRUCTURES}
\subsection{Institutional shifts and structures}
\label{results-1}

% 制度变动综述
% Including the national authorities, the basin management authorities, provinces, cities, and even districts, top-down institutional structures of the YRB started to evolve up to now (\textit{S1 in Supplementary Material}).
% As a pioneer in water governance shifting in China, the YRB started to explore a water allocating scheme since 1970s.
The institutional shifts in 1987 (87-WAS) and 1998 (98-UBR) were two widely recognized milestones in restricting water use among national water governance practices (\textit{Appendix}~\nameref{secS1}).
Until the 87-WAS, stakeholders (the provinces in the YRB) had free access to the YR water resources for development, but there were geographic and temporal differences between freshwater demand and availability.
As a compounded result of development, the provinces such as Shandong, Henan and Inner Mongolia used more water resources in the YRB with larger economies (primarily for irrigation agriculture).
For shrinking water deficits, national authorities proposed in 87-WAS allocating specific water quotas between $10$ provinces (or regions) along the YR basin.
However, the controversial scheme helped little in turning the water depletion around until another strategy attempted to strengthen the responsibilities of the YRCC in integrated water management in 1998 (the 98-UBR).
Therefore, our analysis period spans from 1975 (emergence of river depletion) to 2008 (a further polish of the 98-UBR), with the SESs shifted between three varying institutions (Figure~\ref{structure}).

We selected institutional regulatory documents on water use issued by national ministries (for validation to both watershed and regional agents) and extracted the interactions between the agents involved (\textit{Appendix \nameref{secS1}}).
Before 1987, the YRCC had no links to the provinces regarding water use, and the provinces could link to the Yellow River reaches directly (Figure~\ref{structure}).
However, according to the extracted information from the 87-WAS, the YRCC started to report water use from the provinces.
Furthermore, information from the 98-UBR documents demonstrated that the provinces had to apply their plan for an annual water use licence instead of direct access to the Yellow River water.
Thus, there have been links between the YRCC and the provinces since the strengthening responsibilities of the YRCC in 1998.

\begin{figure*}[!htb]
    \includegraphics[width=\linewidth]{diagrams/diagram.pdf}
	\caption{
		% 黄河流域的制度变迁与经济社会结构差异。
		Institutional shifts and related SES structures in the Yellow River Basin (YRB).
		\textbf{A.} The YBR crosses $10$ provinces or the same-level administrative regions, $8$ of which are highly relying on the water resources from the YRB (see \textit{Appendix \nameref{secS1}} Table~\ref{tab:quota}). The national administrations are the ultimate authority in issuing water governance policies, which are often implemented by basin-level agency (the Yellow River Conservancy Commission, YRCC) and each province-level agency.
		\textbf{B.} Since the YRCC does not use but monitor water, the provincial administrative agencies are the major stakeholders. Since the 87-WAS, with surface water withdraw from the Yellow River restricted by specific quotas, each stakeholder seperatly develop by planning and using fundamental water resources. However, the natural hydrological processes are connected. Although the institutions focus mainly on the surface water (Sur.), it can also influence groundwater inside (Gro.) or water resources outside (Sur. and Gro.'), through systematic socio-hydrological processes within the YRB.
		\textbf{C.} Institutional shifts and following structures changes (details in \textit{Appendix \nameref{secS1}}). (1) From 1975 to 1987, water resources were freely accessible to each stakeholder (denoted by red circles) from connected ecological unit (the reach of Yellow River, denoted by the blue circles). (2) After 1987-WAS, the YRCC (the yellow circles) was monitoring (the dot-line links) river reaches with the water use quota. (3) Since the 98-UBR, stakeholders have to apply water use licences from the YRCC (the connections between the red and yellow circles).
	}
	\label{structure}
\end{figure*}
