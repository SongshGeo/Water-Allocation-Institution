%! Author = songshgeo
%! Date = 2022/3/10

% \subsection{INSTITUTIONAL SHIFTS AND STRUCTURES}
\subsection{Institutional shifts and structures}\label{results-1}

\begin{figure*}[!t]
	\includegraphics[width=\linewidth]{diagrams/diagram.pdf}
	\caption{
		% 黄河流域的制度变迁与经济社会结构差异。
		Institutional shifts and related SES structures in the Yellow River Basin (YRB).
		\textbf{A.} The YRB crosses $10$ provinces or the same-level administrative regions, $8$ of which heavily rely on the water resources from the YRB (Table~\ref{tab:quota}). The national administrations hold ultimate authority in issuing water governance policies, which are often implemented by the basin-level agency (the Yellow River Conservancy Commission, YRCC) and each province-level agency.
		\textbf{B.} Provincial administrative agencies are the major stakeholders. Since the 87-WAS, with surface water withdrawal from the Yellow River restricted by specific quotas, each stakeholder plans and uses water resources for development. However, natural hydrological processes are interconnected. Although the institutions focus mainly on surface water (Sur.), they can also influence groundwater inside (Gro.) or water resources outside (Sur.\ and Gro.') through systematic socio-hydrological processes within the YRB.\\ The YRCC only monitors water withdrawals at that time.
		\textbf{C.} Institutional shifts and subsequent structural changes (details in \textit{\nameref{secS1}}). (1) From 1979 to 1987, water resources were freely accessible to each stakeholder (denoted by red circles) from the connected ecological unit (the reach of the Yellow River, denoted by the blue circles). (2) After 1987-WAS, the YRCC (the yellow circles) monitored (the dot-line links) river reaches with water use quotas. (3) Since the 98-UBR, stakeholders have had to apply for water use licenses from the YRCC (the connections between the red and yellow circles).
	}\label{fig:structure}
\end{figure*}

% 制度变动综述
% The institutional shifts in the YRB in 1987 (87-WAS) and 1998 (98-UBR) were two widely recognized milestones in restricting water use among YRB's water governance practices (\textit{\nameref{sec:yrb}} and \textit{\nameref{secS1}}).
Until the 87-WAS, provincial regions in the YRB had unrestricted access to the Yellow River water resources for development, despite geographic and temporal differences between freshwater demand and availability.
The YRCC had no links to the provinces regarding water use before 1987, and the provinces could connect directly to the Yellow River reaches (Figure~\ref{fig:structure}~C).
Following the 87-WAS, national authorities proposed allocating specific water quotas among the provinces, and the YRCC's duty became to report actual water use volumes in each reach.
As it was the first time the YRCC's responsibilities included water use, this introduced new links between the YRCC and the river (i.e., ecological nodes Figure~\ref{fig:structure}~C).
The 98-UBR further reinforced the YRCC's responsibilities for integrated water use management.
Since $1998$, provinces have been required to submit their annual water use plans for water use licenses to the YRCC instead of freely accessing the Yellow River water.
Consequently, the YRCC has been directly linked to the provinces since then (Figure~\ref{fig:structure}C).
% As result, the two institutional shifts reshaped SES structures, leading to three general structures linked by social actors and ecological nodes (Figure~\ref{fig:structure}~C).
