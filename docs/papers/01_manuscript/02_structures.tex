\subsection{Shifts of institution and SES structure in the YRB}
\label{result-0}

% 制度变动综述
Including the national authorities, the basin management authorities, provinces, cities, and even districts, top-down institutional structures of the YRB started to evolve up to now (\textit{S1 in Supplementary Material}).
As a pioneer in water governance shifting in China, the YRB started to explore the initial water allocating scheme in the 1970s.
The institutional shifts in 1987 (known as the ``87 Water Allocation Scheme, 87-WAS'') and 1998 (Unified Basinal Regulation, 98-UBR) were two widely recognized milestones of water governance.
Until the 87-WAS, stakeholders have free access to the YR water resources, with geographic and temporal differences between freshwater demand and availability.
As the mismatch between demands and supply kept increasing, national authorities proposed in 87-WAS allocating specific water quotas between $10$ provinces (or regions) along the YR basin.
However, this controversial scheme helped slight turn water depletion around until a different strategy expanded the responsibilities of basinal authorities in integrated water management in 1998 (the 98-UBR).
Ten years later, the YRB authorities require to further divide the provinces' quota into cities, counties, and districts.
Therefore, our analysis period spans from 1975 to 2008, with the human-water system shifted between three different institutions, which were different from other large river basins in China.

% 制度结构关系抽象
Because institutions may shape the structure of SESs, describing institutional structure is a first step toward understanding the mechanisms linking structures and outcomes in SESs (Figure~\ref{framework} A).
For example, institutions may create a structure that encourages collaboration between the different actors managing connected ecological components (Figure~\ref{framework} B), leading to sustainable outcomes.
Similarly, institutions for vertical management may enhance multi-layered SES matching by coordinating horizontal relationships (Figure~\ref{framework} C and D).
Empirical studies have suggested that such widespread building blocks in SES are the key to the functioning of structures, and a network model is a widely used way to depict them by abstracting links and nodes.
We thus selected institutional regulatory documents on water use issued by national ministries (for validation to both watershed and regional agents) and extracted the interactions between the agents involved (\textit{Supplementary Material S2}).

Before 1987, the YRCC had no links to the provinces regarding water use, and the provinces could link to the Yellow River reaches directly (Figure~\ref{structure}). However, according to the extracted information from the 87-WAS, the YRCC started to report water use from the provinces. Furthermore, information from the 98-UBR documents demonstrated that the provinces had to apply their plan for an annual water use licence instead of direct access to the Yellow River water. Thus, there were links between the YRCC and the provinces.
Although we abstract the two major institutional shifts from document interpretations, many studies used these two changes as baselines for the YRB's water governance history and widely recognized their importance.
