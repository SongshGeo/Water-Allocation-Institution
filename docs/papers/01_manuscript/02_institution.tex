%! Author = songshgeo
%! Date = 2022/3/10
% 黄河是世界第五长河,它的流域也是中国文明的摇篮。
The Yellow River, whose basin is the cradle of Chinese civilization, is the fifth-longest river in the world. It supports $9.7\%$ of China’s irrigation, with only $2.6\%$ of its total water resources (data from \href{http://www.yrcc.gov.cn}{http://www.yrcc.gov.cn}, last access: 28 February 2021).
% 然而,经过沿江各省多年的免费取水(图~\ref{fig:structure} A和B),到20世纪80年代,黄河地表水耗水量接近径流量的10倍,并不断上升
However, after years of free access to water by provinces along the river (Figure~\ref{fig:structure} A and B), surface water consumption of the Yellow River was close to $80\%$ of its runoff by the 1980s and rising~\cite{wangYellowRiverwater2019,songSedimenttransportincreasing2020}.
% 自1972年以来,黄河径流量的减少破坏了黄河的生态,制约了黄河的经济发展
Reductions in runoff after 1972 damaged the ecology of the YRB and restricted its economic development~\cite{wangYellowRiverwater2019}.
% 因此,在中国典型的自上而下的制度结构下(附录图S1-B),黄河流域不同层次提出了相对完整的水资源分配规定(图1a)。
% Therefore, through typical top-down institutional structures in China (see \textit{Supplementary Material S1}), relatively integrated water allocation regulations were successively proposed across different levels in the YRB.
% % 这些机构包括国家政府、流域管理机构、省、市,甚至地区(见图1 B-D)。
These include the national government, the basin management agency, provinces, cities, and even districts.
% 这些在不同制度发展阶段相继出台的政策,引发了长江经济带经济发展结构的突变,并产生了不同的结果(见附录A和图S1-C)。
These policies at different stages of institutional development triggered abrupt changes in the SES structure of the YRB with different outcomes.

% 0世纪70年代以来,中国政府向黄河水利委员会下达指令,要求黄河水利委员会设计配水方案,同时要求黄河沿岸各省进行水资源规划
In 1982, the Chinese government issued instructions to the Yellow River Water Conservancy Commission (YRCC), the basin agency of the YRB, requiring it to design a water allocation scheme and at the same time requiring the provinces along the Yellow River to carry out water resources planning (see \textit{Supplementary Material S1})~\cite{wangReviewImplementationYellow2019}.
% 经过多方讨论与权衡,中国政府于1987年为相关省份分配了水资源配额,要求沿黄各省(区)贯彻执行
The Chinese government started to assign water quotas to the relevant provinces in 1987, but did not create a unit to coordinate water division between them (Figure~\ref{fig:structure} \textbf{A} and \textbf{C}).
% 这一时期长江水利枢纽的监督任务是编制长江水利枢纽用水量统计公报,并与定额进行对比分析。
The mandate of the YRCC during this period was only to report on and analyze water consumption in the YRB~\cite{wangReviewImplementationYellow2019}.
% 随着断流的进一步恶化,1998年中国政府推进了相关政策的改革,要求所有省份在取用水资源时必须向黄委会申请许可,黄委会得以直接对各省的用水实施监管。
However, since reductions in river flow indicated an unintended SES outcome (Figure~\ref{fig:structure} E), the Chinese government pushed for a policy reform in 1998 that required all provinces to apply for licenses to use water from the YRCC, allowing the council to directly regulate their water use (see \textit{Supplementary Material S1} and Figure~\ref{fig:structure} A and D).
% 由于革新后的政策成功遏制了断流,2008年该分配政策被进一步细化,相关各省都进一步设置了更细致的分配方案,并最终形成了黄河流域如今的水资源分配格局。
The 1998 policy succeeded in curbing water extraction (Figure~\ref{fig:structure} E), and it was further refined in 2008.
The relevant provinces created a more detailed allocation plan and finally formed the present water allocation institutions of the YRB (see \textit{Supplementary Material S1}).
% 因此,在1975-2008长达33年的时间里,从没有分水政策到以两种不同模式监管下的分水政策,黄河流域实际上先后存在着三种不同的社会-生态结构(Fig 1)。
Therefore, in our study period (from 1975 to 2008), the system shifted between three different SES structures (Figure~\ref{fig:structure} \textbf{B} to \textbf{D}).
% 在它们之中,与预期相反地,1987年至1998年的流域SES结构下黄河的生态急剧恶化,表明了制度的失配。
The sharp and unintended decline in the ecological condition of the Yellow River from 1987 to 1998 indicates an institutional mismatch during this period (Figure~\ref{fig:structure} the shadowed time periods).
