
The YRB, cradle of Chinese civilization, locates in north-central China and crosses ten province-level regions whose socio-economic development highly depend on water from the Yellow River~\ref{fig:yrb}.
As semi-arid and arid region, YRB's annual precipitation varies from about $100$ to $1000~mm$ and increases from the northwest to the southeast, while annual pan evaporation varies from about $700$ to $1800~mm$~\cite{wang2022e}.
Taken together, YRB supports $35.63\%$ of China's irrigation and $30\%$ of its population while containing only $2.66\%$ of its water resources (data from \href{http://www.yrcc.gov.cn}{http://www.yrcc.gov.cn}, last access: \today).
Hence, over withdrawing water from the Yellow River became an urgent concern since the river began to dry up since early $1970s$.
Among the policies proposed for solving the problem, a series of water resource allocation institutions aimed to limit water use of each region into specific quotas, remarked as of the most important solutions.
However, few attempts quantitatively assessed how the YRB's water allocation scheme contributed to water governance, while other engineering solutions were carefully evaluated~\cite{long2020}.
% 出现报错是因为最后的大写结尾LaTeX无法判断是缩写还是一个句子结束,这里是句子结束所以加 \\
% https://tex.stackexchange.com/questions/55105/when-should-i-use-intersentence-spacing

% Despite the essential reason for these institutions was the mismatch between the spatial and temporal distribution of water resources as well as social and economic water demands, the direct reason for their introduction was the depletion of the Yellow River.

YRB was the first basin in China for which a water resource allocation institutions were created, and institutional shifts can be traced through several regulating documents released by the Chinese government (at the national level)\cite{wang2019a}:
(1) In $1980s$, the provinces and the Yellow River Water Conservancy Commission (YRCC) are required to develop a water resource plan for the Yellow River~\cite{wang2019, wang2019a}.
(2) In $1987$, implementation of the Allocation Plan. (\href{http://www.gov.cn/zhengce/content/2011-03/30/content_3138.htm#}{http://www.mwr.gov.cn}, last access: \today).
(3) In $1998$, implementation of unified regulation. (\href{http://www.mwr.gov.cn/ztpd/2013ztbd/2013fxkh/fxkhswcbcs/cs/flfg/201304/t20130411_433489.html}{http://www.mwr.gov.cn}, last access: \today).
% 各省按要求编制新的黄河流域水资源规划,将水资源额度分配进一步细化。
(4) In $2008$, provinces are asked to draw up new water resources plans for the YRB to further refine water allocations~\cite{wang2019,wang2019a}.
(5) In $2021$, a call for redesigning the water allocation institution (\href{http://www.ccgp.gov.cn/cggg/zygg/gkzb/202107/t20210721_16591901.htm}{http://www.ccgp.gov.cn}, last access: \today).

Our study period ranges from $1980$ (proposing the water quotas) to $2008$, when it started to completely establish a regulating system with quotas throughout basin, provincial, and district levels.
During the period, two significant institutional shifts can be analyzed by using documents of $1987$ (87-WAS) and $1998$ (98-UBR), which split the study period into three pieces: from 1980 to 1987 (before 87-WAS), from 1988 to 1997 (after 87-WAS and before 98-UBR), and from 1998 to 2007 (after 98-UBR).
% Those efforts led to ecological restoration of wetlands and the estuarine delta. Drying up has been avoided for over $20$ years, which is widely considered a substantial management achievement.
