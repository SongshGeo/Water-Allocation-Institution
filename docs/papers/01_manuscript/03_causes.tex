%! Author = songshgeo
%! Date = 2022/3/10

% \MakeUppercase{\subsection{Mechanism of institutional shift effects}}
\subsection{Mechanism of institutional shift effects}
\label{result-2}
% 结果2部分:
% 1. 加速用水增长的结果 for each province.
% 2. Potential benefits for each stakeholder (economic model - assumption & results), and social-ecological fit holistically. 结合经济模型
% 3. Why some building blocks fits for good outcomes. 结合 building block

Differences between stakeholders in responses to institutional shifts are vital to understanding the mechanism between structures and outcomes.
Our results show that the proportion of accelerated water use in each province after the decade of 87-WAS (the proportion of actual water use exceeding the predicted water use by the model) has a significant correlation ($p<0.05$, see \textbf{Methods}) to the Yellow River water use in each province (Figure~\ref{upset}A).
Furthermore, while no evident impacts for most provinces (no more than $10\%$ differences, the apparent acceleration effects were only prominent in the big water-using provinces (e.g., Neimeng, Henan, and Shandong. Figure~\ref{upset}B).
In particular, Neimeng and Shandong, both provinces that exceeded the prescribed water uses of the 87-WAS, used $44.25\%$ and $25.69\%$ more water uses than the prediction from 1987 to 1998, respectively.
Furthermore, the satisfaction of each province with the water allocation stipulated by the 87-WAS (expressed by the difference between the actual water allocation of each province and the expected planning value) is not significantly correlated with the acceleration.
By contrast, after the 98-UBR, with the exception of Shaanxi (which has always been abundant in water quota) had evident ($17.53\%$) increasd water use, almost all provinces have seen significant declines in water use ($-12.5\%$ on average).
Neither the satisfaction nor Yellow River water use, however, have correlation with the declines.

\begin{figure*}[!h]
    \centering
    \includegraphics[width=32pc]{outputs/upset.jpg}
    \caption{
        \textbf{A.} The partial correlation coefficient between wate uses (WU) of Yellow River (YR), unsatisfied ratio (compared with requirements in water plan and supply in the 87-WAS), and the average accelerated ratio.
        \textbf{B.} Average accelerated ratio of water uses for each province in the YRB during the decade after 87-WAS (from 1987 to 1998).
        \textbf{Mian users:} Major water consumption provinces (over the median).
        \textbf{Overused:} violate the 87-WAS in average water uses.
    }
    \label{upset}
\end{figure*}

% 经济模型与理论解释
%! 这里可以指出他们过去的经济发展路径
We analyzed mathematically why the mismatched structure made win-wins holistically elusive in the institution shift of 87-WAS (\textit{method} and \textit{Supplementary Material S4}).
Our model suggests that for users who are already economically efficient, greater marginal returns from water induce the acceleration of extracting resources for future economic growth (Figure~\ref{economic_model}).
Therefore, isolated stakeholders reacted variety to the similar marginal cost, and the institution of 87-WAS thus triggered an incentive distortion besides the initial intention of sustainable water use.
On the contrary, the presence of central management (by the YRCC in this case, after 1998) can effectively reduce marginal ecological costs as stakeholders take corresponding responsibilities to the YRCC (\textit{Supplementary Material S4}).
The alignments of differences in institutional structures and outcomes echo the hypothesis of successful governance in SES by indirectly (or vertically) creating links between different stakeholders (in the YRB cases, through administration).
%! citation

Authentic the water quota (or the initial water rights) in our case studies went through a stage of "bargaining" among stakeholders (from 1982 to 1987) \cite{wang2019a, wang2019d}, where each province attempted to demonstrate its development potential related to water use.
% 在水资源分配过程中,上下级决策者与利益相关者之间存在信息不对称,当前用水量越大的决策者议价能力越强。
With information asymmetry between upper-level decision-makers and lower-level stakeholders in water use allocation, those with more current water use might have greater bargaining power.
After 1987, the logical next step for provinces was to attempt to justify bargaining for larger quotas rather than immediately adopt resource-conserving transformations, as the trend of the water consumption intensity denotes (\textit{Supplementary Material S4}).
% 研究表明,如果仅考虑经济因素,用水量大户的体量与效率需要超出其配额的水资源
Studies show that the volume and efficiency of large water users need more water than their quota (in the 87-WAS) if only considering the economic factors when designing the institution.
In practice, therefore, although the affected provinces may not have directly encouraged excessive resource use because of the institutional shift, they had a larger incentive to show their economic potential
\cite{krieger1955, ostrom1990}.
As a result, while competing for potential water quotas, the provinces tended to hide the ecological costs behind economic development.

% 毫无疑问,随着资源竞争的日趋激烈,越来越多的SES正依赖着不同形式的制度进行资源分配(如自组织和政府干预),避免“短跑效应”的出现或将成为制度设计的关键。
With increasingly fierce competition for water, worldwide basins are developing institutions for governance (whether through self-organization or government intervention) \cite{andersson2020, wutich2009, cumming2020b}.
% 为了防止公共池资源被过度利用,总配额在强制禁止水资源过度利用的环境规制中发挥着重要作用,从而形成一个长期匹配的水资源分配机制。
Adopting an overall quota plays a vital role in preventing the overuse of Common-pool Resources, like exclusive and competitive water resources.
However, various outcomes following institutional shifts of YRB demonstrate that stakeholders may react to quota with distortion when pursuing returns isolated under mismatched SES structures.
% 历史也证明,在87-WAS之后,各省、尤其是用水大省均对提出了异议,并持续上诉以寻求更大的配额。
History also shows that, after 87-WAS, provinces, especially water-intensive ones, challenged it and continued to appeal for larger quotas.
% 而在用水许可制度之后,这种寻求更大配额的外在诉求转变成了提升用水效率的内部革新。
After YRCC as governing agent coordinated between stakeholders since 98-UBR, the external appeal of provinces for larger quotas turned into internal innovation to improve water efficiency.
% 与生态尺度相匹配的代理人在成功治理的SES中作为 motif 反复出现,无论是在渔业、森林、还是在地下水治理中,表明减少与割裂生态fragment直接相连的独立利益相关者是结构产生良好效果的重要普遍性机制。
Agents matching the ecological scale appear widespread as motifs in SES of successful governance, whether in fisheries, forests, or groundwater management, suggesting that reducing independent stakeholders linked to fragmentation is an essential primary mechanism for a structure to produce good results.
