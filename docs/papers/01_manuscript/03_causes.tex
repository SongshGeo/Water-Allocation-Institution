%! Author = songshgeo
%! Date = 2022/3/10

% \MakeUppercase{\subsection{Mechanism of institutional shift effects}}
\subsection{Mechanism of institutional effects}
\label{result-2}
% 结果2部分:

Differences between stakeholders in responses to institutional shifts are vital to understanding the mechanism between structures and outcomes.
Our results show that the proportion of accelerated water use in each province after the decade of 87-WAS (the proportion of actual water use exceeding the predicted water use by the model) has a significant correlation ($p<0.05$, see \textbf{Methods}) to the Yellow River water use in each province (Figure~\ref{upset}A).
Furthermore, while no evident impacts for most provinces (no more than $10\%$ differences, the apparent acceleration effects were only prominent in the big water-using provinces (e.g., Neimeng, Henan, and Shandong. Figure~\ref{upset}B).
In particular, Neimeng and Shandong, both provinces that exceeded the prescribed water uses of the 87-WAS, used $44.25\%$ and $25.69\%$ more water uses than the prediction from 1987 to 1998, respectively.
Furthermore, the satisfaction of each province with the water allocation stipulated by the 87-WAS (expressed by the difference between the actual water allocation of each province and the expected planning value) has little significant correlation to the acceleration.
By contrast, after the 98-UBR, except Shaanxi (which has always been abundant in water quota) had an evident ($17.53\%$) increase in water use, almost all provinces have seen significant declines in water use ($-12.5\%$ on average).
However, neither the satisfaction nor Yellow River water use correlates with the declines after the 98-UBR.

\begin{figure*}[!h]
    \centering
    \includegraphics[width=32pc]{outputs/upset_87.pdf}
    \caption{
        \textbf{A.} The partial correlation coefficient between wate uses (WU) of Yellow River (YR), unsatisfied ratio (compared with requirements in water plan and supply in the 87-WAS), and the average accelerated ratio.
        \textbf{B.} Average accelerated ratio of water uses for each province in the YRB during the decade after 87-WAS (from 1987 to 1998).
        \textbf{Mian users:} Major water consumption provinces (over the median).
        \textbf{Overused:} violate the 87-WAS in average water uses.
    }
    \label{upset}
\end{figure*}

% 经济模型与理论解释
Differences in the pattern of the response by provinces can demonstrate the influence of social-ecological structures led by the institutional shifts.
We analyzed mathematically why the mismatched structure made limited water use holistically elusive in the institution shift of the 87-WAS but finally achieved by the 98-UBR (\textit{method} and \textit{Supplementary Material S4}).
By taking the structure before and after the two institutional shifts as different basic assumptions (before 87-WAS: free access to water; after 87-WAS but before 98-UBR: decisions on water use under quotas; after 98-UBR: unified regulation), we use the marginal benefit model to analyze the theoretical optimal water consumption of stakeholders in each scenario.
The analysis of the model also shows that 98-UBR can reduce the overall water use of the basin while 87-WAS can increase the water use of the basin when the same parameters are guaranteed but the institutional structure changes.
Before the 98-UBR, the model assumes that the separated ecological units (river reaches) link to stakeholders (related provinces) who use water to pursue their marginal benefits but have a potential political cost if they exceed the quota 87-WAS.
Our model suggests that for users who are already economically efficient (who are already using more water), greater marginal returns from water induce the acceleration of extracting resources for future economic growth (Figure~\ref{economic_model}).
Therefore, isolated stakeholders reacted to the similar marginal cost, and smaller water users have a threshold because of the political cost, so 87-WAS triggered an increased water use for the significant users.
On the contrary, the presence of central management (by the YRCC in this case, after 1998) can effectively reduce marginal ecological costs holistically as stakeholders only take corresponding responsibilities (follow the quota as possible as they can) to the YRCC (\textit{Supplementary Material S4}).
As a result, unified regulating acted the core role after the 98-UBR and reduced water use of all stakeholders (provinces) by irregular ratios.

The alignments of differences in institutional structures and outcomes here echo the hypothesis that successful governance of SES emerged by indirectly (or vertically) creating links between different stakeholders (in the YRB cases, through administration).
When links The water quotas of 87-WAS (or the initial water rights) in our case studies went through a stage of ``bargaining'' among stakeholders (from 1982 to 1987) \cite{wang2019a, wang2019d}, where each province attempted to demonstrate its development potential related to water use.
The bargaining itself was also a process towards matches between their economic volume and water shares, as studies show that the large water users (like Shandong and Henan) need more water than their quota (in the 87-WAS) if only considering the economic equity when designing the institution.
Furthermore, with information asymmetry between upper-level decision-makers and lower-level stakeholders in water use allocation, those with more current water use might have greater bargaining power.
In practice, therefore, although the affected provinces may not have directly encouraged excessive resource use because of the institutional shift, they had a more considerable incentive to show their economic potential
That aligns with the historical records that, even after the 87-WAS had already confirmed the quotas, provinces, especially water-intensive ones, challenged it by appearing to the higher central government for larger quotas.
On the contrary, after YRCC as governing agent coordinated between stakeholders since 98-UBR, the external appeal of provinces for larger quotas turned into internal innovation to improve water efficiency (e.g., drastically increased water-conserving equipment, \textit{Supplementary Material S3})
\cite{krieger1955, ostrom1990}.
Then, the YRCC, the authority for approving water applications from all stakeholders, could adjust water use quotas according to the river conditions of the whole basin.
The 98-UBR led to a structure for achieving social-ecological fits in both basins (between YRCC and the YRB) and regions (between provincial economy and their water shares).
