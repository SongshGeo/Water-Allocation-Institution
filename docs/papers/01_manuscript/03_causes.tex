%! Author = songshgeo
%! Date = 2022/3/10

% \MakeUppercase{\subsection{Mechanism of institutional shift effects}}
\subsection{Mechanism of institutional shift effects}
\label{reason}
Next, we explored the mechanisms linking the structures and the outcomes.
% 87-WAS 后各省响应的差异是理解制度影响机制的关键
Differences in provincial responses to 87-WAS are vital to understanding the mechanism of institutional shifts' impacts.
% 我们的结果表明,各省加速用水的比例(实际用水量超出模型预测用水量的比例)与各省黄河水取用量呈现显著正相关
Our results show that the proportion of accelerated water consumption in each province (the proportion of actual water consumption exceeding the predicted water consumption by the model) has a significant correlation to the Yellow River water consumption in each province (Figure~\ref{upset}A).
% 但明显的加速效应仅突出体现在主要用水省份上,对大多数省份来说,87-WAS 带来的影响并不大,
However, the apparent acceleration effect of the 87-WAS was only prominent in the significant water-using provinces (Neimeng, Henan, and Shandong), and there were no evident impacts for most provinces (Figure~\ref{upset}B).
% 尤其是常年超指标用水的山东与内蒙,分别在1987到1998年的十年间使用了高于模型预测值的xx%和xx%的用水量。
In particular, Shandong and Inner Mongolia, both provinces that exceeded the prescribed water uses of the 87-WAS, used xx\% and XX\% more water uses than predicted by the model from 1987 to 1998, respectively.

\begin{figure*}[!h]
    \centering
    \includegraphics[width=32pc]{outputs/upset.jpg}
    \caption{
        \textbf{A.} The partial correlation coefficient between wate uses (WU) of Yellow River (YR), unsatisfied ratio (compared with requirements in water plan and supply in the 87-WAS), and the average accelerated ratio.
        \textbf{B.} Average accelerated ratio of water uses for each province in the YRB during the decade after 87-WAS (from 1987 to 1998).
        \textbf{Mian users:} Major water consumption provinces (over the median).
        \textbf{Overused:} violate the 87-WAS in average water uses.
    }
    \label{upset}
\end{figure*}

We analyzed mathematically why the mismatched structure made win-wins holistically elusive in the institution shift of 87-WAS (\textit{method} and \textit{Supplementary Material S4}).
% 我们的模型表明,对于已经具有经济效率的用户来说,为了未来的经济增长而加速开采资源是由更大的边际收益引起的
Our model suggests that for users who are already economically efficient, greater marginal returns from water induce the acceleration of extracting resources for future economic growth (Figure~\ref{economic_model}).
% 因此,这种制度引发了一种与可持续用水意图背道而驰的激励扭曲。
Therefore, isolated stakeholders reacted variety to the similar marginal cost, and the institution of 87-WAS thus triggered an incentive distortion besides the initial intention of sustainable water use.
% 相反,中央管理(1998年以后的YRCC)能够有效降低边际生态成本
On the contrary, the presence of central management (by the YRCC in this case, after 1998) can effectively reduce marginal ecological costs as stakeholders take corresponding responsibilities to the YRCC (\textit{Supplementary Material S4}).
% 制度结构和结果的差异呼应了SES成功治理的前景,通过间接(或垂直)在不同的利益相关者之间建立联系(在这种情况下是通过管理)
The alignments of differences in institutional structures and outcomes echo the hypothesis of successful governance in SES by indirectly (or vertically) creating links between different stakeholders (in the YRB cases, through administration).
%! citation

\begin{figure*}[!ht]
    \centering
    \includegraphics[width=24pc]{outputs/economic_model.jpg}
	\caption{
		% \textbf{Assumption 1:} \textit{(Production)} Assuming that water is the only input of the homogenous production function F(x) of each province. Under diminishing marginal returns assumption, and $F(x)$ is continuous, $F'(0)=\infty$, $ F'(\infty)=0$. The production output is under perfect competition, with constant unit price of P.
		% \textbf{Assumption 2:} \textit{(Cost function)} Assuming that the ecology is a unity for the whole basin, the cost of water use is equally assigned to each province under any water use. The unit cost of water is a constant C.
		% \textbf{Assumption 3:} \textit{(Multi-period setting)} There are infinite periods with constant discount factor $\beta$ lying in (0,1) with no cross-period smoothing in water uses.
		\textbf{A.} The relationship of marginal benefits and water use of province i at t = 0 for three different cases (case 1 to case 3, corresponding to the different SES structures in Figure~\ref{structure}, assuming $F(x)=ln(1+x)$, $N=8$, $P=1$, $C=0.5$, and $\beta=0.4$ as an example  (see \textit{Methods}In Case 3, water use by others is taken as a given, equal to the optimal water use for Case 2. The horizontal coordinate of each intersection of marginal benefits and the break-even line represents the optimal water use under each case.
		\textbf{Panel B.} The relation between optimal water use of province i and total quota for Case 3, under time horizon of $T=5$, $T=10$, and an infinite $T$, respectively. The settings are the same as in \textbf{A}.
	}
	\label{economic_model}
\end{figure*}


Authentic the water quota (or the initial water rights) in our case studies went through a stage of "bargaining" among stakeholders (from 1982 to 1987) \cite{wang2019a, wang2019d}, where each province attempted to demonstrate its development potential related to water use.
% 在水资源分配过程中,上下级决策者与利益相关者之间存在信息不对称,当前用水量越大的决策者议价能力越强。
With information asymmetry between upper-level decision-makers and lower-level stakeholders in water use allocation, those with more current water use might have greater bargaining power.
After 1987, the logical next step for provinces was to attempt to justify bargaining for larger quotas rather than immediately adopt resource-conserving transformations, as the trend of the water consumption intensity denotes (\textit{Supplementary Material S4}).
% 研究表明,如果仅考虑经济因素,用水量大户的体量与效率需要超出其配额的水资源
Studies show that the volume and efficiency of large water users need more water than their quota (in the 87-WAS) if only considering the economic factors when designing the institution.
In practice, therefore, although the affected provinces may not have directly encouraged excessive resource use because of the institutional shift, they had a larger incentive to show their economic potential
\cite{krieger1955, ostrom1990}.
As a result, while competing for potential water quotas, the provinces tended to hide the ecological costs behind economic development.

% 毫无疑问,随着资源竞争的日趋激烈,越来越多的SES正依赖着不同形式的制度进行资源分配(如自组织和政府干预),避免“短跑效应”的出现或将成为制度设计的关键。
With increasingly fierce competition for water, worldwide basins are developing institutions for governance (whether through self-organization or government intervention) \cite{andersson2020, wutich2009, cumming2020b}.
% 为了防止公共池资源被过度利用,总配额在强制禁止水资源过度利用的环境规制中发挥着重要作用,从而形成一个长期匹配的水资源分配机制。
Adoption of an overall quota plays a vital role in preventing overuse of Common-pool Resources, like exclusive and competitive water resources.
However, various outcomes following institutional shifts of YRB demonstrates that stakeholders may react to quota with distortion when pursuing returns isolated under mismatched SES structures.
% 历史也证明,在87-WAS之后,各省、尤其是用水大省均对提出了异议,并持续上诉以寻求更大的配额。
History also shows that, after 87-WAS, provinces, especially water-intensive ones, challenged it and continued to appeal for larger quotas.
% 而在用水许可制度之后,这种寻求更大配额的外在诉求转变成了提升用水效率的内部革新。
After YRCC as governing agent coordinated between stakeholders since 98-UBR, the external appeal of provinces for larger quotas turned into internal innovation to improve water efficiency.
% 与生态尺度相匹配的代理人在成功治理的SES中作为 motif 反复出现,无论是在渔业、森林、还是在地下水治理中,表明减少与割裂生态fragment直接相连的独立利益相关者是结构产生良好效果的重要普遍性机制。
Agents matching the ecological scale appear widespread as motifs in SES of successful governance, whether in fisheries, forests, or groundwater management, suggesting that reducing independent stakeholders linked to fragmentation is an essential primary mechanism for a structure to produce good results.
