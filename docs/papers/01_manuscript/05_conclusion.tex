%! Author = songshgeo
%! Date = 2022/3/10

In this investigation of the Yellow River Basin (YRB), we meticulously examined the impacts of two institutional shifts in water governance: the 1987 Water Allocation Scheme (87-WAS) and the 1998 Unified Basin Regulation (98-UBR).
Utilizing the Differenced Synthetic Control (DSC) approach, we were able to quantify the discrete effects of these transitions on water consumption within the basin.
Our findings suggest a paradoxical increase in water use by 5.75\%, attributed to the 87-WAS, defying its original objectives.
Conversely, the 98-UBR efficaciously diminished water usage in line with its intended outcomes.
This analysis unearthed the pivotal role that institutional structural patterns play in determining their efficacy.
Specifically, the misaligned structure of 87-WAS inadvertently fostered increased rivalry and exploitation of water resources.
Meanwhile, the 98-UBR, characterized by its scale-matched, basin-wide coordination and reinforced stakeholder connections, fostered restoration of the Yellow River.

In sum, our study sheds new light on the complex dynamics of institutions within socio-ecological systems (SES) governance, with an emphasis on water allocation.
By unraveling the essential components that govern the triumph or downfall of institutional transformations, we furnish invaluable insights that can guide the crafting of sustainable water governance policies.
These findings beckon further exploration into the multifaceted nature of institutional behavior in SES governance, and how future policy adjustments and institutional metamorphoses might sculpt the efficiency of water utilization and sustainability.
