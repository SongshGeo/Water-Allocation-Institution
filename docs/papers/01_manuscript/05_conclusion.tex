%! Author = songshgeo
%! Date = 2022/3/10

Intense water use in one of the most anthropogenically altered large river basins, the Yellow River Basin (YRB), once led to drying up.
Alterations of institutions eventually successfully restored water governance practices on a decadal time scale.
We propose that the institutional shifts in the YRB (87-WAS and 98-UBR) framed two different SES structures and depicted them as widespread building blocks.
We quantitatively estimate the net effects of these changes in the YRB and analyze the reasons from SES structural perspectives.
Our results show that the historical records, the responses from stakeholders to structural changes, and the theoretical analysis from the marginal benefits model all support that fragmented ecological units linked to separate social actors frames a mismatched SES structure.
Through the quasi-natural experiments of the YRB, we demonstrate that social-ecological fits can lead to successful SESs management worldwide with better sustainability outcomes.
