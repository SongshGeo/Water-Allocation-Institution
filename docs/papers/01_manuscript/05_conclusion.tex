%! Author = songshgeo
%! Date = 2022/3/10

In this study, we examined the effects of two major institutional shifts in water governance within the Yellow River Basin (YRB): the 1987 Water Allocation Scheme (87-WAS) and the 1998 Unified Basin Regulation (98-UBR). By employing a Difference-in-Differences approach with Synthetic Control, we quantified the net effects of these institutional shifts on water use within the YRB.\
Our results showed that the 87-WAS unexpectedly increased water use by $5.75\%$, contrary to its intended goals, while the 98-UBR successfully reduced water use as anticipated. The analysis revealed that the structural patterns of the institutions played a critical role in their effectiveness. The mismatched structure of the 87-WAS led to increased competition and exploitation of water resources, while the 98-UBR, with its scale-matched, basin-wide authority and stronger connections between stakeholders, resulted in improved water governance.

In conclusion, our research contributes to a better understanding of the role of institutions in SES governance, particularly in the context of water management. By identifying the key factors that influence the success or failure of institutional shifts, we provide valuable insights for the design of effective and sustainable water governance policies. Future research should continue to explore the intricacies of institutions in SES governance and investigate the potential impacts of additional policies and institutional shifts on water use and sustainability.
