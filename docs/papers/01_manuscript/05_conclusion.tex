%! Author = songshgeo
%! Date = 2022/3/10

Intense water use in one of the most anthropogenic interfered large river basins, the Yellow River Basin (YRB), once led to overburdened drying up but finally had a successful restoration by sequential water governance practices.
Focusing on two water-demand institutions, 87-WAS and the 98-UBR, we quantitatively analyzed how institutional shifts played a role in the water governance achievement of the YRB.
Shifting throughout different SES structures framed by them, the observed water use of the YRB provinces had an $8.57\%$ increase than expected during the decade after the 87-WAS.
Then, water use significantly decreased by  $4.9$ billions $m^3$ per year since the 98-UBR, while the model still suggests a $10.3$ billions $m^3$ annual increase in expectation.
Finally, as differences in stakeholders' response to the institutional shifts, water use rises after the 87-WAS in provinces with more water uses (e.g., Inner Mongolia, Henan, and Shandong) while shrunk in nearly all provinces after the 98-UBR.
Since the above results closely align with interpretations from a mathematical marginal benefits model, we can link the structures (widespread building blocks) and outcomes (goals of the institution, i.e., limiting water demands) by these quasi-natural experiments of the YRB.
We demonstrate that social-ecological fits lead to successful governance where reducing independent stakeholders linked to fragmentation is an essential primary mechanism for good SES outcomes.
