%! Author = songshgeo
%! Date = 2022/3/10
Our study emphasizes the importance of social-ecological system (SES) structures in effective water resource management. By examining the Yellow River Basin (YRB) in China as a case study, we found that the 1987 Water Allocation Scheme (87-WAS) led to an 5.75\% increase in observed water use, while the 1998 Unified Basinal Regulation (98-UBR) effectively reduced total water use. Our findings highlight the critical role of well-designed SES structures in driving water resource management outcomes. To achieve sustainability, we recommend avoiding fragmented ecological units linked to separated social actors and underscore the need for a deeper understanding of the underlying processes and relationships in SES structures for effective water governance.
