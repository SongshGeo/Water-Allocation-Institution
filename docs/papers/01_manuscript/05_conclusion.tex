%! Author = songshgeo
%! Date = 2022/3/10

Intense water use in one of the most anthropogenic interfered large river basins, the Yellow River Basin (YRB), once led to overburdened drying up but finally had a successful restoration by sequential water governance institutions.
Focusing on two water-demand institutions, the Water Allocation Scheme since 1987 (87-WAS) and the Unified Basinal Regulation since 1998 (98-UBR), we quantitatively analyzed how and how important the institutional shifts played a role in the water governance achievement of the YRB.

First, by abstracting institutions into building blocks of a social-ecological system (SES), we explored the linkages within the YRB before and after two institutional shifts.
Then, we applied Differenced Synthetic Control method to quantify their impacts on water use of the YRB.
The observed water use of the YRB provinces had an $8.57\%$ increase than our estimation in the decade after the 87-WAS but significantly decreased by $0.49 km^3$ per year after the 98-UBR (the model still suggests a $1.03 km^3$ annual increases).
Finally, by analyzing the differences in stakeholders' responses to the institutional shifts, we found that 87-WAS stimulated water use in provinces with more water uses (e.g., Neimeng, Henan, and Shandong) 98-UBR regulated nearly all provinces.

Since the above results closely align with our mathematical marginal benefits model, we can link the structures (widespread building blocks) and outcomes (goals of the institution, i.e., limiting water demands) by the quasi-natural experiments of the YRB.
We demonstrate again that social-ecological fits lead to successful governance, whether in fisheries, forests, or groundwater management; reducing independent stakeholders linked to fragmentation is an essential primary mechanism for a structure to produce good results.
Therefore, insights from the YRB can be a valuable guideline for SESs worldwide facing similar water governance problems.
