drafs.tex

\section{introduction}
% At the same time, undesired and unsustainable outcomes (e.g., failures in environmental regulation of highly polluting industries or the development of “tragedy of the commons” situations when pursuing more water resources), with considerable ecological degradation, have attracted much attention \cite{saylesSocialecologicalnetworkanalysis2019,caiPollutingthyneighbor2016,castilla-rhoSocialtippingpoints2017}.
Two particular weaknesses in existing knowledge include understanding (1) the causal links between SES structures and outcomes; and (2) details of the underlying processes, especially the coordination of the incentives of different participants, that result from an institutional lack of fit. These weaknesses limit our understanding of institutional design, and they may reduce the speed and transfer of new knowledge and experience related to improving the sustainability of comprehensive water resources management.


% In order to disentangle the relationship between SES structure and outcomes, we analyzed a case study to show how an institutional shift led to a structural mismatch that triggered unsustainable water use and unintended ecological deterioration.
% Water is the key exclusive and competitive resource that couples socioeconomic and ecological systems (known as common-pool resources, CPRs) \cite{wutichWaterScarcitySustainability2009,ostromGeneralFrameworkAnalyzing2009,castilla-rhoSocialtippingpoints2017,castilla-rhoGroundwaterCommonPool2020}.
% Conflicts of interest often occur in allocation and competitions of water resources, with water governance policies often leading to long-term changes in human–water relationships and the redistribution of benefits \cite{wangAlignmentsocialecological2019,speedBasinwaterallocation2013,chingManagingsocioecologyvery2015}.

% Two rapid shifts in institutional structure that occurred in 1987 and 1998 (see Supplementary Material S1) provide unique settings to exploit quasi-natural experiments of a large river basin, the Yellow River Basin (YRB) in China \cite{xiaDevelopmentWaterAllocation2012}.

% After a period of severe drying up, an institutional shift implemented in 1987 represented the beginning of attempts to control water use in the YRB through the use of quotas, with the goals of alleviating conflicts between supply and demand and achieving sustainable development.

% Our results show that this initiative actually accelerated water withdrawals, resulting in an unintended “sprint effect”, where institutional mismatches created an even stronger incentive for each resource user to withdraw resources until the next major institutional shift in 1998.
% Our analysis contributes to a deeper understanding of the mechanisms underlying the relationships among institutions, SES structures, and outcomes. By highlighting potential concerns for ecosystem collapse under structural mismatches, our findings are consistent with the urgent calls for a more dynamic design for water use allocation to achieve sustainability.

\section{results}
% % 然后,我们反复比较了两个SES结构(分别自1987年和1998年),以量化是否有显著不同的影响,YRB用水或分配。
% Then we repeatedly compared between two SES structures (since 1987 and 1998, respectively) to quantify if there were significantly different impacts on water uses or allocation (\textbf{methods}).



\section{discussion}
% 我们展示了不匹配的分配制度是如何导致激励失真导致水资源加速耗竭的(即“冲刺效应”)。“冲刺效应”是CPR系统面临的一种特殊情况,在这种情况下,制度的不匹配为每个资源使用者创造了更强的动机(扭曲),促使他们收回资源
We have shown how a mismatched allocation institution can lead to an accelerated depletion of water resources (i.e., the “sprint effect”) caused by incentive distortion. The sprint effect is a special case faced by CPR systems, where institutional mismatches create an even stronger incentive (with distortion) for each resource user to withdraw resources
\cite{ostromRevisitingCommonsLocal1999,ostromGeneralFrameworkAnalyzing2009,castilla-rhoSocialtippingpoints2017}.
% 过往研究指出制度常常是避免公共池塘资源系统的崩溃的关键,但“短跑效应”的出现表明在自上而下进行制度设计所形成的错配SES结构中,制度也可以成为系统加速崩溃的触发者。
Previous studies have suggested that institutions are often the key to avoid the collapse of a CPR system, but the emergence of a sprint effect shows that an institution with structural mismatches can also be the trigger that accelerates system collapse \cite{bodinConservationSuccessFunction2014,bodinCollaborativeenvironmentalgovernance2017,wangAlignmentsocialecological2019}.
The initial formulation of the water quota in our case studies went through a stage of “bargaining” among stakeholders (from 1982 to 1987) \cite{wangReviewImplementationYellow2019, wangThingsCurrentSignificance2019}, where each province attempted to demonstrate its development potential related to water use.
% 为了防止公共池资源被过度利用,总配额在强制禁止水资源过度利用的环境规制中发挥着重要作用,从而形成一个长期匹配的水资源分配机制。
Adoption of an overall quota plays an important role in preventing overuse of CPRs \cite{tilmanLocalizedprosocialpreferences2019}.
However, the negative effects of incentive distortion imply a trade-off between long-term SES benefits and current stability, and the proportion of available resources allocated under quota schemes matters when institutions change \cite{ladeRegimeshiftssocialecological2013}.
According to our analysis of plausible scenario assumptions based on our general economic model, the sprint effect will be reinforced when stakeholders anticipate that technological advances will amplify the benefits of water quotas in the future (see \textit{Supplementary Material S3}).
% 然而,如果有水权转换机制允许利益相关者之间通过交易来弥补 shadow value,当前这种错乱的动机就不会那么强。
However, if an institution allowed stakeholders to compensate for the shadow value (i.e., potential returns sacrificed due to water constraints and water scarcity) \cite{howarthAccountingvalueecosystem2002} of future water use, incentive distortion would be less devastating (e.g., through water rights transfer).
Policymakers can also weaken the sprint effect by increasing the frequency of quota updates, supporting the idea that a more dynamic institution that responds to changing conditions (see \textit{Supplementary Material S3}) will adapt more effectively to its social-ecological context.

% 近年来黄河流域面临的分水制度调整问题也说明了动态设置配额的重要性。
Calls for a redesign of water allocation institutions in the YRB in recent years also illustrate the importance of dynamic quota setting (see \textit{Supplementary Material S1}) \cite{yuAdaptabilityassessmentpromotion2019}. Following the institutional reforms of 1998, the Yellow River has not dried up since 1999. However, given recent changes in the YRB, its rigid resource allocation scheme can no longer meet the new demands of economic development \cite{wangThingsCurrentSignificance2019}. The Chinese government has embarked on an ambitious plan to redesign its decades-old water allocation institution (see \textit{Supplementary Material S1}). Other SESs around the world face similar problems in establishing successful resource allocation institutions \cite{cummingQuantifyingSocialEcologicalScale2020, muneepeerakulStrategicbehaviorsgovernance2017, cummingAdvancingunderstandingnatural2020, leslieOperationalizingsocialecologicalsystems2015}. These initiatives can benefit from our analysis by actively considering and incorporating social-ecological complexity and incentive structures when developing new approaches that avoid unsustainable outcomes. Our research provides a cautionary tale of how institutions can act as a double-edged sword when trying to attain sustainability.


% 我们展示了不匹配的分配制度是如何导致激励失真导致水资源加速耗竭的(即“冲刺效应”)。“冲刺效应”是CPR系统面临的一种特殊情况,在这种情况下,制度的不匹配为每个资源使用者创造了更强的动机(扭曲),促使他们收回资源
\cite{ostromRevisitingCommonsLocal1999,ostromGeneralFrameworkAnalyzing2009,castilla-rhoSocialtippingpoints2017}.
% 过往研究指出制度常常是避免公共池塘资源系统的崩溃的关键,但“短跑效应”的出现表明在自上而下进行制度设计所形成的错配SES结构中,制度也可以成为系统加速崩溃的触发者。
Previous studies have suggested that institutions are often the key to avoiding the collapse of a CPR system, but the emergence of a sprint effect shows that an institution with structural mismatches can also be the trigger that accelerates system collapse \cite{bodinConservationSuccessFunction2014,bodinCollaborativeenvironmentalgovernance2017,wangAlignmentsocialecological2019}.


##
% Linking structures to outcomes need advancement when understanding an SES. Two particular weaknesses in existing knowledge include understanding (1) the causal links between SES structures and outcomes; and (2) details of the underlying processes, especially the coordination of the incentives of different participants, that result from an institutional lack of fit. These weaknesses limit our understanding of institutional design, and they may reduce the speed and transfer of new knowledge and experience related to improving the sustainability of comprehensive water resources management.

The adverse effects of incentive distortion imply a trade-off between long-term SES benefits and current stability, and the proportion of available resources allocated under quota schemes matters when institutions change \cite{lade2013}


% 在我们的研究期间(从1979年到2008年),机构在三种不同的结构之间转换:免费访问、87-WAS和98-UBR。
In our study period, the institution shifted between three different structures (Figure~\ref{structure} A to C).
% 长江水利委员会是长江流域的主要负责人,其任务是报告和分析1998年以前长江流域的用水量(87-WAS之后)。
However, the YRCC, whose primary official response to the river before 1998, the mandate was to conserve the riverway environment, construct and maintain infrastructures, and report on and analyze water consumption (after the 87-WAS) in the YRB, i.e., connecting the ecological nodes (different river reaches) horizontally ~\cite{wang2019a}.


Water governance tends to shift to institutional solutions within a complex basinal system, where societal drivers impact through water use and related technical interventions \cite{fischer2020}.

% YR水治理机构的转变重构了人与水之间的相互作用,并产生了长期的级联效应,为理解这种相互作用留下了两个准自然实验。
% 利用差分综合控制方法,对制度变迁的净效应进行了分析,结果表明:制度不匹配促进了98-WAS后资源使用者的取水速度。
% 由于很少有大型河流流域经历过如此剧烈的结构变化,我们对年度制度变迁的定量分析,通过将自然和人类干预脱钩,促使我们对水的可持续治理有了宝贵的理解。
% % !这还要继续凝练我们的贡献
% 通过强调结构不匹配下生态系统崩溃的潜在担忧,我们的研究结果与迫切需要为水治理机构提供更动态的设计以实现可持续性的呼吁相一致。
The shifts in the water governance institution of YR refactored the interplays between humans and water with long-term cascading effects, leaving two quasi-natural experiments for understanding the interactions.
By Differenced Synthetic Control method, our analysis of the net effects of institutional shift shows that institutional mismatches contributed to the acceleration of water withdrawals for resource users after 98-WAS.
As few large river basins have experienced such radical structural changes several times, our quantitative analysis of institutional shifts in the YR induces a valuable understanding of water sustainable governance by decoupling natural and human interferences.
%! 这还要继续凝练我们的贡献
By highlighting potential concerns for ecosystem collapse under structural mismatches, our findings align with the urgent calls for a more dynamic design for water governance institutions to achieve sustainability.


% 毫无疑问,随着资源竞争的日趋激烈,越来越多的SES正依赖着不同形式的制度进行资源分配(如自组织和政府干预),避免“短跑效应”的出现或将成为制度设计的关键。
With increasingly fierce competition for water, worldwide basins are developing institutions for governance (whether through self-organization or government intervention) \cite{andersson2020, wutich2009, cumming2020b}.
% 为了防止公共池资源被过度利用,总配额在强制禁止水资源过度利用的环境规制中发挥着重要作用,从而形成一个长期匹配的水资源分配机制。
Adopting an overall quota plays a vital role in preventing the overuse of Common-pool Resources, like exclusive and competitive water resources.
However, various outcomes following institutional shifts of YRB demonstrate that stakeholders may react to quota with distortion when pursuing returns isolated under mismatched SES structures.


## methods
\subsection{Placebo Test}
% 作为一种稳健性检验,安慰剂测试的必要性体现在两个原因上。
For robustness, we conducted a placebo test because the synthetic control method neglects the influences of overall changes in factors in the same year by simply dividing time periods according to institutional shifts. Three steps were required to apply the placebo test:
% (1) 对目标组中的每个省份$i$,计算所有潜在控制组与它之间特征向量的欧氏距离
(1) For each province in the target group, we calculated the Euclidean distance of vectors between all provinces in the potential control group.
% (2) 将距离由小到大排序后,选取特征向量最相似的三个省份,所有特征的均值作为 $i$ 的替代
(2) After ranking the distances, the three provinces with the most similar economic context were used to generate an average paired treatment target unit.
% (3) 对这个配对目标同样实施控制合成法(潜在控制组排除构成它的三个省)
(3) We performed the same synthetic control analysis for this paired target (i.e., the potential control group excluding the three provinces in step 2).
% 经过上述步骤,我们在理论上构建了一个相似的“区域”并实施了同样的控制合成实验。
In this way, we theoretically constructed a pseudo-treated unit and performed the same synthetic control treatments. Because these placebo tests were directed at units unaffected by the institutional shifts, the results can be regarded as a reasonable baseline expectation or null model from which to assess the changes caused by other factors.


%\begin{figure}
%	\centering
%	\includegraphics[width=16pc]{diagrams/framework.jpg}
%	\caption{
%		Framework for understanding linkages between SES structures and outcomes.
%		\textbf{a.} The general framework for analyzing social-ecological systems (SESs) (adapted from Ostrom \cite{ostromGeneralFrameworkAnalyzing2009}). Institutions embedded in SESs may reshape structures by changing the interactions between core subsystems, resulting in different outcomes.
%        Three typical types of abstracted SES structures are shown as \textbf{b.}, \textbf{c.} and \textbf{d.} (adapted from Bodin, 2017)\cite{bodinCollaborativeenvironmentalgovernance2017}. Red circles indicate social actors, and green ones indicate ecological components. Connection (ties between two ecological components), collaboration (ties between two social actors), or management (ties between a social actor and an ecological component) exist when two units are linked by gray lines. The gray dashed lines show aligned SES structures that are more likely to result in a desirable outcome according to empirical evidence.
%	}
%    \label{framework}
%\end{figure}
