drafs.tex

\section{introduction}
% At the same time, undesired and unsustainable outcomes (e.g., failures in environmental regulation of highly polluting industries or the development of “tragedy of the commons” situations when pursuing more water resources), with considerable ecological degradation, have attracted much attention \cite{saylesSocialecologicalnetworkanalysis2019,caiPollutingthyneighbor2016,castilla-rhoSocialtippingpoints2017}.
Two particular weaknesses in existing knowledge include understanding (1) the causal links between SES structures and outcomes; and (2) details of the underlying processes, especially the coordination of the incentives of different participants, that result from an institutional lack of fit. These weaknesses limit our understanding of institutional design, and they may reduce the speed and transfer of new knowledge and experience related to improving the sustainability of comprehensive water resources management.

% 机构可以塑造社会经济体系的结构,对其进行抽象是机械地理解结构和结果之间联系的第一步。
Because institutions may shape the structure of SESs, describing institutional structure is a first step toward understanding the mechanisms linking structures and outcomes in SESs (Figure~\ref{framework} A).
% 例如,机构可以创建一种被确定为与良好的社会经济效益相关联的水平匹配结构,如果它鼓励管理相连生态成分的不同行动者之间的协作(图1B)。
For example, institutions may create a structure that encourages collaboration between the different actors managing connected ecological components (Figure~\ref{framework} B), leading to sustainable outcomes.
% 例如,机构可以创建一个结构,鼓励不同行为者之间的合作,管理相连的生态成分
Similarly, institutions for vertical management may enhance multi-layered SES matching by coordinating horizontal relationships (Figure~\ref{framework} C and D).

% 从国家官员到流域管理官员,从省、市到区,长江流域自上而下的制度结构开始演变至今
Including the national authorities, the basin management authorities, provinces, cities, and even districts, top-down institutional structures of the YRB started to evolve up to now (\textit{S1 in Supplementary Material}).

Our analysis period, therefore, spans from 1975 to 2008, with the human-water system shifted between three different institutional structures (Figure~\ref{structure}).


% In order to disentangle the relationship between SES structure and outcomes, we analyzed a case study to show how an institutional shift led to a structural mismatch that triggered unsustainable water use and unintended ecological deterioration.
% Water is the key exclusive and competitive resource that couples socioeconomic and ecological systems (known as common-pool resources, CPRs) \cite{wutichWaterScarcitySustainability2009,ostromGeneralFrameworkAnalyzing2009,castilla-rhoSocialtippingpoints2017,castilla-rhoGroundwaterCommonPool2020}.
% Conflicts of interest often occur in allocation and competitions of water resources, with water governance policies often leading to long-term changes in human–water relationships and the redistribution of benefits \cite{wangAlignmentsocialecological2019,speedBasinwaterallocation2013,chingManagingsocioecologyvery2015}.

% Two rapid shifts in institutional structure that occurred in 1987 and 1998 (see Supplementary Material S1) provide unique settings to exploit quasi-natural experiments of a large river basin, the Yellow River Basin (YRB) in China \cite{xiaDevelopmentWaterAllocation2012}.

% After a period of severe drying up, an institutional shift implemented in 1987 represented the beginning of attempts to control water use in the YRB through the use of quotas, with the goals of alleviating conflicts between supply and demand and achieving sustainable development.

% Our results show that this initiative actually accelerated water withdrawals, resulting in an unintended “sprint effect”, where institutional mismatches created an even stronger incentive for each resource user to withdraw resources until the next major institutional shift in 1998.
% Our analysis contributes to a deeper understanding of the mechanisms underlying the relationships among institutions, SES structures, and outcomes. By highlighting potential concerns for ecosystem collapse under structural mismatches, our findings are consistent with the urgent calls for a more dynamic design for water use allocation to achieve sustainability.

\section{Institutions and SES structures}
% 机构可以塑造社会经济体系的结构,对其进行抽象是机械地理解结构和结果之间联系的第一步。
Because institutions may shape the structure of SESs, describing institutional structure is a first step toward understanding the mechanisms linking structures and outcomes in SESs (Figure~\ref{fig:framework}A).
% 例如,机构可以创建一种被确定为与良好的社会经济效益相关联的水平匹配结构,如果它鼓励管理相连生态成分的不同行动者之间的协作(图1B)。
For example, institutions may create a structure that encourages collaboration between the different actors managing connected ecological components (Figure~\ref{fig:framework}B), leading to sustainable outcomes.
% 例如,机构可以创建一个结构,鼓励不同行为者之间的合作,管理相连的生态成分
Similarly, institutions for vertical management may enhance multi-layered SES matching by coordinating horizontal relationships (Figure~\ref{fig:framework}C and D).
% 在实践中,一个大型、复杂的河流流域的制度变化将创造或摧毁数百种不同的联系。这些局部变化的更广泛影响可以从系统的整体行为中看到。
% 因此,我们通过黄河流域的准自然实验,探讨了社会经济结构与可持续发展(结果)之间的因果关系,为两个主要原因提供了一个有益的案例研究。
We thus explored the causal linkages between the SES structures and sustainability (outcomes) in quasi-natural experiments of the YRB, which provides an informative case study for two main reasons.
% 首先,长江流域管理的急剧结构变化使我们能够定量估计高层制度设计变化对用水的净影响。决定水分配的制度包括自下而上的协议或社会规范,以及自上而下的配额或法规,它们对社会经济结构有不同的影响;自上而下的监管可能会立即引发制度转变和SES的剧烈结构性变化。通过与由自下而上的制度转变引起的更渐进的变化相比,探索自上而下变化的影响,在对社会经济地位的定量分析中,极大地减少了来自不可观测因素的潜在干扰,提高了社会经济地位结构和结果之间因果关系的清晰度。
First, the sharp structural shifts in YRB management enabled us to quantitatively estimate the net effects of changes in high-level institutional design on water use. Institutions that determine water allocation include bottom-up agreements or social norms as well as top-down quotas or regulations, with different effects on SES structure \cite{wangAlignmentsocialecological2019,speedBasinwaterallocation2013}; top-down regulations can trigger immediate institutional shifts, and sharp SES structural changes \cite{speedBasinwaterallocation2013,rolandUnderstandinginstitutionalchange2004}.
In comparison with investigations of more gradual changes induced by bottom-up institutional shifts, exploring the impacts of a top-down change substantially diminishes potential problems of omitted variables in the quantitative analysis of SES and improves the clarity of the causal link between SES structure and outcome.
% 其次,通过比较长江流域两次制度变迁所分裂的三种不同制度结构的净效应,我们也可以更深入地理解“盆地固定效应”下结构格局的影响。尽管流域内的社会经济单位从世界各地的大型河流流域和许多地区的水资源中受益,但很少有流域多次经历过如此激进的社会经济结构变化。
Second, by comparing the net effects of three different institutional structures split by two institutional shifts in the YRB, we can also better understand of the influence of structural alignments under a fixed basin. Although socioeconomic units within a basin benefit from water resources in large river basins all over the world and many locations have shown increased levels of regulation, few basins have experienced such radical SES structural changes several times (see \textit{Supplementary Material} S1). Thus, the YRB provides a valuable setting for understanding the direct impacts of changes in SES institutional structure.
% 因此,黄河流域的“流域固定效应”为SES结构的自我比较提供了宝贵的机会。
% Thus, the YRB provides valuable settings for understanding the direct impacts of changes in SES institutional structure.


\section{results}
% % 然后,我们反复比较了两个SES结构(分别自1987年和1998年),以量化是否有显著不同的影响,YRB用水或分配。
% Then we repeatedly compared between two SES structures (since 1987 and 1998, respectively) to quantify if there were significantly different impacts on water uses or allocation (\textbf{methods}).



\section{discussion}
% 我们展示了不匹配的分配制度是如何导致激励失真导致水资源加速耗竭的(即“冲刺效应”)。“冲刺效应”是CPR系统面临的一种特殊情况,在这种情况下,制度的不匹配为每个资源使用者创造了更强的动机(扭曲),促使他们收回资源
We have shown how a mismatched allocation institution can lead to an accelerated depletion of water resources (i.e., the “sprint effect”) caused by incentive distortion. The sprint effect is a special case faced by CPR systems, where institutional mismatches create an even stronger incentive (with distortion) for each resource user to withdraw resources
\cite{ostromRevisitingCommonsLocal1999,ostromGeneralFrameworkAnalyzing2009,castilla-rhoSocialtippingpoints2017}.
% 过往研究指出制度常常是避免公共池塘资源系统的崩溃的关键,但“短跑效应”的出现表明在自上而下进行制度设计所形成的错配SES结构中,制度也可以成为系统加速崩溃的触发者。
Previous studies have suggested that institutions are often the key to avoid the collapse of a CPR system, but the emergence of a sprint effect shows that an institution with structural mismatches can also be the trigger that accelerates system collapse \cite{bodinConservationSuccessFunction2014,bodinCollaborativeenvironmentalgovernance2017,wangAlignmentsocialecological2019}.
The initial formulation of the water quota in our case studies went through a stage of “bargaining” among stakeholders (from 1982 to 1987) \cite{wangReviewImplementationYellow2019, wangThingsCurrentSignificance2019}, where each province attempted to demonstrate its development potential related to water use.
% 为了防止公共池资源被过度利用,总配额在强制禁止水资源过度利用的环境规制中发挥着重要作用,从而形成一个长期匹配的水资源分配机制。
Adoption of an overall quota plays an important role in preventing overuse of CPRs \cite{tilmanLocalizedprosocialpreferences2019}.
However, the negative effects of incentive distortion imply a trade-off between long-term SES benefits and current stability, and the proportion of available resources allocated under quota schemes matters when institutions change \cite{ladeRegimeshiftssocialecological2013}.
According to our analysis of plausible scenario assumptions based on our general economic model, the sprint effect will be reinforced when stakeholders anticipate that technological advances will amplify the benefits of water quotas in the future (see \textit{Supplementary Material S3}).
% 然而,如果有水权转换机制允许利益相关者之间通过交易来弥补 shadow value,当前这种错乱的动机就不会那么强。
However, if an institution allowed stakeholders to compensate for the shadow value (i.e., potential returns sacrificed due to water constraints and water scarcity) \cite{howarthAccountingvalueecosystem2002} of future water use, incentive distortion would be less devastating (e.g., through water rights transfer).
Policymakers can also weaken the sprint effect by increasing the frequency of quota updates, supporting the idea that a more dynamic institution that responds to changing conditions (see \textit{Supplementary Material S3}) will adapt more effectively to its social-ecological context.

% 近年来黄河流域面临的分水制度调整问题也说明了动态设置配额的重要性。
Calls for a redesign of water allocation institutions in the YRB in recent years also illustrate the importance of dynamic quota setting (see \textit{Supplementary Material S1}) \cite{yuAdaptabilityassessmentpromotion2019}. Following the institutional reforms of 1998, the Yellow River has not dried up since 1999. However, given recent changes in the YRB, its rigid resource allocation scheme can no longer meet the new demands of economic development \cite{wangThingsCurrentSignificance2019}. The Chinese government has embarked on an ambitious plan to redesign its decades-old water allocation institution (see \textit{Supplementary Material S1}). Other SESs around the world face similar problems in establishing successful resource allocation institutions \cite{cummingQuantifyingSocialEcologicalScale2020, muneepeerakulStrategicbehaviorsgovernance2017, cummingAdvancingunderstandingnatural2020, leslieOperationalizingsocialecologicalsystems2015}. These initiatives can benefit from our analysis by actively considering and incorporating social-ecological complexity and incentive structures when developing new approaches that avoid unsustainable outcomes. Our research provides a cautionary tale of how institutions can act as a double-edged sword when trying to attain sustainability.


% 我们展示了不匹配的分配制度是如何导致激励失真导致水资源加速耗竭的(即“冲刺效应”)。“冲刺效应”是CPR系统面临的一种特殊情况,在这种情况下,制度的不匹配为每个资源使用者创造了更强的动机(扭曲),促使他们收回资源
\cite{ostromRevisitingCommonsLocal1999,ostromGeneralFrameworkAnalyzing2009,castilla-rhoSocialtippingpoints2017}.
% 过往研究指出制度常常是避免公共池塘资源系统的崩溃的关键,但“短跑效应”的出现表明在自上而下进行制度设计所形成的错配SES结构中,制度也可以成为系统加速崩溃的触发者。
Previous studies have suggested that institutions are often the key to avoiding the collapse of a CPR system, but the emergence of a sprint effect shows that an institution with structural mismatches can also be the trigger that accelerates system collapse \cite{bodinConservationSuccessFunction2014,bodinCollaborativeenvironmentalgovernance2017,wangAlignmentsocialecological2019}.


##
% Linking structures to outcomes need advancement when understanding an SES. Two particular weaknesses in existing knowledge include understanding (1) the causal links between SES structures and outcomes; and (2) details of the underlying processes, especially the coordination of the incentives of different participants, that result from an institutional lack of fit. These weaknesses limit our understanding of institutional design, and they may reduce the speed and transfer of new knowledge and experience related to improving the sustainability of comprehensive water resources management.

The adverse effects of incentive distortion imply a trade-off between long-term SES benefits and current stability, and the proportion of available resources allocated under quota schemes matters when institutions change \cite{lade2013}


% 在我们的研究期间(从1975年到2008年),机构在三种不同的结构之间转换:免费访问、87-WAS和98-UBR。
In our study period, the institution shifted between three different structures (Figure~\ref{structure} A to C).
% 长江水利委员会是长江流域的主要负责人,其任务是报告和分析1998年以前长江流域的用水量(87-WAS之后)。
However, the YRCC, whose primary official response to the river before 1998, the mandate was to conserve the riverway environment, construct and maintain infrastructures, and report on and analyze water consumption (after the 87-WAS) in the YRB, i.e., connecting the ecological nodes (different river reaches) horizontally ~\cite{wang2019a}.

% 为此,长江流域作为中国水治理转型的先驱者,从上世纪70年代开始探索初步的配水方案,到1998年找到了解决水资源枯竭问题的成功方案,并从2008年开始全面推广。
% 1987年(被称为“87年水分配计划,87- was”)和1998年(统一流域管理,98-UBR)的制度转变是水治理的两个被广泛认可的里程碑。
% 相反,在自98-UBR以来在恢复河流枯竭方面取得显著成就之前,第一个限制87-WAS用水的诱惑被认为是对体制转变的不满足的期望。
% 在87-WAS之前,利益攸关方可以免费使用YR水资源,但淡水需求和可用性之间存在地理和时间差异。
% 由于需求和供应之间的不匹配不断增加,国家当局在87-WAS中建议在YR盆地沿线10个省(或区域)之间分配具体的水配额。
% 然而,这个有争议的计划在扭转水资源枯竭方面帮助不大,直到1998年一项不同的战略扩大了流域当局在综合水资源管理方面的责任,并开始在恢复方面取得进展。
For that, as a pioneer in water governance shifting in China, the YRB started to explore the initial water allocating scheme in the 1970s, then found a successful solution of dring-up in 1998, and promoted entirely since 2008.
The institutional shifts in 1987 (known as the ``87 Water Allocation Scheme, 87-WAS'') and 1998 (Unified Basinal Regulation, 98-UBR) were two widely recognized milestones of water governance.
Before the remarkable achievement in the restoration of river depletion since the 98-UBR, on the contrary, the first temptation to restrict water uses in the 87-WAS was recognized as a not fulfilling expectation of institutional shift.
Until the 87-WAS, stakeholders have free access to the YR water resources, with geographic and temporal differences between freshwater demand and availability.
As the mismatch between demands and supply kept increasing, national authorities proposed in 87-WAS allocating specific water quotas between $10$ provinces (or regions) along the YR basin.
However, this controversial scheme helped little in turning water depletion around until a different strategy expanded the responsibilities of basinal authorities in integrated water management in 1998 and started progress in the restoration.

Water governance tends to shift to institutional solutions within a complex basinal system, where societal drivers impact through water use and related technical interventions \cite{fischer2020}.

% YR水治理机构的转变重构了人与水之间的相互作用,并产生了长期的级联效应,为理解这种相互作用留下了两个准自然实验。
% 利用差分综合控制方法,对制度变迁的净效应进行了分析,结果表明:制度不匹配促进了98-WAS后资源使用者的取水速度。
% 由于很少有大型河流流域经历过如此剧烈的结构变化,我们对年度制度变迁的定量分析,通过将自然和人类干预脱钩,促使我们对水的可持续治理有了宝贵的理解。
% % !这还要继续凝练我们的贡献
% 通过强调结构不匹配下生态系统崩溃的潜在担忧,我们的研究结果与迫切需要为水治理机构提供更动态的设计以实现可持续性的呼吁相一致。
The shifts in the water governance institution of YR refactored the interplays between humans and water with long-term cascading effects, leaving two quasi-natural experiments for understanding the interactions.
By Differenced Synthetic Control method, our analysis of the net effects of institutional shift shows that institutional mismatches contributed to the acceleration of water withdrawals for resource users after 98-WAS.
As few large river basins have experienced such radical structural changes several times, our quantitative analysis of institutional shifts in the YR induces a valuable understanding of water sustainable governance by decoupling natural and human interferences.
%! 这还要继续凝练我们的贡献
By highlighting potential concerns for ecosystem collapse under structural mismatches, our findings align with the urgent calls for a more dynamic design for water governance institutions to achieve sustainability.
