%! Author = songshgeo
%! Date = 2022/3/10

\subsection{Novel insights and policy implications}
\label{insights}

% 由于87-WAS和98-UBR分别带来的两种结构在很多的SES中都是反复出现的构件,我们提出的机制对理解这类耦合系统至关重要
Since the structures introduced by 87-WAS and 98-UBR are recurring motifs in many SES, our proposed mechanism is crucial to understanding such coupled systems.
% 因此,我们通过黄河流域的准自然实验,探讨了社会经济结构与可持续发展(结果)之间的因果关系,为两个主要原因提供了一个有益的案例研究。
Furthermore, we explored the causal linkages between the SES structures and sustainability (outcomes) in quasi-natural experiments of the YRB, which provides an informative case study for two main reasons.
% 首先,长江流域管理的急剧结构变化使我们能够定量估计高层制度设计变化对用水的净影响。
First, the sharp structural shifts in YRB management enabled us to quantitatively estimate the net effects of changes in high-level institutional design on water use. Institutions that determine water allocation include bottom-up agreements or social norms as well as top-down quotas or regulations, with different effects on SES structure \cite{wang2019d,speed2013}; top-down regulations can trigger immediate institutional shifts, and sharp SES structural changes \cite{speed2013,roland2004}.
In comparison with investigations of more gradual changes induced by bottom-up institutional shifts, exploring the impacts of a top-down change substantially diminishes potential problems of omitted variables in the quantitative analysis of SES and clarifies the causal link between SES structure and outcome.
% 其次,通过比较长江流域两次制度变迁所分裂的三种不同制度结构的净效应,我们也可以更深入地理解“盆地固定效应”下结构格局的影响。尽管流域内的社会经济单位从世界各地的大型河流流域和许多地区的水资源中受益,但很少有流域多次经历过如此激进的社会经济结构变化。
Second, we can better understand the influence of structural alignments under a fixed basin by comparing the net effects of three different institutional structures split by two institutional shifts in the YRB. Although socioeconomic units within a basin benefit from water resources in large river basins all over the world, and many locations have shown increased levels of regulation, few basins have experienced such radical SES structural changes several times (see \textit{Supplementary Material} S1). Thus, the YRB provides a valuable setting for understanding the direct impacts of changes in the SES institutional structure.
% 最后,我们的研究也存在一定局限性,那就是很难排除同期其他政策的效应。
Finally, one of the limitations of our method is that it is difficult to rule out the effects of other policies over the same time breakpoints.
% 然而,鉴于学者对于87-WAS与98-UBR两次制度转变的重要性基本达成共识,其结果差异仍对理解水治理有重要借鉴意义。
However, since scholars have reached a consensus on the importance of the two institutional shifts of 87-WAS and 98-UBR, the differences in their results still provide important references for understanding water governance.

% 我们的模型结果与机制探讨加深了SES结构的理解,强化了孤立利益相关者形成的结构不利于制度解决环境问题的基本认识
Our results and discussion deepen the understanding of SES structure and strengthen the basic understanding that the mismatched structure formed by isolated stakeholders is not conducive to institutional solutions to environmental sustainability.
% 而随后 98-UBR 的成功从理论和实际上都佐证了制度尺度匹配的重要性,即需要强调与生态单元的尺度(这里是流域尺度)相符的代理人建立利益相关者之间的横向联系。
Moreover, the subsequent success of 98-UBR has proved the importance of institutional scale matching both theoretically and practically; that is, it is necessary to emphasize the establishment of potential connections between stakeholders by agents consistent with the scale of the ecological system (in this case, basinal scale and the YRCC).
Furthermore, according to our analysis of plausible scenario assumptions based on our model, when stakeholders anticipate that technological advances will amplify the benefits of water quotas in the future, the incentive distortion will be reinforced (see \textit{Supplementary Material S4}).
However, an institution allowed stakeholders to compensate for the shadow value (i.e., potential returns sacrificed due to water constraints and water scarcity) \cite{howarth2002} of future water use would weaken incentive distortion (e.g., through water rights transfer) (see \textit{Supplementary Material S4}).
Policymakers can also propose a more dynamic institution by increasing the frequency of quota updates that responds to changing conditions and will adapt more effectively to its social-ecological context (see \textit{Supplementary Material S4}).

% 未来的政策建议
% 近年来黄河流域面临的分水制度调整问题也说明了动态设置配额的重要性。
Calls for a redesign of water allocation institutions in the YRB in recent years also illustrate the importance of dynamic quota setting (see \textit{Supplementary Material S1}) \cite{yu2019}. Following the institutional reforms of 1998, the Yellow River has not dried up since 1999. However, given recent changes in the YRB, its rigid resource allocation scheme can no longer meet the new demands of economic development \cite{wang2019a}. As a result, the Chinese government has embarked on an ambitious plan to redesign its decades-old water allocation institution (see \textit{Supplementary Material S1}). Other SESs around the world face similar problems in establishing successful institutions for governance \cite{cumming2020b, muneepeerakul2017, cumming2020a, leslie2015}. These initiatives can benefit from our analysis by actively considering and incorporating social-ecological complexity and incentive structures when developing new approaches that avoid unsustainable outcomes. Our research provides a cautionary tale of how a mismatched structure of institutions can be a double-edged sword when attaining sustainability.
