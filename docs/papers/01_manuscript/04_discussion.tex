%! Author = songshgeo
%! Date = 2022/3/10
% 我们展示了不匹配的分配制度是如何导致激励失真导致水资源加速耗竭的(即“冲刺效应”)。“冲刺效应”是CPR系统面临的一种特殊情况,在这种情况下,制度的不匹配为每个资源使用者创造了更强的动机(扭曲),促使他们收回资源
We have shown how a mismatched allocation institution can lead to an accelerated depletion of water resources (i.e., the “sprint effect”) caused by incentive distortion. The sprint effect is a special case faced by CPR systems, where institutional mismatches create an even stronger incentive (with distortion) for each resource user to withdraw resources
\cite{ostromRevisitingCommonsLocal1999,ostromGeneralFrameworkAnalyzing2009,castilla-rhoSocialtippingpoints2017}.
% 过往研究指出制度常常是避免公共池塘资源系统的崩溃的关键,但“短跑效应”的出现表明在自上而下进行制度设计所形成的错配SES结构中,制度也可以成为系统加速崩溃的触发者。
Previous studies have suggested that institutions are often the key to avoid the collapse of a CPR system, but the emergence of a sprint effect shows that an institution with structural mismatches can also be the trigger that accelerates system collapse \cite{bodinConservationSuccessFunction2014,bodinCollaborativeenvironmentalgovernance2017,wangAlignmentsocialecological2019}.
The initial formulation of the water quota in our case studies went through a stage of “bargaining” among stakeholders (from 1982 to 1987) \cite{wangReviewImplementationYellow2019, wangThingsCurrentSignificance2019}, where each province attempted to demonstrate its development potential related to water use.
% 在高层决策者与低层利益相关者之间存在信息不对称的用水分配中,当前用水量越大的利益相关者议价能力越大。1987年以后,对各省来说,合乎逻辑的下一步是试图证明争取更大配额的合理性,而不是立即采取资源节约型改革。在实践中,尽管受影响的省份可能没有直接鼓励过度使用资源,但由于激励扭曲,他们有更大的动机对资源提取开绿灯。因此,在争夺潜在的水资源配额时,各省往往将生态成本隐藏在经济发展背后。
In water use allocation with information asymmetry between upper-level decision-makers and lower-level stakeholders, those with more current water use might have greater bargaining power. After 1987, the logical next step for provinces was to attempt to justify bargaining for larger quotas rather than immediately adopt resource-conserving transformations. In practice, although the affected provinces may not have directly encouraged excessive resource use, they had a greater incentive to give the green light to resource withdrawals because of incentive distortions \cite{kriegerProgressGroundWater1955, ostromGoverningCommonsEvolution1990}. As a result, while competing for potential water quotas, the provinces tended to hide the ecological costs behind economic development.

% 毫无疑问,随着资源竞争的日趋激烈,越来越多的SES正依赖着不同形式的制度进行资源分配(如自组织和政府干预),避免“短跑效应”的出现或将成为制度设计的关键。
There is no doubt that with increasingly fierce competition for water, more and more SESs are developing new institutions for water allocation (whether through self-organization or government intervention) \cite{anderssonVoluntaryleadershipemergence2020, wutichWaterScarcitySustainability2009}.
% 为了防止公共池资源被过度利用,总配额在强制禁止水资源过度利用的环境规制中发挥着重要作用,从而形成一个长期匹配的水资源分配机制。
Adoption of an overall quota plays an important role in preventing overuse of CPRs \cite{tilmanLocalizedprosocialpreferences2019}.
However, the negative effects of incentive distortion imply a trade-off between long-term SES benefits and current stability, and the proportion of available resources allocated under quota schemes matters when institutions change \cite{ladeRegimeshiftssocialecological2013}.
According to our analysis of plausible scenario assumptions based on our general economic model, the sprint effect will be reinforced when stakeholders anticipate that technological advances will amplify the benefits of water quotas in the future (see \textit{Supplementary Material S3}).
% 然而,如果有水权转换机制允许利益相关者之间通过交易来弥补 shadow value,当前这种错乱的动机就不会那么强。
However, if an institution allowed stakeholders to compensate for the shadow value (i.e., potential returns sacrificed due to water constraints and water scarcity) \cite{howarthAccountingvalueecosystem2002} of future water use, incentive distortion would be less devastating (e.g., through water rights transfer).
Policymakers can also weaken the sprint effect by increasing the frequency of quota updates, supporting the idea that a more dynamic institution that responds to changing conditions (see \textit{Supplementary Material S3}) will adapt more effectively to its social-ecological context.

% 近年来黄河流域面临的分水制度调整问题也说明了动态设置配额的重要性。
Calls for a redesign of water allocation institutions in the YRB in recent years also illustrate the importance of dynamic quota setting (see \textit{Supplementary Material S1}) \cite{yuAdaptabilityassessmentpromotion2019}. Following the institutional reforms of 1998, the Yellow River has not dried up since 1999. However, given recent changes in the YRB, its rigid resource allocation scheme can no longer meet the new demands of economic development \cite{wangThingsCurrentSignificance2019}. The Chinese government has embarked on an ambitious plan to redesign its decades-old water allocation institution (see \textit{Supplementary Material S1}). Other SESs around the world face similar problems in establishing successful resource allocation institutions \cite{cummingQuantifyingSocialEcologicalScale2020, muneepeerakulStrategicbehaviorsgovernance2017, cummingAdvancingunderstandingnatural2020, leslieOperationalizingsocialecologicalsystems2015}. These initiatives can benefit from our analysis by actively considering and incorporating social-ecological complexity and incentive structures when developing new approaches that avoid unsustainable outcomes. Our research provides a cautionary tale of how institutions can act as a double-edged sword when trying to attain sustainability.