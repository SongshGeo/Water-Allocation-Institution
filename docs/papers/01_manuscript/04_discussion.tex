%! Author = songshgeo
%! Date = 2022/3/10

% \subsection{CAUSES OF INSTITUTIONAL IMPACTS}
% \subsection{}
% discussion-1:
% 用水量的上升、下降-结果解读
% 制度对社会生态系统的结果产生影响在世界范围内都很普遍,
The influences of institutional shifts on governing effects of social-ecological systems (SESs) were widely reported worldwide, but few attempts to quantify their net effects~\cite{cumming2020a}.
Our results show that while 98-UBR decreased water use in the YRB, 87-WAS increased it by $5.75\%$.
The results challenged previous analyses (i.e., suggesting that 87-WAS ``had little practical effect'') because theoretically, there should be few gaps between actual and synthetic water use in the YRB if no effect is present~\cite{abadie2015,hill2021}.
However, the significant net effect indicated by our analysis suggests 87-WAS was followed by more water use even after controlling for environmental and economic variables (see \textit{\nameref{secS2}}~Table~\ref{tab:variables}).
On the contrary, the 98-UBR reduced surface water competition, so many studies attributed the streamflow restoration mainly to the successful introduction of this institution~\cite{chen2021,huangang2002,an2007}.

% discussion-2: 87的增加
% 结果与制度的目的完全相反的87-WAS与许多其它的SES失败治理案例类似,表明不匹配的社会生态结构可能促进了对公共资源的掠夺式开采。
The above comparison suggests that the 87-WAS, whose results were contrary to the purpose of the institution, is similar to many other SES governance failures, supporting that mismatched socio-ecological structures can deterioration of common resources~\cite{kellenberg2009,cai2016,barnes2019}.
The increased water use after 87-WAS aligns with concerns about frequently scrambling for water in some provinces during this period~\cite{mao2000, bouckaert2022}.
Although reasons for the non-ideal effect of 87-WAS had been widely discussed~\cite{huangang2002} (such as enforcement, feasibility, and equity), however, structural change has received limited attention.
Our results show that the correlation between current water use and changed (increased or decreased) water use was significant after 87-WAS (Figure~\ref{fig:regulating}).
This ``major users use more'' pattern supports the hypothesis that separated stakeholders (individual provinces) will respond to structure by maximizing utility (interpreted in our structure-based model, see Figure~\ref{fig:model}).

The validity of our theoretical analysis is supported by two facts:
(1) The water quotas of 87-WAS (or the initial water rights) went through a stage of ``bargaining'' among stakeholders (from 1982 to 1987)~\cite{wang2019a, wang2019d}, where each province attempted to demonstrate its development potential related to water use.
The bargaining was also a process for matching water shares to economic volume because the major water users (like Shandong and Henan) needed more water than their original quota (if only considering economic potentials when designing the institution)~\cite{zuo2020}.
(2) Provinces with higher current water use might have greater bargaining power in water use allocation because of information asymmetry between decision-makers and stakeholders.
Therefore, stakeholders had considerable incentives to prevent water quotas from hindering their economic potential, which aligned with their appeals to the higher central government for larger shares~\cite{wang2019a, wang2019d}.

On the other hand, social-ecological matches can also be supported by structural effects.
After 98-UBR, the YRCC could adjust surface water use quotas according to climate conditions for the whole YRB.
When the YRCC began to coordinate among stakeholders, the external appeals of provinces for larger quotas turned into internal innovation to improve water efficiency (e.g., drastically increased water-conserving equipment)
\cite{krieger1955, ostrom1990}.
During this period, proportional decreased water use of provinces indicated expected river regulating results (see \nameref{result-3}).
Meanwhile, our model also demonstrates that in this case, the unified scale-matched institution was indispensable for decreased water use.
However, since the 98-UBR only regulated surface water use, many clues suggested the institution shift may cause broader influences (cascading effects) because of unsatisfied water demands.
For example, literature estimated increased groundwater withdrawals after 98-UBR in many intensive irrigation regions, despite little eligible data on groundwater use and stakeholders being accessible~\cite{sun2022b}.
Since the 21st century, similar water quota policies started to be implemented nationally and altering the relationship between social actors and resource units.
As both institutional shifts examined here induced unexpected changes or cascading effects within SESs, better governance calls for more institutional analysis of coupled human and natural systems in the future.

The structural pattern we depicted here (Figure~\ref{fig:structure}) have also been reported in other SESs worldwide~\cite{kluger2020,guerrero2015,bodin2012}.
Before 98-UBR, SES structure (i.e., fragment ecological units linked to separate social actors) was more likely to be mismatched because isolated actors generally struggle to maintain interconnected ecosystems holistically~\cite{sayles2017,sayles2019,cai2016,bergsten2019}.
Institutional re-alignments since 98-UBR improved the authority of the YRCC and helped it match the scale of resource provisioning in the YRB, leading to enhanced effectiveness regarding runoff restoration~\cite{cumming2020a,wang2019d}.
The comparison demonstrates again the challenge of finding win-win situations in coupled human-nature systems~\cite{hegwood2022}, and the need to more deeply understand the role of social-ecological structures~\cite{bergsten2019, sayles2019}.
% Therefore, the YRB cases can provide further explanation of the matching and mismatching of the previous SES building blocks, linking SES structure and outcomes by plausible reasons of causality and underlying processes.

% \subsection{LIMITATION, INSIGHTS AND IMPLICATIONS}
% \subsection{Limitation, insights and implications}
% \label{discussion-4}
% discussion-3: 启示、未来的展望

Our approach has some inevitable limitations.
First, the contributions of economic growth and institutional shifts are difficult to distinguish because of intertwined causality (institutional changes can also influence the relative economic variables);
and second, when applying the DSC method, it is difficult to rule out the effects of other policies over the same time breakpoints (1987 and 1998).
Our quasi-experiment approach nonetheless provides evidence supporting the view that there was a change in water use trajectory following the YRB's unique institutional shifts and offers insights into water governance (and particularly the importance of having a scale-matched, basin-wide authority for water allocation solutions~\cite{bodin2017b, ostrom2009, reyers2018})
Moreover, the ultimate success of the 98-UBR institutional shift theoretically and practically proved the importance of social-ecological fit.
For sustainability in the future, therefore, it is necessary to emphasize the necessity of strengthening connections between stakeholders by agents consistent with the scale of the ecological system.
From these perspectives, two scenarios based on the marginal benefit analysis (see \textit{\nameref{secS5}}) can inspire institutional design on how to reduce mismatches.
For example, water rights transfers may be another way to build horizontal links between stakeholders that also have the potential to result in better water governance.
In addition, policymakers can propose more dynamic and flexible institutions to increase the adaptation of stakeholders to a changing SES context~\cite{reyers2018}.

The structural pattern that led to different effectiveness widespread in global SESs, so our proposed mechanism is crucial to governing such coupled systems.
Calls for a redesign of water allocation institutions in the YRB in recent years also illustrate the importance of institutional solutions for sustainability (see \textit{\nameref{secS1}})~\cite{yu2019, niu2022}.
Given the changing environmental context, outdated and inflexible water quotas can no longer meet the demands of sustainable development~\cite{wang2019a}.
Thus, the Chinese government has embarked on a plan to redesign its decades-old water allocation scheme (see \textit{\nameref{secS1}}).
Our analysis suggests that these initiatives can lead to distortions because of mismatched structure when shifting to another institution~\cite{bodin2017b}.
Therefore, our research provides a cautionary tale of how institutions can change perverse incentives~\cite{hegwood2022}, while insights from the YRB can provide guidelines for SESs management worldwide~\cite{muneepeerakul2017, leslie2015}.
