%! Author = songshgeo
%! Date = 2022/3/10

The impacts of institutional shifts on the governing effects of social-ecological systems (SESs) have attracted global attention, yet efforts to quantify their net effects remain sparse~\cite{cumming2020a}.
Our investigation of the YRB's water governance reveals vary effects of nuanced-differences institutional shifts: while the 98-UBR led to an expected decrease in total water use, the 87-WAS surprisingly increased it by $5.75\%$.
This comparison offers insightful perspectives on the effectiveness of governance because it suggests a significant net effect on increased water use following the implementation of this policy, in addition to the previous reports and comments suggesting that the 87-WAS was ``out-of-control''~\cite{wang2019d, departmentofearthsciences1999}.
In contrast, the 98-UBR reduced surface water competition, so many studies attributed the streamflow restoration mainly to the successful introduction of it~\cite{chen2021,huangang2002,an2007}.

The unanticipated consequence of the 87-WAS policy echoes the structural challenges reported in many other SES governance failures.
This suggests a general pattern where specific misaligned structures can precipitate the rapid depletion of common resources~\cite{kellenberg2009,cai2016,barnes2019}.
These structure-based failures often occur when social actors have unregulated access to linked resource units, a feature prevalent in the institution prior to 1987~\cite{wang2019c}.
When the central government attempted to curtail this free access by introducing water quotas, they were met with water demands from stakeholders' proposals that far exceeded expectations (Table~\ref{tab:quota}).
A previous study attributed the suboptimal effect of 87-WAS to the lack of enforcement and control mechanisms~\cite{huangang2002}.
Taken together, it underpins a hypothesis that in the absence of enforcement, stakeholders might have exploited the system by increasing water withdrawals to secure more water quotas for their economic prospects.

This hypothesis can be further substantiated by two reported facts:
(1) There were not only surges of total water uses following the 87-WAS, but also scrambles for water reported in several provinces during this period~\cite{mao2000, bouckaert2022}.
(2) From 1983 to the 1990s, the stakeholders persistently argued for increased the water quotas, when is a stage of ``bargaining''~\cite{wang2019e, wang2019d};
(3) During this ``bargaining'' stage, the stakeholders who had more economic profits submitted appeals to the higher central government for larger shares~\cite{wang2019e, wang2019d}.

Our results also corroborate some intuitive deductions of the hypothesis.
Firstly, we found significant correlations between current and changed water use after the 87-WAS, which suggests that the key stakeholders (such as Neimeng, Henan, and Shandong), were more likely to be affected by the institutional change.\
Secondly, a theoretical marginal benefit analysis (see \textit{\ref{secS4}}) suggests that this ``major users are effected more'' pattern can be inferred from a simple assumption that stakeholders anticipate future value in water quotas, thereby lending further support to the above hypothesis.
Finally, since the YRCC could forcibly coordinate stakeholders by water quota licenses for the entire YRB after 98-UBR, the external appeals of provinces for larger quotas turned into internal innovation to improve water efficiency (e.g., drastically increased water-conserving equipment)~\cite{krieger1955, ostrom1990}.

On the flip side, the apparent success of the 98-UBR institutional transformation has received consistent acclaim, particularly for its role in restoring the previously dry river~\cite{wang2019e, wang2019d}.
Our findings suggest that the 98-UBR led to a proportional decrease in water use across provinces, indeed indicative of an immediate and tangible effect.
However, it's essential to recognize that the 98-UBR focused solely on regulating surface water use, which hints at potential broader implications.
Notably, some evidence suggests that this institutional shift might have resulted in increased groundwater withdrawals in regions with intensive water usage following the 98-UBR~\cite{sun2022b}.
Unfortunately, the limited availability of eligible data on groundwater use constrains a comprehensive assessment, leaving this aspect beyond the scope of the current study.
Nonetheless, this consideration remains highly relevant, especially as similar water quota policies have begun to be implemented nationally since the turn of the 21st century.

To provide an intuitive understanding of the profound impact of the Institutional shifts, we can turn to the insights shared by a representative of the Hetao Irrigation District in Neimeng.
As a primary stakeholder, the district's representative voiced the struggle to adapt under the 98-UBR policy which strictly enforced water quotas in our surveys.
``The water allocated to us is far from enough'', he revealed with a desire on more water quota: ``And it's not like in the past when we could actually over use, it is very strictly controlled now.''
``Under a limited quota, of course there are conflicts between users time to time, which depends on leaderships of the water-user associates'', he reflected: ``-farmers may have their own solutions, such as switching to sunflower, which is more water-efficient, or using shallow groundwater when is available.''
Simultaneously, the district looked forward to future projects, such as the ``South-to-North Water Diversion'' Western Route Project, which they hoped would increase their water quotas and allow for expansion of their irrigation area.
The desire of water in Neimeng wasn't without controversy. Stakeholders in other lower reaches argued that the Hetao Irrigation District was consuming too much water from the Yellow River.

The above analysis with a real-world example emphasizes the vital role of institutions in shaping the socio-ecological systems (SES) structures of water governance.
The structural pattern we have depicted (Figure~\ref{fig:structure}), mirrored in other SESs worldwide~\cite{kluger2020,guerrero2015,bodin2012}, illustrates how fragmented ecological units linked to isolated social actors can lead to inefficiencies.
Before the 98-UBR, this fragmentation resulted in lower effectiveness, as disconnected actors struggled to maintain holistic ecosystems~\cite{sayles2017,sayles2019,cai2016,bergsten2019}.
After the 98-UBR, institutional realignments enhanced basin-scale authority (YRCC), fostering effectiveness in runoff restoration—a phenomenon often termed scale or institutional match in SESs~\cite{cumming2020a,wang2019d}.
This comparison underscores the complex challenges of crafting win-win scenarios in SES and accentuates the importance of understanding institutional roles in water governance~\cite{hegwood2022,bergsten2019, sayles2019}.

Our approach acknowledges certain limitations, such as the difficulty in quantifying contributions from economic growth, and challenges in isolating the effects from other concurrent policies in 1987 and 1998.
Despite these constraints, our quasi-experimental methodology elucidates the change in water use following the YRB's unique institutional shifts.
It offers critical insights into water governance, emphasizing scale-matched, basin-wide authority for water allocation solutions~\cite{bodin2017b, ostrom2009, reyers2018}.
The success of the 98-UBR shift underscores the need for social-ecological alignment, fostering sustainable governance.
Future endeavors must focus on strengthening stakeholder connections, exploring alternative solutions like water rights transfers, and embracing more dynamic and adaptable institutional frameworks to respond to evolving SES contexts~\cite{reyers2018}.

The diverse effectiveness of structural patterns, as observed in global SESs, underscores the necessity for nuanced governance in coupled systems. The potential for unexpected outcomes due to institutional mismatches calls for thorough institutional analysis. As China seeks to overhaul its water allocation schemes, our research serves as a timely beacon, highlighting how nuanced institutional interplays can shape successful river basin governance, resonating with the global challenge of socio-hydrological complexities~\cite{muneepeerakul2017, leslie2015, hegwood2022}.
