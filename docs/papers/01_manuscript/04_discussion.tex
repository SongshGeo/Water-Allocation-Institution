%! Author = songshgeo
%! Date = 2022/3/10

% \subsection{CAUSES OF INSTITUTIONAL IMPACTS}
% \subsection{}
\label{discussion-1}
% discussion-1: 主要介绍结果的意义、合理性

Besides environmental background, our forecast by DSC takes economic factors into account under the assumptions that the production function between economic volume and water uses remained unchanged (\textit{S2 in Supplementary Material}).
It means the forecast of water use includes the part caused by the increased economic volume, while the outcomes of the economy (GDP in different sectors) of the YRB maintained a parallel trend with other regions during the period (\textit{S3 in Supplementary Material} Figure~\ref{S3-1}).
Therefore, 87-WAS did not ``have little effect'' as previous analyses suggested (cites) but led to increased water use because the difference between prediction and observation will be trivial when the shift was just a blank policy by applying the DSC method. %! Citation
Water-use intensity is another crucial factor in interpreting the differences besides the economic factors (e.g., irrigated areas and industrial outcomes) considered and controlled by the method.
In addition to the expansion of irrigation area after the 87-WAS, water uses per unit of irrigation area also rapidly widened the gap with the average level of the rest provinces. However, the industry water use intensity hardly changed (\textit{S3 in Supplementary Material} Figure~\ref{S3-2}).
As a previous report sigh: although the key to alleviating the drought is saving water in the irrigated areas, the tragedy of scrambling for water appeared in provinces and irrigated areas %! \cite{mao2000}.
In terms of the average ratio of water-saving irrigation area (refer to drip or sprinkler irrigation systems and canal lining), although there was a significant increase in the whole country after 1987, the YRB did not rapidly open a noticeable gap until about 1994 (Figure~\ref{S3-3}).
As a result, despite the irrigation area expanding, scrambling for water resources without any incentive to optimize production per unit of water resources accelerated holistic water use.
This accelerated water use was contrary to the original intention of the 87-WAS in conserving the limited water, and the failure was a barrier to the sustainability %! \cite{huangang2002}.

% 过去的研究总结出87-WAS收效甚微的几个因素:
Previous studies have summarised factors that contribute to the non-ideal effect of 87-WAS: (1) The YRCC had no right to punish the provinces for over-exploitation; (2) the water quotas were annual values, causing provinces to rob water in the dry season; (3) The YRCC can make statistics on water use in the mainstream but cannot on the tributaries, so provinces water use underreport %! \cite{huangang2002}.
However, the effects of the two institutional shifts (the 87-WAS and the 98-UBR) were significantly different, which the above reasons cannot fully explain.
Between the 98-UBR and the further refinement of the unified regulation in 2008, there was still a lack of a temporary water allocation scheme and effective monitoring of tributaries.
Moreover, without any actual punishment, provinces with high water consumption (such as Inner Mongolia and Shandong) continued to exceed the quota after 98-UBR.
As we have analyzed (Figure~\ref{structure}), the difference between the two institutional shifts is mainly reflected in the structure of linkages between social actors.
Until the institutional shift of 98-UBR, with no necessity to apply for a water permit from YRCC, there were no horizontal connections (cooperations or agreements) between the various stakeholders (provinces) directly connected to the ecological units.
Make it clear that the YRCC was responsible for regulating provincial water use; that is, each province has made it clear that in the long run, water resources are not ``internal'' but ``dependent'' on YRCC.
In that way, the YRCC, whose authority scale matches the whole river basin, also took the primary responsibilities to the river, and literature recognized the structure as a social-ecological fit that usually led to good outcomes.
Empirical studies in many different fields also indicate that the structure before 98-UBR (i.e., fragment ecological units are linked to separate social actors) is likely to be mismatched as isolated stakeholders struggle with holistically maintaining interconnected ecosystems
\cite{sayles2017,sayles2019,cai2016,bergsten2019}.
The effect of the institutional shifts once again demonstrated that it is not easy to have a win-win situation of environment and interests in complex coupled human-nature systems \cite{hegwood2022} which calls for exceptional understanding and caution to the structure of hampering sustainability \cite{bergsten2019, sayles2019}.

% \subsection{LINKING STRUCTURES WITH ISS OUTCOMES}
% \subsection{Linking structures with ISs outcomes}
\label{discussion-2}
% discussion-2: 机制解释
% 经济模型与理论解释
Differences in the pattern of the response by provinces can demonstrate the influence of social-ecological structures led by the institutional shifts.
We analyzed mathematically why the mismatched structure made limited water use holistically elusive in the institution shift of the 87-WAS but finally achieved by the 98-UBR (\textit{method} and \textit{Supplementary Material S4}).
By taking the structure before and after the two institutional shifts as different basic assumptions (before 87-WAS: free access to water; after 87-WAS but before 98-UBR: decisions on water use under quotas; after 98-UBR: unified regulation), we use the marginal benefit model to analyze the theoretical optimal water consumption of stakeholders in each scenario.
The analysis of the model also shows that 98-UBR can reduce the overall water use of the basin while 87-WAS can increase the water use of the basin when the same parameters are guaranteed but the institutional structure changes.
Before the 98-UBR, the model assumes that the separated ecological units (river reaches) link to stakeholders (related provinces) who use water to pursue their marginal benefits but have a potential political cost if they exceed the quota 87-WAS.
Our model suggests that for users who are already economically efficient (who are already using more water), greater marginal returns from water induce the acceleration of extracting resources for future economic growth (Figure~\ref{economic_model}).
Therefore, isolated stakeholders reacted to the similar marginal cost, and smaller water users have a threshold because of the political cost, so 87-WAS triggered an increased water use for the significant users.
On the contrary, the presence of central management (by the YRCC in this case, after 1998) can effectively reduce marginal ecological costs holistically as stakeholders only take corresponding responsibilities (follow the quota as possible as they can) to the YRCC (\textit{Supplementary Material S4}).
As a result, unified regulating acted the core role after the 98-UBR and reduced water use of all stakeholders (provinces) by irregular ratios.

The alignments of differences in institutional structures and outcomes here echo the hypothesis that successful governance of SES emerged by indirectly (or vertically) creating links between different stakeholders (in the YRB cases, through administration).
When links The water quotas of 87-WAS (or the initial water rights) in our case studies went through a stage of ``bargaining'' among stakeholders (from 1982 to 1987) \cite{wang2019a, wang2019d}, where each province attempted to demonstrate its development potential related to water use.
The bargaining itself was also a process towards matches between their economic volume and water shares, as studies show that the large water users (like Shandong and Henan) need more water than their quota (in the 87-WAS) if only considering the economic equity when designing the institution.
Furthermore, with information asymmetry between upper-level decision-makers and lower-level stakeholders in water use allocation, those with more current water use might have greater bargaining power.
In practice, therefore, although the affected provinces may not have directly encouraged excessive resource use because of the institutional shift, they had a more considerable incentive to show their economic potential
That aligns with the historical records that, even after the 87-WAS had already confirmed the quotas, provinces, especially water-intensive ones, challenged it by appearing to the higher central government for larger quotas.
On the contrary, after YRCC as governing agent coordinated between stakeholders since 98-UBR, the external appeal of provinces for larger quotas turned into internal innovation to improve water efficiency (e.g., drastically increased water-conserving equipment, \textit{Supplementary Material S3})
\cite{krieger1955, ostrom1990}.
Then, the YRCC, the authority for approving water applications from all stakeholders, could adjust water use quotas according to the river conditions of the whole basin.
The 98-UBR led to a structure for achieving social-ecological fits in both basins (between YRCC and the YRB) and regions (between provincial economy and their water shares).

% \subsection{LIMITATION, INSIGHTS AND IMPLICATIONS}
\subsection{Limitation, insights and implications}
\label{discussion-3}
% discussion-3: 启示、未来的展望

Agents matching the ecological scale appear widespread as motifs in SES of successful governance, whether in fisheries, forests, or groundwater management, suggesting that reducing independent stakeholders linked to fragmentation is an essential primary mechanism for a structure to produce good results.
% 由于87-WAS和98-UBR分别带来的两种结构在很多的SES中都是反复出现的构件,我们提出的机制对理解这类耦合系统至关重要
Since the structures introduced by 87-WAS and 98-UBR are recurring motifs in many SES, our proposed mechanism is crucial to understanding such coupled systems.
% 因此,我们通过黄河流域的准自然实验,探讨了社会经济结构与可持续发展(结果)之间的因果关系,为两个主要原因提供了一个有益的案例研究。
Furthermore, we explored the causal linkages between the SES structures and sustainability (outcomes) in quasi-natural experiments of the YRB, which provides an informative case study for two main reasons.
% 首先,长江流域管理的急剧结构变化使我们能够定量估计高层制度设计变化对用水的净影响。
First, the sharp structural shifts in YRB management enabled us to quantitatively estimate the net effects of changes in high-level institutional design on water use. Institutions that determine water allocation include bottom-up agreements or social norms as well as top-down quotas or regulations, with different effects on SES structure \cite{wang2019d,speed2013}; top-down regulations can trigger immediate institutional shifts and sharp SES structural changes \cite{speed2013,roland2004}.
In comparison with investigations of more gradual changes induced by bottom-up institutional shifts, exploring the impacts of a top-down change substantially diminishes potential problems of omitted variables in the quantitative analysis of SES and clarifies the causal link between SES structure and outcome.
% 其次,通过比较长江流域两次制度变迁所分裂的三种不同制度结构的净效应,我们也可以更深入地理解“盆地固定效应”下结构格局的影响。尽管流域内的社会经济单位从世界各地的大型河流流域和许多地区的水资源中受益,但很少有流域多次经历过如此激进的社会经济结构变化。
Second, we can better understand the influence of structural alignments under a fixed basin by comparing the net effects of three different institutional structures split by two institutional shifts in the YRB. Although socioeconomic units within a basin benefit from water resources in large river basins all over the world, and many locations have shown increased levels of regulation, few basins have experienced such radical SES structural changes several times (see \textit{Supplementary Material} S1). Thus, the YRB provides a valuable setting for understanding the direct impacts of changes in the SES institutional structure.
Finally, one of the limitations of our method is that it is difficult to rule out the effects of other policies over the same time breakpoints.
However, since scholars have reached a consensus on the importance of the two institutional shifts of 87-WAS and 98-UBR, the differences in their results still provide important insights for understanding water governance.

% 我们的模型结果与机制探讨加深了SES结构的理解,强化了孤立利益相关者形成的结构不利于制度解决环境问题的基本认识
Our results and discussion deepen the understanding of SES structure and strengthen the basic understanding that the mismatched structure formed by isolated stakeholders is not conducive to institutional solutions; -and then reported how another social-ecological fit structure contributed to successful water governance and sustainability.
Moreover, the subsequent success of 98-UBR has proved the importance of institutional scale matching both theoretically and practically. Therefore, it is necessary to emphasize the establishment of potentially connected building blocks between stakeholders by agents consistent with the scale of the ecological system (in this case, the basinal scale and the YRCC).
Furthermore, we applied several scenarios based on the marginal benefit model (see \textit{Supplementary Material S4}) for some further insights into sustainable water governance.
For example, water rights transfers can be another way to emerge horizontal links between stakeholders that also have the potential in resulting in better water governance.
In addition, the policymakers can also propose a more dynamic and flexible institution by increasing the frequency of quota updates that responds to changing conditions and will adapt more effectively to its SES context.

% 未来的政策建议
Calls for a redesign of water allocation institutions in the YRB in recent years also illustrate the importance of dynamic quota setting (see \textit{Supplementary Material S1}) \cite{yu2019}. Following the institutional reforms of 1998, the Yellow River has not dried up since 1999. However, given recent changes in the YRB, its rigid resource allocation scheme can no longer meet the new demands of economic development \cite{wang2019a}. As a result, the Chinese government has embarked on an ambitious plan to redesign its decades-old water allocation institution (see \textit{Supplementary Material S1}). These initiatives can benefit from our analysis by actively considering and incorporating social-ecological complexity and incentive structures when developing new approaches that avoid unsustainable outcomes. Our research provides a cautionary tale of how institutions can be a double-edged sword in attaining sustainability. Therefore, insights from the YRB can be a valuable guideline for SESs around the world facing similar governance problems \cite{cumming2020b, muneepeerakul2017, cumming2020a, leslie2015}.
