%! Author = songshgeo
%! Date = 2022/3/10

% \MakeUppercase{\subsection{Mechanism of institutional shift effects}}
\subsection{Mechanism of institutional shift effects}
\label{reason}
Next, we explored the mechanisms linking the structures and the outcomes.
% 87-WAS 后各省响应的差异是理解制度影响机制的关键
Differences in provincial responses to 87-WAS are vital to understanding the mechanism of institutional shifts' impacts.
% 我们的结果表明,各省加速用水的比例(实际用水量超出模型预测用水量的比例)与各省黄河水取用量呈现显著正相关
Our results show that the proportion of accelerated water consumption in each province (the proportion of actual water consumption exceeding the predicted water consumption by the model) has a significant correlation to the Yellow River water consumption in each province (Figure~\ref{fig:upset}A).
% 但明显的加速效应仅突出体现在主要用水省份上,对大多数省份来说,87-WAS 带来的影响并不大,
However, the apparent acceleration effect of the 87-WAS was only prominent in the significant water-using provinces (Neimeng, Henan, and Shandong), and there were no evident impacts for most provinces (Figure~\ref{fig:upset}B).
% 尤其是常年超指标用水的山东与内蒙,分别在1987到1998年的十年间使用了高于模型预测值的xx%和xx%的用水量。
In particular, Shandong and Inner Mongolia, both of which exceed the prescribed water uses of the 87-WAS, used xx\% and XX\% more water uses than predicted by the model from 1987 to 1998, respectively.

\begin{figure}[!h]
    \centering
    \includegraphics[width=16pc]{outputs/upset.jpg}
    \caption{
        \textbf{A.} The partial correlation coefficient between wate uses (WU) of Yellow River (YR), unsatisfied ratio (compared with requirements in water plan and supply in the 87-WAS), and the average accelerated ratio.
        \textbf{B.} Average accelerated ratio of water uses for each province in the YRB during the decade after 87-WAS (from 1987 to 1998).
        \textbf{Mian users:} Major water consumption provinces (over the median).
        \textbf{Overused:} violate the 87-WAS in average water uses.
    }
    \label{fig:upset}
\end{figure}

Then, we analyzed mathematically why the mismatched structure made win-wins elusive in the institution shift of 87-WAS (\textit{method} and \textit{Supplementary Material S4}).
Theoretically, our model suggests that different kinds of institutional shifts should lead to different optimal water uses (Figure~\ref{fig:economic_model}).
% 在不匹配条件下,边际成本减去边际收益增加,导致用水量增加,xx的差异反映了水配额模拟的非预期冲刺效应。
The unintended accelerations from 1987 to 1998 was caused by both declining marginal costs (a shift from a fixed unit cost to an irrelevant cost) and increasing marginal returns due to future water use benefits (\textit{Supplementary Material S4}).
% 对于已经拥有较高经济效率的用水者(如这里的黄河用水大户),更大的边际收益诱使他们为了未来的经济增长而开采资源的“冲刺效应”
For users who are already economically efficient, the "sprint effect" of extracting resources for future economic growth is is induced by greater marginal returns.
% 因此,这种制度引发了一种与可持续用水意图背道而驰的激励扭曲。
The institution (87-WAS) thus triggered an incentive distortion that ran counter to the intention of sustainable water use.
% 中央管理的存在(1998年后由YRCC进行管理)可以有效降低边际生态成本(详细的数学公式见表~\ref{tab:cases}的\textbf{methods})。
On the contrary, the presence of central management (by the YRCC in this case, after 1998) can effectively reduce marginal ecological costs (\textit{Supplementary Material S4}).

% 制度失效背后的资源竞争
Linking structures to outcomes are in need of advancement when understanding a SES.
% 我们展示了不匹配的分配制度是如何导致激励失真导致水资源加速耗竭的(即“冲刺效应”)。“冲刺效应”是CPR系统面临的一种特殊情况,在这种情况下,制度的不匹配为每个资源使用者创造了更强的动机(扭曲),促使他们收回资源
Here, we have shown how a mismatched structure induced by institutional shift can lead to accelerated depletion of water resources (i.e., the "sprint effect") caused by incentive distortion. The ``sprint effect'' is a particular case faced by CPR systems, where institutional mismatches create an even stronger incentive (with distortion) for each resource user to withdraw resources
\cite{ostromRevisitingCommonsLocal1999,ostromGeneralFrameworkAnalyzing2009,castilla-rhoSocialtippingpoints2017}.
% 过往研究指出制度常常是避免公共池塘资源系统的崩溃的关键,但“短跑效应”的出现表明在自上而下进行制度设计所形成的错配SES结构中,制度也可以成为系统加速崩溃的触发者。
Previous studies have suggested that institutions are often the key to avoiding the collapse of a CPR system, but the emergence of a sprint effect shows that an institution with structural mismatches can also be the trigger that accelerates system collapse \cite{bodinConservationSuccessFunction2014,bodinCollaborativeenvironmentalgovernance2017,wangAlignmentsocialecological2019}.
The initial formulation of the water quota in our case studies went through a stage of "bargaining" among stakeholders (from 1982 to 1987) \cite{wangReviewImplementationYellow2019, wangThingsCurrentSignificance2019}, where each province attempted to demonstrate its development potential related to water use.
% 在高层决策者与低层利益相关者之间存在信息不对称的用水分配中,当前用水量越大的利益相关者议价能力越大。1987年以后,对各省来说,合乎逻辑的下一步是试图证明争取更大配额的合理性,而不是立即采取资源节约型改革。在实践中,尽管受影响的省份可能没有直接鼓励过度使用资源,但由于激励扭曲,他们有更大的动机对资源提取开绿灯。因此,在争夺潜在的水资源配额时,各省往往将生态成本隐藏在经济发展背后。
With information asymmetry between upper-level decision-makers and lower-level stakeholders in water use allocation, those with more current water use might have greater bargaining power. After 1987, the logical next step for provinces was to attempt to justify bargaining for larger quotas rather than immediately adopt resource-conserving transformations. In practice, although the affected provinces may not have directly encouraged excessive resource use, they had a greater incentive to give the green light to resource withdrawals because of incentive distortions \cite{kriegerProgressGroundWater1955, ostromGoverningCommonsEvolution1990}. As a result, while competing for potential water quotas, the provinces tended to hide the ecological costs behind economic development.

% 毫无疑问,随着资源竞争的日趋激烈,越来越多的SES正依赖着不同形式的制度进行资源分配(如自组织和政府干预),避免“短跑效应”的出现或将成为制度设计的关键。
There is no doubt that with increasingly fierce competition for water, more and more SESs are developing new institutions for water allocation (whether through self-organization or government intervention) \cite{anderssonVoluntaryleadershipemergence2020, wutichWaterScarcitySustainability2009}.
% 为了防止公共池资源被过度利用,总配额在强制禁止水资源过度利用的环境规制中发挥着重要作用,从而形成一个长期匹配的水资源分配机制。
Adoption of an overall quota plays a vital role in preventing overuse of CPRs \cite{tilmanLocalizedprosocialpreferences2019}.
However, the adverse effects of incentive distortion imply a trade-off between long-term SES benefits and current stability, and the proportion of available resources allocated under quota schemes matters when institutions change \cite{ladeRegimeshiftssocialecological2013}.
According to our analysis of plausible scenario assumptions based on our general economic model, the ``sprint effect'' will be reinforced when stakeholders anticipate that technological advances will amplify the benefits of water quotas in the future (see \textit{Supplementary Material S3}).

\subsection{Novel insights and policy implications}
\label{insights}

% 然而,如果有水权转换机制允许利益相关者之间通过交易来弥补 shadow value,当前这种错乱的动机就不会那么强。
However, if an institution allowed stakeholders to compensate for the shadow value (i.e., potential returns sacrificed due to water constraints and water scarcity) \cite{howarthAccountingvalueecosystem2002} of future water use, incentive distortion would be less devastating (e.g., through water rights transfer).
Policymakers can also weaken the sprint effect by increasing the frequency of quota updates, supporting the idea that a more dynamic institution that responds to changing conditions (see \textit{Supplementary Material S3}) will adapt more effectively to its social-ecological context.

% 重视制度的影响,人水耦合
% 机构可以塑造社会经济体系的结构,对其进行抽象是机械地理解结构和结果之间联系的第一步。
Because institutions may shape the structure of SESs, describing institutional structure is a first step toward understanding the mechanisms linking structures and outcomes in SESs (Figure~\ref{fig:framework}A).
% 例如,机构可以创建一种被确定为与良好的社会经济效益相关联的水平匹配结构,如果它鼓励管理相连生态成分的不同行动者之间的协作(图1B)。
For example, institutions may create a structure that encourages collaboration between the different actors managing connected ecological components (Figure~\ref{fig:framework}B), leading to sustainable outcomes.
% 例如,机构可以创建一个结构,鼓励不同行为者之间的合作,管理相连的生态成分
Similarly, institutions for vertical management may enhance multi-layered SES matching by coordinating horizontal relationships (Figure~\ref{fig:framework}C and D).
% 在实践中,一个大型、复杂的河流流域的制度变化将创造或摧毁数百种不同的联系。这些局部变化的更广泛影响可以从系统的整体行为中看到。
% 因此,我们通过黄河流域的准自然实验,探讨了社会经济结构与可持续发展(结果)之间的因果关系,为两个主要原因提供了一个有益的案例研究。
We thus explored the causal linkages between the SES structures and sustainability (outcomes) in quasi-natural experiments of the YRB, which provides an informative case study for two main reasons.
% 首先,长江流域管理的急剧结构变化使我们能够定量估计高层制度设计变化对用水的净影响。决定水分配的制度包括自下而上的协议或社会规范,以及自上而下的配额或法规,它们对社会经济结构有不同的影响;自上而下的监管可能会立即引发制度转变和SES的剧烈结构性变化。通过与由自下而上的制度转变引起的更渐进的变化相比,探索自上而下变化的影响,在对社会经济地位的定量分析中,极大地减少了来自不可观测因素的潜在干扰,提高了社会经济地位结构和结果之间因果关系的清晰度。
First, the sharp structural shifts in YRB management enabled us to quantitatively estimate the net effects of changes in high-level institutional design on water use. Institutions that determine water allocation include bottom-up agreements or social norms as well as top-down quotas or regulations, with different effects on SES structure \cite{wangAlignmentsocialecological2019,speedBasinwaterallocation2013}; top-down regulations can trigger immediate institutional shifts, and sharp SES structural changes \cite{speedBasinwaterallocation2013,rolandUnderstandinginstitutionalchange2004}.
In comparison with investigations of more gradual changes induced by bottom-up institutional shifts, exploring the impacts of a top-down change substantially diminishes potential problems of omitted variables in the quantitative analysis of SES and improves the clarity of the causal link between SES structure and outcome.
% 其次,通过比较长江流域两次制度变迁所分裂的三种不同制度结构的净效应,我们也可以更深入地理解“盆地固定效应”下结构格局的影响。尽管流域内的社会经济单位从世界各地的大型河流流域和许多地区的水资源中受益,但很少有流域多次经历过如此激进的社会经济结构变化。
Second, by comparing the net effects of three different institutional structures split by two institutional shifts in the YRB, we can also better understand of the influence of structural alignments under a fixed basin. Although socioeconomic units within a basin benefit from water resources in large river basins all over the world and many locations have shown increased levels of regulation, few basins have experienced such radical SES structural changes several times (see \textit{Supplementary Material} S1). Thus, the YRB provides a valuable setting for understanding the direct impacts of changes in SES institutional structure.
% 因此,黄河流域的“流域固定效应”为SES结构的自我比较提供了宝贵的机会。
% Thus, the YRB provides valuable settings for understanding the direct impacts of changes in SES institutional structure.

% 未来的政策建议
% 近年来黄河流域面临的分水制度调整问题也说明了动态设置配额的重要性。
Calls for a redesign of water allocation institutions in the YRB in recent years also illustrate the importance of dynamic quota setting (see \textit{Supplementary Material S1}) \cite{yuAdaptabilityassessmentpromotion2019}. Following the institutional reforms of 1998, the Yellow River has not dried up since 1999. However, given recent changes in the YRB, its rigid resource allocation scheme can no longer meet the new demands of economic development \cite{wangThingsCurrentSignificance2019}. The Chinese government has embarked on an ambitious plan to redesign its decades-old water allocation institution (see \textit{Supplementary Material S1}). Other SESs around the world face similar problems in establishing successful resource allocation institutions \cite{cummingQuantifyingSocialEcologicalScale2020, muneepeerakulStrategicbehaviorsgovernance2017, cummingAdvancingunderstandingnatural2020, leslieOperationalizingsocialecologicalsystems2015}. These initiatives can benefit from our analysis by actively considering and incorporating social-ecological complexity and incentive structures when developing new approaches that avoid unsustainable outcomes. Our research provides a cautionary tale of how institutions can act as a double-edged sword when trying to attain sustainability.
