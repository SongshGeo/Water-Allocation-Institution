%! Author = songshgeo
%! Date = 2022/3/10

% \subsection{CAUSES OF INSTITUTIONAL IMPACTS}
% \subsection{}
\label{discussion-1}
% discussion-1:
% 用水量的上升、下降-结果解读
Our results show that while 98-UBR decreased water use in the YRB, 87-WAS increased it.
The results challenged previous analyses (i.e., suggesting that 87-WAS ''had little practical effect'') because theoretically, there should be few gaps between actual and synthetic YRB's water use if no effect is present \cite{abadie2015,hill2021}.
However, the significant net effect indicated by our analysis suggests 87-WAS was followed by more water use even after controlling for environmental and economic variables (see \textit{Appendix~\nameref{secS2}}~Table~\ref{tab:variables}).
Increased water use aligns with concerns about frequently scrambling for water that were voiced in some provinces during this period \cite{mao2000}.
On the contrary, the 98-UBR reduced water competition, so many studies attributed the restoration mainly to the successful introduction of this institution \cite{chen2021,huangang2002,an2007}.

% 我们认为制度结构的变化是被长期忽略的重要因素
We proposed that for these institutional shifts' divergent outcomes regarding water use, the key factors long ignored can be institutional structure changes.
The structural building blocks we depicted in Figure~\ref{structure}, which were also reported in various types of SESs worldwide \cite{kluger2020,guerrero2015,bodin2012}, and here twice reframed by the institutional shifts.
Before the 98-UBR, the SES structure (i.e., fragment ecological units are linked to separate social actors) is more likely to be mismatched because isolated actors generally struggle with holistically maintaining interconnected ecosystems \cite{sayles2017,sayles2019,cai2016,bergsten2019}.
On the contrary, institutional alignments since the 98-UBR improved YRCC's authority in water use regulation to match the YRB in scale, thus leading to social-ecological fit and good outcomes -a ``scale-matched'' example \cite{cumming2020a,wang2019d}.
Despite other reasons for the non-ideal effect of the 87-WAS being widely discussed \cite{huangang2002} (such as enforcement, feasibility, and equity), few radical improvements in the 98-UBR suggest an over-attention paid to non-structural impacts.

% \subsection{Linking structures with ISs outcomes}
\label{discussion-2}
% discussion-2: 机制解释
Different individual provinces' water-use responses to the institutional shifts also support influences from SES structures.
Our results show that the correlation between current water use and changed (increased or decreased) water use was significant after the 87-WAS.
This ``major users use more'' pattern supports the hypothesis that separated stakeholder (individual province here) decides on linked ecological units for output maximization (water use).
Therefore, we theoretically abstracted the SES structures and analyzed how different structures shape different outcomes with provincial heterogeneity.

Thus, our model suggests that a unified scale-matched institution is indispensable for sustainable water use.

% The effect of the institutional shifts once again demonstrated that it is not easy to have a win-win situation of environment and interests in complex coupled human-nature systems \cite{hegwood2022}, which calls for exceptional understanding and caution to the structure of hampering sustainability \cite{bergsten2019, sayles2019}.
Our above theoretical discussion also aligns with the evolution of the institutions in the YRB.
When links The water quotas of 87-WAS (or the initial water rights) went through a stage of ``bargaining'' among stakeholders (from 1982 to 1987) \cite{wang2019a, wang2019d}, where each province attempted to demonstrate its development potential related to water use.
The bargaining was also a process for matching their water shares to economic volume because the major water users (like Shandong and Henan) need more water than their quota (if only considering economic potentials when designing the institution) \cite{zuo2020}.
Furthermore, those with more current water use might have greater bargaining power in water use allocation because of information asymmetry between decision-makers and stakeholders.
Therefore, stakeholders had considerable incentives to prevent water quotas from hindering their economic potential, which aligns with the fact that they appeared to the higher central government for larger shares \cite{wang2019a, wang2019d}.
On the contrary, after YRCC as governing agent coordinated between stakeholders since 98-UBR, the external appeal of provinces for larger quotas turned into internal innovation to improve water efficiency (e.g., drastically increased water-conserving equipment)
\cite{krieger1955, ostrom1990}.
Then, the YRCC, the authority for approving water applications from all stakeholders, could adjust water use quotas according to the river conditions of the whole basin.
The 98-UBR thus led to expected institutional outcomes at a basin scale, indicating that successful governance of SES emerged by indirectly (or vertically) creating links between different stakeholders.

\subsection{LIMITATION, INSIGHTS AND IMPLICATIONS}
% \subsection{Limitation, insights and implications}
\label{discussion-3}
% discussion-3: 启示、未来的展望

We explored these causal linkages between the SES structures and sustainability (outcomes) by quasi-natural experiments (the institutional shifts) of the YRB, which provides a unique case for two main reasons:
(1) The top-down institutional shifts induced sharp changes in SES structures in the YRB, enabling us to estimate their net effects quantitatively.
(2) Since few basins experienced such radical institutional shifts more than once, the YRB provides comparable settings for understanding the structural changes of SESs.

In complex coupled human-nature systems, however, our approach also has inevitable limitations:
(1) The contributions of economic growth and institutional shifts are still incomparable because of intertwined causality (institutional changes can also influence the relative economic variables).
Still, our quasi-experiment approach focused on and proved that a change in water use trajectory follows the YRB's unique institutional shifts.
(2) When applying the DSC method, it is difficult to rule out the effects of other policies over the same time breakpoints (1987 and 1998).
However, since the analysis had agreed on the importance of the two institutional shifts of 87-WAS and 98-UBR in water governance \cite{wang2012b}, our results on their effects still provide insights for understanding water governance.

% 我们的模型结果与机制探讨加深了SES结构的理解,强化了孤立利益相关者形成的结构不利于制度解决环境问题的基本认识
Our results and discussion deepen the understanding of SES structure and strengthen the basic knowledge that the mismatched structure (isolated stakeholders with the fragmentation of ecology) is not conducive to institutional solutions \cite{bodin2017b, ostrom2009, reyers2018}.
Moreover, we report how another structure contributed to successful governance -the subsequent success of the 98-UBR institutional shift theoretically and practically proved the importance of social-ecological fit again.
For sustainability in the future, therefore, it is necessary to emphasize the necessity of strengthening connections between stakeholders by agents consistent with the scale of the ecological system.
From these perspectives, we give two scenarios based on the marginal benefit model (see \textit{Appendix \nameref{secS5}}), which can inspire institutional design on how to reduce mismatches.
For example, water rights transfers can be another way to emerge horizontal links between stakeholders that also have the potential to result in better water governance.
In addition, the policymakers can propose more dynamic and flexible institutions to increase the adaptation of stakeholders to respond to changing SES context \cite{reyers2018}.

The structural building blocks that led to different outcomes are recurring motifs in global SESs, so our proposed mechanism is crucial to governing such coupled systems.
Calls for a redesign of water allocation institutions in the YRB in recent years also illustrate the importance of institutional solutions to sustainability (see \textit{Appendix \nameref{secS1}}) \cite{yu2019}.
Given the changing environmental context, outdated and inflexible water quotas can no longer meet the demands of sustainable development \cite{wang2019a}.
Thus, the Chinese government has embarked on a plan to redesign its decades-old water allocation institution (see \textit{Appendix~\nameref{secS1}}).
These initiatives can benefit from our analysis by actively incorporating social-ecological matched building blocks when developing a new institutional shift for sustainability \cite{bodin2017b}.
Moreover, our research provides a cautionary tale of how institutions can be double-edged \cite{hegwood2022}, while insights from the YRB can be a valuable guideline for SESs worldwide \cite{muneepeerakul2017, leslie2015}.
