%! Author = songshgeo
%! Date = 2022/3/10

% \subsection{CAUSES OF INSTITUTIONAL IMPACTS}
% \subsection{}
% discussion-1:
% 用水量的上升、下降-结果解读
% 制度对社会生态系统的结果产生影响在世界范围内都很普遍,
The influences of institutional shifts on governing effects of social-ecological systems (SESs) were widely reported worldwide, but few attempts to quantify their net effects~\cite{cumming2020a}.
Our case study of the YRB's water governance suggests that while 98-UBR decreased total water use as expected, 87-WAS unexpected increased it by $5.75\%$, comparison of which can produce insights on effectiveness of governance.
Firstly, the results challenged previous analyses (i.e., suggesting that 87-WAS ``had little practical effect'') because theoretically, there should be few gaps between actual and synthetic water use in the YRB if no effect is present~\cite{abadie2015,hill2021}.
However, the significant net effect indicated by our analysis suggests 87-WAS was followed by more water use even after controlling for environmental and economic variables (see \textit{\nameref{secS2}}~Table~\ref{tab:variables}).
On the contrary, the 98-UBR reduced surface water competition, so many studies attributed the streamflow restoration mainly to the successful introduction of this institution~\cite{chen2021,huangang2002,an2007}.

Examining the unexpected 87-WAS policy, we found it framed a similar structure to many other SES governance failures, supporting the hypothesis that specific mismatched structures can rapidly exhaust common resources~\cite{kellenberg2009,cai2016,barnes2019}.
Generally, these structure-based failures when social actors freely access linked resource units (like the institution before 1987), while the monitoring duty of YRCC after 87-WAS was a sign that water quota was valuable to pursue (between 1987 and 1998).
This conjecture aligns with the increased water use after 87-WAS and the concerns about frequently scrambling for water in some provinces during this period~\cite{mao2000, bouckaert2022}.
Previous study analyzed reasons for the non-ideal effect of 87-WAS~\cite{huangang2002} where core concerns were lack of enforcement and controlling approaches, while major stakeholders kept arguing they need more quotas from 1983 to 1990s.
It indicates that it was reasonable for stakeholders to pursue more water quota by withdrawing more water, beyond their economic growth.
Our results stand in line with this hypothesis since the correlation between current water use and changed (increased or decreased) water use was significant after the 87-WAS but not after the 98-UBR (Figure~\ref{fig:regulating}).
In addition, through a theoretically marginal benefit analysis, this ``major users use more'' pattern can be inferred from a simple assumption that stakeholder can expect water quota's value in the short future, also supporting the above hypothesis (see Figure~\ref{fig:model}).

Besides our results, the above hypothesis also keeps in line with two reported facts:
(1) The water quotas of 87-WAS (or the initial water rights) went through a stage of ``bargaining'' among stakeholders (from 1982 to 1987) and the bargaining arguments even last years after 1987~\cite{wang2019a, wang2019d}.
It was the process where each province attempted to demonstrate its development potential related to water use, for matching water shares to economy because the major water users (like Shandong and Henan) needed more water than their original quota (if only considering economic potentials when designing the institution)~\cite{zuo2020}.
(2) During the ``bargaining'', more significant stakeholders had considerable incentives to pursue more water quotas, which aligned with those major water users submitted appeals to the higher central government for larger shares~\cite{wang2019a, wang2019d}.
It means provinces with higher current water use have greater bargaining power in water use allocation.

On the other hand, since the YRCC could coordinate stakeholders by water quotas licenses according to water conditions for the whole YRB after 98-UBR, the external appeals of provinces for larger quotas turned into internal innovation to improve water efficiency (e.g., drastically increased water-conserving equipment)~\cite{krieger1955, ostrom1990}.
Again, proportional decreased water use of provinces and the theoretical minimal water use of marginal benefit model indicated this policy lead to successful governance as expected (see \nameref{result-3}).
However, since the 98-UBR only regulated surface water use, many clues suggested the institution shift may cause broader influences, including estimated increased groundwater withdrawals after 98-UBR in many intensive water use regions~\cite{sun2022b}.
With little eligible data on groundwater use, related assessment is out of this study but quite important, as similar water quota policies started to be implemented nationally since the 21st century.

The structural pattern we depicted here (Figure~\ref{fig:structure}) have also been reported in other SESs worldwide~\cite{kluger2020,guerrero2015,bodin2012}.
Before 98-UBR, fragment ecological units linked to separate social actors, which can more likely lead to lower effectiveness because isolated actors generally struggle to maintain interconnected ecosystems holistically~\cite{sayles2017,sayles2019,cai2016,bergsten2019}.
Institutional re-alignments since 98-UBR enhanced the responsibilities of the basin-scale authority (YRCC) and led to effectiveness in runoff restoration, which are usually named scale match or institutional match of SESs~\cite{cumming2020a,wang2019d}.
Taken together, the comparison demonstrates again the challenge of finding win-win situations in coupled social-ecological systems~\cite{hegwood2022}, and the need to more deeply understand the role of institutions in water governance~\cite{bergsten2019, sayles2019}.
% Therefore, the YRB cases can provide further explanation of the matching and mismatching of the previous SES building blocks, linking SES structure and outcomes by plausible reasons of causality and underlying processes.

% \subsection{LIMITATION, INSIGHTS AND IMPLICATIONS}
% \subsection{Limitation, insights and implications}
% \label{discussion-4}
% discussion-3: 启示、未来的展望

Our approach has some inevitable limitations.
First, the contributions of economic growth and institutional shifts are difficult to distinguish because of intertwined causality (institutional changes can also influence the relative economic variables);
and second, when applying the DSC method, it is difficult to rule out the effects of other policies over the same time breakpoints (1987 and 1998).
Our quasi-experiment approach nonetheless provides evidence supporting the view that there was a change in water use trajectory following the YRB's unique institutional shifts and offers insights into water governance (and particularly the importance of having a scale-matched, basin-wide authority for water allocation solutions~\cite{bodin2017b, ostrom2009, reyers2018})
Moreover, the ultimate success of the 98-UBR institutional shift theoretically and practically proved the importance of social-ecological fit.
For sustainability in the future, therefore, it is necessary to emphasize the necessity of strengthening connections between stakeholders by agents consistent with the scale of the ecological system.
From these perspectives, two scenarios based on the marginal benefit analysis (see \textit{\nameref{secS5}}) can inspire institutional design on how to reduce mismatches.
For example, water rights transfers may be another way to build horizontal links between stakeholders that also have the potential to result in better water governance.
In addition, policymakers can propose more dynamic and flexible institutions to increase the adaptation of stakeholders to a changing SES context~\cite{reyers2018}.

The structural pattern that led to different effectiveness widespread in global SESs, so our proposed mechanism is crucial to governing such coupled systems.
As both institutional shifts examined here induced unexpected changes or cascading effects within SESs, better governance calls for more institutional analysis of coupled human and natural systems in the future.
Calls for a redesign of water allocation institutions in the YRB in recent years also illustrate the importance of institutional solutions for sustainability (see \textit{\nameref{secS1}})~\cite{yu2019, niu2022}.
Given the changing environmental context, outdated and inflexible water quotas can no longer meet the demands of sustainable development~\cite{wang2019a}.
Thus, the Chinese government has embarked on a plan to redesign its decades-old water allocation scheme (see \textit{\nameref{secS1}}).
Our analysis suggests that these initiatives can lead to unexpected result because of mismatched structure when shifting to another institution~\cite{bodin2017b}.
Therefore, our research provides insights of how institutions can effect achieving successful river basin governance when socio-hydrological interplays framing  SES structures~\cite{muneepeerakul2017, leslie2015, hegwood2022}.
