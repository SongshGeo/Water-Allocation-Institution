%! Author = songshgeo
%! Date = 2022/3/10

% \subsection{CAUSES OF INSTITUTIONAL IMPACTS}
% \subsection{}
\label{discussion-1}
% discussion-1: 主要介绍结果的意义、合理性
In addition to quantitatively demonstrating the regulatory effect of 98-UBR by previous studies, our study also found that 87-WAS would increase overall water use of the YRB.
The results challenged the previous analyses suggesting that the 87-WAS ``has little effect'' because the difference between prediction and observation will be trivial when the institutional shift was just a blank policy by applying the DSC method. %! (cites)
Fixing the environmental background, the forecast by DSC takes economic factors into account under the assumptions that the production function between economic volume and water uses remained unchanged (\textit{S2 in Supplementary Material}).
As the accumulations of economy volume (GDP in different sectors) in the YRB maintained a parallel trend with other regions throughout (\textit{Appendix S3} Figure~\ref{S3-1}), differences after the institutional shifts suggest water use changing per unit of production, especially in agriculture (\textit{S3 in Supplementary Material} Figure~\ref{S3-2}).
This fact is in line with the sigh from then: although the key to alleviating the drought is saving water in the irrigated areas, the tragedy of frequently scrambling for water appeared in some provinces %! \cite{mao2000}.
Since the 98-UBR improved tragedy of water competition greatly, many studies attributed the restoration from river depletion mainly to the successful institutional shift.

% 过去的研究总结出87-WAS收效甚微的几个因素:
Although previous studies summarised reasons for the non-ideal effect of 87-WAS, few improvements in the 98-UBR indicate that they underestimated influences from the structural changes (\textit{S3 in Supplementary Material} Figure~\ref{fig:S3-3}).
As we have depicted (Figure~\ref{structure}), the institutional shifts twice framed the structure of SESs in the YRB and led to different building blocks, which were also reported in various types of SESs worldwide.
The empirical studies in many different fields also indicate that the structure before 98-UBR (i.e., fragment ecological units are linked to separate social actors) is likely to be mismatched as isolated stakeholders struggle with holistically maintaining interconnected ecosystems
\cite{sayles2017,sayles2019,cai2016,bergsten2019}.
On the contrary, the YRCC, whose authority matched the YRB in scale after the 98-UBR, led to a well-recognized structure for institutional alignments to social-ecological fit and good outcomes.
The effect of the institutional shifts once again demonstrated that it is not easy to have a win-win situation of environment and interests in complex coupled human-nature systems \cite{hegwood2022} which calls for exceptional understanding and caution to the structure of hampering sustainability \cite{bergsten2019, sayles2019}.

% \subsection{LINKING STRUCTURES WITH ISS OUTCOMES}
% \subsection{Linking structures with ISs outcomes}
\label{discussion-2}
% discussion-2: 机制解释
% 经济模型与理论解释
% Differences between stakeholders in responses to institutional shifts are vital to understanding the mechanism between structures and outcomes.
Differences in the pattern of the response by provinces can demonstrate the influence of social-ecological structures led by the institutional shifts.
We analyzed mathematically why the mismatched structure made limited water use holistically elusive in the institution shift of the 87-WAS but finally achieved by the 98-UBR (\textit{method} and \textit{Supplementary Material S4}).
By taking the structure before and after the two institutional shifts as different basic assumptions (before 87-WAS: free access to water; after 87-WAS but before 98-UBR: decisions on water use under quotas; after 98-UBR: unified regulation), we use the marginal benefit model to analyze the theoretical optimal water consumption of stakeholders in each scenario.
The analysis of the model also shows that 98-UBR can reduce the overall water use of the basin while 87-WAS can increase the water use of the basin when the same parameters are guaranteed but the institutional structure changes.
Before the 98-UBR, the model assumes that the separated ecological units (river reaches) link to stakeholders (related provinces) who use water to pursue their marginal benefits but have a potential political cost if they exceed the quota 87-WAS.
Our model suggests that for users who are already economically efficient (who are already using more water), greater marginal returns from water induce the acceleration of extracting resources for future economic growth (\textit{Supplementary Material S4}).
Therefore, isolated stakeholders reacted to the similar marginal cost, and smaller water users have a threshold because of the political cost, so 87-WAS triggered an increased water use for the significant users.
On the contrary, the presence of central management (by the YRCC in this case, after 1998) can effectively reduce marginal ecological costs holistically as stakeholders only take corresponding responsibilities (follow the quota as possible as they can) to the YRCC (\textit{Supplementary Material S4}).
As a result, unified regulating acted the core role after the 98-UBR and reduced water use of all stakeholders (provinces) by irregular ratios.

The alignments of differences in institutional structures and outcomes here echo the hypothesis that successful governance of SES emerged by indirectly (or vertically) creating links between different stakeholders (in the YRB cases, through administration).
When links The water quotas of 87-WAS (or the initial water rights) in our case studies went through a stage of ``bargaining'' among stakeholders (from 1982 to 1987) \cite{wang2019a, wang2019d}, where each province attempted to demonstrate its development potential related to water use.
The bargaining itself was also a process towards matches between their economic volume and water shares, as studies show that the large water users (like Shandong and Henan) need more water than their quota (in the 87-WAS) if only considering the economic equity when designing the institution.
Furthermore, with information asymmetry between upper-level decision-makers and lower-level stakeholders in water use allocation, those with more current water use might have greater bargaining power.
In practice, therefore, although the affected provinces may not have directly encouraged excessive resource use because of the institutional shift, they had a more considerable incentive to show their economic potential
That aligns with the historical records that, even after the 87-WAS had already confirmed the quotas, provinces, especially water-intensive ones, challenged it by appearing to the higher central government for larger quotas.
On the contrary, after YRCC as governing agent coordinated between stakeholders since 98-UBR, the external appeal of provinces for larger quotas turned into internal innovation to improve water efficiency (e.g., drastically increased water-conserving equipment, \textit{Supplementary Material S3})
\cite{krieger1955, ostrom1990}.
Then, the YRCC, the authority for approving water applications from all stakeholders, could adjust water use quotas according to the river conditions of the whole basin.
The 98-UBR led to a structure for achieving social-ecological fits in both basins (between YRCC and the YRB) and regions (between provincial economy and their water shares).

\subsection{LIMITATION, INSIGHTS AND IMPLICATIONS}
% \subsection{Limitation, insights and implications}
\label{discussion-3}
% discussion-3: 启示、未来的展望

Matching social and ecological scales appears widespread as building blocks (or motifs) in successfully governed SES, whether in fisheries, forests, or groundwater management.
Since the building blocks introduced by 87-WAS and 98-UBR are recurring motifs in many SES, our proposed mechanism is crucial to understanding such coupled systems.
We explored these causal linkages between the SES structures and sustainability (outcomes) by quasi-natural experiments (the institutional shifts) of the YRB, which provides an informative case study for two main reasons.
First, different from gradual changes following bottom-up emergence, the top-down institutional shifts induced sharp changes in SES structures in the YRB, enabling us to estimate their net effects quantitatively.
Second, as few basins experienced such radical institutional shifts more than once, the YRB provides comparable settings for understanding the impacts of structural changes in SESs.
However, one of the inevitable limitations of our method is that it is difficult to rule out the effects of other policies over the same time breakpoints.
Since scholars have reached a consensus on the importance of the two institutional shifts of 87-WAS and 98-UBR, the differences in their results still provide important insights for understanding water governance.

% 我们的模型结果与机制探讨加深了SES结构的理解,强化了孤立利益相关者形成的结构不利于制度解决环境问题的基本认识
Our results and discussion deepen the understanding of SES structure and strengthen the basic understanding that the mismatched structure (isolated stakeholders with the fragmentation of ecology) is not conducive to institutional solutions.
Moreover, we report how another institutional shift contributed to successful water governance  -the subsequent success of 98-UBR has proved the importance of social-ecological fit again, theoretically and practically.
For sustainability in the future, therefore, it is necessary to emphasize the necessity of strengthening connections between stakeholders by agents consistent with the scale of the ecological system (in this case, the basinal scale and the YRCC).
From these starting points, several other scenarios given by a marginal benefit model (see \textit{Supplementary Material S4}) can provide plausible insights into water governance.
For example, water rights transfers can be another way to emerge horizontal links between stakeholders that also have the potential to result in better water governance.
In addition, the policymakers can propose more dynamic and flexible institutions to increase the adaptation of stakeholders to respond to changing SES context.

% 未来的政策建议
Calls for a redesign of water allocation institutions in the YRB in recent years also illustrate the importance of institutional solutions to sustainability (see \textit{Supplementary Material S1}) \cite{yu2019}.
Given recent changes in the YRB, outdated and inflexible water quota can no longer meet the new demands of economic development \cite{wang2019a}.
As a result, the Chinese government has embarked on a plan to redesign its decades-old water allocation institution (see \textit{Supplementary Material S1}).
These initiatives can benefit from our analysis by actively incorporating social-ecological matched building blocks when developing a new institutional shift toward sustainability.
Moreover, our research provides a cautionary tale of how institutions can be double-edged, while insights from the YRB can be a valuable guideline for SESs worldwide \cite{cumming2020b, muneepeerakul2017, cumming2020a, leslie2015}.
