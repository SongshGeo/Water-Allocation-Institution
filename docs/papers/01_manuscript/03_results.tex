%! Author = songshgeo
%! Date = 2022/3/10
% \MakeUppercase{\subsection{Cascading effects of the institutional shifts}}

% \subsection{ISS IMPACT ON WATER USE OF THE YRB}
\subsection{ISs impact on water use of the YRB}
\label{result-2}
% 结果一:展示制度转变带来的用水量变化

\label{result-1-p2}
Here, we use Differenced Synthetic Control (DSC) method, which considers economic growth and natural background, to estimate theoretical water use scenarios without basinal policy interferences (\textbf{Methods}; \textit{S2 in Supplementary Material}).
Our results suggest that the institutional shift in 1987 (87-WAS) stimulated the provinces to withdraw more water than would have been used without the interference (Figure~\ref{main_results}A).
From 1988 to 1998, while the estimation of water use only suggests $956.38 km^3$, the observed water use of the YRB provinces reached $1038.36 km^3$ in sum, $8.57\%$ increased.
However, after the institution shifted again in 1998 (98-UBR), the trend of increasing water use appeared to be effectively suppressed. From 1998 to 2008, the total observed water use decreased by $0.49 km^3$ per year, while the estimation of water use still suggests $1.03 km^3$ increases (Figure~\ref{main_results} B).
The increased water uses after 87-WAS aligns with the fact that badly drying-up of the surface streamflow from 1987 to 1998, which was an obvious touchstone of river degradation and environmental crisis (Figure~\ref{main_results}C).
On the other hand, the environmental crisis of river drying up was effectively resolved after the 98-UBR, though the density of droughts still increased for decades (from $0.47$ after 87-WAS to $0.62$ after 98-UBR on average) (Figure~\ref{main_results}C).
In line with previous literature had reported; therefore, the institution shift of 98-UBR contributed a lot to the successful water governance. %! citation

\begin{figure*}[!h]
    \centering
    \includegraphics[width=32pc]{outputs/main_results2.pdf}
    \caption{
        Effects of two institutional shifts on water resources use and allocation in the Yellow River Basin (YRB).
        \textbf{A.} water uses of the YRB before and after the institutional shift in 1987 (87-WAS);
        \textbf{B.} water uses of the YRB before and after the institutional shift in 1998 (98-UBR). While the blue lines are statistic water use data, the grey ones are the estimation from the Differenced Synthetic Control method with economic and environmental background controlled.
        \textbf{C.} Drought intensity in the YRB and drying up events of the Yellow River. The size of the grey bubbles denotes the length of a drying upstream.
    }
    \label{main_results}
\end{figure*}


% \subsection{REGIONAL DIFFERENCES IN RESPONSES TO THE IS}
\subsection{Regional differences in responses to the ISs}
\label{result-3}
% 结果2部分:展示区域相应差异

Differences between stakeholders in responses to institutional shifts are vital to understanding the mechanism between structures and outcomes.
Our results show that the proportion of accelerated water use in each province after the decade of 87-WAS (the proportion of actual water use exceeding the predicted water use by the model) has a significant correlation ($p<0.05$, see \textbf{Methods}) to the Yellow River water use in each province (Figure~\ref{upset}A).
Furthermore, while no evident impacts for most provinces (no more than $10\%$ differences, the apparent acceleration effects were only prominent in the big water-using provinces (e.g., Neimeng, Henan, and Shandong. Figure~\ref{upset}B).
In particular, Neimeng and Shandong, both provinces that exceeded the prescribed water uses of the 87-WAS, used $44.25\%$ and $25.69\%$ more water uses than the prediction from 1987 to 1998, respectively.
Furthermore, the satisfaction of each province with the water allocation stipulated by the 87-WAS (expressed by the difference between the actual water allocation of each province and the expected planning value) has little significant correlation to the acceleration.
By contrast, after the 98-UBR, except Shaanxi (which has always been abundant in water quota) had an evident ($17.53\%$) increase in water use, almost all provinces have seen significant declines in water use ($-12.5\%$ on average).
However, neither the satisfaction nor Yellow River water use correlates with the declines after the 98-UBR.

\begin{figure*}[!h]
    \centering
    \includegraphics[width=32pc]{outputs/upset_87.pdf}
    \caption{
        \textbf{A.} The partial correlation coefficient between wate uses (WU) of Yellow River (YR), unsatisfied ratio (compared with requirements in water plan and supply in the 87-WAS), and the average accelerated ratio.
        \textbf{B.} Average accelerated ratio of water uses for each province in the YRB during the decade after 87-WAS (from 1987 to 1998).
        \textbf{Mian users:} Major water consumption provinces (over the median).
        \textbf{Overused:} violate the 87-WAS in average water uses.
    }
    \label{upset}
\end{figure*}
