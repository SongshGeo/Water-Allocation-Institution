%! Author = songshgeo
%! Date = 2022/3/10
% \MakeUppercase{\subsection{Cascading effects of the institutional shifts}}

%\subsection{INSTITUTIONAL SHIFTS IMPACT ON WATER USE OF THE YRB}
\subsection{Institutional shifts impact on water use}
\label{result-2}
% 结果一:展示制度转变带来的用水量变化

\begin{figure*}[!htb]
	\centering
	\includegraphics[width=0.9\linewidth]{outputs/main_results2.pdf}
	\caption{
		Effects of two institutional shifts on water resources use and allocation in the Yellow River Basin (YRB).
		\textbf{A.} water uses of the YRB before and after the institutional shift in 1987 (87-WAS);
		\textbf{B.} water uses of the YRB before and after the institutional shift in 1998 (98-UBR). Blue lines are statistics derived from water use data; grey lines are estimates from the Differenced Synthetic Control method with economic and environmental background controlled;
		\textbf{C.} Drought intensity in the YRB and drying up events of the Yellow River. The size of the grey bubbles denotes the length of drying upstream.
	}
	\label{fig:main_results}
\end{figure*}


\label{result-1-p2}
Our estimation of theoretical water use suggests that the institutional shift in 1987 (87-WAS) stimulated the provinces to withdraw more water than would have been used without an institutional shift (Figure~\ref{fig:main_results}A).
From 1988 to 1998, on average, while the estimation of annual water use only suggests $974.34$ billion $m^3$, the observed water use of the YRB provinces reached $1038.36$ billion $m^3$ (an increase of $6.57\%$).
However, after the institutional change in 1998 (98-UBR), trends of increasing water use appeared to be effectively suppressed. From 1998 to 2008, the total observed water use decreased by $0.49$ billion $m^3/yr$ per year, while the estimation of water use still suggests $0.82$ billion $m^3/yr$ increases (Figure~\ref{fig:main_results} B).
The increased water uses after 87-WAS aligns with the severe drying-down of the surface streamflow from 1987 to 1998, an obvious indicator of river degradation and environmental crisis (Figure~\ref{fig:main_results}C).
On the other hand, the 98-UBR ended river depletion, despite subsequent increases in drought intensity (from $0.47$ after 87-WAS to $0.62$ after 98-UBR on average) (Figure~\ref{fig:main_results}C).



%\subsection{REGIONAL DIFFERENCES IN RESPONSES TO INSTITUTIONAL SHIFTS}
\subsection{Heterogeneous effects and interpretation}
\label{result-3}

\begin{figure*}[!htb]
	\centering
	\includegraphics[width=0.9\linewidth]{outputs/fig3.pdf}
	\caption{
		Regulating differences for provinces in the YRB.
		Red (the 87-WAS) and green (the 98-UBR) bars denote an increased or decreased ratio for actual water use relative to the estimate from the model in the decade after the institutional shift.
		The grey bars indicate the proportions of actual water use for each province relative to their total water use in the decade after the institutional shift.
		The triangles mark the water quotas assigned under the institution, converted to ratios by dividing by their sum.
	}
	\label{fig:regulating}
\end{figure*}

% 结果2部分:展示区域相应差异
Our results also suggest differences between patterns of provinces in their responses to the two institutional regulating.
During the decade after the 87-WAS, the major water-using provinces (e.g., Inner Mongolia, Henan, Shandong) had apparent accelerations (Figure~\ref{fig:regulating}).
The proportion of increased (or decreased) water use for each province (over the estimated water use by the model) correlated significantly (partial correlation coefficient is $0.77$, $p<0.05$) with actual water use from the Yellow River.
On average, the major water users (Shandong, Inner Mongolia, Henan, and Ningxia) used $32.14\%$ more water than predicted from 1987 to 1998.
By contrast, after the 98-UBR (from 1998 to 2008), almost all provinces have seen declines ($-16.54\%$ on average) in water use.
Furthermore, the regulated water use of provinces was unrelated (partial correlation coefficient is $0.33$, $p>0.1$) to their proportional water use from the Yellow River.



% \subsection{Structure-based marginal benefit analysis}
% \label{result-4}

\begin{figure}[!htb]
	\centering
	\includegraphics[width=0.6\linewidth]{outputs/economic_model.pdf}
	\caption{
		The proposed relationship of marginal benefits and water use of individual province under varying cases (case 1 to case 3, corresponding to the different SES structures in Figure~\ref{fig:structure}~C) Major water users' theoretically optimal water use is also larger (see \nameref{sec:model} and \textit{\nameref{secS4}}).}
	\label{fig:model}
\end{figure}

For interpretation of the pattern, we compared the theoretical marginal returns and optimal water use under three different structural cases (case 1 to case 3, corresponding to different SES structures in Figure~\ref{fig:structure}~C, see \nameref{sec:model}~Figure~\ref{fig:model}, detailed derivation in \textit{\nameref{secS4}}).
Assuming that water is the factor input with decreasing marginal output of each province, results show that varying incentives for water use in each province derive from the relationship between the benefits and costs of water use.
As a benchmark, case 1 analogy to a decentralized stakeholders situation and lead to medium-level water use.
In case 2, each stakeholder expects that current water use helps bargain for a favorable water quota in the face of institutional shift (see \textit{\nameref{secS4}}), which can intensify the incentive to use water, leading to higher water use.
Furthermore, the water users with higher capability are more stimulated by the institutional shift and away from the theoretically optimal water use under a unified allocation.
After water-use decisions are consolidated into unified management (case 3), marginal benefits analysis suggests the lowest water use among the cases.
