%! Author = songshgeo
%! Date = 2022/3/10
% \MakeUppercase{\subsection{Cascading effects of the institutional shifts}}

%\subsection{INSTITUTIONAL SHIFTS IMPACT ON WATER USE OF THE YRB}
\subsection{Institutional shifts impact on water use}
\label{result-2}
% 结果一:展示制度转变带来的用水量变化

\label{result-1-p2}
Our estimation of theoretical water use suggests that the institutional shift in 1987 (87-WAS) stimulated the provinces to withdraw more water than would have been used without an institutional shift (Figure~\ref{main_results}A).
From 1988 to 1998, on average, while the estimation of annual water use only suggests $956.38$ billion $m^3$, the observed water use of the YRB provinces reached $1038.36$ billion $m^3$ in sum, $8.57\%$ increased.
However, after the institution shifted again in 1998 (98-UBR), the trend of increasing water use appeared to be effectively suppressed. From 1998 to 2008, the total observed water use decreased by $4.9$ billion $m^3/yr$ per year, while the estimation of water use still suggests $10.3$ billion $m^3/yr$ increases (Figure~\ref{main_results} B).
The increased water uses after 87-WAS aligns with the fact that badly drying-up of the surface streamflow from 1987 to 1998, which was an obvious touchstone of river degradation and environmental crisis (Figure~\ref{main_results}C).
On the other hand, the 98-UBR ended river depletion, despite the intensity of droughts still increasing for decades (from $0.47$ after 87-WAS to $0.62$ after 98-UBR on average) (Figure~\ref{main_results}C).

\begin{figure*}[!tb]
    \centering
    \includegraphics[width=32pc]{outputs/main_results2.pdf}
    \caption{
        Effects of two institutional shifts on water resources use and allocation in the Yellow River Basin (YRB).
        \textbf{A.} water uses of the YRB before and after the institutional shift in 1987 (87-WAS);
        \textbf{B.} water uses of the YRB before and after the institutional shift in 1998 (98-UBR). While the blue lines are statistic water use data, the grey ones are the estimation from the Differenced Synthetic Control method with economic and environmental background controlled.
        \textbf{C.} Drought intensity in the YRB and drying up events of the Yellow River. The size of the grey bubbles denotes the length of a drying upstream.
    }
    \label{main_results}
\end{figure*}


%\subsection{REGIONAL DIFFERENCES IN RESPONSES TO INSTITUTIONAL SHIFTS}
\subsection{Institutional effects on regulating differences}
\label{result-3}
% 结果2部分:展示区域相应差异
Our results also suggest differences between patterns of provinces in their responses to the two institutional regulating.
During the decade after the 87-WAS, the major water-using provinces (e.g., Inner Mongolia, Henan, Shandong) had apparent accelerations (Figure~\ref{regulating}).
The proportion of increased (or decreased) water use for each province (over the estimated water use by the model) has a significant correlation (partial correlation coefficient is $0.84$, $p<0.05$) to the actual water use from the Yellow River.
In particular, Inner Mongolia and Shandong, both provinces that exceeded the prescribed water uses of the 87-WAS, used $44.25\%$ and $25.69\%$ more water uses than the prediction from 1987 to 1998, respectively.
By contrast, after the 98-UBR, except Shanxi (whose water quota has always been far abundant over its actual water use since the 87-WAS) had an evident ($17.53\%$) increase in water use, almost all provinces have seen evident declines in water use ($-12.5\%$ on average).
Furthermore, the regulated water use of provinces was not correlated (partial correlation coefficient is $-0.03$, $p>0.1$) with their water use from the Yellow River in proportions.

\begin{figure*}[!tb]
    \centering
    \includegraphics[width=32pc]{outputs/fig3.pdf}
    \caption{
        Regulating differences for provinces in the YRB.
        Red and green bars denote actual water use over the estimation from the model in a decade after the institutional shift -the 87-WAS and the 98-UBR, respectively.
        The grey bars indicate the proportions of actual water use for each province to total water use of the provinces in a decade after the institutional shift.
        The triangles mark the water quotas assigned in the institution, scaled into ratios by the same total actual water use, too.
    }
    \label{regulating}
\end{figure*}
