%! Author = songshgeo
%! Date = 2022/3/10
% \MakeUppercase{\subsection{Cascading effects of the institutional shifts}}

% \subsection{}

%\subsection{INSTITUTIONAL SHIFTS IMPACT ON WATER USE OF THE YRB}
\subsection{Institutional shifts impact on water use}\label{result-2}
% 结果一:展示制度转变带来的用水量变化

\begin{figure*}[!htb]
	\centering
	\includegraphics[width=0.9\linewidth]{outputs/main_results2.pdf}
	\caption{
	Effects of two institutional shifts on water resources use and allocation in the Yellow River Basin (YRB).
	\textbf{A.} Water uses of the YRB before and after the institutional shift in 1987 (87-WAS);
	\textbf{B.} Water uses of the YRB before and after the institutional shift in 1998 (98-UBR). Blue lines are statistics derived from water use data; grey lines are estimates from the Differenced Synthetic Control method with economic and environmental background controlled;
	\textbf{C.} Drought intensity in the YRB and drying up events of the Yellow River. The size of the grey bubbles denotes the length of drying upstream.
	}\label{fig:main_results}
\end{figure*}

% 黄河流域的总用水量在反事实推断模型和实际观测值在两次制度变化后呈现出差异显著,在之前此差异则较小且不显著(见图\ref{ch5:fig:main_results}A和B),这表明其用水变化的估计重建良好。
The total water use of the YRB exhibited a significant difference between the counterfactual prediction and the actual observed value after the two institutional shifts, while the difference was small and insignificant before (see Figures~\ref{fig:main_results}A and B). This indicates that the estimated reconstruction of water use change was robust.
Figure~\ref{fig:main_results}A suggests that the 87-WAS prompted the provinces to withdraw even more water than would have been used without an institutional shift (Figure~\ref{fig:main_results}A).
From 1988 to 1998, on average, while the estimation of annual water use only suggests $887.05~km^3$ billion $m^3$, the observed water use of the YRB provinces reached $938.06$ billion $m^3$ (an increase of $5.75\%$).
However, after the 98-UBR, trends of increasing water use appeared to be effectively suppressed.
From 1998 to 2008, the total observed water use decreased by $6.6$ billion $m^3/yr$ per year, while the estimation of water use still suggests $5.5$ billion $m^3/yr$ increases (Figure~\ref{fig:main_results} B).
The increased water uses after 87-WAS align with the severe dry-up of the surface streamflow from $1987$ to $1998$, a clear indicator of river degradation and environmental crisis (Figure~\ref{fig:main_results}C).
On the other hand, the 98-UBR ended river depletion, despite subsequent increases in drought intensity (from $0.47$ after 87-WAS to $0.62$ after 98-UBR on average) (Figure~\ref{fig:main_results}C).

%\subsection{REGIONAL DIFFERENCES IN RESPONSES TO INSTITUTIONAL SHIFTS}
\subsection{Heterogeneous effects and interpretation}\label{result-3}

\begin{figure*}[!htb]
	\centering
	\includegraphics[width=0.9\linewidth]{outputs/fig3.pdf}
	\caption{
		Regulating differences for provinces in the YRB.\\
		Red (the 87-WAS) and green (the 98-UBR) bars denote an increased or decreased ratio for actual water use relative to the estimate from the model in the decade after the institutional shift.
		The grey bars indicate the proportions of actual water use for each province relative to their total water use in the decade after the institutional shift.
		The triangles mark the water quotas assigned under the institution, converted to ratios by dividing by their sum.
	}\label{fig:regulating}
\end{figure*}

Our results demonstrate that there are differences in the response patterns of the two changes in the water resources allocation system.
In Figure~\ref{fig:regulating}, the red bar chart (87-WAS) and the green bar chart (98-UBR) respectively represent the increase or decrease ratio of actual water consumption compared to the estimated water use of the DSC model within ten years after the institutional shifts.
The gray bar chart shows the ratio of actual water use by provinces to their total water use in the decade after the two changes; The triangle marks indicate the ratio of the theoretical water resource quota of the province to the total available water in the YRB.\
In the ten years after the 87-WAS, the proportion of water consumption increase (or decrease) compared to that estimated by the DSC model was positively correlated with the proportion of water consumption taken from the YRB at present (partial correlation coefficient was $0.64$, Figure~\ref{fig:regulating}).
From 1987 to 1998, some provinces with high water consumption (e.g., Inner Mongolia, Henan, and Shandong) also showed significant increases in water consumption (Figure~\ref{fig:regulating}), with the average water consumption in Shandong, Inner Mongolia, Henan, and Ningxia exceeding the predicted value by $32.14\%$.
However, from 1998 to 2008, almost all provinces experienced a decrease in water consumption (by an average of $16.54\%$).
In addition, the water consumption of each province has a negative correlation with the proportion of water taken from the Yellow River Basin (partial correlation coefficient is $-0.51$).
