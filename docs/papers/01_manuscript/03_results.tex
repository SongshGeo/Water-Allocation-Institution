%! Author = songshgeo
%! Date = 2022/3/10
% \MakeUppercase{\subsection{Cascading effects of the institutional shifts}}

%\subsection{INSTITUTIONAL SHIFTS IMPACT ON WATER USE OF THE YRB}
\subsection{Institutional shifts impact on water use}
\label{result-2}
% 结果一:展示制度转变带来的用水量变化

\label{result-1-p2}
Our estimation of theoretical water use suggests that the institutional shift in 1987 (87-WAS) stimulated the provinces to withdraw more water than would have been used without an institutional shift (Figure~\ref{main_results}A).
From 1988 to 1998 on average, while the estimation of annual water use only suggests $956.38$ billion $m^3$, the observed water use of the YRB provinces reached $1038.36$ billion $m^3$ in sum, $8.57\%$ increased.
However, after the institution shifted again in 1998 (98-UBR), the trend of increasing water use appeared to be effectively suppressed. From 1998 to 2008, the total observed water use decreased by $4.9$ billion $m^3/yr$ per year, while the estimation of water use still suggests $10.3$ billion $m^3/yr$ increases (Figure~\ref{main_results} B).
The increased water uses after 87-WAS aligns with the fact that badly drying-up of the surface streamflow from 1987 to 1998, which was an obvious touchstone of river degradation and environmental crisis (Figure~\ref{main_results}C).
On the other hand, the 98-UBR ended river depletion, despite the density of droughts still increasing for decades (from $0.47$ after 87-WAS to $0.62$ after 98-UBR on average) (Figure~\ref{main_results}C).

\begin{figure*}[!h]
    \centering
    \includegraphics[width=32pc]{outputs/main_results2.pdf}
    \caption{
        Effects of two institutional shifts on water resources use and allocation in the Yellow River Basin (YRB).
        \textbf{A.} water uses of the YRB before and after the institutional shift in 1987 (87-WAS);
        \textbf{B.} water uses of the YRB before and after the institutional shift in 1998 (98-UBR). While the blue lines are statistic water use data, the grey ones are the estimation from the Differenced Synthetic Control method with economic and environmental background controlled.
        \textbf{C.} Drought intensity in the YRB and drying up events of the Yellow River. The size of the grey bubbles denotes the length of a drying upstream.
    }
    \label{main_results}
\end{figure*}


%\subsection{REGIONAL DIFFERENCES IN RESPONSES TO INSTITUTIONAL SHIFTS}
\subsection{Institutional effects on regulating outcomes}
\label{result-3}
% 结果2部分:展示区域相应差异
Our results also suggest huge differences between patterns of provinces in their responses to the two institutional shifts.
During the decade after the 87-WAS, the proportion of accelerated water use for each province (the proportion of actual water use exceeding the predicted water use by the model) has a significant correlation ($p<0.05$) to the Yellow River water use (Figure~\ref{upset}A).
While non-obvious impacts (less than $10\%$ in proportion) for most provinces, the apparent acceleration was prominent for the central water-using provinces (e.g., Inner Mongolia, Henan, and Shandong. Figure~\ref{upset}B).
In particular, Inner Mongolia and Shandong, both provinces that exceeded the prescribed water uses of the 87-WAS, used $44.25\%$ and $25.69\%$ more water uses than the prediction from 1987 to 1998, respectively.
In addition, the deficits of the water quota for each province (expressed by the difference between the water quota in 87-WAS and an expectation in water demands from a beforehand plan) have no significant ($p>0.05$) correlation to the acceleration.
By contrast, after the 98-UBR, except Shanxi (whose water quota has always been abundant since the 87-WAS) had an evident ($17.53\%$) increase in water use, almost all provinces have seen evident declines in water use ($-12.5\%$ on average).
Furthermore, neither the Yellow River water use nor the deficits correlate with the decreased proportions.

\begin{figure*}[!h]
    \centering
    \includegraphics[width=32pc]{outputs/fig3.pdf}
    \caption{Responses of the provinces to the institutional shifts (upper: the 87-WAS, lower: the 98-UBR).
        \textbf{A.} The partial correlation coefficient between water use (WU) of Yellow River (YR), unsatisfied ratio (compared with requirements in water plan and supply in the 87-WAS), and the average accelerated ratio.
        \textbf{B.} Average accelerated ratio of water uses for each province in the YRB during the decade after 87-WAS (from 1987 to 1998).
        \textbf{Mian users:} Major water consumption provinces (over the median).
        \textbf{Overused:} violate the 87-WAS in average water uses.
    }
    \label{upset}
\end{figure*}
