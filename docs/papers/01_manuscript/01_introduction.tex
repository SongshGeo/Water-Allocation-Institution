% 水竞争的重要性
Widespread freshwater scarcity and overuse challenge the sustainability of large river basins, resulting in systematic risks to economies, societies, and ecosystems globally \cite{distefano2017, dolan2021, xu2020b, mekonnen2016}.
With steadily increasing demand, competition for water causes depletion of freshwater globally and calls for an urgent transformation of the governance system by considering water use conservation \cite{gleick2010, ziolkowska2016, wang2019d}.
Despite worldwide trying to govern water, however, degradation of large river basins is not easily reversible because of few alignments between practice and theory in successful water governance cases.
\cite{giuliani2013, falkenmark2019, jaeger2019}.

% 黄河的介绍
The Yellow River Basin (YRB), the fifth large river worldwide, is known for its irreplaceable role in the social-economic development of China, and thus also drastically interference by anthropogenic stress.
Supporting $35.63\%$ irrigation and $30\%$ population with only $2.66\%$ water resources of China (data from \href{http://www.yrcc.gov.cn}{http://www.yrcc.gov.cn}, last access: \today), the overburdened Yellow River dried up in consecutive years, resulting in substantial ecological, economic and social crisis (e.g., wetland shrink, agriculture reduction, and scramble for water).
Intense water use, accounting for about $80\%$ of Yellow River surface runoff in the 1980s, was remarked as the significant reason for the degradation.
Furthermore, human interferences such as soil conservation and water conservancy project boosted water withdrawal and then stressed the water scarcity of the Yellow River.
In the context of future climate change, the contradiction between supplies and demands of water resources in the YRB will become more prominent.
%! Citation: Hao Wang presentation.
Therefore, balancing ecological and developing demands in such a human-dominated basin is a problem for China in terms of water governance throughout and for large rivers worldwide.

% 第三段 过去管理的经验,断流的严重程度,连续多少没有断流,少有的解决断流的大河。治理成功的流域?分析背后的机制,基础。往用水上靠。
% 为了缓解用水压力
Chinese authorities implemented several ambitious water management practices in the YRB in the last century to relieve the water stress, such as reservoir regulation, South-to-north Water Diversion Project (WDP), and Water Allocation Scheme since 1987 (87-WAS), and the Unified Basinal Regulation since 1998 (98-UBR).
Through those efforts, ecological restoration of wetland and estuary delta in the YRB without drying up for over $20$ is widely considered a considerable river management achievement.
Different from the engineering of WDP provides further water supply or reservoir matches water supply and demands, institutional strategies like the 87-WAS (assign water quotas for provinces in the YRB) and the 98-UBR (the provinces had to be allowed to use water by the Yellow River Conservancy Commission, YRCC) mainly focused on limiting demands of water use.
Such institutions (policies, laws, and norms) can influence regional sustainability by changing the structure of the coupled human and natural system, including interplays between social actors, ecological units, or between social and ecological system elements \cite{young2008,cumming2020b,lien2020, bodin2017b}.
Understanding those complex interlinkages, therefore, is crucial for developing strategies to effectively manage natural resources and enhance the resilience of social-ecological systems (SES) \cite{kluger2020}.
However, while literature had well evaluated and quantified the effects of engineering solutions beforehand, there are few attempts to assess institutional contributions to successful water governance.

% 第四段 目标 制度变化。是不是制度变化产生的,如果是的话怎么影响。
In addition to widespread recognition of the rising importance of institutions as an approach to water sustainable use within large river basins (especially transboundary rivers like the YRB), their specific effects are still in open discussion \cite{agrawal2003, persha2011, agrawal2001}.
Effective (``matched'' or ``fit'') institutions operate at appropriate spatial, temporal, and functional scales to manage and balance different relationships and interactions between human and water systems, therefore, supporting (but do not guarantee) sustainability of SES \cite{epstein2015, wang2019d}.
Some institutional shifts have desirable water governance outcomes (e.g., the Ecological Water Diversion Project in Heihe River Basin, China \cite{wang2019d} and collaborative water governance systems in Europe \cite{green2013}).
However, shifting institutions in a large, complex river basin may create or destroy hundreds of different connections between social agents and ecological units, where matched social-ecological structures are not ubiquitous.
Therefore, the role of institutional shifts in the water governance achievement of the YRB and their impacts on water use is still uncertain without an understanding of SES structures.
% 通过抽象文件变化
Here, by abstracting changes in official documents following institutional shifts (the 87-WAS and the 98-UBR), we depicted the SES structures of the YRB from 1975 to 2008.
Then, we use Differenced Synthetic Control (DSC) method, which considers economic growth and natural background, to estimate theoretical water use scenarios without the institutional shifts (\textit{\hyperref[{sec:methods}]{Methods}} and \textit{Appendix~\nameref{secS3}}).
By further interpreting the differences of the effects in the YRB, we explored the mechanisms linking SESs structure and outcomes for a deeper understanding of institutions' role in water governance worldwide.
