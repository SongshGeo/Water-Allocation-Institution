% 水竞争的重要性
Widespread freshwater scarcity and overuse challenge the sustainability of large river basins, resulting in systematic risks to economies, societies, and ecosystems globally \cite{distefano2017, dolan2021, xu2020b, mekonnen2016}.
In the context of future climate change, the gap between supply and demand for water resources in large river basins is expected to become increasingly more prominent \cite{florke2018, yoon2021}.
Those river basin systems successfully supporting sustainable water resource use are structurally well-aligned with water provisioning and social-ecological demands, without inefficient competition or overuses \cite{wang2019d}.
However, balancing the water demands of ecosystems and development in heavily human-dominated river basins is a challenge because human activities and water are intertwined in their structures as complex social-ecological systems (SES) \cite{huggins2022,konar2019}.

For governing river basin systems, their SES structures can be reshaped by institutions, such as policies, laws, and norms \cite{young2008,cumming2020b}.
Representing all relative governance practices, institutions include interplays between social actors, ecological units, or between social and ecological system elements
\cite{lien2020, bodin2017b}.
Understanding how these complex interplays are crucial for developing strategies to effectively manage natural resources and enhance the resilience of social-ecological systems \cite{kluger2020}.
Effective (``matched'' or ``fit'') institutions operate at appropriate spatial, temporal, and functional scales to manage and balance different relationships and interactions between human and water systems, supporting (but not guaranteeing) the sustainability of SES \cite{epstein2015, wang2019d}.
Some institutional advances have had desirable water governance outcomes (e.g., the Ecological Water Diversion Project in Heihe River Basin, China \cite{wang2019d}, and collaborative water governance systems in Europe \cite{green2013}).
However, imposing institutional changes on a large, complex river basin may create or destroy hundreds of connections between social agents and ecological units, where matched social-ecological structures are not ubiquitous.
Two particular weaknesses in existing knowledge of institutional matches include understanding: (i) the causal links between SES structures and outcomes; (ii) details of the underlying processes, and especially the coordination of the incentives of different participants, that result from an institutional lack of matches.
These weaknesses limit understanding of institutional design and hinder approaches toward institutional matches for improving the sustainability of river basin systems.

To better understand how water governance institutions match their social-ecological context, we take the Yellow River Basin (YRB), China, as an example \textit{\nameref{sec:yrb}} to dive into causal links between SES structures and outcomes.
Specifically, we focused on two institutional shifts in water allocation of the YRB: the 1987 Water Allocation Scheme (87-WAS), and the 1998 Unified Basinal Regulation (98-UBR), which reframed SES structures significantly.
The YRB provides an informative case for two main reasons:
(1) The top-down institutional shifts induced sharp changes in SES structures, enabling us to estimate their net effects quantitatively.
(2) Since few large river basins have experienced such radical institutional shifts more than once, this case study provides comparable natural experiments for understanding the impacts of structural changes in SESs on natural resources.

We explored causal linkages between SES structures and sustainability-related outcomes by quasi-natural experiments (institutional shifts imposed by central government) in the YRB.
Firstly, we used data on changes in official documents following two institutional shifts to describe comparable changes in the SES structures associated with the YRB from 1979 to 2008, by abstracting them into SES structures motifs (or building blocks, see \textit{\nameref{sec:structures}}).
We then used a method called `Differenced Synthetic Control (DSC)' \cite{arkhangelsky2021}, which considers economic growth and natural background, to estimate theoretical water use scenarios without institutional shifts (\textit{\nameref{sec:DSC}} and \textit{\nameref{secS2}}).
This approach allowed us to create a counterfactual against which to explore the mechanisms linking SESs structure and outcomes for a deeper understanding of the potential role of institutions in water governance worldwide.
Finally, we further developed an approach for marginal benefits analysis, to interpret the underlying processes of the match and mismatched institutions based on SESs structures (\textit{\nameref{sec:model}}).
