% 水竞争的重要性
Widespread freshwater scarcity and overuse challenge the sustainability of large river basins, resulting in systematic risks to economies, societies, and ecosystems globally~\cite{distefano2017, dolan2021, xu2020b, mekonnen2016}.
In the context of climate change, the mismatches between supply and demand for water resources are expected to become increasingly more prominent~\cite{florke2018, yoon2021}.
Hence, large river basins are increasingly seeking for effective water governance solutions by coordinating stakeholders, providing water resources, and ensuring the sustainable allocation of shared water resources~\cite{wang2019d}.
As results, hydrological processes are tightly intertwined with societies, shown as an social-ecological system (SES) at a basin scale with complex socio-hydrological feedback.

Institutions encompass the interplay between social actors, ecological units, and their interactions~\cite{lien2020, bodin2017b, wang2022g}.
These interactions consist a type of structure of SES, where effective institutions operate at appropriate spatial, temporal, and functional scales to manage and balance different interactions, contributing to the sustainability~\cite{epstein2015, wang2019d} (Figure~\ref{fig:framework}~\textbf{a}).
Governing river basin systems involves reshaping their SES structures through institutions such as policies, laws, and norms~\cite{young2008,cumming2020b}.
While some institutional advances have led to effective water governance outcomes (e.g., the Ecological Water Diversion Project in Heihe River Basin, China~\cite{wang2019d}, and collaborative water governance systems in Europe~\cite{green2013}),
imposing institutional shifts may create or destroy connections and effectiveness are not ubiquitous~\cite{loos2022}.
For example, Colorado River once had sever water shortage and institutions led to various shortage magnitude for different stakeholders even under the same water demands level~\cite{hadjimichael2020}.
Therefore, examining when and how an institution leads to effective water governance can bring crucial insights for sustainability of river basins.

% match & mismatch
Recent studies explored diverse effects of institutions on river basins governance~\cite{bouckaert2022, vallury2022, loch2020, kirchhoff2016}, while the current analysis is more about interpreting outcomes after the institutional changes, but it cannot compares how the scenarios would be without these institutional changes.
Additionally, understanding how the different SESs structures influence institutional effectiveness is challenging due to the complexity and dynamics of socio-hydrological systems~\cite{bodin2017b}.
Thus, knowledge gaps lies in the limited understanding of effective alignments between institutional shifts and SES structures, hinders designing effective policies to promote sustainable river basins governance.
To fill these knowledge gaps, we study the Yellow river basin, the fifth-largest river worldwide and one of the most anthropogenically altered river basins, to quantitatively measure the effects of changing SES structure.

\begin{figure}[!ht]
	\centering
	\includegraphics[width=0.5\linewidth]{diagrams/framework.png}
	\caption{
		Framework for understanding institutional shifts and SES structures. \textbf{a.} In the general framework for analyzing social-ecological systems (SESs), (Adapted from Ostrom, 2008~\cite{ostrom2009}). Institutional shifts can change interactions within the SES and reframe the structures.  \textbf{b.} We aim to investigate how institutional shifts impacts effective governance by structuring SES.}\label{fig:framework}
\end{figure}

% 黄河的介绍
In the $1980s$, intense water use, accounting for about $80\%$ of Yellow River surface water, caused consecutive drying-up crises of runoff, leading to wetland shrinkage, agriculture reduction, and scrambles for water~\cite{wohlfart2016}.
Then, Chinese authorities implemented several ambitious water management policies in the YRB to relieve water stress, such as the South-to-north Water Diversion Project and the Water Allocation Institutions~\cite{long2020, wang2019d}.
In this study, we specifically examined two significant institutional shifts in water allocation of the YRB\: the 1987 Water Allocation Scheme (87-WAS) and the 1998 Unified Basinal Regulating (98-UBR).
Instead of engineering and increasing water supply, 87-WAS (which assigned water quotas for provinces in the YRB) and the 98-UBR (under which provinces had to obtain permits from the Yellow River Conservancy Commission, YRCC, authority at a basin-level) focused mainly on limiting demands for water~\cite{bouckaert2022, speed2013}.
These institutional shifts can offer valuable insights because of two main reasons:
(1) the top-down institutional shifts suddenly led to transforms of SES structures, allowing us to quantitatively estimate their net effects; and (2) the two institutional shifts within the same river basin provide rarely comparable quasi-natural experiments.

Here, we portrayed changes of SES structures throughout the YRB's institutional shifts (the 87-WAS and the 98-UBR) and quantitatively investigated their consequences, then including an discussion of the effectiveness of institutional shifts.
Specifically, we first use the descriptions of official documents following the two institutional shifts to abstract the interactions between main stakeholders and their river segment units for interpreting SES structures changes between 1979 and 2008.
Next, and perhaps most importantly, we use the ``Differenced Synthetic Control (DSC)'' method~\cite{arkhangelsky2021}, which accounts for economic growth and natural background, to estimate theoretical water use volumes under the absence of institutional shifts scenarios.
Finally in discussion, we linked the effectiveness of institutional shifts to the portrayed structures, by comparing YRB's case to previous SES structure studies and developing a marginal benefits analysis.
