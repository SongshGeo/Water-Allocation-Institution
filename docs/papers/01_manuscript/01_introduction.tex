% 水竞争的重要性
Widespread freshwater scarcity and overuse challenge the sustainability of large river basins, resulting in systematic risks to economies, societies, and ecosystems globally \cite{distefano2017, dolan2021, xu2020b, mekonnen2016}.
With steadily increasing demand, competition for water causes depletion of freshwater globally. Therefore, it creates a strong need for an urgent transformation of the water governance system to improve water use conservation \cite{gleick2010, ziolkowska2016, wang2019d}.
Despite worldwide efforts to govern water sustainably, overuse and the resulting degradation of large river basins are not easily reversible. Thus, there have been relatively few successful governance practices and theory re-alignments
\cite{giuliani2013, falkenmark2019, jaeger2019}.
In the context of future climate change, the gap between supply and demand for water resources in large river basins is expected to become increasingly more prominent \cite{florke2018, yoon2021}.
Balancing the water demands of ecosystems and development in heavily human-dominated river basins is a challenge not just for China but also across many large river basins worldwide.

% 黄河的介绍
The Yellow River Basin (YRB), the fifth-largest river basin worldwide, is known for its vital role in the socio-economic development of China.
It supports $35.63\%$ of China's irrigation and $30\%$ of its population while containing only $2.66\%$ of its water resources (data from \href{http://www.yrcc.gov.cn}{http://www.yrcc.gov.cn}, last access: \today).
In the 1980s, intense water use, accounting for about $80\%$ of Yellow River surface runoff, combined with other forms of human interference (e.g., soil conservation and water conservancy projects), caused consecutive drying events and substantial ecological, economic, and social crises (e.g., wetland shrinkage, agriculture reduction, and a scramble for water).

% 第三段 过去管理的经验,断流的严重程度,连续多少没有断流,少有的解决断流的大河。治理成功的流域?分析背后的机制,基础。往用水上靠。
In response, Chinese authorities implemented several ambitious water management practices in the YRB to relieve water stress, such as reservoir regulation, the South-to-north Water Diversion Project (WDP), the 1987 Water Allocation Scheme (87-WAS), and the 1998 Unified Basinal Regulation (98-UBR) \cite{long2020, wang2019d}.
Those efforts led to ecological restoration of wetlands and estuarine delta, and drying up has been avoided for over 20 years, widely considered considerable management achievements.
Unlike engineering providing further water supply, institutional strategies like the 87-WAS (which assigned water quotas for provinces in the YRB) and the 98-UBR (under which provinces had to obtain permits from the Yellow River Conservancy Commission, YRCC, authority at a basin-level) focused mainly on limiting demands of water use.
Such institutions (policies, laws, and norms) can influence regional sustainability by changing the structure of the coupled human and natural system, including interplays between social actors, ecological units, or between social and ecological system elements
\cite{young2008,cumming2020b,lien2020, bodin2017b}.
Therefore, understanding those complex interplays is crucial for developing strategies to effectively manage natural resources and enhance the resilience of social-ecological systems (SES) \cite{kluger2020}.

While researchers have carefully evaluated and quantified the effects of engineering solutions\cite{long2020}, there have been few attempts to assess institutional contributions to successful water governance in the YRB.
% 第四段 目标 制度变化。是不是制度变化产生的,如果是的话怎么影响。
In addition to widespread recognition of the rising importance of governmental institutions for sustainable water use within large river basins (especially in the case of transboundary basins like the YRB), the best approach to designing effective institutions remains an open question \cite{agrawal2003, persha2011, agrawal2001}.
Effective (``matched'' or ``fit'') institutions operate at appropriate spatial, temporal, and functional scales to manage and balance different relationships and interactions between human and water systems, supporting (but not guaranteeing) the sustainability of SES \cite{epstein2015, wang2019d}.
Some institutional advances have had desirable water governance outcomes (e.g., the Ecological Water Diversion Project in Heihe River Basin, China \cite{wang2019d}, and collaborative water governance systems in Europe \cite{green2013}).
However, imposing institutional changes on a large, complex river basin may create or destroy hundreds of connections between social agents and ecological units, where matched social-ecological structures are not ubiquitous.

To better understand how water management institutions can be designed to fit their social-ecological context, we used data on changes in official documents following institutional shifts (the 87-WAS and the 98-UBR) to describe changes in the SES structures associated with the YRB from 1979 to 2008.
We then used Differenced Synthetic Control (DSC) method \cite{arkhangelsky2021}, which considers economic growth and natural background, to estimate theoretical water use scenarios without institutional shifts (\textit{\hyperref[{sec:methods}]{Methods}} and \textit{\nameref{secS2}}).
It allowed us to create a counterfactual against which to explore the mechanisms linking SESs structure and outcomes for a deeper understanding of the potential role of institutions in water governance worldwide.
