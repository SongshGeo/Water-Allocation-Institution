% 水竞争的重要性
Widespread freshwater scarcity and overuse challenge the sustainability of large river basins, resulting in systematic risks to economies, societies, and ecosystems globally~\cite{distefano2017, dolan2021, xu2020b, mekonnen2016}.
In the context of climate change, the mismatches between supply and demand for water resources are expected to become increasingly more prominent~\cite{florke2018, yoon2021}.
Hence, large river basins are increasingly seeking for effective water governance solutions by coordinating stakeholders, providing water resources, and ensuring the sustainable allocation of shared water resources~\cite{wang2019d}.
As results, hydrological processes are tightly intertwined with societies, shown as an social-ecological system (SES) at a basin scale where complex socio-hydrological feedback merges SES structures.

Governing river basin systems involves reshaping their SES structures through institutions such as policies, laws, and norms~\cite{young2008,cumming2020b}.
Institutions encompass the interplay between social actors, ecological units, and their interactions~\cite{lien2020, bodin2017b}.
Effective institutions operate at appropriate spatial, temporal, and functional scales to manage and balance different relationships and interactions between human and water systems, contributing to the sustainability of SES~\cite{epstein2015, wang2019d} (Figure~\ref{fig:framework}~\textbf{a}).
While some institutional advances have led to desirable water governance outcomes (e.g., the Ecological Water Diversion Project in Heihe River Basin, China~\cite{wang2019d}, and collaborative water governance systems in Europe~\cite{green2013}),
imposing institutional shifts may create or destroy connections and effectiveness are not ubiquitous~\cite{loos2022}. 
For example, Colorado River once had sever water shortage and costly water competition until confirming water rights with following water institutional system~\cite{hadjimichael2020}.  % TODO check this expression. 科罗拉多河的这个例子还要举得更好一些
Therefore, examining when and how an institution leads to effective water governance can bring crucial insights for sustainability in global river basins.

% match & mismatch
Recent studies explored various effects of institutions on river basins governance, while quantitative assessments with controlled comparisons are remaining scarce~\cite{bouckaert2022, vallury2022, loch2020, kirchhoff2016}.
Additionally, understanding how different SESs structures influence institutional effectiveness is challenging due to the complexity and dynamics of these systems~\cite{bodin2017b}.
Thus, the knowledge gap lies in the limited understanding of aligning institutions with SES structures to promote sustainable river basins governance.
It hinders designing effective institutions that address the diverse challenges and opportunities in various social-ecological contexts, ultimately impacting long-term sustainability.

\begin{figure}[!ht]
	\centering
	\includegraphics[width=0.5\linewidth]{diagrams/framework.png}
	\caption{
		Framework for understanding institutional shifts and SES structures. \textbf{a.} In the general framework for analyzing social-ecological systems (SESs), (Adapted from Ostrom, 2008~\cite{ostrom2009}). Institutional shifts can change interactions within the SES and reframe the structures.  \textbf{b.} We aim to investigate how institutional shifts impacts effective governance by structuring SES.}\label{fig:framework}
\end{figure}

To gain a deeper understanding of how institutions align or misalign with their social-ecological context, we use the Yellow River Basin (YRB), China, as a case study to examine the consequences and incentives of two significant institutional shifts in water allocation: the 1987 Water Allocation Scheme (87-WAS) and the 1998 Unified Basinal Regulation (98-UBR).
These shifts considerably transformed the SES structures within the basin, and offered valuable insights for two main reasons:
(1) the top-down institutional shifts led to substantial changes in SES structures, allowing us to quantitatively estimate their net effects; and (2) the basin's unique experience of multiple radical institutional shifts provides a rare opportunity to compare natural experiments and understand the impacts of SES structural changes on water governance.

Our investigation of the consequences of YRB's institutional shifts involves analyzing quasi-natural experiments with causal inference approach.
First, we use data from official documents following the two institutional shifts to describe the YRB's SES structures' comparable changes between 1979 and 2008.
Next, we employ the ``Differenced Synthetic Control (DSC)'' method~\cite{arkhangelsky2021}, which accounts for economic growth and natural background, to estimate theoretical water use scenarios in the absence of institutional shifts.
This approach enables us to establish a counterfactual basis for exploring the consequences and incentives related to SES structures, thereby enhancing our understanding of institutions' potential role in global water governance.
% Finally, we develop a marginal benefits analysis approach to interpret the underlying processes of misaligned institutions based on SES structures.
