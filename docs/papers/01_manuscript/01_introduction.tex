% 水竞争的重要性
Widespread freshwater scarcity and overuse, resulting in systematic risks to economies, societies, and ecosystems globally, are critical environmental challenges to sustainable development \cite{distefano2017, dolan2021, xu2020b, mekonnen2016}.
With steadily increasing demand, competition for water is an urgent problem in water governance, where policies often lead to long-term changes in human–water relationships and the redistribution of benefits \cite{gleick2010, ziolkowska2016, wang2019d}.
Despite governments worldwide trying to resolve competition for water through deliberate institutions at large river basins, cascading effects of these initiatives are poorly understood
\cite{giuliani2013, falkenmark2019, jaeger2019}.

% 制度的意义
Institutions (such as policies, laws, and norms) can influence regional sustainability by changing the structure of the coupled system, including interplays between social actors, ecological units, or between social and ecological system elements \cite{young2008,cumming2020b,lien2020, bodin2017b}.
Understanding the complex interlinkages is crucial for developing strategies to effectively manage natural resources and enhance the resilience of social-ecological systems (SES) \cite{kluger2020}.
Effective (“matched”) institutions operate at appropriate spatial, temporal, and functional scales to manage and balance these different relationships and interactions, therefore, supporting (but do not guarantee) sustainability of SES \cite{epstein2015, wang2019d}.
Water governance tends to shift to institutional solutions within a complex basinal system, where societal drivers impact through water use and related technical interventions \cite{fischer2020}.
Some institutional shifts have desirable water governance outcomes (e.g., the Ecological Water Diversion Project in Heihe River Basin, China \cite{wang2019d} and collaborative water governance systems in Europe \cite{green2013}).
However, shifting institutions in a large, complex river basin may create or destroy hundreds of different connections, where matched human-water relationships are not ubiquitous.
%TODO 这里应该梳理 Institutional shift --> structures / building blocks.
Therefore, despite widespread recognition of the rising importance of institutions as an approach to water sustainable use within large river (especially transboundary river) basins, broader cascading effects of these changes are still in open discussion \cite{agrawal2003, persha2011, agrawal2001}.
% TODO 根据武老师的意见,这个KG应该再明确

%! 这一段梳理一下 building blocks 的作用机制
% Among

% 黄河的制度介绍
Supporting $35.63\%$ irrigation and $xx\%$ population with only $2.66\%$ of water resources in China (data from \href{http://www.yrcc.gov.cn}{http://www.yrcc.gov.cn}, last access: 28 February 2021), the overburdened Yellow River (YR) once dried up in consecutive years.
For that, as a pioneer in water governance shifting in China, the YRB started to explore the initial water allocating scheme in the 1970s, then found a successful solution of dring-up in 1998, and promoted entirely since 2008.
The institutional shifts in 1987 (known as the ``87 Water Allocation Scheme, 87-WAS'') and 1998 (Unified Basinal Regulation, 98-UBR) were two widely recognized milestones of water governance.
Before the remarkable achievement in the restoration of river depletion since the 98-UBR, on the contrary, the first temptation to restrict water uses in the 87-WAS was recognized as a not fulfilling expectation of institutional shift.
Until the 87-WAS, stakeholders have free access to the YR water resources, with geographic and temporal differences between freshwater demand and availability.
As the mismatch between demands and supply kept increasing, national authorities proposed in 87-WAS allocating specific water quotas between $10$ provinces (or regions) along the YR basin.
However, this controversial scheme helped little in turning water depletion around until a different strategy expanded the responsibilities of basinal authorities in integrated water management in 1998 and started progress in the restoration.

% 我们的研究
The shifts in the water governance institution of YR refactored the interplays between humans and water with long-term cascading effects, leaving two quasi-natural experiments for understanding the interactions.
By Differenced Synthetic Control method, our analysis of the net effects of institutional shift shows that institutional mismatches contributed to the acceleration of water withdrawals for resource users after 98-WAS.
As few large river basins have experienced such radical structural changes several times, our quantitative analysis of institutional shifts in the YR induces a valuable understanding of water sustainable governance by decoupling natural and human interferences.
%! 这还要继续凝练我们的贡献
By highlighting potential concerns for ecosystem collapse under structural mismatches, our findings align with the urgent calls for a more dynamic design for water governance institutions to achieve sustainability.
