% 水竞争的重要性
Widespread freshwater scarcity and overuse challenge the sustainability of large river basins, resulting in systematic risks to economies, societies, and ecosystems globally~\cite{distefano2017, dolan2021, xu2020b, mekonnen2016}.
Amidst climate change, mismatches between supply and demand for water resources are expected to become increasingly more prominent~\cite{florke2018, yoon2021}.
Consequently, large river basins are progressively seeking effective water governance solutions by coordinating stakeholders, providing water resources, and ensuring the sustainable allocation of shared water resources~\cite{wang2019d}.
In this way, hydrological processes are tightly intertwined with societies, forming a social-ecological system (SES) at a basin scale with complex socio-hydrological feedback.

Institutions encompass the interplay between social actors, ecological units, and their interactions~\cite{young2008, lien2020, bodin2017b, wang2022g} (Figure~\ref{fig:framework}~a).
These interactions constitute a type of SES structure, where effective institutions operate at appropriate spatial, temporal, and functional scales to manage and balance different interactions, contributing to sustainability~\cite{epstein2015, wang2019d} (Figure~\ref{fig:framework}~b).
While some institutional advances have led to effective water governance outcomes (e.g., the Ecological Water Diversion Project in Heihe River Basin, China~\cite{wang2019d}, and collaborative water governance systems in Europe~\cite{green2013}), imposing institutional shifts may create or destroy connections and effectiveness is not ubiquitous~\cite{loos2022}.
For example, the Colorado River once experienced severe water shortage, and institutions led to various shortage magnitudes for different stakeholders even under the same water demand levels~\cite{hadjimichael2020}.
Therefore, examining when and how an institution leads to effective water governance can bring crucial insights for the sustainability of river basins.

% match & mismatch
Recent research has delved into the multifaceted effects of institutions on river basin governance, shedding light on diverse consequences and interactions~\cite{bouckaert2022, vallury2022, loch2020, kirchhoff2016}.
Primarily due to the intricate dynamics within socio-hydrological systems, understanding the manner in which different SES structures influence institutional effectiveness remains a complex challenge~\cite{bodin2017b}.
The current study contributes to this understanding by interpreting outcomes following institutional changes, though it does not explore hypothetical scenarios without such changes.
Thus, knowledge gaps lie in the limited understanding of effective alignments between institutional shifts and SES structures, hindering the design of effective policies to promote sustainable river basin governance.
To fill these knowledge gaps, we study the fifth-largest river worldwide and one of the most anthropogenically altered river basins, the Yellow River Basin (YRB) in China, to quantitatively measure the effects of changing SES structures.

\begin{figure}[!ht]
	\centering
	\includegraphics[width=0.5\linewidth]{diagrams/framework.png}
	\caption{
		Illustration for understanding institutional shifts and SES structural changes. \textbf{a.} In the general framework for analyzing social-ecological systems (SESs), (Adapted from Ostrom, 2008~\cite{ostrom2009}). Institutional shifts can change interactions within the SES and reframe the structures.  \textbf{b.} We aim to examine how institutional shifts effect river basin governance by structuring SES.}\label{fig:framework}
\end{figure}

% 黄河的介绍
In the 1980s, intense water use, accounting for about $80\%$ of the Yellow River surface water, caused consecutive drying-up crises of runoff, leading to wetland shrinkage, agriculture reduction, and scrambles for water~\cite{wohlfart2016}.
To alleviate water stress, Chinese authorities implemented several ambitious water management policies in the Yellow River Basin (YRB), such as the South-to-North Water Diversion Project and the Water Resources Allocation Institutions~\cite{long2020, wang2019d}.
In this study, we specifically examined two significant institutional shifts in water allocation of the YRB\: the 1987 Water Allocation Scheme (87-WAS) and the 1998 Unified Basinal Regulating (98-UBR).
Instead of focusing on engineering and increasing water supply, the 87-WAS (which assigned water quotas for provinces in the YRB) and the 98-UBR (under which provinces had to obtain permits from the Yellow River Conservancy Commission, YRCC, an authority at a basin level) mainly aimed to limit water demands~\cite{bouckaert2022, speed2013}.
These institutional shifts can offer valuable insights for two main reasons:
(1) the top-down institutional shifts suddenly led to transformations of SES structures, allowing us to quantitatively estimate their net effects; and (2) the two institutional shifts within the same river basin provide rare comparable quasi-natural experiments.

In this study, we portrayed changes of SES structures throughout the YRB's institutional shifts (the 87-WAS and the 98-UBR) and quantitatively investigated their consequences, followed by a discussion on the effectiveness of institutional shifts.
Specifically, we first used the descriptions of official documents following the two institutional shifts to abstract the interactions between main stakeholders and their river segment units for interpreting SES structure changes between 1979 and 2008.
Next, and perhaps most importantly, we employed the ``Differenced Synthetic Control (DSC)'' method~\cite{arkhangelsky2021}, which accounts for economic growth and natural background, to estimate theoretical water use volumes under scenarios absent of institutional shifts.
Finally, in the discussion, we linked the effectiveness of institutional shifts to the portrayed structures, by comparing the YRB's case to previous SES structure studies and developing a marginal benefits analysis.
