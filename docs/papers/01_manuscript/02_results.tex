%! Author = songshgeo
%! Date = 2022/3/10
% \MakeUppercase{\subsection{Cascading effects of the institutional shifts}}
\subsection{Institutional shifts effect on water use}
\label{effects}
% 从国家官员到流域管理官员,从省、市到区,长江流域自上而下的制度结构开始演变至今
Including the national authorities, the basin management authorities, provinces, cities, and even districts, top-down institutional structures of the YRB started to evolve up to now (\textit{S1 in Supplementary Material}).
% 从70年代黄河断流开始摸索87-WAS的初步治理方案,到98-UBR成功解决断流后于2008年强化该制度,黄河一直是中国水治理制度转变的先行者。
As a pioneer in water governance shifting in China, the YRB started to explore the initial water allocating scheme in the 1970s, then found a successful solution of dring-up in 1998, and promoted entirely since 2008.
% 1987年(87-WAS)和1998年(98-UBR)的制度变迁是YRB中两个被广泛认可的制度演变里程碑(图~\ref{structure} \textit{S1 in  Supplementary  Methods})。
Throughout, the institutional shifts in 1987 (87-WAS) and 1998 (98-UBR) were two widely recognized milestones of water governance (Figure~\ref{structure} and \textit{S1 in Supplementary Material}).
Our analysis period, therefore, spans from 1975 to 2008, with the human-water system shifted between three different institutional structures (Figure~\ref{structure}).

Here, we use Differenced Synthetic Control method, which considers economic growth and natural background, to estimate theoretical water uses scenarios without these policy interferences (\textbf{Methods}; \textit{S2 in Supplementary Material}).
% 我们的结果表明,1987年(87-WAS)的制度转变实际上刺激了各省比没有干扰时使用更多的水,观测到比预期增加了10(图~\ref{main_results}A和B)。
Our results suggest that the institutional shift in 1987 (87-WAS) stimulated the provinces to withdraw more water than would have been used without the interference (Figure~\ref{main_results}A).
% 在1998年制度再次改变后,用水量增加的趋势似乎被有效地抑制了,观察到的总用水量比预期减少了10(图~\ref{main_results} C和D)。
However, after the institution shifted again in 1998 (98-UBR), the trend of increasing water use appeared to be effectively suppressed, with total observed water consumption decreasing by $260\%$ relative to the estimation (Figure~\ref{main_results} B).
% 87-WAS后用水量的增加与1987 - 1998年地表径流严重干涸的事实相吻合,这是河流退化和环境危机的一个明显的试金石
The increased water uses after 87-WAS align with the fact that badly drying-up of the surface streamflow from 1987 to 1998, which was an obvious touchstone of river degradation and environmental crisis (Figure~\ref{main_results}C).
% 尽管在1998年制度转变后的十年内,干旱程度有增无减,但断流的环境危机却得到了有效解决。
On the other hand, although the density of droughts increased in the decade after the 98-UBR, the environmental crisis of river drying up was effectively resolved (Figure~\ref{main_results}C).
% 正如许多研究指出的,这主要是大规模制度变迁的成功
As literature has suggested, this institution shift contributed a lot to the successful water governance.

\begin{figure*}[!h]
    \centering
    \includegraphics[width=32pc]{outputs/main_results2.jpg}
    \caption{
        Effects of two institutional shifts on water resources use and allocation in the Yellow River Basin (YRB).
        \textbf{A.} water uses of the YRB before and after the institutional shift in 1987 (87-WAS);
        \textbf{B.} water uses of the YRB before and after the institutional shift in 1998 (98-UBR). While the blue lines are statistic water use data, the grey ones are the estimation from the Differenced Synthetic Control method with economic and environmental background controlled.
        \textbf{C.} Drought intensity in the YRB and drying up events of the Yellow River. Size of the grey bubbles denotes the length of a drying up stream.
    }
    \label{main_results}
\end{figure*}

% 然而,87-WAS 也非大多分析报告所认为的“没有起到效果”,因为制度变化完全没有产生影响时,流域的用水量将更接近我们的模型预测。
Furthermore, 87-WAS did not ``have no effect'' as previous analyses suggested because water use will be closer to what our models predicted when the shift was a blank policy.
% 除了环境背景外,我们的预测还考虑了经济因素,假设经济总量与用水量之间的生产函数保持不变。
Besides environmental background, our forecast takes economic factors into account under the assumptions that the production function between economic volume and water uses remained unchanged (\textit{S2 in Supplementary Material}).
% 由于单位水资源的生产效率在这一时期实际上保持了相似的趋势,这意味着流域用水量的加速增长不是经济增长的必然结果。
However, the production efficiency of unit water resources kept similar trends between provinces of the YRB and others (Figure \textit{S3 in Supplementary Material}), which means that the accelerated growth of basinal water uses was not the inevitable result of economic growth.
% 但实际上,我们的结果表明相对于正常的经济发展进程来说,黄河流域的水资源利用反而更像是受到了刺激。
However, our results suggest that water use (especially for irrigation) in the YRB was more likely to be partly stimulated by the shift in 1987 besides economic growth.
% 这种用水量的加速增长与制度设计时遏制用水、保护黄河流域有限的水资源的初衷相悖。
This accelerated growth in water use was contrary to the original intention of the 87-WAS in conserving the limited water.

% 两次制度转变带来的级联效应存在巨大差异
The cascading effects of the two institutional shifts were very different, while their reframed SES structures have direct differences.
% 在我们的研究期间(从1975年到2008年),机构在三种不同的结构之间转换:免费访问、87-WAS和98-UBR。
In our study period, the institution shifted between three different structures (Figure~\ref{structure} A to C).
% 长江水利委员会是长江流域的主要负责人,其任务是报告和分析1998年以前长江流域的用水量(87-WAS之后)。
However, the YRCC, whose primary official response to the river before 1998, the mandate was to conserve the riverway environment, construct and maintain infrastructures, and report on and analyze water consumption (after the 87-WAS) in the YRB, i.e., connecting the ecological nodes (different river reaches) horizontally ~\cite{wang2019a}.
% 这意味着,而与生态节点直接联系的各利益相关者之间没有横向连接。
Thus, until the institutional shift of 98-UBR, with no necessity to apply for a water permit from YRCC, there were no horizontal connections (cooperations or agreements) between the various stakeholders (provinces) directly connected to the ecological sections.
% 许多不同领域的经验研究指出这种结构很可能是错配的,因为这些利益相关者难以从全局出发维护好互相联系的生态系统。
Empirical studies in many different fields indicate that this structure is likely to be mismatched, as isolated stakeholders struggle with holistically maintaining interconnected ecosystems
\cite{sayles2017,sayles2019,cai2016,bergsten2019}.
% 87-WAS带来的效果再一次说明了这种阻碍可持续性的治理结构需要得到重视。
The cascading effect of the 87-WAS once again demonstrates that the more stakeholders, the more difficult it is to have a win-win situation of environment and interests \cite{hegwood2022} which calls for exceptional understanding and caution to the structure of hampering sustainability \cite{bergsten2019, sayles2019}.
