%! Author = songshgeo
%! Date = 2022/3/10
% \MakeUppercase{\subsection{Cascading effects of the institutional shifts}}
\subsection{Institutional shifts effect on water use}
\label{result-1}
% 结果一:
% 1. 展示制度转变带来的用水量变化
% 2. 解释结果的意义、合理性、
% 3. 建立结果与结构之间的联系

\label{result-1-p2}
Here, we use Differenced Synthetic Control (DSC) method, which considers economic growth and natural background, to estimate theoretical water use scenarios without basinal policy interferences (\textbf{Methods}; \textit{S2 in Supplementary Material}).
Our results suggest that the institutional shift in 1987 (87-WAS) stimulated the provinces to withdraw more water than would have been used without the interference (Figure~\ref{main_results}A).
From 1988 to 1998, while the estimation of water use only suggests $956.38 km^3$, the observed water use of the YRB provinces reached $1038.36 km^3$ in sum, $8.57\%$ increased.
However, after the institution shifted again in 1998 (98-UBR), the trend of increasing water use appeared to be effectively suppressed. From 1998 to 2008, the total observed water use decreased by $0.49 km^3$ per year, while the estimation of water use still suggests $1.03 km^3$ increases (Figure~\ref{main_results} B).
The increased water uses after 87-WAS aligns with the fact that badly drying-up of the surface streamflow from 1987 to 1998, which was an obvious touchstone of river degradation and environmental crisis (Figure~\ref{main_results}C).
On the other hand, the environmental crisis of river drying up was effectively resolved after the 98-UBR, though the density of droughts still increased for decades (from $0.47$ after 87-WAS to $0.62$ after 98-UBR on average) (Figure~\ref{main_results}C).
In line with previous literature had reported; therefore, the institution shift of 98-UBR contributed a lot to the successful water governance. %! citation

\begin{figure*}[!h]
    \centering
    \includegraphics[width=32pc]{outputs/main_results2.pdf}
    \caption{
        Effects of two institutional shifts on water resources use and allocation in the Yellow River Basin (YRB).
        \textbf{A.} water uses of the YRB before and after the institutional shift in 1987 (87-WAS);
        \textbf{B.} water uses of the YRB before and after the institutional shift in 1998 (98-UBR). While the blue lines are statistic water use data, the grey ones are the estimation from the Differenced Synthetic Control method with economic and environmental background controlled.
        \textbf{C.} Drought intensity in the YRB and drying up events of the Yellow River. The size of the grey bubbles denotes the length of a drying upstream.
    }
    \label{main_results}
\end{figure*}

% P2: 主要介绍结果的意义、合理性
\label{result-1-p2}
Besides environmental background, our forecast by DSC takes economic factors into account under the assumptions that the production function between economic volume and water uses remained unchanged (\textit{S2 in Supplementary Material}).
It means the forecast of water use includes the part caused by the increased economic volume, while the outcomes of the economy (GDP in different sectors) of the YRB maintained a parallel trend with other regions during the period (\textit{S3 in Supplementary Material} Figure~\ref{S3-1}).
Therefore, 87-WAS did not ``have little effect'' as previous analyses suggested (cites) but led to increased water use because the difference between prediction and observation will be trivial when the shift was just a blank policy by applying the DSC method. %! Citation
Water-use intensity is another crucial factor in interpreting the differences besides the economic factors (e.g., irrigated areas and industrial outcomes) considered and controlled by the method.
In addition to the expansion of irrigation area after the 87-WAS, water uses per unit of irrigation area also rapidly widened the gap with the average level of the rest provinces. However, the industry water use intensity hardly changed (\textit{S3 in Supplementary Material} Figure~\ref{S3-2}).
As a previous report sigh: although the key to alleviating the drought is saving water in the irrigated areas, the tragedy of scrambling for water appeared in provinces and irrigated areas %! \cite{mao2000}.
In terms of the average ratio of water-saving irrigation area (refer to drip or sprinkler irrigation systems and canal lining), although there was a significant increase in the whole country after 1987, the YRB did not rapidly open a noticeable gap until about 1994 (Figure~\ref{S3-3}).
As a result, despite the irrigation area expanding, scrambling for water resources without any incentive to optimize production per unit of water resources accelerated holistic water use.
This accelerated water use was contrary to the original intention of the 87-WAS in conserving the limited water, and the failure was a barrier to the sustainability %! \cite{huangang2002}.

% 过去的研究总结出87-WAS收效甚微的几个因素:
Previous studies have summarised factors that contribute to the non-ideal effect of 87-WAS: (1) The YRCC had no right to punish the provinces for over-exploitation; (2) the water quotas were annual values, causing provinces to rob water in the dry season; (3) The YRCC can make statistics on water use in the mainstream but cannot on the tributaries, so provinces water use underreport %! \cite{huangang2002}.
However, the effects of the two institutional shifts (the 87-WAS and the 98-UBR) were significantly different, which the above reasons cannot fully explain.
Between the 98-UBR and the further refinement of the unified regulation in 2008, there was still a lack of a temporary water allocation scheme and effective monitoring of tributaries.
Moreover, without any actual punishment, provinces with high water consumption (such as Inner Mongolia and Shandong) continued to exceed the quota after 98-UBR.
As we have analyzed (Figure~\ref{structure}), the difference between the two institutional shifts is mainly reflected in the structure of linkages between social actors.
Until the institutional shift of 98-UBR, with no necessity to apply for a water permit from YRCC, there were no horizontal connections (cooperations or agreements) between the various stakeholders (provinces) directly connected to the ecological units.
Make it clear that the YRCC was responsible for regulating provincial water use; that is, each province has made it clear that in the long run, water resources are not ``internal'' but ``dependent'' on YRCC.
In that way, the YRCC, whose authority scale matches the whole river basin, also took the primary responsibilities to the river, and literature recognized the structure as a social-ecological fit that usually led to good outcomes.
Empirical studies in many different fields also indicate that the structure before 98-UBR (i.e., fragment ecological units are linked to separate social actors) is likely to be mismatched as isolated stakeholders struggle with holistically maintaining interconnected ecosystems
\cite{sayles2017,sayles2019,cai2016,bergsten2019}.
The effect of the institutional shifts once again demonstrated that it is not easy to have a win-win situation of environment and interests in complex coupled human-nature systems \cite{hegwood2022} which calls for exceptional understanding and caution to the structure of hampering sustainability \cite{bergsten2019, sayles2019}.
