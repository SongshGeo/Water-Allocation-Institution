%! Author = songshgeo
%! Date = 2022/3/10

% 为了量化制度变迁为黄河流域用水带来的影响,我们按附图1所示的技术路线执行了分析过程
We first abstract the SES structures of water using in the YRB from 1975 to 2008, where two institutional shifts split the period into three pieces. We then estimated the net effects of two institutional shifts on total water uses, changing trends, and differences of the YRB's provinces, by Differenced Synthetic Control (DSC) method \cite{arkhangelsky2021}. Finally, for discussion, we created an economic model based on marginal revenue to provide a theoretical interpretation for the observed water use outcomes.

\subsection{Portraying structures}
% 制度结构关系抽象
We apply network approach to portray SES structures by abstracting relationships between ecological units (river reaches), stakeholders (provinces), govern unit (the YRCC) into general building blocks (or motifs) (see Figure~\ref{framework}), from the official documents.
For example, institutions may create a horizontal connected structure that encourages collaboration between the different stakeholders managing ecological components (Figure~\ref{framework} c).
Similarly, institutions for vertical management may enhance multi-layered SES matching by coordinating through the higher-level governing node (Figure~\ref{framework} d).
Empirical studies have suggested that such widespread building blocks in SES are the key to the functioning of structures, and a network-based description is a widely used way to depict them by abstracting links and nodes \cite{bodin2017a,kluger2020,guerrero2015}.
In this study, we examined the official documents of the two institutional shifts of concern (87-WAS and 98-UBR, see \textit{Appendix \nameref{secS1}} for details), in which the agents are abstracted as nodes and the required interactions between agents are abstracted as links.
The resulting structure is mainly used to inference the mechanism of institutional impact in the discussion, and to give an overview understanding of the institutional shifts.
We try to approve that focusing on SES structures rather than institutional details, can well interpret the differences caused by institutional shifts in the YRB, and keeps consistent with previous studies theoretically.

\subsection{Differenced Synthetic Control}
We estimate water use without institutional shifts effect by using the Differenced Synthetic Control (DSC) method.
The DSC method is a improved version of the Synthetic Control, an effective identification strategy for estimating the net effect of historical events or policy interventions on aggregate units (such as cities, regions, and countries) by constructing a comparable control unit \cite{abadie2010, abadie2015, hill2021}.
This method assumes that the outcome of the treated unit can be explained in terms of a set of control units that were themselves not affected by the intervention (i.e., institutional shifts here).
By using data-driven weights to balance pre-treatment outcomes for treated and control units, the DSC method imputes post-treatment control outcomes for the treated unit(s) by constructing a synthetic version of the treated unit(s) equal to a convex combination of control units with a new estimator with improved bias properties.

In practice, each of the units (i.e., provinces) in the treated group were affected by institutional shifts in 1987 and 1998, each of which was taken as the ``shifted'' time $t_0$ within two individually analyzed periods $T$: 1975-1998; 1987-2008.
Including each province in the YRB ($n=8$, see \textit{\nameref{sec:dataset}}) as the only treated unit, we consider the $J+1$ units observed in time periods $T = {1,2 \cdots , T}$ with the remaining $J=20$ units are untreated provinces from outside.
We define $T_0$ to represent the number of pre-treatment periods ($1,\cdots,t_0$) and $T_1$ the number post-treatment periods ($t_0,\cdots,T$), such that $T = T_0+ T_1$.
That is, treated unit is exposed to the institutional shift in every post-treatment period $T_0$, and unaffected by the institutional shift in all preceding periods $T_1$.
Then, any weighted average of the control units is a synthetic control and can be represented by a ($J * 1$) vector of weights $\mathbf{W} = (w_{1},...,w_{J})$, with $w_j \in (0, 1)$.
Among them, by introduce a ($k * k$) diagonal, semidefinite matrix $\mathbf{V}$ that signifies the relative importance of each covariate, the DSC method procedure for finding the optimal synthetic control ($W$) is expressed as follows:

\begin{equation}
    \mathbf{W^{*}(V)}=\underset{\mathbf{W} \in \mathcal{W}}{\operatorname{minimize}}\left(\mathbf{X}_{\mathbf{1}}-\mathbf{X}_{\mathbf{0}} \mathbf{W}\right)^{\prime} \mathbf{V}\left(\mathbf{X}_{\mathbf{1}}-\mathbf{X}_{\mathbf{0}} \mathbf{W}\right)
\end{equation}

where $\mathbf{W}^{*}(V)$ is the vector of weights $\mathbf{W}$ that minimizes the difference between the pre-treatment characteristics of the treated unit and the synthetic control, given $\mathbf{V}$. That is, $\mathbf{W^{*}}$ depends on the choice of $\mathbf{V}$ –hence the notation $\mathbf{W*(V)}$. We choose $\mathbf{V^{*}}$ to be the $\mathbf{V}$ that results in $\mathbf{W*(V)}$ that minimizes the following expression:

\begin{equation}
    \mathbf{V}^{*}=\underset{\mathbf{V} \in \mathcal{V}}{\operatorname{argmin}}\left(\mathbf{Z}_{1}-\mathbf{Z}_{0} \mathbf{W}^{*}(\mathbf{V})\right)^{\prime}\left(\mathbf{Z}_{1}-\mathbf{Z}_{0} \mathbf{W}^{*}(\mathbf{V})\right)
\end{equation}

That is the minimum difference between the outcome of the treated unit and the synthetic control in the pre-treatment period, where $\mathbf{Z}_{1}$ is a ($1*T_0$) matrix containing every observation of the outcome for the treated unit in the pre-treatment period. Similarly, let $\mathbf{Z}_{0}$ is a ($k * T_0$) matrix containing the outcome for each control unit in the pre-treatment period, $k$ is number of variables in the datasets.
The DSC method generalizes the difference-in-differences estimator and allows for time-varying individual-specific unobserved heterogeneity, with double robustness properties \cite{billmeier2013, smith2015}.

\subsection{Dataset and variables}\label{sec:dataset}
In this study, we aim to compare on actural and estimated water use of the YRB.
The actual water uses are accessible in China’s provincial annual water consumption dataset from the National Water Resources Utilization Survey, whose details are accessible from Zhou (2020) \cite{zhou2020}.
To estimate the water use of the YRB by assuming there were not effects from institutional shifts, we focused on variables from five categories (environmental, economic, domestic, and technological) water use factors. Their specific items and origins are listed in Table~\ref{tab:variables}.

Among the total $31$ data-accessible provinces (or regions) directly affected by the 87-WAS and the 98-UBR, we dropped Sichuan, Tianjin and Beijing because they influenced by the institutions, but trivial in their water use (see \textit{Appendix}~Table~\ref{tab:quota}). We then divided the dataset into a ``target group'' and a ``control group'', treating provinces involved in water quota as the target group $(n=8)$ and other provinces as the control group $(n=20)$, for applying the DSC.

\subsection{Economic model}\label{sec:model}
In order to inference the mechanisms underlying the results, we developed a economic model based on marginal revenue to analyze how institutional shift could have led to differences in water use.

\begin{ass}
    (Water-dependent production) Because of irreplaceability, water is assumed to be the only input of the production function with two types of production efficiency. The production function of a high-incentive province is $A_HF(x)$, and the production function of a low-incentive province is $A_LF(x)$ ($A_H>A_L$). F(x) is continuous, $F'(0)=\infty$, $ F'(\infty)=0$, $F'(x)>0$, and $F''(x)<0$. The production output is under perfect competition, with a constant unit price of $P$.
\end{ass}

\begin{ass}
    (Ecological cost allocation) Under the assumption that the ecology is a single entity for the whole basin involved in $N$ provinces, the cost of water use is equally assigned to each province under any water use. The unit cost of water is a constant $C$.
\end{ass}

\begin{ass}
    (Multi-period settings) There are infinite periods with a constant discount factor $\beta$ lying in (0,1). There is no cross-period smoothing in water use.
\end{ass}

Under the above simplified assumptions, we can demonstrate three cases -corresponding the abstracted SES structures (Figure~\ref{structure}), inference stakeholders in a whole basin to simulate their water use patterns by optimization of marginal revenue.
As one of the possible interpretation for the causity between SES structure and institutional effects, the derivation of the model based on the above three assumptions can be found in \textit{Appendix~\nameref{secS4}}, and some simple model-based extensions are involved in \textit{Appendix~\nameref{secS5}}.
