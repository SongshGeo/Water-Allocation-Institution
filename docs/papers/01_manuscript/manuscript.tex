% chktex-file 46
\documentclass[preprint, 12pt]{elsarticle}

%% Use the option review to obtain double line spacing
%% \documentclass[authoryear,preprint,review,12pt]{elsarticle}

%% Use the options 1p,twocolumn; 3p; 3p,twocolumn; 5p; or 5p,twocolumn
%% for a journal layout:
%% \documentclass[final,1p,times]{elsarticle}
%% \documentclass[final,1p,times,twocolumn]{elsarticle}
%% \documentclass[final,3p,times]{elsarticle}
%% \documentclass[final,3p,times,twocolumn]{elsarticle}
%% \documentclass[final,5p,times]{elsarticle}
%% \documentclass[final,5p,times,twocolumn]{elsarticle}

%% For including figures, graphicx.sty has been loaded in
%% elsarticle.cls. If you prefer to use the old commands
%% please give \usepackage{epsfig}

%% The amssymb package provides various useful mathematical symbols
\usepackage{amssymb}
\usepackage{amsmath, graphicx, array}
\usepackage{dcolumn, soul}%
\let\openbox\relax
\usepackage{amsthm}
\usepackage[figuresright]{rotating}%
\usepackage{algorithm, algorithmicx, algpseudocode}
\usepackage{listings}%
\usepackage{hyperref}
\usepackage{geometry}

\geometry{a4paper,left=2.0cm,right=2.0cm,top=2.5cm,bottom=2.5cm}
% 行距
\linespread{1.5}
% \usepackage[nofiglist,notablist]{endfloat}
% \def\uns{\ifmmode\,\else$\,$\fi}%
\newtheorem{defn}{Definition}
\newtheorem{thm}{Theorem}
\newtheorem{ass}{Assumption}
\newtheorem{prop}{Proposition}
\newtheorem{fig}{Fig.}
\newtheorem{case}{Case}
\newtheorem{case_appendix}{Case}
\newtheorem{example}{Example}[section]
\renewcommand{\proofname}{\textbf{Proof}}
\newtheorem{property}{Property}
\newtheorem{remark}{Remark}
%\usepackage{enumerate}
\usepackage{enumitem}
\usepackage{float}
\usepackage{multirow}
% \usepackage{lineno}
\usepackage{booktabs}
\usepackage{diagbox}
%%

% \jvol{XX}
% \jnum{X}
% \jyear{Year}
% \doi{10.1093/nsr/XXXX}
% \received{XX XX Year}
% \revised{XX XX Year}
% \accepted{XX XX Year}

% \markboth{One, Two, and Three}{One, Two, and Three}
\graphicspath{{../../../figs/}}
\journal{Jounral of Hydrology}

\begin{document}
\begin{frontmatter}
% \normalsize
% \dhead{RESEARCH ARTICLE}
% \subhead{EARTH SCIENCES}

% \bibliographystyle{../nsr}
% 社会水文学专刊地址
% https://www.sciencedirect.com/journal/journal-of-hydrology/about/call-for-papers#grounded-sociohydrology
\title{Achieving Effective River Basin Governance through Institutional Shifts: Examining the Yellow River Basin's Social-Ecological Systems in China}


\author[inst1]{Shuang Song}
\author[inst2]{Huiyu Wen}
\author[inst1]{*Shuai Wang}
\author[inst1]{Xutong Wu}

\author[inst3]{Graeme S. Cumming}
\author[inst1]{Bojie Fu}


\affiliation[inst1]{
     State Key Laboratory of Earth Surface Processes and Resource Ecology,
     Faculty of Geographical Science,
     Beijing Normal University,
     Beijing 100875,
     P.R. China}

% \affil{stitute of Land Surface System and Sustainability,
%      Faculty of Geographical Science,
%      Beijing Normal University,
%      Beijing 100875,
%      P.R. China}

\affiliation[inst2]{School of Finance,
     Renmin University of China,
     Beijing 100875,
     P.R. China}

\affiliation[inst3]{
     ARC Centre of Excellence for Coral Reef Studies,
     James Cook University,
     Townsville 4811,
     QLD, Australia}

% \affil{e research for this article was financed by the National Natural Science Foundation of China (CN) (Grant Nos. NSFC 42041007). A supplementary online appendix is available with this article at the \em{National Science Review} website.}

% \authornote{\textbf{Corresponding authors.} Email: shuaiwang@bnu.edu.cn}
%\authornote{Shuang Song and Huiyu Wen equally contributed to this work.}
% \renewcommand\linenumberfont{\normalfont\bfseries\small}
% \linenumbers
\begin{abstract}
     Global freshwater depletion necessitates efficient water governance that accounts for social-ecological system (SES) structures. To improve our understanding of the relationships between SES structures and water governance, we investigated the Yellow River Basin (YRB) in China, which has experienced significant anthropogenic alterations. We analyzed two institutional shifts, the 1987 Water Allocation Scheme (87-WAS) and the 1998 Unified Basinal Regulation (98-UBR), that re-framed SES structures. Our counterfactual identification showed that water use in the YRB increased by 5.75\% more than expected following the 87-WAS, while the 98-UBR reduced total water use. These heterogeneous effects align with our marginal benefits analysis, confirming the importance of well-designed SES structures in water governance. Our quasi-natural experiment on the YRB highlights the need to avoid fragmented ecological units and separated social actors for sustainable water governance, while also emphasizing the significance of understanding the underlying processes and relationships in SES structures.
\end{abstract}


%%Graphical abstract
% \begin{graphicalabstract}
%      \includegraphics{outputs/main_results2.pdf}
% \end{graphicalabstract}

%%Research highlights
% \begin{highlights}
%      \item Research highlight 1
%      \item Research highlight 2
% \end{highlights}

\begin{keyword}
     water use~\sep~water governance~\sep~social-ecological system~\sep~institutions~\sep~Yellow River
\end{keyword}

\end{frontmatter}

\section{INTRODUCTION}\label{sec:introduction}
% 水竞争的重要性
Widespread freshwater scarcity and overuse challenge the sustainability of large river basins, resulting in systematic risks to economies, societies, and ecosystems globally \cite{distefano2017, dolan2021, xu2020b, mekonnen2016}.
In the context of future climate change, the gap between supply and demand for water resources in large river basins is expected to become increasingly more prominent \cite{florke2018, yoon2021}.
Those river basin systems successfully supporting sustainable water resource use are structurally well-aligned with water provisioning and social-ecological demands, without inefficient competition or overuses \cite{wang2019d}.
However, balancing the water demands of ecosystems and development in heavily human-dominated river basins is a challenge because human activities and water are intertwined in their structures as complex social-ecological systems (SES) \cite{huggins2022,konar2019}.

For governing river basin systems, their SES structures can be reshaped by institutions, such as policies, laws, and norms \cite{young2008,cumming2020b}.
Representing all relative governance practices, institutions include interplays between social actors, ecological units, or between social and ecological system elements
\cite{lien2020, bodin2017b}.
Understanding how these complex interplays is crucial for developing strategies to effectively manage natural resources and enhance the resilience of social-ecological systems \cite{kluger2020}.
However, the best approach to designing effective institutions remains an open question \cite{agrawal2003, persha2011, agrawal2001}.

Effective (``matched'' or ``fit'') institutions operate at appropriate spatial, temporal, and functional scales to manage and balance different relationships and interactions between human and water systems, supporting (but not guaranteeing) the sustainability of SES \cite{epstein2015, wang2019d}.
Some institutional advances have had desirable water governance outcomes (e.g., the Ecological Water Diversion Project in Heihe River Basin, China \cite{wang2019d}, and collaborative water governance systems in Europe \cite{green2013}).
However, imposing institutional changes on a large, complex river basin may create or destroy hundreds of connections between social agents and ecological units, where matched social-ecological structures are not ubiquitous.
Two particular weaknesses in existing knowledge of institutional matches include understanding: (i) the causal links between SES structures and outcomes; (ii) details of the underlying processes, and especially the coordination of the incentives of different participants, that result from an institutional lack of match.
These weaknesses limit understanding of institutional design, and hinder approaches toward institutional matches for improving the sustainability of river basin systems.

To better understand how water governance institutions can be designed to match their social-ecological context, we take the Yellow River Basin (YRB), China, as an example \textit{\nameref{sec:yrb}} to dive into causal links between SES structures and outcomes.
We used data on changes in official documents following two institutional shifts (the 87-WAS and the 98-UBR) to describe comparable changes in the SES structures associated with the YRB from 1979 to 2008 \textit{\nameref{sec:structures}}.
We then used a method called `Differenced Synthetic Control (DSC)' \cite{arkhangelsky2021}, which considers economic growth and natural background, to estimate theoretical water use scenarios without institutional shifts (\textit{\nameref{sec:DSC}} and \textit{\nameref{secS2}}).
This approach allowed us to create a counterfactual against which to explore the mechanisms linking SESs structure and outcomes for a deeper understanding of the potential role of institutions in water governance worldwide.
Finally, we further developed an approach for marginal benefits analysis, to interpret the underlying processes of match and mismatched institutions based on SESs structures (\textit{\nameref{sec:model}}).


\section{MATERIALS AND METHODS}\label{sec:methods}
%! Author = songshgeo
%! Date = 2022/3/10

% 为了量化制度变迁为黄河流域用水带来的影响,我们按附图1所示的技术路线执行了分析过程
We first abstract the SES structures of water using in the YRB from 1975 to 2008, where two institutional shifts split the period into three pieces. We then estimated the net effects of two institutional shifts on total water uses, changing trends, and differences of the YRB's provinces, by Differenced Synthetic Control (DSC) method \cite{arkhangelsky2021}. Finally, for discussion, we created an economic model based on marginal revenue to provide a theoretical interpretation for the observed water use outcomes.

\subsection{Portraying structures}
% 制度结构关系抽象
We apply network approach to portray SES structures by abstracting relationships between ecological units (river reaches), stakeholders (provinces), govern unit (the YRCC) into general building blocks (or motifs) (see Figure~\ref{framework}), from the official documents.
For example, institutions may create a horizontal connected structure that encourages collaboration between the different stakeholders managing ecological components (Figure~\ref{framework} c).
Similarly, institutions for vertical management may enhance multi-layered SES matching by coordinating through the higher-level governing node (Figure~\ref{framework} d).
Empirical studies have suggested that such widespread building blocks in SES are the key to the functioning of structures, and a network-based description is a widely used way to depict them by abstracting links and nodes \cite{bodin2017a,kluger2020,guerrero2015}.
In this study, we examined the official documents of the two institutional shifts of concern (87-WAS and 98-UBR, see \textit{Appendix \nameref{secS1}} for details), in which the agents are abstracted as nodes and the required interactions between agents are abstracted as links.
The resulting structure is mainly used to inference the mechanism of institutional impact in the discussion, and to give an overview understanding of the institutional shifts.
We try to approve that focusing on SES structures rather than institutional details, can well interpret the differences caused by institutional shifts in the YRB, and keeps consistent with previous studies theoretically.

\subsection{Differenced Synthetic Control}
We estimate water use without institutional shifts effect by using the Differenced Synthetic Control (DSC) method.
The DSC method is a improved version of the Synthetic Control, an effective identification strategy for estimating the net effect of historical events or policy interventions on aggregate units (such as cities, regions, and countries) by constructing a comparable control unit \cite{abadie2010, abadie2015, hill2021}.
This method assumes that the outcome of the treated unit can be explained in terms of a set of control units that were themselves not affected by the intervention (i.e., institutional shifts here).
By using data-driven weights to balance pre-treatment outcomes for treated and control units, the DSC method imputes post-treatment control outcomes for the treated unit(s) by constructing a synthetic version of the treated unit(s) equal to a convex combination of control units with a new estimator with improved bias properties.

In practice, each of the units (i.e., provinces) in the treated group were affected by institutional shifts in 1987 and 1998, each of which was taken as the ``shifted'' time $t_0$ within two individually analyzed periods $T$: 1975-1998; 1987-2008.
Including each province in the YRB ($n=8$, see \textit{\nameref{sec:dataset}}) as the only treated unit, we consider the $J+1$ units observed in time periods $T = {1,2 \cdots , T}$ with the remaining $J=20$ units are untreated provinces from outside.
We define $T_0$ to represent the number of pre-treatment periods ($1,\cdots,t_0$) and $T_1$ the number post-treatment periods ($t_0,\cdots,T$), such that $T = T_0+ T_1$.
That is, treated unit is exposed to the institutional shift in every post-treatment period $T_0$, and unaffected by the institutional shift in all preceding periods $T_1$.
Then, any weighted average of the control units is a synthetic control and can be represented by a ($J * 1$) vector of weights $\mathbf{W} = (w_{1},...,w_{J})$, with $w_j \in (0, 1)$.
Among them, by introduce a ($k * k$) diagonal, semidefinite matrix $\mathbf{V}$ that signifies the relative importance of each covariate, the DSC method procedure for finding the optimal synthetic control ($W$) is expressed as follows:

\begin{equation}
    \mathbf{W^{*}(V)}=\underset{\mathbf{W} \in \mathcal{W}}{\operatorname{minimize}}\left(\mathbf{X}_{\mathbf{1}}-\mathbf{X}_{\mathbf{0}} \mathbf{W}\right)^{\prime} \mathbf{V}\left(\mathbf{X}_{\mathbf{1}}-\mathbf{X}_{\mathbf{0}} \mathbf{W}\right)
\end{equation}

where $\mathbf{W}^{*}(V)$ is the vector of weights $\mathbf{W}$ that minimizes the difference between the pre-treatment characteristics of the treated unit and the synthetic control, given $\mathbf{V}$. That is, $\mathbf{W^{*}}$ depends on the choice of $\mathbf{V}$ –hence the notation $\mathbf{W*(V)}$. We choose $\mathbf{V^{*}}$ to be the $\mathbf{V}$ that results in $\mathbf{W*(V)}$ that minimizes the following expression:

\begin{equation}
    \mathbf{V}^{*}=\underset{\mathbf{V} \in \mathcal{V}}{\operatorname{argmin}}\left(\mathbf{Z}_{1}-\mathbf{Z}_{0} \mathbf{W}^{*}(\mathbf{V})\right)^{\prime}\left(\mathbf{Z}_{1}-\mathbf{Z}_{0} \mathbf{W}^{*}(\mathbf{V})\right)
\end{equation}

That is the minimum difference between the outcome of the treated unit and the synthetic control in the pre-treatment period, where $\mathbf{Z}_{1}$ is a ($1*T_0$) matrix containing every observation of the outcome for the treated unit in the pre-treatment period. Similarly, let $\mathbf{Z}_{0}$ is a ($k * T_0$) matrix containing the outcome for each control unit in the pre-treatment period, $k$ is number of variables in the datasets.
The DSC method generalizes the difference-in-differences estimator and allows for time-varying individual-specific unobserved heterogeneity, with double robustness properties \cite{billmeier2013, smith2015}.

\subsection{Dataset and variables}\label{sec:dataset}
In this study, we aim to compare on actural and estimated water use of the YRB.
The actual water uses are accessible in China’s provincial annual water consumption dataset from the National Water Resources Utilization Survey, whose details are accessible from Zhou (2020) \cite{zhou2020}.
To estimate the water use of the YRB by assuming there were not effects from institutional shifts, we focused on variables from five categories (environmental, economic, domestic, and technological) water use factors. Their specific items and origins are listed in Table~\ref{tab:variables}.

Among the total $31$ data-accessible provinces (or regions) directly affected by the 87-WAS and the 98-UBR, we dropped Sichuan, Tianjin and Beijing because they influenced by the institutions, but trivial in their water use (see \textit{Appendix}~Table~\ref{tab:quota}). We then divided the dataset into a ``target group'' and a ``control group'', treating provinces involved in water quota as the target group $(n=8)$ and other provinces as the control group $(n=20)$, for applying the DSC.

\subsection{Economic model}\label{sec:model}
In order to inference the mechanisms underlying the results, we developed a economic model based on marginal revenue to analyze how institutional shift could have led to differences in water use.

\begin{ass}
    (Water-dependent production) Because of irreplaceability, water is assumed to be the only input of the production function with two types of production efficiency. The production function of a high-incentive province is $A_HF(x)$, and the production function of a low-incentive province is $A_LF(x)$ ($A_H>A_L$). F(x) is continuous, $F'(0)=\infty$, $ F'(\infty)=0$, $F'(x)>0$, and $F''(x)<0$. The production output is under perfect competition, with a constant unit price of $P$.
\end{ass}

\begin{ass}
    (Ecological cost allocation) Under the assumption that the ecology is a single entity for the whole basin involved in $N$ provinces, the cost of water use is equally assigned to each province under any water use. The unit cost of water is a constant $C$.
\end{ass}

\begin{ass}
    (Multi-period settings) There are infinite periods with a constant discount factor $\beta$ lying in (0,1). There is no cross-period smoothing in water use.
\end{ass}

Under the above simplified assumptions, we can demonstrate three cases -corresponding the abstracted SES structures (Figure~\ref{structure}), inference stakeholders in a whole basin to simulate their water use patterns by optimization of marginal revenue.
As one of the possible interpretation for the causity between SES structure and institutional effects, the derivation of the model based on the above three assumptions can be found in \textit{Appendix~\nameref{secS4}}, and some simple model-based extensions are involved in \textit{Appendix~\nameref{secS5}}.


\section{RESULTS}\label{sec:results}
%! Author = songshgeo
%! Date = 2022/3/10

% \subsection{INSTITUTIONAL SHIFTS AND STRUCTURES}
\subsection{Institutional shifts and structures}
\label{results-1}

% 制度变动综述
% Including the national authorities, the basin management authorities, provinces, cities, and even districts, top-down institutional structures of the YRB started to evolve up to now (\textit{S1 in Supplementary Material}).
% As a pioneer in water governance shifting in China, the YRB started to explore a water allocating scheme since 1970s.
The institutional shifts in 1987 (87-WAS) and 1998 (98-UBR) were two widely recognized milestones in restricting water use among national water governance practices (\textit{Appendix}~\nameref{secS1}).
Until the 87-WAS, stakeholders (the provinces in the YRB) had free access to the YR water resources for development, but there were geographic and temporal differences between freshwater demand and availability.
As a compounded result of development, the provinces such as Shandong, Henan and Inner Mongolia used more water resources in the YRB with larger economies (primarily for irrigation agriculture).
For shrinking water deficits, national authorities proposed in 87-WAS allocating specific water quotas between $10$ provinces (or regions) along the YR basin.
However, the controversial scheme helped little in turning the water depletion around until another strategy attempted to strengthen the responsibilities of the YRCC in integrated water management in 1998 (the 98-UBR).
Therefore, our analysis period spans from 1975 (emergence of river depletion) to 2008 (a further polish of the 98-UBR), with the SESs shifted between three varying institutions (Figure~\ref{structure}).

We selected institutional regulatory documents on water use issued by national ministries (for validation to both watershed and regional agents) and extracted the interactions between the agents involved (\textit{Appendix \nameref{secS1}}).
Before 1987, the YRCC had no links to the provinces regarding water use, and the provinces could link to the Yellow River reaches directly (Figure~\ref{structure}).
However, according to the extracted information from the 87-WAS, the YRCC started to report water use from the provinces.
Furthermore, information from the 98-UBR documents demonstrated that the provinces had to apply their plan for an annual water use licence instead of direct access to the Yellow River water.
Thus, there have been links between the YRCC and the provinces since the strengthening responsibilities of the YRCC in 1998.

\begin{figure*}[!htb]
    \includegraphics[width=\linewidth]{diagrams/diagram.pdf}
	\caption{
		% 黄河流域的制度变迁与经济社会结构差异。
		Institutional shifts and related SES structures in the Yellow River Basin (YRB).
		\textbf{A.} The YBR crosses $10$ provinces or the same-level administrative regions, $8$ of which are highly relying on the water resources from the YRB (see \textit{Appendix \nameref{secS1}} Table~\ref{tab:quota}). The national administrations are the ultimate authority in issuing water governance policies, which are often implemented by basin-level agency (the Yellow River Conservancy Commission, YRCC) and each province-level agency.
		\textbf{B.} Since the YRCC does not use but monitor water, the provincial administrative agencies are the major stakeholders. Since the 87-WAS, with surface water withdraw from the Yellow River restricted by specific quotas, each stakeholder seperatly develop by planning and using fundamental water resources. However, the natural hydrological processes are connected. Although the institutions focus mainly on the surface water (Sur.), it can also influence groundwater inside (Gro.) or water resources outside (Sur. and Gro.'), through systematic socio-hydrological processes within the YRB.
		\textbf{C.} Institutional shifts and following structures changes (details in \textit{Appendix \nameref{secS1}}). (1) From 1975 to 1987, water resources were freely accessible to each stakeholder (denoted by red circles) from connected ecological unit (the reach of Yellow River, denoted by the blue circles). (2) After 1987-WAS, the YRCC (the yellow circles) was monitoring (the dot-line links) river reaches with the water use quota. (3) Since the 98-UBR, stakeholders have to apply water use licences from the YRCC (the connections between the red and yellow circles).
	}
	\label{structure}
\end{figure*}

%! Author = songshgeo
%! Date = 2022/3/10
% \MakeUppercase{\subsection{Cascading effects of the institutional shifts}}

% \subsection{ISS IMPACT ON WATER USE OF THE YRB}
\subsection{ISs impact on water use of the YRB}
\label{result-2}
% 结果一:展示制度转变带来的用水量变化

\label{result-1-p2}
Here, we use Differenced Synthetic Control (DSC) method, which considers economic growth and natural background, to estimate theoretical water use scenarios without basinal policy interferences (\textbf{Methods}; \textit{S2 in Supplementary Material}).
Our results suggest that the institutional shift in 1987 (87-WAS) stimulated the provinces to withdraw more water than would have been used without the interference (Figure~\ref{main_results}A).
From 1988 to 1998, while the estimation of water use only suggests $956.38 km^3$, the observed water use of the YRB provinces reached $1038.36 km^3$ in sum, $8.57\%$ increased.
However, after the institution shifted again in 1998 (98-UBR), the trend of increasing water use appeared to be effectively suppressed. From 1998 to 2008, the total observed water use decreased by $0.49 km^3$ per year, while the estimation of water use still suggests $1.03 km^3$ increases (Figure~\ref{main_results} B).
The increased water uses after 87-WAS aligns with the fact that badly drying-up of the surface streamflow from 1987 to 1998, which was an obvious touchstone of river degradation and environmental crisis (Figure~\ref{main_results}C).
On the other hand, the environmental crisis of river drying up was effectively resolved after the 98-UBR, though the density of droughts still increased for decades (from $0.47$ after 87-WAS to $0.62$ after 98-UBR on average) (Figure~\ref{main_results}C).
In line with previous literature had reported; therefore, the institution shift of 98-UBR contributed a lot to the successful water governance. %! citation

\begin{figure*}[!h]
    \centering
    \includegraphics[width=32pc]{outputs/main_results2.pdf}
    \caption{
        Effects of two institutional shifts on water resources use and allocation in the Yellow River Basin (YRB).
        \textbf{A.} water uses of the YRB before and after the institutional shift in 1987 (87-WAS);
        \textbf{B.} water uses of the YRB before and after the institutional shift in 1998 (98-UBR). While the blue lines are statistic water use data, the grey ones are the estimation from the Differenced Synthetic Control method with economic and environmental background controlled.
        \textbf{C.} Drought intensity in the YRB and drying up events of the Yellow River. The size of the grey bubbles denotes the length of a drying upstream.
    }
    \label{main_results}
\end{figure*}


% \subsection{REGIONAL DIFFERENCES IN RESPONSES TO THE IS}
\subsection{Regional differences in responses to the ISs}
\label{result-3}
% 结果2部分:展示区域相应差异

Differences between stakeholders in responses to institutional shifts are vital to understanding the mechanism between structures and outcomes.
Our results show that the proportion of accelerated water use in each province after the decade of 87-WAS (the proportion of actual water use exceeding the predicted water use by the model) has a significant correlation ($p<0.05$, see \textbf{Methods}) to the Yellow River water use in each province (Figure~\ref{upset}A).
Furthermore, while no evident impacts for most provinces (no more than $10\%$ differences, the apparent acceleration effects were only prominent in the big water-using provinces (e.g., Neimeng, Henan, and Shandong. Figure~\ref{upset}B).
In particular, Neimeng and Shandong, both provinces that exceeded the prescribed water uses of the 87-WAS, used $44.25\%$ and $25.69\%$ more water uses than the prediction from 1987 to 1998, respectively.
Furthermore, the satisfaction of each province with the water allocation stipulated by the 87-WAS (expressed by the difference between the actual water allocation of each province and the expected planning value) has little significant correlation to the acceleration.
By contrast, after the 98-UBR, except Shaanxi (which has always been abundant in water quota) had an evident ($17.53\%$) increase in water use, almost all provinces have seen significant declines in water use ($-12.5\%$ on average).
However, neither the satisfaction nor Yellow River water use correlates with the declines after the 98-UBR.

\begin{figure*}[!h]
    \centering
    \includegraphics[width=32pc]{outputs/upset_87.pdf}
    \caption{
        \textbf{A.} The partial correlation coefficient between wate uses (WU) of Yellow River (YR), unsatisfied ratio (compared with requirements in water plan and supply in the 87-WAS), and the average accelerated ratio.
        \textbf{B.} Average accelerated ratio of water uses for each province in the YRB during the decade after 87-WAS (from 1987 to 1998).
        \textbf{Mian users:} Major water consumption provinces (over the median).
        \textbf{Overused:} violate the 87-WAS in average water uses.
    }
    \label{upset}
\end{figure*}


\section{DISCUSSION}\label{sec:discussion}
%! Author = songshgeo
%! Date = 2022/3/10

% \subsection{CAUSES OF INSTITUTIONAL IMPACTS}
% \subsection{}
\label{discussion-1}
% discussion-1: 主要介绍结果的意义、合理性

Besides environmental background, our forecast by DSC takes economic factors into account under the assumptions that the production function between economic volume and water uses remained unchanged (\textit{S2 in Supplementary Material}).
It means the forecast of water use includes the part caused by the increased economic volume, while the outcomes of the economy (GDP in different sectors) of the YRB maintained a parallel trend with other regions during the period (\textit{S3 in Supplementary Material} Figure~\ref{S3-1}).
Therefore, 87-WAS did not ``have little effect'' as previous analyses suggested (cites) but led to increased water use because the difference between prediction and observation will be trivial when the shift was just a blank policy by applying the DSC method. %! Citation
Water-use intensity is another crucial factor in interpreting the differences besides the economic factors (e.g., irrigated areas and industrial outcomes) considered and controlled by the method.
In addition to the expansion of irrigation area after the 87-WAS, water uses per unit of irrigation area also rapidly widened the gap with the average level of the rest provinces. However, the industry water use intensity hardly changed (\textit{S3 in Supplementary Material} Figure~\ref{S3-2}).
As a previous report sigh: although the key to alleviating the drought is saving water in the irrigated areas, the tragedy of scrambling for water appeared in provinces and irrigated areas %! \cite{mao2000}.
In terms of the average ratio of water-saving irrigation area (refer to drip or sprinkler irrigation systems and canal lining), although there was a significant increase in the whole country after 1987, the YRB did not rapidly open a noticeable gap until about 1994 (Figure~\ref{S3-3}).
As a result, despite the irrigation area expanding, scrambling for water resources without any incentive to optimize production per unit of water resources accelerated holistic water use.
This accelerated water use was contrary to the original intention of the 87-WAS in conserving the limited water, and the failure was a barrier to the sustainability %! \cite{huangang2002}.

% 过去的研究总结出87-WAS收效甚微的几个因素:
Previous studies have summarised factors that contribute to the non-ideal effect of 87-WAS: (1) The YRCC had no right to punish the provinces for over-exploitation; (2) the water quotas were annual values, causing provinces to rob water in the dry season; (3) The YRCC can make statistics on water use in the mainstream but cannot on the tributaries, so provinces water use underreport %! \cite{huangang2002}.
However, the effects of the two institutional shifts (the 87-WAS and the 98-UBR) were significantly different, which the above reasons cannot fully explain.
Between the 98-UBR and the further refinement of the unified regulation in 2008, there was still a lack of a temporary water allocation scheme and effective monitoring of tributaries.
Moreover, without any actual punishment, provinces with high water consumption (such as Inner Mongolia and Shandong) continued to exceed the quota after 98-UBR.
As we have analyzed (Figure~\ref{structure}), the difference between the two institutional shifts is mainly reflected in the structure of linkages between social actors.
Until the institutional shift of 98-UBR, with no necessity to apply for a water permit from YRCC, there were no horizontal connections (cooperations or agreements) between the various stakeholders (provinces) directly connected to the ecological units.
Make it clear that the YRCC was responsible for regulating provincial water use; that is, each province has made it clear that in the long run, water resources are not ``internal'' but ``dependent'' on YRCC.
In that way, the YRCC, whose authority scale matches the whole river basin, also took the primary responsibilities to the river, and literature recognized the structure as a social-ecological fit that usually led to good outcomes.
Empirical studies in many different fields also indicate that the structure before 98-UBR (i.e., fragment ecological units are linked to separate social actors) is likely to be mismatched as isolated stakeholders struggle with holistically maintaining interconnected ecosystems
\cite{sayles2017,sayles2019,cai2016,bergsten2019}.
The effect of the institutional shifts once again demonstrated that it is not easy to have a win-win situation of environment and interests in complex coupled human-nature systems \cite{hegwood2022} which calls for exceptional understanding and caution to the structure of hampering sustainability \cite{bergsten2019, sayles2019}.

% \subsection{LINKING STRUCTURES WITH ISS OUTCOMES}
% \subsection{Linking structures with ISs outcomes}
\label{discussion-2}
% discussion-2: 机制解释
% 经济模型与理论解释
Differences in the pattern of the response by provinces can demonstrate the influence of social-ecological structures led by the institutional shifts.
We analyzed mathematically why the mismatched structure made limited water use holistically elusive in the institution shift of the 87-WAS but finally achieved by the 98-UBR (\textit{method} and \textit{Supplementary Material S4}).
By taking the structure before and after the two institutional shifts as different basic assumptions (before 87-WAS: free access to water; after 87-WAS but before 98-UBR: decisions on water use under quotas; after 98-UBR: unified regulation), we use the marginal benefit model to analyze the theoretical optimal water consumption of stakeholders in each scenario.
The analysis of the model also shows that 98-UBR can reduce the overall water use of the basin while 87-WAS can increase the water use of the basin when the same parameters are guaranteed but the institutional structure changes.
Before the 98-UBR, the model assumes that the separated ecological units (river reaches) link to stakeholders (related provinces) who use water to pursue their marginal benefits but have a potential political cost if they exceed the quota 87-WAS.
Our model suggests that for users who are already economically efficient (who are already using more water), greater marginal returns from water induce the acceleration of extracting resources for future economic growth (Figure~\ref{economic_model}).
Therefore, isolated stakeholders reacted to the similar marginal cost, and smaller water users have a threshold because of the political cost, so 87-WAS triggered an increased water use for the significant users.
On the contrary, the presence of central management (by the YRCC in this case, after 1998) can effectively reduce marginal ecological costs holistically as stakeholders only take corresponding responsibilities (follow the quota as possible as they can) to the YRCC (\textit{Supplementary Material S4}).
As a result, unified regulating acted the core role after the 98-UBR and reduced water use of all stakeholders (provinces) by irregular ratios.

The alignments of differences in institutional structures and outcomes here echo the hypothesis that successful governance of SES emerged by indirectly (or vertically) creating links between different stakeholders (in the YRB cases, through administration).
When links The water quotas of 87-WAS (or the initial water rights) in our case studies went through a stage of ``bargaining'' among stakeholders (from 1982 to 1987) \cite{wang2019a, wang2019d}, where each province attempted to demonstrate its development potential related to water use.
The bargaining itself was also a process towards matches between their economic volume and water shares, as studies show that the large water users (like Shandong and Henan) need more water than their quota (in the 87-WAS) if only considering the economic equity when designing the institution.
Furthermore, with information asymmetry between upper-level decision-makers and lower-level stakeholders in water use allocation, those with more current water use might have greater bargaining power.
In practice, therefore, although the affected provinces may not have directly encouraged excessive resource use because of the institutional shift, they had a more considerable incentive to show their economic potential
That aligns with the historical records that, even after the 87-WAS had already confirmed the quotas, provinces, especially water-intensive ones, challenged it by appearing to the higher central government for larger quotas.
On the contrary, after YRCC as governing agent coordinated between stakeholders since 98-UBR, the external appeal of provinces for larger quotas turned into internal innovation to improve water efficiency (e.g., drastically increased water-conserving equipment, \textit{Supplementary Material S3})
\cite{krieger1955, ostrom1990}.
Then, the YRCC, the authority for approving water applications from all stakeholders, could adjust water use quotas according to the river conditions of the whole basin.
The 98-UBR led to a structure for achieving social-ecological fits in both basins (between YRCC and the YRB) and regions (between provincial economy and their water shares).

% \subsection{LIMITATION, INSIGHTS AND IMPLICATIONS}
\subsection{Limitation, insights and implications}
\label{discussion-3}
% discussion-3: 启示、未来的展望

Agents matching the ecological scale appear widespread as motifs in SES of successful governance, whether in fisheries, forests, or groundwater management, suggesting that reducing independent stakeholders linked to fragmentation is an essential primary mechanism for a structure to produce good results.
% 由于87-WAS和98-UBR分别带来的两种结构在很多的SES中都是反复出现的构件,我们提出的机制对理解这类耦合系统至关重要
Since the structures introduced by 87-WAS and 98-UBR are recurring motifs in many SES, our proposed mechanism is crucial to understanding such coupled systems.
% 因此,我们通过黄河流域的准自然实验,探讨了社会经济结构与可持续发展(结果)之间的因果关系,为两个主要原因提供了一个有益的案例研究。
Furthermore, we explored the causal linkages between the SES structures and sustainability (outcomes) in quasi-natural experiments of the YRB, which provides an informative case study for two main reasons.
% 首先,长江流域管理的急剧结构变化使我们能够定量估计高层制度设计变化对用水的净影响。
First, the sharp structural shifts in YRB management enabled us to quantitatively estimate the net effects of changes in high-level institutional design on water use. Institutions that determine water allocation include bottom-up agreements or social norms as well as top-down quotas or regulations, with different effects on SES structure \cite{wang2019d,speed2013}; top-down regulations can trigger immediate institutional shifts and sharp SES structural changes \cite{speed2013,roland2004}.
In comparison with investigations of more gradual changes induced by bottom-up institutional shifts, exploring the impacts of a top-down change substantially diminishes potential problems of omitted variables in the quantitative analysis of SES and clarifies the causal link between SES structure and outcome.
% 其次,通过比较长江流域两次制度变迁所分裂的三种不同制度结构的净效应,我们也可以更深入地理解“盆地固定效应”下结构格局的影响。尽管流域内的社会经济单位从世界各地的大型河流流域和许多地区的水资源中受益,但很少有流域多次经历过如此激进的社会经济结构变化。
Second, we can better understand the influence of structural alignments under a fixed basin by comparing the net effects of three different institutional structures split by two institutional shifts in the YRB. Although socioeconomic units within a basin benefit from water resources in large river basins all over the world, and many locations have shown increased levels of regulation, few basins have experienced such radical SES structural changes several times (see \textit{Supplementary Material} S1). Thus, the YRB provides a valuable setting for understanding the direct impacts of changes in the SES institutional structure.
Finally, one of the limitations of our method is that it is difficult to rule out the effects of other policies over the same time breakpoints.
However, since scholars have reached a consensus on the importance of the two institutional shifts of 87-WAS and 98-UBR, the differences in their results still provide important insights for understanding water governance.

% 我们的模型结果与机制探讨加深了SES结构的理解,强化了孤立利益相关者形成的结构不利于制度解决环境问题的基本认识
Our results and discussion deepen the understanding of SES structure and strengthen the basic understanding that the mismatched structure formed by isolated stakeholders is not conducive to institutional solutions; -and then reported how another social-ecological fit structure contributed to successful water governance and sustainability.
Moreover, the subsequent success of 98-UBR has proved the importance of institutional scale matching both theoretically and practically. Therefore, it is necessary to emphasize the establishment of potentially connected building blocks between stakeholders by agents consistent with the scale of the ecological system (in this case, the basinal scale and the YRCC).
Furthermore, we applied several scenarios based on the marginal benefit model (see \textit{Supplementary Material S4}) for some further insights into sustainable water governance.
For example, water rights transfers can be another way to emerge horizontal links between stakeholders that also have the potential in resulting in better water governance.
In addition, the policymakers can also propose a more dynamic and flexible institution by increasing the frequency of quota updates that responds to changing conditions and will adapt more effectively to its SES context.

% 未来的政策建议
Calls for a redesign of water allocation institutions in the YRB in recent years also illustrate the importance of dynamic quota setting (see \textit{Supplementary Material S1}) \cite{yu2019}. Following the institutional reforms of 1998, the Yellow River has not dried up since 1999. However, given recent changes in the YRB, its rigid resource allocation scheme can no longer meet the new demands of economic development \cite{wang2019a}. As a result, the Chinese government has embarked on an ambitious plan to redesign its decades-old water allocation institution (see \textit{Supplementary Material S1}). These initiatives can benefit from our analysis by actively considering and incorporating social-ecological complexity and incentive structures when developing new approaches that avoid unsustainable outcomes. Our research provides a cautionary tale of how institutions can be a double-edged sword in attaining sustainability. Therefore, insights from the YRB can be a valuable guideline for SESs around the world facing similar governance problems \cite{cumming2020b, muneepeerakul2017, cumming2020a, leslie2015}.


\section{CONCLUSION}\label{sec:conclusion}
%! Author = songshgeo
%! Date = 2022/3/10

Intense water use in one of the most anthropogenic interfered large river basins, the Yellow River Basin (YRB), once led to overburdened drying up but finally had a successful restoration by sequential water governance practices.
Focusing on two water-demand institutions, 87-WAS and the 98-UBR, we quantitatively analyzed how institutional shifts played a role in the water governance achievement of the YRB.
Shifting throughout different SES structures framed by them, the observed water use of the YRB provinces had an $8.57\%$ increase than expected during the decade after the 87-WAS.
Then, water use significantly decreased by  $4.9$ billions $m^3$ per year since the 98-UBR, while the model still suggests a $10.3$ billions $m^3$ annual increase in expectation.
Finally, as differences in stakeholders' response to the institutional shifts, water use rises after the 87-WAS in provinces with more water uses (e.g., Inner Mongolia, Henan, and Shandong) while shrunk in nearly all provinces after the 98-UBR.
Since the above results closely align with interpretations from a mathematical marginal benefits model, we can link the structures (widespread building blocks) and outcomes (goals of the institution, i.e., limiting water demands) by these quasi-natural experiments of the YRB.
We demonstrate that social-ecological fits lead to successful governance where reducing independent stakeholders linked to fragmentation is an essential primary mechanism for good SES outcomes.


% Authors Contribution
% SW and BF designed this research. SS performed the study and analysed data. SS and HW wrote the paper. XW, GC, and HW revised and polished the manuscript and gave significant advice.

% Acknowledgments
% This research has been supported by the National Natural Science Foundation of China (grant no. 42041007) and the Fundamental Research Funds for the Central Universities

\bibliography{../mybib}
\bibliographystyle{elsarticle-num}\label{bib}

%%%%%%%% -----  02_appendix -------- %%%%%%%%%%
\newpage
\appendix\label{appendix}

\section{Contexts of institutional shifts}\label{secS1}
\renewcommand{\thefigure}{A\arabic{figure}}
\renewcommand{\thetable}{A\arabic{table}}
\setcounter{figure}{0}
\setcounter{table}{0}
%! Author = songshgeo
%! Date = 2022/3/10

We aim to abstract the water allocating institutions from the description in official documents with necessary context into SES building blocks (Figure~\ref{framework})
Widespread building blocks in SES are the key to the functioning of structures, and a network-based description is a widely used way to depict them by abstracting links and nodes \cite{bodin2017a,kluger2020,guerrero2015}.

\begin{figure}
	\centering
	\includegraphics[width=0.9\linewidth]{diagrams/framework.jpg}
	\caption{
		Framework for understanding linkages between SES structures and outcomes.
		\textbf{a.} The general framework for analyzing social-ecological systems (SESs) (adapted from Ostrom \cite{ostrom2009}). Institutions embedded in SESs may reshape structures by changing the interactions between core subsystems, resulting in different outcomes.
        Three typical types of abstracted SES structures are shown as \textbf{b.}, \textbf{c.} and \textbf{d.} (adapted from Bodin, 2017)\cite{bodin2017b}. Red circles indicate social actors, and green ones indicate ecological components. Connection (ties between two ecological components), collaboration (ties between two social actors), or management (ties between a social actor and an ecological component) exist when gray lines link two units. According to empirical evidence, the gray dashed lines show aligned SES structures that are more likely to achieve a desirable outcome.
        }
    \label{framework}
\end{figure}

% 水资源分配方案在全世界范围内都是流域管理的普遍制度。
Water allocation institutions are widespread in large river basin management programs throughout the world (see \textit{Appendix} Figure~\ref{fig:world}) \cite{speed2013}.
This was the first basin in China for which a water resource allocation institution was created, and institutional shifts can be traced through several documents released by the Chinese government (at the national level)\cite{wang2019a}:
\begin{itemize}
    \item \textbf{1982}: The provinces and the Yellow River Water Conservancy Commission (YRCC) are required to develop a water resource plan for the Yellow River \cite{wang2019, wang2019a}.
    \item \textbf{1987}: Implementation of the Allocation Plan. (\href{http://www.gov.cn/zhengce/content/2011-03/30/content_3138.htm#}{http://www.mwr.gov.cn}, last access: \today).
    \item \textbf{1998}: Implementation of unified regulation. (\href{http://www.mwr.gov.cn/ztpd/2013ztbd/2013fxkh/fxkhswcbcs/cs/flfg/201304/t20130411_433489.html}{http://www.mwr.gov.cn}, last access: \today).
    % 各省按要求编制新的黄河流域水资源规划,将水资源额度分配进一步细化。
    \item \textbf{2008}: Provinces are asked to draw up new water resources plans for the YRB to further refine water allocations \cite{wang2019,wang2019a}.
    \item \textbf{2021}: A call for redesigning the water allocation institution (\href{http://www.ccgp.gov.cn/cggg/zygg/gkzb/202107/t20210721_16591901.htm}{http://www.ccgp.gov.cn}, last access: \today).
\end{itemize}

% 在上述文件中,1982年的文件标志着设计分水制度尝试的开始,2008年标志着该制度走向成熟(完全建立起流域-省-市区的多级水资源分配和统一调度)。
Since 1982, administrations attemptted to design a quota institution, and the 2008 document marked the maturity of the scheme (complete establishment of basin-level, provincial, and district water quotas).
Between the period, two significant institutional shits can be analyzed by using the 1987 (87-WAS) and 1998 (98-UBR) documents.

% It is worth noting that, although the essential reason for these institutions was the mismatch between the spatial and temporal distribution of water resources as well as social and economic water demands, the direct reason for their introduction was the depletion of the Yellow River.

The official documents in 1987 (\href{http://www.gov.cn/zhengce/content/2011-03/30/content_3138.htm#}{http://www.mwr.gov.cn}, last access: \today) convey the following key points:

\begin{itemize}
	% 该政策面向的目标是各省(区域),黄委会没有被提及
	\item The policy is aimed at related provinces (or regions at the same administrative level).
	% 政策制定的首要考虑是解决断流问题
	\item Depletion of the river is identified as the first consideration of this institution.
	% 各省被鼓励在此配额下制定自己的用水计划
	\item Provinces are encouraged to develop their water use plans based on a quota system.
	% 水资源供给无法满足需求对相关省(地区)是普遍现象。
	\item Water in short supply is a common phenomenon in relevant provinces (regions).
\end{itemize}

The official documents in 1998
(\href{http://www.mwr.gov.cn/ztpd/2013ztbd/2013fxkh/fxkhswcbcs/cs/flfg/201304/t20130411_433489.html}{http://www.mwr.gov.cn}, last access: \today) convey the following key points:

\begin{itemize}
	% 除了说明政策针对的各省区之外,明确指出其用水需要黄河水利委员会进行申报,并由其组织和监管
	\item The document points out that not only provinces and autonomous regions involved in water resources management (see \textit{Article 3}), the provinces’ and regions’ water use shall be declared, organized, and supervised by the YRCC (\textit{Article 11 and Chapter III to Chapter V, and Chapter VII}).
	% 本研究(\textit{ Article 1})首先考虑的是上、中、下游用水的总体规划。
	\item Creating the overall plan of water use in the upper, middle, and lower reaches is identified as the first consideration of this institution (\textit{Article 1}).
	% 各省需要
	\item With the same quota as used in the 1987 policy, provinces were encouraged to further distribute their quota into lower-level administrations (see \textit{Article 6 and Article 41}).
	% 强调以总量确定供给,以供给决定需求。
	\item They emphasize that supply is determined by total quantity, and water use should not exceed the quota proposed in 1987 (see \textit{Article 2}).
\end{itemize}

\begin{figure*}[!htb]
    \centering
    \includegraphics[width=12cm]{/Users/songshgeo/Documents/Pycharm/WAInstitution_YRB_2021/figs/diagrams/world_institutions}
	\caption{
		Overview of water allocation institutions.
		% 世界已有水资源分配制度的大河流域,其中黄河流域最早于1987年提出资源分配方案,后于1998年更改为统一调度方案。
		\textbf{A.} Major river basins in the world with water resource allocation systems (shaded red); the YRB first proposed a resource allocation scheme in 1987 (designed since 1983) and then changed to a unified regulation scheme in 1998 (designed in 1997 but implemented in 1998) \cite{speed2013}.
		% 不同的水资源分配制度设计模式,中国黄河流域是典型的自上而下。
		\textbf{B.} Different water resource allocation system design patterns; the YRB is typical of a top-down system.
		% 流域分水制度的演化。这种多层次的制度设计有其历史变化过程。
		\textbf{C.} The four periods of institutional evolution of water allocation of the YRB.
	}
    \label{fig:world}
\end{figure*}

% 基于上述分析,我们抽象出了两次制度转变之后的SES结构变化如正文的图1C所示。
Based on the above documents, we abstracted the structural changes of SES (see \textit{Appendix S2}) after the two institutional changes, as shown in Figure~\ref{fig:structure}~C.

\begin{table*}
    \centering
    \small
    \caption{Water quotas assigned in the 87-WAS}\label{tab:quota}
	\resizebox{\linewidth}{!}{
    \begin{tabular}{p{0.24\linewidth}llllp{0.1\linewidth}lllll}
	\hline
	Items (water volume, billion $m^3$)                    & Qinghai & Sichuan & Gansu   & Ningxia & Inner Mongolia & Shanxi  & Shaanxi & Henan   & Shandong & Jinji  \\
	\hline
	Demands in water plan                                                & 35.7    & 0       & 73.5    & 60.5    & 148.9          & 115     & 60.8    & 111.8   & 84       & 6      \\
	Quota designed in 1983                                               & 14      & 0       & 30      & 40      & 62             & 43      & 52      & 58      & 75       & 0      \\
	Quota assigned in 1987                                               & 14.1    & 0.4     & 30.4    & 40.0    & 58.6           & 38.0    & 43.1    & 55.4    & 70.0     & 20     \\
	Average water consumption from the Yellow River from 1987-2008       & 12.03   & 0.25$^a$   & 25.80   & 36.58   & 61.97          & 21.16   & 11.97   & 34.30   & 77.87    & 5.85$^a$  \\
	Proportion of water from the Yellow River in total water consumption & 48.12\% & 0.10$^b$\%  & 30.79\% & 58.45\% & 47.82\%        & 73.55\% & 44.39\% & 24.77\% & 34.41\%  & 3.11\%$^b$ \\
    \hline
    \end{tabular}}
	\footnotesize[a]\leftline{{Calculated by data from 2004 to 2017.}}\\
	\footnotesize[b]{\leftline{The share is too small, thus the provinces (or region) Sichuan and Jinji not to be considered in this study.}}
\end{table*}


\section{Robustness of DSC method}\label{secS2}
\renewcommand{\thefigure}{B\arabic{figure}}
\renewcommand{\thetable}{B\arabic{table}}
\setcounter{figure}{0}
\setcounter{table}{0}
%! Author = songshgeo
%! Date = 2022/3/19


% % 找到具解释力的变量是构造合成控制法稳健的关键。
% Explanatory variables are the key to constructing a robust synthetic control method.
% % 我们共使用了用水量密切相关的26个变量,这些变量的数据集已在先前的研究中被用来解释中国的用水量变化
% We used a total of $24$ variables related to water consumption Table~\ref{tab:variables}, which datasets have been used in previous studies to explain changes in water use in China \cite{zhou2020}.
% % 由于这些变量间存在自相关,我们通过肘部法供选择了5个主成分作为DSC的输入,前人研究表明PCA方法的结合能够增强合成控制法的稳健性

% In addition, we selected $5$ principal components as input by the elbow method because selection in autocorrelated variables reduces dimensions and then enhances the robustness of the DSC (Figure~\ref{fig:elbow}).

% There are two approaches to validity testing of the DSC: (1) comparing the post-treated and pre-treated reconstructions and (2) testing robustness through placebo analysis.
% For (1), differences between each province and their synthetic are significant in post-treated periods and small in pre-treated periods (Figure~\ref{fig:87panel} and figure~\ref{fig:98panel}), which show good reconstructions of their water use changes' estimation.
% For (2), we applied the in-place placebo analysis described by \cite{abadie2010}. In most provinces, ratios of post-MSPE to pre-MSPE are higher than the median of other placebo units, which suggests the institutional shifts in treated time (1987 and 1998 here) influenced them more than most of the other provinces (figure~\ref{fig:87placebo}, figure~\ref{fig:98placebo}, Table~\ref{tab:DSC_summary}).

% \begin{figure*}[!bh]
%     \includegraphics[width=0.9\linewidth]{outputs/87panel.pdf}
%     \centering
%     \caption{Comparations between YRB' provinces and their synthetic controls around the 87-WAS.}
%     \label{fig:87panel}
% \end{figure*}

% \begin{figure*}
%     \includegraphics[width=0.9\linewidth]{outputs/98panel.pdf}
%     \centering
%     \caption{Comparations between YRB' provinces and their synthetic controls around the 98-UBR.}
%     \label{fig:98panel}
% \end{figure*}


% \begin{figure*}
%     \includegraphics[width=0.9\linewidth]{outputs/87placebo.pdf}
%     \centering
%     \caption{Gaps in change in water use between provinces outside the YRB and their synthetic control, around the 87-WAS, excluding the provinces with high pre-treatment RMSPE (more than $3$ times of treated units' RMSPE).}
%     \label{fig:87placebo}
% \end{figure*}

% \begin{figure*}
%     \includegraphics[width=0.9\linewidth]{outputs/98placebo.pdf}
%     \centering
%     \caption{Gaps in change in water use between provinces outside the YRB and their synthetic control, around the 98-UBR, excluding the provinces with high pre-treatment RMSPE (more than $3$ times of treated units' RMSPE)}
%     \label{fig:98placebo}
% \end{figure*}


\begin{table*}[!ht]
	\caption{Variables and their categories for water use predictions}
	\scriptsize
	\label{tab:variables}
	\resizebox{\linewidth}{!}{
	\begin{tabular}{lllll}
	\hline
	Sector &
	  Category &
	  Unit &
	  Description &
	  Variables \\ \hline
	Agriculture &
	  Irrigation Area &
	  thousand ha &
	  \begin{tabular}[c]{@{}l@{}}Area equipped for irrgiation by different \\ crop:\end{tabular} &
	  \begin{tabular}[c]{@{}l@{}}Rice, \\ Wheat, \\ Maize, \\ Fruits, \\ Others.\end{tabular} \\ \hline
	Industry &
	  \begin{tabular}[c]{@{}l@{}}Industrial gross \\ value added\end{tabular} &
	  Billion Yuan &
	  Industrial GVA by industries &
	  \begin{tabular}[c]{@{}l@{}}Textile, \\ Papermaking, \\ Petrochemicals, \\ Metallurgy, \\ Mining, \\ Food, \\ Cements, \\ Machinery, \\ Electronics, \\ Thermal electrivity, \\ Others.\end{tabular} \\
	 &
	  \begin{tabular}[c]{@{}l@{}}Industrial water \\ use efficiency\end{tabular} &
	  \% &
	  \begin{tabular}[c]{@{}l@{}}The ratio of recycled water and evaporated \\ water to total industrial water use\end{tabular} &
	  \begin{tabular}[c]{@{}l@{}}Ratio of industrial water recycling, \\ Ratio of industrial water evaporated.\end{tabular} \\ \hline
	Services &
	  \begin{tabular}[c]{@{}l@{}}Services gross \\ value added\end{tabular} &
	  Billion Yuan &
	  GVA of service activities &
	  Services GVA \\ \hline
	Domestic &
	  Urban population &
	  Million Capita &
	  Population living in urban regions. &
	  Urban pop \\
	 &
	  Rural population &
	  Million Capita &
	  Population living in rural regions. &
	  Rural pop \\
	 &
	  Livestock population &
	  Billion KJ &
	  \begin{tabular}[c]{@{}l@{}}Livestock commodity calories summed from \\ 7 types of animal.\end{tabular} &
	  Livestock \\ \hline
	Environment &
		  Temperature & $K$ & Near surface air temperature & Temperature \\
			& Precipitation & $mm$ & Annual accumulated precipitation & Precipitation \\ \hline
	\end{tabular}}
\end{table*}


\begin{figure*}[!h]
    \includegraphics[width=0.9\linewidth]{outputs/elbow.pdf}
    \centering
    \caption{Choose number of pricipal components by Elbow method, $5$ pricipal components already capture $89.63\%$ explained variance.}\label{fig:elbow}
\end{figure*}


% \section{S2: Methods in details}\label{secS3}
% \graphicspath{{../../../figs/}}

\begin{figure*}
    \includegraphics[width=0.7\linewidth]{outputs/economy.pdf}
    \centering
    \caption{test}
    \label{S3-1}
\end{figure*}


\begin{figure*}
    \includegraphics[width=0.7\linewidth]{outputs/S3_WUI.pdf}
    \centering
    \caption{test}
    \label{S3-2}
\end{figure*}


\begin{figure*}
    \includegraphics[width=0.7\linewidth]{outputs/S3_wci.pdf}
    \centering
    \caption{test}
    \label{S3-3}
\end{figure*}


\section{Optimization model for water use}\label{secS4}
\renewcommand{\thefigure}{C\arabic{figure}}
\renewcommand{\thetable}{C\arabic{table}}
\setcounter{figure}{0}
\setcounter{table}{0}
%! Author = songshgeo
%! Date = 2022/3/19


% \subsection{Structure-based marginal benefit analysis}
% \label{result-4}

For interpretation of the pattern of provincial water uses, we compared the theoretical marginal returns and optimal water use under three different structural cases (case 1 to case 3, corresponding to different SES structures in Figure~\ref{fig:structure}~C).

Assuming that water is the factor input with decreasing marginal output of each province, results show that varying incentives for water use in each province derive from the relationship between the benefits and costs of water use.
As a benchmark, case 1 analogy to a decentralized stakeholders situation and lead to medium-level water use.
In case 2, each stakeholder expects that current water use helps bargain for a favorable water quota in the face of institutional shift (see \textit{\nameref{secS4}}), which can intensify the incentive to use water, leading to higher water use.
Furthermore, the water users with higher capability are more stimulated by the institutional shift and away from the theoretically optimal water use under a unified allocation.
After water-use decisions are consolidated into unified management (case 3), marginal benefits analysis suggests the lowest water use among the cases.


\begin{figure}[!htb]
	\centering
	\includegraphics[width=0.6\linewidth]{outputs/economic_model.pdf}
	\caption{
		The proposed relationship of marginal benefits and water use of individual province under varying cases (case 1 to case 3, corresponding to the different SES structures in Figure~\ref{fig:structure}~C) Major water users' theoretically optimal water use is also larger (see the proofs below.)}
\end{figure}

Below are the detailed theoretical model derivation process, where we started from proposing three intuitive and general assumptions:

\begin{ass}
(Water-dependent production) Because of irreplaceability, water is assumed to be the only input of the production function with two types of production efficiency. The production function of a high-incentive province is $A_HF(x)$, and the production function of a low-incentive province is $A_LF(x)$ ($A_H>A_L$). F(x) is continuous, $F'(0)=\infty$, $ F'(\infty)=0$, $F'(x)>0$, and $F''(x)<0$. The production output is under perfect competition, with a constant unit price of $P$.
\end{ass}

\begin{ass}
 (Ecological cost allocation) Under the assumption that the ecology is a single entity for the whole basin involved in $N$ provinces, the cost of water use is equally assigned to each province under any water use. The unit cost of water is a constant $C$.
\end{ass}
\begin{ass}
(Multi-period settings) There are infinite periods with a constant discount factor $\beta$ lying in (0,1). There is no cross-period smoothing in water use.
\end{ass}

Under the above assumptions, we can demonstrate three cases consisting of local governments in a whole basin to simulate their water use decision-making and water use patterns.

 %case1
\begin{case} before 1987: This case corresponds to a situation without any high-level water allocation institution.

When each province independently decides on its water use, the optimal water use $x_i^*$ in province $i$ satisfies:

 $AF'(x)=\frac{C}{P}$,

 where $A_H$ and $A_L$ denote high-incentive and low-incentive provinces, respectively.

 When the decisions in different periods are independent, for $t$ = $0, 1, 2 \cdots$, then:

 $x_{it}^* = x_i^*$

 \end{case}

 %case2
 \begin{case} from 1987 to 1998: This case corresponds to an SES structure where fragmented stakeholders are linked to unified river reaches.

 The water quota is determined at $t$=0 and imposed in $t$=1,2, \ldots Under the subjective expectation of each province that current water use may influence the future water allocation determined by high-level authorities, the total quota is a constant denoted as Q, and the quota for province $i$ is determined in a proportional form:

 $Q_i=Q \cdot \frac{x_i}{x_i + \begin{matrix}\sum{x_{-i}} \end{matrix}}$.

Under a scenario with decentralized decision-making with a water quota, given other provinces' decisions on water use remain unchanged, the optimal water use of province $i$ at $t$=0 satisfies:

$AF'(x_{i,0})=\frac{C}{P \cdot N} - \frac{\beta}{1-\beta} \cdot A \cdot f(Q \cdot \frac{x_{i,0}}{\begin{matrix} x_{i,0} + \sum x_{-i,0} \end{matrix}}) \cdot Q \cdot \frac{\begin{matrix} \sum x_{-i,0} \end{matrix}}{(\begin{matrix} x_{i,0} + \sum x_{-i,0} \end{matrix})^2}$,

where $A_H$ denotes a high-incentive province and $A_L$ denotes a low-incentive province.

\end{case}

 %case3
\begin{case} after 1998: This case corresponds to the institution under which water use in a basin is centrally managed.

 When the $N$ provinces decide on water use as a unified whole (e.g., the central government completely decides and controls the water use in each province), the optimal water use $x_i^*$ of province $i$ satisfies:

$F'(x)=\frac{C}{P}$.

\end{case}

We propose Proposition 1 and Proposition 2:

Proposition 1: Compared with the decentralized institution, a institution with unified management decreases total water use.

The optimal water use under the three cases implies that mismatched institutions cause incentive distortions and lead to resource overuse.


Proposition 2: Water overuse is higher among provinces with high water use incentives than low- water use incentives under a mismatched institution.

The intuition for this proposition is straightforward in that all provinces would use up their allocated quota under a relatively small $Q$. As production efficiency increases, the marginal benefits of a unit quota increase, and the quota would provide higher future benefits for a pre-emptive water use strategy. Provinces with high production efficiency have higher optimal water use values under the decentralized decision. The divergence in water use would be exaggerated when the water quota is expected to be implemented with greater competition.

%Appendix的模型和proposition推导细节部分
%放在Appendix

%case1

When the N provinces decide on water uses as a unity, the marginal cost is C, equal to its fixed unit cost.
The water use of province $i$ aims to maximize $P\cdot A\cdot F(x)-C$.
Hence, $x_i^*$ satisfies $P \cdot A\cdot F'(x)=C$, i.e., $AF'(x)=\frac{C}{P}$, where A denotes $A_H$ for a high-incentive province and $A_L$ for a low-incentive province.

When each of the N provinces independently decides on its water use, the marginal cost of water use would be $\frac{C}{N}$ as a result of cost-sharing with others.
Hence, the optimal water use in province i at period t, denoted as $\hat x_i^*$, satisfies $P \cdot A \cdot F'(x_{it})=\frac{C}{N}$, i.e., $A \cdot F'(x)=\frac{C}{P \cdot N}$.
Since $F'$ is monotonically decreasing, $\hat x_{it}^*>x_i^*$.

When the water quota would constrain future water use, the dynamic optimization problem of province i is shown as follows. In $t=1,2,\cdots$, there would be no relevant cost when the quota is bound that each province takes ongoing costs of $\frac{P \cdot Q}{N}$ regardless of the allocation. Therefore, it is sufficient to consider only the total water quota is less than total water use in Case 2 since a ``too large'' quota doesn't make sense for ecological policies.

$max  \quad P \cdot A \cdot F(x_{i,0})-\frac{C \cdot \begin{matrix} \sum x_{i,0} + x_{-i,0} \end{matrix}}{N}+\beta P \cdot A \cdot F(x_{i,1})+\beta^2 P \cdot A \cdot F(x_{i,2})+...$

$=P \cdot A \cdot F(x_{i,0})-C \cdot \frac{x_{i,0} + \begin{matrix} \sum x_{-i,0} \end{matrix}}{N}+\frac{\beta}{1-\beta} P \cdot A \cdot F(Q \cdot \frac{x_{i,0}}{x_{i,0} + \begin{matrix} \sum x_{-i,0} \end{matrix}})$

First-order condition: $P \cdot A \cdot F'(x_{i,0})-\frac{C}{N}+\frac{\beta}{1-\beta}[P \cdot A \cdot f(Q \cdot \frac{x_{i,0}}{x_{i,0} + \begin{matrix} \sum x_{-i,0} \end{matrix}}) \cdot Q \cdot \frac{\begin{matrix} \sum x_{-i,0} \end{matrix}}{(x_{i,0}+\begin{matrix} \sum  x_{-i,0} \end{matrix})^2}]=0$

where $f(\cdot)$ is the differential function of $F(\cdot)$.

The optimal water use in province i at t=0 $\widetilde x_{i,0}^*$ satisfies $P \cdot A \cdot F'(x_{i,0})=\frac{C}{N}-\frac{\beta}{1-\beta} \cdot P \cdot A \cdot f(Q \cdot \frac{x_{i,0}}{x_{i,0} + \begin{matrix} \sum x_{-i,0} \end{matrix}}) \cdot Q \cdot \frac{\begin{matrix} \sum x_{-i,0} \end{matrix}}{(x_{i,0} + \begin{matrix} \sum x_{-i,0} \end{matrix})^2}$,
i.e.,
$A \cdot F'(x_{i,0})=\frac{C}{P \cdot N} - \frac{\beta}{1-\beta} \cdot A \cdot f(Q \cdot \frac{x_{i,0}}{x_{i,0} + \begin{matrix} \sum x_{-i,0} \end{matrix}}) \cdot Q \cdot \frac{\begin{matrix} \sum x_{-i,0} \end{matrix}}{(x_{i,0} + \begin{matrix} \sum x_{-i,0} \end{matrix})^2}$.

Since $F'>0$ and $F''<0$, $\widetilde x_i^*>\hat x_i^*>x_i^*$, taken others' water use $x_{-i,0}$ as given. Since the provincial water use decisions are exactly symmetric, total water use would increase when each province has higher incentives for current water use.

%Proposition 1
Proof of Proposition 1:

Because $F'>0$ and $F''(x)<0$ is monotonically decreasing, based on a comparison of costs and benefits for stakeholders (provinces) in the three cases,

$\widetilde x_i^*>\hat x_i^*>x_i^*$.

The result of $\hat x_i^*>x_i^*$ indicates that individual rationality would deviate from collective rationality under unclear property rights where a water user is fully responsible for the relevant costs. The result of $\hat x_i^*>x_i^*$

The difference between $ x_i^*$ and $\hat x_i^*$ stems from two parts: the effect of the marginal returns and the effect of the marginal costs. First, the ``shadow value'' provides additional marginal returns of water use in $t$ = 0, which increases the incentives of water overuse by encouraging bargaining for a larger quota. Second, the future cost of water use would be degraded from $\frac{P}{N}$ to an irrelevant cost.

%Proposition 2
Proof of Proposition 2:

Since $A_H>A_L$, $F'(x_H)<F'(x_L)$,
Eq.(xxx) %此处引用:$AF'(x_{i,0})=\frac{C}{P \cdot N} - \frac{\beta}{1-\beta} \cdot A \cdot f(Q \cdot \frac{x_{i,0}}{\begin{matrix} x_{i,0} + \sum x_{-i,0} \end{matrix}}) \cdot Q \cdot \frac{\begin{matrix} \sum x_{-i,0} \end{matrix}}{(\begin{matrix} x_{i,0} + \sum x_{-i,0} \end{matrix})^2}$
implies a positive relation between $x_{i0}$ and A, when $\beta, P, C, Q$, and other provinces' water use are taken as given.

The difference between $\widetilde x_i^*$ and $\hat x_i^*$ (i.e., $\frac{\beta}{1-\beta} \cdot A \cdot f(Q \cdot \frac{x_{i,0}}{x_{i,0} + \begin{matrix} \sum x_{-i,0} \end{matrix}}) \cdot Q \cdot \frac{\begin{matrix} \sum x_{-i,0} \end{matrix}}{(x_{i,0} + \begin{matrix} \sum x_{-i,0} \end{matrix})^2}$) represents the incentive of water overuse derived from an expectation of water quota allocation. The incentive of water overuse increases by A.


\section{Model extensions}\label{secS5}
\renewcommand{\thefigure}{D\arabic{figure}}
\renewcommand{\thetable}{D\arabic{table}}
\setcounter{figure}{0}
\setcounter{table}{0}
%Model extension部分
Further analysis of the general economic model

Using the general economic model (see the Methods section in the main text), we also explored the response of stakeholders to water quota policies. We considered two additional scenarios for stakeholders, one that considered technology growth and one that considered different valuations through time (via the discount rate) of economic benefits and ecological costs. In the following scenarios, the cost is assumed to be untransferable, which could be fully allocated to the one incurring the water use. Explaining plausible scenarios for these stakeholders will help us better understand the causes of the water overuse and potential solutions. We argue that the water overuse remains robust even if a complete and equitable system.

% technology growth
    \begin{case_appendix}Forward-looking decentralized decision, taken ecology cost into considerations

    Even if the negative externality of water overuse is eliminated by``fair" ecology cost of $\frac{x_{i,0}}{x_{i,0} + \begin{matrix} \sum x_{-i,0} \end{matrix}} \cdot Q \cdot C$, it is possible that the future growth opportunities and ``remote" ecological costs provide enough incentive for the sprint.  The water overuse has the value of future economical benefits by slacking the water use constraint in the future. The heterogeneous production efficiency is omitted in this section, and we set A=1.

(a) technology growth

Assume that there is an exogenous technology growth rate of g in the scenario of $N$ provinces bargaining for water use under total quota $Q$, with unit price of output $P$, unit cost $C$, and discount factor $\beta$. For simplicity, consider a finite-period water use optimization:

$ max \quad P \cdot (1+g)^t ln(1+x_{i,0})-\frac{C}{N}+\beta^t \begin{matrix} \sum_{t=1}^T [P \cdot (1+g)^t ln(x_{i,t}+1)-C \cdot x_{i,t}] \end{matrix}$

$s.t. \quad x_{i,t} \leq Q \cdot \frac{x_{i,0}}{x_{i,0} + \begin{matrix} \sum x_{-i,0} \end{matrix}} \quad for \quad \forall t$

We depict the relationship of multi-period profit and water use $x_{i,0}$ in different horizons in Figure 4%%此处引用图片
,and thus find out the optimal water use pattern under technology growth. The higher marginal output of water might create enough incentive to set off the untransferable cost since a higher allocated quota provides growth option value. As the provincial decision is under a longer horizon, there is a greater sprint effect due to higher accumulated yield and relatively tighter water use constraints over time.

% \begin{figure}[H]
%     \centering
%     \includegraphics[width=0.7\linewidth]{Fig3.jpg}
% \end{figure}

% \begin{fig}
% Multi-period optimization of optimal water use under technology growth

% \tiny Notes: The figure depicts the relationship of multi-period benefits of province $i$ and water use under Case 3 with technology growth. Assume F(x)=ln(1+x), N=8, P=1, C=0.5, $\beta$=0.7, g=0.2, and Q=8.
% \end{fig}

(b) Economic benefits and ``remote'' ecological costs with different discount factors

Assume that there is a high discount rate for economical benefits and a low discount rate for ecological costs, in the scenario of $N$ provinces bargaining for water use under total quota $Q$, with unit price of output $P$, unit cost $C$, discount factor $\beta^{economy}$ and $\beta^{ecology}$ ($\beta^{economy} > \beta^{ecology}$). For simplicity, consider the following finite-period water use optimization, noting the water use of province $i$ at period $t$:

$ max \quad P \cdot ln(1+x_{i,0})-\frac{C}{N}+\beta_1^t \begin{matrix} \sum_{t=1}^T [P \cdot ln(x_{i,t}+1)]  \end{matrix} - \beta_2^t \begin{matrix} \sum_{t=1}^T [C \cdot x_{i,t}] \end{matrix}$

$s.t. \quad x_{i,t} \leq Q \cdot \frac{x_{i,0}}{x_{i,0} + \begin{matrix} \sum x_{-i,0} \end{matrix}} \quad for \quad \forall t$

We depict the relationship of multi-period net income and water use $x_{i,0}$ in different horizons in Figure 4%此处引用图片
, and thus find out the optimal water use pattern under ``remote'' ecological costs. The higher discounted ecological costs might create enough incentive to set off the untransferable cost. As the provincial decision is under a longer horizon, there is a greater sprint effect due to a higher accumulated yield.

% \begin{figure}[H]
%     \centering
%     \includegraphics[width=0.7\linewidth]{Fig4.jpg}
% \end{figure}

% \begin{fig}
% Multi-period optimization of water use under ``remote'' ecological cost

% \tiny Notes: The figure depicts the relationship of multi-period benefits of province $i$ and water use under Case 3 with ``remote'' ecological cost. Assume F(x)=ln(1+x), N=8, P=1, C=0.5, $\beta_{economy}$=0.7, $\beta_{ecology}$=0.3, and Q=8.
% \end{fig}

\end{case_appendix}


\end{document}
